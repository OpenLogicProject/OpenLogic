% Part: computability
% Chapter: recursive-functions
% Section: pr-relations

\documentclass[../../include/open-logic-section]{subfiles}

\begin{document}

\olfileid{cmp}{rec}{prr}
\olsection{Primitive Recursive Relations}


\begin{defn}
A relation $R(\vec x)$ is said to be primitive recursive if its characteristic
function,
\[
\Char{R}(\vec x) = \left\{
  \begin{array}{ll}
  1 & \mbox{if $R(\vec x)$} \\
  0 & \mbox{otherwise}
  \end{array}
\right.
\]
is primitive recursive.
\end{defn}

In other words, when one speaks of a primitive
recursive relation $R(\vec x)$, one is referring to a relation of the
form $\Char{R}(\vec x) = 1$, where $\Char{R}$ is a primitive recursive
function which, on any input, returns either 1 or 0. For example, the
relation $\fn{Zero}(x)$, which holds if and only if $x = 0$,
corresponds to the function $\Char{\fn{Zero}}$, defined using primitive
recursion by
\[
\Char{\fn{Zero}}(0) = 1, \quad \Char{\fn{Zero}}(x+1) = 0.
\]

It should be clear that one can compose relations with other primitive
recursive functions. So the following are also primitive recursive:
\begin{enumerate}
\item The equality relation, $x = y$, defined by $\fn{Zero}(\left|x -
  y\right|)$
\item The less-than relation, $x \leq y$, defined by $\fn{Zero}(x
  \tsub y)$
\end{enumerate}
Furthermore, the set of primitive recursive relations is closed under
boolean operations:
\begin{enumerate}
\item Negation, $\lnot P$
\item Conjunction, $P \land Q$
\item Disjunction, $P \lor Q$
\item Implication, $P \lif Q$
\end{enumerate}
are all primitive recursive, if $P$ and $Q$ are.

One can also define relations using bounded quantification:
\begin{enumerate}
\item Bounded universal quantification: if $R(x, \vec z)$ is a
primitive recursive relation, then so is the relation
\[
\lforall[x < y][R(x, \vec z)]
\]
which holds if and only if $R(x,\vec z)$ holds for every $x$ less than
$y$.
\item Bounded existential quantification: if $R(x, \vec z)$ is a
primitive recursive relation, then so is
\[
\lexists[x < y][R(x, \vec z)].
\]
\end{enumerate}
By convention, we take expressions of the form $\lforall[x < 0][R(x,\vec
z)]$ to be true (for the trivial reason that there \emph{are} no $x$
less than $0$) and $\lexists[x < 0][R(x,\vec z)]$ to be false. A universal
quantifier functions just like a finite product; it can also be
defined directly by
\[
g(0,\vec z) = 1, \quad g(y+1,\vec z) = \Char{\fn{and}}(g(y,\vec
z),\Char{R}(y,\vec z)).
\]
Bounded existential quantification can similarly be defined using
$\fn{or}$. Alternatively, it can be defined from bounded universal
quantification, using the equivalence, $\lexists[x < y][!A(x)] \liff \lnot
\lforall[x < y][\lnot !A(x)]$. Note that, for example, a bounded quantifier
of the form $\lexists[x \leq y]$ is equivalent to $\lexists[x < y+1]$.

Another useful primitive recursive function is:
\begin{enumerate}
\item The conditional function, $\fn{cond}(x,y,z)$, defined by
\[
\fn{cond}(x,y,z) = \left\{ \begin{array}{ll}
  y & \mbox{if $x$ = 0} \\
  z & \mbox{otherwise}
\end{array}\right.
\]
\end{enumerate}
This is defined recursively by
\[
\fn{cond}(0,y,z) = y, \quad \fn{cond}(x+1,y,z) = z.
\]
One can use this to justify:
\begin{enumerate}
\item Definition by cases: if $g_0(\vec x), \dots, g_m(\vec x)$ are
functions, and $R_1(\vec x), \dots, R_{m-1}(\vec x)$ are relations, then
the function $f$ defined by
\[
f(\vec x) = \left\{\begin{array}{ll}
    g_0(\vec x) & \mbox{if $R_0(\vec{x})$} \\
    g_1(\vec x) & \mbox{if $R_1(\vec{x})$ and not $R_0(\vec{x})$} \\
    \vdots & \\
    g_{m-1}(\vec x) & \mbox{if $R_{m-1}(\vec{x})$ and none of the
      previous hold}
    \\
    g_m(\vec x) & \mbox{otherwise}
\end{array}\right.
\]
is also primitive recursive.
\end{enumerate}
When $m = 1$, this is just the function defined by
\[
f(\vec x) = \fn{cond}(\Char{\lnot R_0}(\vec x),g_0(\vec x),g_1(\vec
 x)).
\]
For $m$ greater than $1$, one can just compose definitions of this
form.

\end{document}
