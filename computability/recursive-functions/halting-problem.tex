% Part: computability
% Chapter: recursive-functions
% Section: halting-problem

\documentclass[../../include/open-logic-section]{subfiles}

\begin{document}

\olfileid{cmp}{rec}{hlt}
\olsection{The Halting Problem}

The \emph{halting problem} in general is the problem of deciding,
given the specification~$e$ (e.g., program) of a computable function
and a number~$n$, whether the computation of the function on input~$n$
halts, i.e., produces a result.  Famously, Alan Turing proved that
this problem itself cannot be solved by a computable function, i.e.,
the function
\[
h(e, n) =
\begin{cases}
1 & \text{if computation $e$ halts on input $n$}\\
0 & \text{otherwise,}
\end{cases}
\]
is not computable.

In the context of partial recursive functions, the role of the
specification of a program may be played by the index~$e$ given in
Kleene's normal form theorem.  If $f$ is a partial recursive function,
any $e$~for which the equation in the normal form theorem holds, is an
index of~$f$. Given a number~$e$, the normal form theorem states that
\[
\cfind{e}(x) \simeq U(\mu s \; T(e, x, s))
\]
is partial recursive, and for every partial recursive $f\colon \Nat
\to \Nat$, there is an $e \in \Nat$ such that $\cfind{e}(x) \simeq
f(x)$ for all~$x \in \Nat$.  In fact, for each such $f$ there is not
just one, but infinitely many such~$e$.  The \emph{halting function}~$h$
is defined by
\[
h(e, x) =
\begin{cases}
1 & \text{if $\cfind{e}(x) \defined$}\\
0 & \text{otherwise.}
\end{cases}
\]
Note that $h(e, x) = 0$ if $\cfind{e}(x) \undefined$, but also
when~$e$ is not the index of a partial recursive function at all.

\begin{thm}
\ollabel{thm:halting-problem}
The halting function~$h$ is not partial recursive.
\end{thm}

\begin{proof}
If $h$
were partial recursive, we could define
\[
d(y) =
\begin{cases}
1 & \text{if $h(y, y) = 0$}\\
\umin{x}{x \neq x} & \text{otherwise.}
\end{cases}
\]
From this definition it follows that
\begin{enumerate}
\item $d(y) \defined$ iff $\cfind{y}(y) \undefined$ or $y$ is
  not the index of a partial recursive function.
\item $d(y) \undefined$ iff $\cfind{y}(y) \defined$.
\end{enumerate}
If $h$ were partial recursive, then $d$ would be partial recursive as
well.  Thus, by the Kleene normal form theorem, it has an index
$e_d$. Consider the value of $h(e_d, e_d)$. There are two
possible cases, 0 and 1.
\begin{enumerate}
\item If $h(e_d, e_d) = 1$ then $\cfind{e_d}(e_d) \defined$.  But
  $\cfind{e_d} \simeq d$, and $d(e_d)$ is defined iff $h(e_d, e_d) = 0$.
  So $h(e_d, e_d) \neq 1$.
\item If $h(e_d, e_d) = 0$ then either $e_d$ is not the index of a
  partial recursive function, or it is and $\cfind{e_d}(e_d)
  \undefined$. But again, $\cfind{e_d} \simeq d$, and $d(e_d)$ is undefined
  iff $\cfind{e_d}(e_d) \defined$.
\end{enumerate}
The upshot is that $e_d$ cannot, after all, be the index of a partial
recursive function.  But if $h$ were partial recursive, $d$ would be
too, and so our definition of $e_d$ as an index of it would be
admissible.  We must conclude that $h$ cannot be partial recursive.
\end{proof}

\end{document}


