% Part: computability
% Chapter: computability-theory
% Section: russells-paradox

\documentclass[../../include/open-logic-section]{subfiles}

\begin{document}

\olfileid{cmp}{thy}{rus}
\olsection{Comparison with Russell's Paradox}

It is instructive to compare and contrast the arguments in
this section with Russell's paradox:
\begin{enumerate}
\item Russell's paradox: let $S = \Setabs{x}{x \notin x}$. Then $x
  \in S$ if and only if $x \notin S$, a contradiction.
  
  \emph{Conclusion:} There is no such set~$S$. Assuming the existence of a
  ``set of all sets'' is inconsistent with the other axioms of set
  theory.

\item A modification of Russell's paradox: let $F$ be the ``function''
  from the set of all functions to $\{ 0, 1 \}$, defined by
  \[
  F(f) = 
  \begin{cases}
    1 & \text{if $f$ is in the domain of $f$, and $f(f) = 0$} \\
    0 & \text{otherwise}
  \end{cases}
  \]
  A similar argument shows that $F(F) = 0$ if and only if $F(F) = 1$,
  a contradiction.

  \emph{Conclusion:} $F$ is not a function. The ``set of all
  functions'' is too big to be the domain of a function.
  
\item The diagonalization argument: let $f_0$, $f_1$, \dots be the
  enumeration of the partial computable functions, and let $G \colon \Nat \to
  \{ 0, 1 \}$ be defined by
  \[
  G(x) = 
  \begin{cases}
    1 & \text{if $f_x(x)\downarrow = 0$} \\
    0 & \text{otherwise}
  \end{cases}
  \]
  If $G$ is computable, then it is the function $f_k$ for some
  $k$. But then $G(k) = 1$ if and only if $G(k) = 0$, a contradiction.

  \emph{Conclusion:} $G$ is not computable. Note that according to the
  axioms of set theory, $G$ is still a function; there is no paradox
  here, just a clarification.
\end{enumerate}

That talk of partial functions, computable functions,
partial computable functions, and so on can be confusing. The set of
all partial functions from $\Nat$ to $\Nat$ is a big collection of
objects. Some of them are total, some of them are computable, some are
both total and computable, and some are neither. Keep in mind that
when we say ``function,'' by default, we mean a total function. Thus we
have:
\begin{enumerate}
\item computable functions
\item partial computable functions that are not total
\item functions that are not computable
\item partial functions that are neither total nor computable
\end{enumerate}
To sort this out, it might help to draw a big square representing all
the partial functions from $\Nat$ to $\Nat$, and then mark off two
overlapping regions, corresponding to the total functions and the
computable partial functions, respectively. It is a good exercise to
see if you can describe an object in each of the resulting regions in
the diagram.

\end{document}
