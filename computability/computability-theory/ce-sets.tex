% Part: computability
% Chapter: computability-theory
% Section: ce-sets

\documentclass[../../include/open-logic-section]{subfiles}

\begin{document}

\olfileid{cmp}{thy}{ces}
\olsection{Computably Enumerable Sets}

\begin{defn}
A set is \emph{computably enumerable} if it is empty or the range of a
computable function.
\end{defn}

\begin{history}
Computably enumarable sets are also called \emph{recursively
  enumerable} instead. This is the original terminology, and today
both are commonly used, as well as the abbreviations ``c.e.'' and
``r.e.'' 
\end{history}

\begin{explain}
You should think about what the definition means, and why the
terminology is appropriate. The idea is that if $S$ is the range of
the computable function~$f$, then
\[
S = \{ f(0), f(1), f(2), \dots \},
\]
and so $f$ can be seen as ``enumerating'' the elements of $S$. Note
that according to the definition, $f$ need not be an increasing
function, i.e., the enumeration need not be in increasing order. In
fact, $f$ need not even be injective, so that the constant function
$f(x) = 0$ enumerates the set $\{ 0 \}$.
\end{explain}

Any computable set is computably enumerable. To see this, suppose
$S$~is computable. If $S$ is empty, then by definition it is
computably enumerable. Otherwise, let $a$ be any element of
$S$. Define $f$ by
\[
f(x) = 
\begin{cases}
x & \text{if $\Char{S}(x) = 1$} \\
a & \text{otherwise.}
\end{cases}
\]
Then $f$ is a computable function, and $S$ is the range of~$f$.

\end{document}
