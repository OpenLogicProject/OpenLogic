% Part: computability
% Chapter: lambda-calculus
% Section: reduction

\documentclass[../../include/open-logic-section]{subfiles}

\begin{document}

\olfileid{cmp}{lam}{red}
\olsection{Reduction of Lambda Terms}

What can one do with lambda terms? Simplify them. If $M$ and $N$ are
any lambda terms and $x$ is any variable, we can use $\Subst{M}{N}{x}$ to
denote the result of substituting $N$ for~$x$ in~$M$, after renaming
any bound variables of $M$ that would interfere with the free
variables of~$N$ after the substitution. For example,
\[
\Subst{(\lambd[w][xxw])}{yyz}{x} = \lambd[w][(yyz)(yyz)w].
\]

\begin{digress}
Alternative notations for substitution are $[N/x]M$, $M[N/x]$, and
also $M[x/N]$. Beware!
\end{digress}

Intuitively, $(\lambd[x][M])N$ and $\Subst{M}{N}{x}$ have the same
meaning; the act of replacing the first term by the second is called
$\beta$-conversion. More generally, if it is possible convert a
term $P$ to~$P'$ by $\beta$-conversion of some subterm, one says
\emph{$P$ $\beta$-reduces to $P'$ in one step}. If $P$ can be
converted to~$P'$ with any number of one-step reductions (possibly
none), then $P$ \emph{$\beta$-reduces} to~$P'$. A term that cannot be
$\beta$-reduced any further is called \emph{$\beta$-irreducible}, or
$\beta$-normal. I will say ``reduces'' instead of ``$\beta$-reduces,''
etc., when the context is clear.

Let us consider some examples.
\begin{enumerate}
\item We have
\begin{align*}
(\lambd[x][xxy]) \lambd[z][z] & \redone (\lambd[z][z])(\lambd[z][z]) y \\
& \redone (\lambd[z][z]) y \\
& \redone y
\end{align*}
\item ``Simplifying'' a term can make it more complex:
\begin{align*}
(\lambd[x][xxy])(\lambd[x][xxy]) & \redone (\lambd[x][xxy])(\lambd[x][xxy])y \\
& \redone (\lambd[x][xxy])(\lambd[x][xxy])yy \\
& \redone \dots
\end{align*}
\item It can also leave a term unchanged:
\[
(\lambd[x][xx])(\lambd[x][xx]) \redone (\lambd[x][xx])(\lambd[x][xx])
\]
\item Also, some terms can be reduced in more than one way; for
  example,
\[
(\lambd[x][(\lambd[y][yx]) z)] v \redone (\lambd[y][yv]) z
\]
by contracting the outermost application; and
\[
(\lambd[x][(\lambd[y][yx]) z)] v \redone (\lambd[x][zx]) v
\]
by contracting the innermost one. Note, in this case, however, that
both terms further reduce to the same term, $zv$.
\end{enumerate}

The final outcome in the last example is not a coincidence, but rather
illustrates a deep and important property of the lambda calculus, known as the
``Church-Rosser property.''

\end{document}
