% Part: first-order-logic
% Chapter: beyond
% Section: modal-logics

\documentclass[../../include/open-logic-section]{subfiles}

\begin{document}

\olfileid{fol}{byd}{mod}

\olsection{Modal Logics}

Consider the following example of a conditional sentence:
\begin{quote}
  If Jeremy is alone in that room, then he is drunk and naked and
  dancing on the chairs.
\end{quote}
This is an example of a conditional assertion that may be materially
true but nonetheless misleading, since it seems to suggest that there
is a stronger link between the antecedent and conclusion other than
simply that either the antecedent is false or the consequent
true. That is, the wording suggests that the claim is not only true in
this particular world (where it may be trivially true, because Jeremy
is not alone in the room), but that, moreover, the conclusion
\emph{would have} been true \emph{had} the antecedent been true. In
other words, one can take the assertion to mean that the claim is true
not just in this world, but in any ``possible'' world; or that it is
\emph{necessarily} true, as opposed to just true in this particular
world.

Modal logic was designed to make sense of this kind of necessity. One
obtains modal propositional logic from ordinary
propositional logic by adding a box operator; which is to say, if
$!A$ is a !!{formula}, so is $\Box !A$.  Intuitively, $\Box !A$
asserts that $!A$ is \emph{necessarily} true, or true in any possible
world. $\Diamond !A$ is usually taken to be an abbreviation for
$\lnot \Box \lnot !A$, and can be read as asserting that $!A$ is
\emph{possibly} true. Of course, modality can be added to predicate
logic as well.

Kripke !!{structure}s can be used to provide a semantics for modal
logic; in fact, Kripke first designed this semantics with modal logic
in mind. Rather than restricting to partial orders, more generally one
has a set of ``possible worlds,'' $P$, and a binary ``accessibility''
relation $\Atom{R}{x,y}$ between worlds. Intuitively, $\Atom{R}{p,q}$
asserts that the world~$q$ is compatible with~$p$; i.e., if we are
``in'' world~$p$, we have to entertain the the possibility that the
world could have been like~$q$.

Modal logic is sometimes called an ``intensional'' logic, as opposed
to an ``extensional'' one. The intended semantics for an extensional
logic, like classical logic, will only refer to a single world, the
``actual'' one; while the semantics for an ``intensional'' logic
relies on a more elaborate ontology. In addition to !!{structure}ing
necessity, one can use modality to !!{structure} other linguistic
constructions, reinterpreting $\Box$ and $\Diamond$ according to the
application. For example:
\begin{enumerate}
\item In provability logic, $\Box !A$ is read ``$!A$ is provable''
  and $\Diamond !A$ is read ``$!A$ is consistent.''
\item In epistemic logic, one might read $\Box !A$ as ``I know
  $!A$'' or ``I believe $!A$.''
\item In temporal logic, one can read $\Box !A$ as ``$!A$ is always
  true'' and $\Diamond !A$ as ``$!A$ is sometimes true.''
\end{enumerate}

One would like to augment logic with rules and axioms dealing with
modality. For example, the system $\Log{S4}$ consists of the ordinary
axioms and rules of propositional logic, together with the following
axioms:
\begin{align*}
& \Box (!A \lif !B) \lif (\Box !A \lif \Box !B)\\
& \Box !A \lif !A \\
& \Box !A \lif \Box \Box !A
\intertext{as well as a rule, ``from $!A$ conclude $\Box !A$.''
  $\Log{S5}$ adds the following axiom:}
& \Diamond !A \lif \Box \Diamond !A
\end{align*}
Variations of these axioms may be suitable for different applications;
for example, S5 is usually taken to characterize the notion of logical
necessity. And the nice thing is that one can usually find a semantics
for which the proof system is sound and complete by restricting the
accessibility relation in the Kripke !!{structure}s in natural
ways. For example, $\Log{S4}$ corresponds to the class of Kripke
!!{structure}s in which the accessibility relation is reflexive and
transitive. $\Log{S5}$ corresponds to the class of Kripke
!!{structure}s in which the accessibility relation is {\em universal},
which is to say that every world is accessible from every other; so
$\Box !A$ holds if and only if $!A$ holds in every world.

\end{document}
