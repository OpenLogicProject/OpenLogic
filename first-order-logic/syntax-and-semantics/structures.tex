% Part: first-order-logic
% Chapter: syntax-and-semantics
% Section: Structures

\documentclass[../../include/open-logic-section]{subfiles}

\begin{document}

\olfileid{fol}{syn}{str}
\olsection{Structures for First-order Languages}

\begin{explain}
First-order languages are, by themselves, \emph{uninterpreted:} the
constants, functions, and predicates have no specific meaning attached
to them.  Meanings are given by specifying a \emph{structure}
(sometimes called an \emph{interpretation}).  It specifies the
\emph{domain}, i.e., the objects which the constats pick out, the
functions operate on, and the quantifiers range over. In addition, it
specifies which constants pick out which objects, how a function maps
objects to objects, and which objects the predicates apply to.
Structures are the basis for \emph{semantic} notions in logic, e.g.,
the notion of consequence, validity, satisfiablity.
\end{explain}

\begin{defn}[Structure]
A \emph{structure}, $\Struct M$, for a language $\Lang{L}$ of
first-order logic consists of the following elements:
\begin{enumerate}
\item \emph{Domain:} a non-empty set, $\Domain M$ 
\item \emph{Name assignment:} for each !!{constant} $c$ of
  $\Lang{L}$, an element $\Assign{c}{M} \in \Domain M$
\item \emph{Relations:} for each $n$-place !!{predicate} $R$ of
  $\Lang{L}$ (other than $\eq$), an $n$-ary relation $\Assign{R}{M}
  \subseteq \Domain{M}^n$
\item \emph{Functions:} for each $n$-place !!{function} $f$ of
  $\Lang{L}$, an $n$-ary function $\Assign{f}{M} \colon
  \Domain{M}^n \to \Domain{M}$
  
\end{enumerate}
\end{defn}

\begin{explain}
Recall that a term is \emph{closed} if it is closed or if it is a
function of constants only (without !p{variable}, whether free or bound).
\end{explain}

\begin{defn}[Denotation of closed terms]
If $t$ is a closed term of the langage~$\Lang L$ and $\Struct M$ is a
!!{structure} for~$\Lang L$, the \emph{denotation}~$\Value{t}{M}$ is
defined as follows:
\begin{enumerate}
\item If $t$ is just the constant $c$, then $\Value{c}{M} = \Assign{c}{M}$.
\item If $t$ is of the form $\Atom{f}{t_1, \ldots, t_n}$, then
  $\Value{t}{M}$ is $\Assign{f}{M}(\Value{t_1}{M}, \ldots,
  \Value{t_n}{M})$.
\end{enumerate}
\end{defn}

\begin{defn}[Covered !!{structure}]
A !!{structure} is \emph{covered} if every element of the domain is the
denotation of some closed term.
\end{defn}

\begin{ex}
Let ~$\Lang L$ be the language with !p{constant} $\mathsf{Zero},
\mathsf{One}, \mathsf{Two}, \ldots$, the binary predicate symbols $=$
and $<$, and the binary !p{function} $+$ and $\times$.  Then a
!!{structure} $\Struct M$ for $\Lang L$ is the one with domain
$\Domain M = \{0, 1, 2, \ldots \}$ and name assignment
$\Assign{\mathsf{Zero}}{M} = 0$, $\Assign{\mathsf{One}}{M} = 1$,
$\Assign{\mathsf{Two}}{M} = 2$, and so forth. For the binary relation
symbol $<$, the set $\Assign{<}{M}$ is the set of all pairs $\langle
c_1, c_2 \rangle \in \Domain{M}^2$ such that the integer $c_1$ is less
than the integer $c_2$: for example, $\langle 1, 3 \rangle \in
\Value{<}{M}$ but $\langle 2, 2 \rangle \notin \Value{<}{M}$. For the
binary !!{function} $+$, define $\Assign{+}{M}$ in the usual way ---
for example, $\Assign{+}{M}(2,3)$ maps to $5$, and similarly for the
binary !!{function} $\times$. Hence, the denotation of $\mathsf{Four}$
is just 4, and the denotation of $\times(\mathsf{Two},
+(\mathsf{Three},\mathsf{Zero}))$ (or in infix notation, $\mathsf{Two}
\times (\mathsf{Three} + \mathsf{Zero})$ ) is
\begin{align*} 
\Value{\times(\mathsf{Two}, +(\mathsf{Three},\mathsf{Zero})}{M}&=
\Assign{\times}{M}(\Value{\mathsf{Two}}{M}, \Value{\mathsf{Two}, 
+(\mathsf{Three}, \mathsf{Zero})}{M})\\
&= \Assign{\times}{M}(\Value{\mathsf{Two}}{M}, \Assign{+}{M}(\Value{\mathsf{Three}}{M}, 
\Value{\mathsf{Zero}}{M})) \\
&= \Assign{\times}{M}(\Assign{\mathsf{Two}}{M}, \Assign{+}{M}(\Assign{\mathsf{Three}}{M}, 
\Assign{\mathsf{Zero}}{M})) \\
&= \Assign{\times}{M}(2, \Assign{+}{M}(3, 0)) \\
&= \Assign{\times}{M}(2, 3) \\
&= 5
\end{align*}
\end{ex}

\begin{digress}
The stipulations we make as to what counts as a !!{structure} impact
our logic. For example, the choice to prevent empty domains ensures,
given the usual account of satisfaction (or truth) for quantified
sentences, that $\lexists[x][(!A(x) \lor \lnot !A(x))]$ is valid ---
that is, a logical truth. And the stipulation that all names must
refer to an object in the domain ensures that the existential
generalization is a sound pattern of inference: $!A(a)$, therefore
$\lexists[x][!A(x)]$. If we allowed names to refer outside the domain,
or to not refer, then we would be on our way to a \emph{free logic},
in which existential generalization requires an additional premise:
$!A(a)$ and $\lexists[x][\eq[x][a]]$, therefore $\lexists[x][!A(x)]$.
\end{digress}

\end{document}
