% Part: first-order-logic
% Chapter: syntax-and-semantics
% Section: main-connective

\documentclass[../../include/open-logic-section]{subfiles}

\begin{document}

\olfileid{fol}{syn}{mai}

\olsection{Main Connective of a Formula}

\begin{explain}
It is often useful to talk about the last operator used in
constructing a "{formula}~$!A$.  This operator is called the \emph{main
  connective} of ~$!A$. Intuitively, it is the ``outermost'' operator
of $!A$. For example, the main connective of $(\lnot !A)$ is $\lnot$,
the main connective of $(!A \lor !B)$ is $\lor$, etc.
\end{explain}


\begin{defn}[!S{main connective}]
The \emph{!!{main connective}} of a !!{formula}~$!A$ is induictively
defined as follows:
\begin{enumerate}
\item Atomic !p{formula} have no main connective.
\item If $!A$ is of the form $!B \land !C$, then the main connective of
  $!A$ is $\land$.
\item If $!A$ is of the form $!B \lor !C$, then the main connective of
  $!A$ is $\lor$.
\item If $!A$ is of the form $!B \lif !C$, then the main connective of
  $!A$ is $\lif$.
\item If $!A$ is of the form $!B \liff !C$, then the main connective of
  $!A$ is $\liff$.
\end{enumerate}
\end{defn}

We give names to !p{formula} with each type of main connective:

\begin{figure}[!h]
\centering
\begin{tabular}{| c | c | c |}
\hline
Main connective & Type of !!{formula} & Example\\
\hline
$\lnot$ & negation & $\lnot !A$ \\
$\land$ & conjunction & $!A \land !B$ \\
$\lor$ & disjunction & $!A \lor !B$ \\
$\lif$ & !!{conditional} & $!A \lif !B$ \\
$\liff$ & !!{biconditional} & $!A \liff !B$ \\
$\lforall[][]$ & universal & $\lforall[x][!A]$ \\
$\lexists[][]$ & existential & $\lexists[x][!A]$\\ \hline
\end{tabular}
\end{figure}

\end{document}
