% Part: first-order-logic
% Chapter: syntax-and-semantics
% Section: syntax

\documentclass[syntax-and-semantics]{subfiles}

\begin{document}

\section{Syntax}

%This still needs to be macro'd.

The basic language of first-order logic contains

\begin{itemize}
\item Logical connectives: $\lnot, \land, \lor, \forall, \exists, \rightarrow$
\item The propositional constant for falsity, $\bot$
\item The two-place equality predicate, $=$
\item A denumerable set of $n$-place predicates for each $n>0$: $A^n_0, A^n_1, A^n_2,\ldots$
\item A denumerable set of constants: $f^0_0, f^0_1, f^0_2,\ldots$
\item A denumerable set of $n$-place function symbols for each $n>0$: $f^n_0, f^n_1, f^n_2,\ldots$
\item A denumerable set of variables: $v_0, v_1, v_2,\ldots$
\end{itemize}

% Explanation of using ~ vs. $\lnot$, ampersand versus wedge, arrow vs. horseshoe, quantifiers, etc. compared to LPL and Logic Book
% Perhaps an explanation for "wordy" of 0-place function symbols counting as constants

\begin{defn}[Term]
The \emph{terms} of the language of first-order logic are defined inductively:
\begin{enumerate}
\item Variables and constants are terms.
\item If $f$ is an $n$-place function symbol and $t_1,\ldots,t_n$ are terms, then $f(t_1 \ldots t_n)$ is a term.
\item Nothing else is a term.
\end{enumerate}
A term containing no variables is a \emph{closed term}.
\end{defn}

\begin{defn}[Formula]
The \emph{formulae} (or \emph{well-formed formulae}, abbreviated wff) of the language of first-order logic are defined inductively:
\begin{enumerate}
\item If $R$ is an $n$-place predicate and $t_1,\ldots,t_n$ are terms, then $R(t_1\ldots t_n)$ is an (atomic) formula.
\item If $t_1$ and $t_2$ are terms, then $=(t_1,t_2)$ is an atomic formula.
\item If $!A$ is a formula, then $(\lnot !A)$ is a formula.
\item If $!A$ and $!B$ are formulae, then $(!A \land !B)$ is a formula.
\item If $!A$ and $!B$ are formulae, then $(!A \lor !B)$ is a formula.
\item If $!A$ is a formula and $x$ is a variable, then $(\forall x !A)$ is a formula.
\item If $!A$ is a formula and $x$ is a variable, then $(\exists x !A)$ is a formula.
\item Nothing else is a formula.
\end{enumerate}
\end{defn}

%Vacuous quantification?

\begin{defn}[Subformula]
Given a formula $!A$, the set of \emph{subformulae} of $!A$, $\mathsf{SBF}(!A)$, is defined inductively as follows:
\begin{enumerate}
\item If $!A$ is atomic, then $\mathsf{SBF}(!A)=\{!A\}$
\item $\mathsf{SBF}(\lnot !A) = \{\lnot !A\} \cup \mathsf{SBF}(!A)$
\item If $!B$ is also a formula, then $\mathsf{SBF}(!A \land !B) = \{!A \land !B\} \cup \mathsf{SBF}(!A) \cup \mathsf{SBF}(!B)$
\item If $!B$ is also a formula, then $\mathsf{SBF}(!A \lor !B) = \{!A \lor !B\} \cup \mathsf{SBF}(!A) \cup \mathsf{SBF}(!B)$
\item If $x$ is a variable, then $\mathsf{SBF}(\forall x !A) = \{\forall x !A\} \cup \mathsf{SBF}(!A)$
\item If $x$ is a variable, then $\mathsf{SBF}(\exists x !A) = \{\exists x !A\} \cup \mathsf{SBF}(!A)$
\end{enumerate}
The set of all \emph{proper} subformulae of $!A$ is $\mathsf{SBF}(!A) - \{!A\}$.
\end{defn}

%Wordy explanation of subformula in here

\begin{defn}[Bound/free variable]
A variable $x$ occurring in a formula $!A$ is \emph{bound} if and only if there is some subformula of $!A$ of the form $\forall x !B$ or $\exists x !B$ for $!B\in\mathsf{SBF}(!A)$. A variable $x$ is \emph{free} (or \emph{unbound}) in $!A$ if and only if it is not bound in $!A$. \end{defn}

\begin{defn}[Sentence]
A formula $!A$ is a \emph{sentence} if and only if it is a (well-formed) formula and contains no free variables.\end{defn}

%"Wordy" explanation of bound variable in here

\begin{defn}[Substitution of a term for a variable]
If $!A$ is a formula, $x$ is a variable, and $t$ is a term, $A(x/t)$ is the result of replacing all free occurrences of $x$ in $!A$ with $t$. Substitution may be vacuous ($x$ need not occur in $!A$). \end{defn}





















\end{document}
