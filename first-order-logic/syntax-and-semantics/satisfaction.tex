% Part: first-order-logic
% Chapter: syntax-and-semantics
% Section: satisfaction

\documentclass[../../include/open-logic-section]{subfiles}

\begin{document}

\olfileid{fol}{syn}{sat}
\olsection{Satisfaction of a Formula in a Structure}

\begin{explain}
The basic notion that relates expressions such as terms and "p{formula},
on the one hand, and "p{structure} on the other, are those of
\emph{denotation} and \emph{satisfaction}.  Informally, a term
\emph{denotes} an element of a "{structure}---if the term is just a
constant, it denotes the object assigned to the constant by the
"{structure}, and if it is built up using "p{function}, the
denotation is computed from the values of constants and the functions
assigned to the functions in the term.  A "{formula} is \emph{satisfied}
in a "{structure} if the interpretation given to the predicates makes the
"{formula} true in the domain of the "{structure}. This notion of
satisfaction is specified inductively: the specification of the
"{structure} directly states when atomic "p{formula} are satisfied, and we
define when a complex "{formula} is satisfied depending on the main
connective or quantifier and whether or not the immediate subformulas
are satisfied. The case of the quantifiers here is a bit tricky, as
the immediate subformula of a quantified "{formula} has a free "{variable},
and "p{structure} don't specify what "p{variable} denote.  In order to deal
with this difficulty, we also introduce \emph{variable assignments}
and define satisfaction not with respect to a "{structure} alone, but
with respect to a "{structure} plus a "{variable} assignment.
\end{explain}

\begin{defn}[Variable Assignment]
A \emph{variable assignment}~$s$ for a "{structure}~$\Struct{M}$ is a
function which maps each "{variable} to an element of~$\Domain M$, i.e.,
$s\colon \Var \to \Domain M$.
\end{defn}

\begin{defn}[Denotation of terms]
If $t$ is a term of the language~$\Lang L$, $\Struct M$ is a "{structure}
for~$\Lang L$, and $s$ is a "{variable} assignment for~$\Struct M$, the
\emph{denotation}~$\Value{t}{M}[s]$ is defined as follows:
\begin{enumerate}
\item If $t$ is just the constant $c$, then $\Value{c}{M}[s] = \Assign{c}{M}$.
\item If $t$ is just the "{variable}~$x$, then $\Value{c}{M}[s] = s(x)$.
\item If $t$ is of the form $\Atom{f}{t_1, \ldots, t_n}$, then
\[
\Value{t}{M}[s] = \Assign{f}{M}(\Value{t_1}{M}[s], \ldots,
\Value{t_n}{M}[s]).
\]
\end{enumerate}
\end{defn}

\begin{defn}[$x$-Variant]
If $s$ is a "{variable} assignment for a "{structure}~$\Struct M$, then any
"{variable} assignment $s'$ for $\Struct M$ which differs from $s$ only
in what it assigns to $x$ is called an \emph{$x$-variant} of~$s$.  If
$s'$ is an $x$-variant of $s$ we write $s \sim_x s'$.
\end{defn}

\begin{defn}[Satisfaction]
Satisfaction of a "{formula}~$!A$ in a "{structure}~$\Struct M$ relative to
a "{variable} assignment~$s$, $\Sat{M}{!A}[s]$ is defined inductively as
follows. If not $\Sat{M}{!A}[s]$ we write $\Sat/{M}{!A}[s]$.
\begin{enumerate}
\item Never $\Sat{M}{\lfalse}[s]$.
\item Always $\Sat{M}{\ltrue}[s]$.
\item For any $n$-place predicate~$R$ and terms $t_1$, \dots, $t_n$,
  $\Sat{M}{\Atom{R}{t_1, \dots, t_n}}[s]$ iff $\langle \Value{t_1}{M}[s],
  \dots, \Value{t_n}{M}[s] \rangle \in \Assign{R}{M}$.
\item For any terms $t$, $t'$, $\Sat{M}{\eq[t_1][t_2]}[s]$ iff
  $\Value{t}{M}[s] = \Value{t'}{M}[s]$.
\item $\Sat{M}{\lnot !A}[s]$ iff $\Sat/{M}{!A}[s]$
\item $\Sat{M}{!A \land !B}[s]$ iff $\Sat{M}{!A}[s]$ and $\Sat{M}{!B}[s]$.
\item $\Sat{M}{!A \lor !B}[s]$ iff $\Sat{M}{!A}[s]$ or
  $\Sat{M}{!B}[s]$ (or both).
\item $\Sat{M}{!A \lif !B}[s]$ iff $\Sat/{M}{!A}[s]$ or
  $\Sat{M}{!B}[s]$ (or both).
\item $\Sat{M}{\lexists[x][!A]}[s]$ iff there is an $x$-variant $s'$
  of $s$ so that $\Sat{M}{!A}[s']$.
\item $\Sat{M}{\lforall[x][!A]}[s]$ iff for every $x$-variant $s'$ of
  $s$, $\Sat{M}{!A}[s']$.
\end{enumerate}
\end{defn}

\begin{explain}
A "{variable} assignment~$s$ provide a value for \emph{every} variable in
the language. This is of course not necessary: whether or not a
"{formula}~$!A$ is satisfied in a "{structure} with respect to~$s$ only
depends on the assignments~$s$ makes to the free "p{variable} that
actually occur in~$!A$.  This is the content of the next theorem.  We
require "{variable} assignments to assign values to all "p{variable} simply
because it makes things a lot easier.
\end{explain}

\begin{prop}
If $x_1$, \dots, $x_n$ are the only free "p{variable} in $!A$ and $s(x_i)
= s'(x_i)$ for $i = 1$, \dots, $n$, then $\Sat{M}{!A}[s]$ iff
$\Sat{M}{!A}[s']$.
\end{prop}

\begin{proof}

Prove by induction on the complexity of $!A$. For the base case, where
$!A$ is atomic, $!A$ can be: $\ltrue$, $\lfalse$, $R(t_1 \ldots t_k)$
for a $k$-place predicate $R$ and terms $t_1,\ldots,t_k$, or
$\eq[t_1][t_2]$ for terms $t_1$ and $t_2$.

When $!A$ is $\ltrue$, since $\ltrue$ is valid, both $\Sat{M}{!A}[s]$
and $\Sat{M}{!A}[s']$. Similarly, when $!A$ is $\lfalse$, both
$\Sat!{M}{!A}[s]$ and $\Sat!{M}{!A}[s']$.

For the case when $!A$ is $\Atom{R}{t_1, \ldots, t_k}$, let
$\Sat{M}{!A}[s]$. Then $\langle \Value{t_1}{M}[s], \ldots,
\Value{t_k}{M}[s] \rangle \in \Assign{R}{M}$. For $1 \eq i \eq k$,
if $t_i$ is a constant, then $\Value{t_i}{M}[s] = \Value{t_i}{M} =
\Value{t_i}{M}[s']$. If $t_i$ is a free "{variable}, then since the
mappings $s$ and $s'$ agree on all free "p{variable}, $\Value{t_i}{M}[s]
= s(t_i) = s'(t_i) = \Value{t_i}{M}[s']$. Similarly, if $t_i$ is of
the form $f(t'_1,\ldots,t'_j)$, we will also get $\Value{t_i}{M}[s] =
\Value{t_i}{M}[s']$. Hence, $\Value{t_i}{M}[s] = \Value{t_i}{M}[s']$
for any term $t_i$ for $1 \eq i \eq k$, so we also have $\langle
\Value{t_i}{M}[s'], \ldots, \Value{t_k}{M}[s'] \rangle \in
\Assign{R}{M}$. Similarly, when $!A$ is $\eq[t_1][t_2]$, and when
$\Sat{M}{\eq[t_1][t_2]}[s]$, $\Value{t_1}{M}[s'] = \Value{t_1}{M}[s] =
\Value{t_2}{M}[s] = \Value{t_2}{M}[s']$, so $\Sat{M}{\eq[t_1][t_2]}[s']$.

Let $\Sat{M}{!B}[s] \Leftrightarrow \Sat{M}{!B}[s']$ for all "p{formula}
$!B$ and $!C$ that are less complex than $!A$. The induction step
proceeds in cases determined by the main connective of $!A$. In each
case, only the forward direction of the "{biconditional} is demonstrated
since the proof of the reverse direction is symmetrical.

If $!A$ is $\lnot !B$ and $\Sat{M}{\lnot !B}[s]$, then
$\Sat!{M}{!B}[s]$, so by the induction hypothesis, $\Sat!{M}{!B}[s']$,
hence $\Sat{M}{\lnot !B}[s']$.

If $!A$ is $!B \land !C$ and $\Sat{M}{!B \land !C}[s]$, then
$\Sat{M}{!B}[s]$ and $\Sat{M}{!C}[s]$, so by the induction hypothesis,
$\Sat{M}{!B}[s']$ and $\Sat{M}{!C}[s']$. Hence, $\Sat{M}{!B \land
  !C}[s']$.

If $!A$ is $!B \lor !C$ and $\Sat{M}{!B \lor !C}[s]$, then
$\Sat{M}{!B}[s]$ or $\Sat{M}{!C}[s]$. By the induction hypothesis,
$\Sat{M}{!B}[s']$ or $\Sat{M}{!C}[s']$, so $\Sat{M}{!B \lor !C}[s']$.

If $!A$ is $!B \lif !C$ and $\Sat{M}{!B \lif !C}[s]$, then
$\Sat{M}{!B}[s]$ or $\Sat{M}{!C}[s]$. By the induction hypothesis,
$\Sat{M}{!B}[s']$ or $\Sat{M}{!C}[s']$, so $\Sat{M}{!B \lif !C}[s']$.

If $!A$ is $\lexists[x][!B]$ and $\Sat{M}{\lexists[x][!B]}[s]$, there
is an $x$-variant $\bar{s}$ of $s$ so that $\Sat{M}{!B}[\bar{s}]$. Let
$\bar{s}'$ denote the $x$-variant of $s'$ that assigns the same thing
to $x$ as does $\bar{s}$: then by the induction hypothesis,
$\Sat{M}{!B}[\bar{s}']$. Hence, there is an $x$-variant of $s'$ that
satisfies $!B$, so $\Sat{M}{\lexists[x][!B]}[s']$.

If $!A$ is $\lforall[x][!B]$ and $\Sat{M}{\lforall[x][!B]}[s]$, then
for every $x$-variant $\bar{s}$ of $s$, $\Sat{M}{!B}[\bar{s}]$. Hence,
if $\bar{s}'$ is the $x$-variant of $s'$ that assigns the same thing
to $x$ as does $\bar{s}$, then we have $\Sat{M}{!B}[\bar{s}']$. Hence,
every $x$-variant of $s'$ satisfies $!B$, so
$\Sat{M}{\lforall[x][!B]}[s']$.

By induction, we get that $\Sat{M}{!A}[s]$ iff $\Sat{M}{!A}[s']$
whenever $x_1$, \dots, $x_n$ are the only free "p{variable} in $!A$ and
$s(x_i)=s'(x_i)$ for $i=1$, \dots,~$n$.
\end{proof}
 
\begin{defn}
If $!A$ is a sentence, we say that a "{structure}~$\Struct M$
satisfies~$!A$, $\Sat{M}{!A}$, iff $\Sat{M}{!A}[s]$ for all "{variable}
assignments~$s$.
\end{defn}

%N: is this what you had in mind for extensionality? I just want to make 
%sure that the statement of the thm is ok before I type up a proof.

\begin{cor}[Extensionality]
Let $!A$ be a sentence, and $\Struct M$ and $\Struct M'$ be "p{structure}. 
If $\Assign{c}{M} = \Assign{c}{M'}$, $\Assign{R}{M}=\Assign{R}{M'}$, and 
$\Assign{f}{M} = \Assign{f}{M'}$ for every "{constant} $c$, relation 
symbol $R$, and "{function} $f$ occurring in $!A$, then $\Sat{M}{!A}$ 
iff $\Sat{M'}{!A}$.
\end{cor}

\begin{prob}
Show that if $!A$ is a sentence, $\Sat{M}{!A}$ iff there is a "{variable}
assignment~$s$ so that $\Sat{M}{!A}[s]$.
\end{prob}

\end{document}
