% Part: first-order-logic
% Chapter: syntax-and-semantics
% Section: first-order-languages

\documentclass[../../include/open-logic-section]{subfiles}

\begin{document}

\olfileid{fol}{syn}{fol}

\olsection{First-Order Languages}


Expressions of first-order logic are built up from a basic vocabulary
containing \emph{!p{variable}}, \emph{!p{constant}},
\emph{!p{predicate}} and sometimes \emph{!p{function}}.  From them,
together with logical connectives, quantifiers, and punctuation
symbols such as parentheses and commas, \emph{terms} and
\emph{!p{formula}} are formed.

\begin{explain}
Informally, !p{predicate} are names for properties and relations,
!p{constant} are names for individual objects, and !p{function} are names
for mappings.  These, except for the !!{identity}~$\eq$, are the
\emph{non-logical symbols} and together make up a language.  Any
first-order language~$\Lang L$ is determined by its non-logical
symbols.  In the most general case, $\Lang L$ contains infinitely
many symbols of each kind.
\end{explain}

In the general case, we make use of the following symbols in
first-order logic:

\begin{enumerate}
\item Logical symbols
\begin{enumerate}
\item Logical connectives: $\lnot$ (negation), $\land$ (conjunction),
  $\lor$ (disjunction), $\lif$ (!!{conditional}), $\liff$ (!!{biconditional}),
  $\lforall$ (universal quantifier), $\lexists$ (existential
  quantifier).
\item The propositional constant for !!{falsity}~$\lfalse$, called ``bottom'' or ``\emph{falsum}''.
\item The propositional constant for $\ltrue$, called ``top'' or ``\emph{verum}''.
% Need to add in !!{true}~ into the above
\item The two-place !!{identity}~$\eq$.
\item A "p{denumerable} set of "p{variable}: $\Obj v_0$, $\Obj v_1$, $\Obj
  v_2$, \dots
\end{enumerate}
\item Non-logical symbols, making up the \emph{standard
  language} of first-order logic
\begin{enumerate}
\item A "p{denumerable} set of $n$-place !p{predicate} for each $n>0$: $\Obj
  A^n_0$, $\Obj A^n_1$, $\Obj A^n_2$, \dots
\item A "p{denumerable} set of !p{constant}: $\Obj c_0$, $\Obj c_1$, $\Obj
  c_2$, \dots.
\item A "p{denumerable} set of $n$-place !p{function} for each $n>0$:
  $\Obj f^n_0$, $\Obj f^n_1$, $\Obj f^n_2$, \dots
\end{enumerate}
\item Punctuation marks: (, ), and the comma.
\end{enumerate}

% Alternate symbols

\begin{intro}
You may be familiar with different symbols than the ones we use
above. Logic texts (and teachers) commonly use either $\sim$ or $\neg$
for negation, $\wedge$ or $\&$ for conjunction, and $\rightarrow$ or
$\supset$ for the !!{conditional}. Other alternatives include $\cdot$ for
conjunction, $\equiv$ for !!{biconditional} and !~for negation. It is very
common to use lower case letters (e.g. $a$, $b$, $c$) from the
beginning of the Latin alphabet for !p{constant} (sometimes called
names), and lower case letters from the end (e.g. $x$, $y$, $z$)
for "p{variable}. Quantifier variations include $\forall$ and $(x)$
(where $x$ is a "{variable}) for the universal quantifier and
$\exists$ and $(Ex)$ for the existential quantifier.
\end{intro}

%Polish notation?

\begin{explain}
We are treating the propositional operators and both quantifiers as
primitive symbols of the language. We might instead choose a smaller
stock of primitive symbols and treat the other logical operators as
defined. ``Truth functionally complete'' sets of Boolean operators
include $\{ \lnot, \lor \}$, $\{ \lnot, \land \}$, and $\{ \lnot,
\lif\}$ --- these can be combined with either quantifier for an
expressively complete first-order language.

You may be familiar with two other logical operators: the Sheffer
stroke, $\uparrow$ (named after Henry Sheffer), and Peirce's arrow
$\downarrow$, also known as Quine's dagger. When given their usual
readings of ``nand'' and ``nor'' (respectively), these operators are
truth functionally complete by themselves.
\end{explain}

\end{document}
