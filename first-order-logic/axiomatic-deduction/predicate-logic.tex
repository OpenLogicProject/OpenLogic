% Part: first-order-logic 
% Chapter: axiomatic-proofs 
% Section: predicate-logic

\documentclass[../../include/open-logic-section]{subfiles}

\begin{document}

\olfileid{fol}{axp}{prl}

\olsection{Predicate Logic}

\begin{explain}
The language $\Lang L_1$ of classical predicate logic comprises
the !!{operator}s $\lnot$ and $\lif$, the universal quantifier
$\lforall$, parentheses $($ and $)$ as well as:
\begin{itemize}
\item denumerably many individual !!{variable}s $v_0,v_1,\ldots$;
\item !!{enumerably} many (i.e., finitely or !!{denumerably} many)
individual
  !!{constant}s $c_0,c_1,\ldots$;
\item for each $n>0$, !!{enumerably} many $n$-place !!{predicate}s,
including at least the
  $2$-place symbol $\doteq$ for !!{identity};
\item for each $n>0$, !!{enumerably}  many   $n$-place !!{function}s.
\end{itemize}
\end{explain}

\begin{defn}[Terms and Formulas]
We define the sets comprising the terms, atomic !!{formula}s and
!!{formula}s
of $\Lang L_1$:
  \begin{itemize}
  \item The set $T_1$ of the terms $\Lang L_1$ is defined as the
    smallest set containing the !!{constant}s, the !!{variable}s, and such
    that if $t_1,\ldots,t_n$ are terms and $f$ is an $n$-place
    !!{function}, then $ft_1\ldots t_n$ is also a term.
  \item The set $\mathsf{At}_1$ of the atomic !!{formula}s comprises all
    expressions of the form $Pt_1\ldots t_n$, where $t_1,\ldots,t_n$
    are terms and $P$ is an $n$-place !!{predicate} as well as all
    expressions of the form $t_1 \doteq t_2$.
    \item The set $F_1$ of the !!{formula}s of $\Lang L_1$ is defined
      as the smallest set containing the atomic !!{formula}s and such that
      if $!A$ and $!B$ are !!{formula}s and $x$ is a variable, then
$(\lnot !A)$, $(!A \lif !B)$ and $\lforall[x][!A]$ are !!{formula}s (the
!!{formula} $!A$ is the \emph{scope}
      of the quantifier $\lforall[x]$).
  \end{itemize}
\end{defn}

\begin{explain}
We adopt the same conventions for dropping parentheses as in the
propositional case as well as the same abbreviations for $(!A
\land !B)$ and $(!A \lor !B)$. Moreover, we abbreviate
$\lnot \lforall[x][\lnot !A]$ by $\lexists[x][!A]$. Just as in
the propositional case, we have a \emph{principle of induction} on
both terms and !!{formula}s, as well as a \emph{principle of definition by
  recursion} (also on both terms and !!{formula}s).
\end{explain}

\begin{defn}[Variables in Formulas]
The following notions relate to the occurrence of
  !!{variable}s in !!{formula}s:
  \begin{enumerate}
  \item A variable $x$ occurs \emph{free} in a !!{formula} $!A$ if it
    does not fall within the scope of a quantifier $\lforall[x]$. This
    can be defined recursively on the complexity of $!A$: $x$ is
    always free in $!A$ if $!A \in \mathsf{At}_1$; $x$ is
    free in $\lnot !B$ or $!B \lif !C$ if it is free in
    $!B$ or $!C$; and $x$ is free in $\lforall[y][!B]$ if it is
    free in $!B$ and not the same !!{variable} as $y$.
  \item If $!A$ is a !!{formula} and $x_1,\ldots,x_n$ are distinct
    !!{variable}s, we denote by $!A(x_1,\ldots,x_n)$ the fact that
    all of the !!{variable}s occurring free in $!A$ are among
    $x_1,\ldots,x_n$.
  \item If no variable occurs free in $!A$, then $!A$ is a
    \emph{!!{sentence}} . 
  \end{enumerate}
\end{defn}

\begin{defn}[Substitution I]
  We define $\Subst{t}{t'}{x}$, the result of replacing $t'$ for
  every  occurrence of $x$ in $t$, recursively on the complexity of
  $t$: if $t$ is !!a{constant} $c$ or !!a{variable} other than $x$ then
  $\Subst{t}{t'}{x}$ is just $t$; if $t$ is $x$ then $\Subst{t}{t'}{x}$
  is $t'$; and if $t$ is $ft_1\ldots t_n$ then $\Subst{t}{t'}{x}$ is
  $f\Subst{t_1}{t'}{x}\ldots \Subst{t_n}{t'}{x}$.
\end{defn}

\begin{defn}[Substitution II]
  We define $\Subst{!A}{t}{x}$, the result of replacing $t$ for
  every free occurrence of $x$ in $!A$, recursively on the
  complexity of $!A$:
  \begin{itemize}
  \item if $!A$ is $Pt_1\ldots t_n$ or $t_1 \doteq t_2$ then
    $\Subst{!A}{t}{x}$ is $Pf \Subst{t_1}{t'}{x}\ldots
\Subst{t_n}{t'}{x}$ or $\Subst{t_1}{t'}{x} \doteq \Subst{t_2}{t'}{x}$,
respectively;
    \item if $!A$ is $\lnot !B$ or $!B \lif !C$ then
      $\Subst{!A}{t}{x}$ is  $\lnot(\Subst{!B}{t'}{x})$ or
      $\Subst{!B}{t'}{x} \lif \Subst{!C}{t'}{x}$, respectively;
    \item if $!A$ is $\lforall[y][!B]$ (where $y$ is a variable
      other than $x$), then $\Subst{!A}{t}{x}$ is $\lforall
      [y][(\Subst{!A}{t}{x})]$;
    \item if $!A$ is $\lforall[x][!B]$ then  $\Subst{!A}{t}{x}$
      is just $!A$.
  \end{itemize}
\end{defn}

\begin{defn}
  A term $t$ is \emph{!!{free for} $x$ in $!A$} if $x$ does not occur in
  $!A$ within the scope of a quantifier $\lforall[y]$ binding a
  variable $y$ occurring in $t$.
\end{defn}

\noindent
Needless to say, the previous definition can be more precisely given
as a recursion on $!A$.

\end{document}