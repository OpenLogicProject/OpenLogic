% Part: first-order-logic 
% Chapter: axiomatic-deduction 
% Section:provability

% verification of properties of provability needed for maximally 
% consistent sets in the completeness chapter.

\documentclass[../../include/open-logic-section]{subfiles}

\begin{document}

\olfileid{fol}{axd}{prv} \olsection{Properties of
\usetoken{S}{derivability}}

\begin{prop}[Monotony] if $\Gamma \subseteq \Delta$ and $\Gamma \Proves !A$
then also $\Delta \Proves !A$ \end{prop}

\begin{proof} Any finite $\Gamma_0 \subseteq \Gamma$ is also a finite
subset of~$\Delta$, so !!a{derivation} of $\Gamma_0 \Proves !A$ also shows
$\Delta \Proves !A$. \end{proof}

\begin{prop}\ollabel{prop:provability} $\Gamma \Proves!A$ if and only if
$\Gamma\cup \{\lnot!A \}$ is inconsistent. \end{prop}

\begin{prop} \begin{enumerate}
\item \ollabel{prop:provability-contr} If $\Gamma \Proves !A$ and $\Gamma
\cup \{ !A\} \Proves \lfalse$, then $\Gamma$ is inconsistent.

\item \ollabel{prop:provability-lnot} If $\Gamma \cup \{!A\}
\Proves \lfalse$, then $\Gamma \Proves \lnot !A$.

\item \ollabel{prop:provability-exhaustive} If $\Gamma \cup \{!A\}
\Proves \lfalse$ and $\Gamma \cup \{\lnot !A\} \Proves
\lfalse$, then $\Gamma \Proves \lfalse$.

\tagitem{prvOr}{\ollabel{prop:provability-lor-left} If $\Gamma \cup \{!A\}
\Proves \lfalse$ and $\Gamma \cup \{!B\} \Proves
\lfalse$, then $\Gamma \cup \{!A \lor !B\} \Proves \lfalse$.}{}

\tagitem{prvOr}{\ollabel{prop:provability-lor-right} If $\Gamma
\Proves !A$ or $\Gamma \Proves !B$, then $\Gamma
\Proves !A \lor !B$.}{}

\tagitem{prvAnd}{\ollabel{prop:provability-land-left} If $\Gamma
\Proves !A \land !B$ then $\Gamma \Proves !A$ and
$\Gamma \Proves !B$.}{}

\tagitem{prvAnd}{\ollabel{prop:provability-land-right} If $\Gamma
\Proves !A$ and $\Gamma \Proves !B$, then $\Gamma
\Proves !A \land !B$.}{}

\tagitem{prvIf}{\ollabel{prop:provability-mp} If $\Gamma \Proves
!A$ and $\Gamma \Proves !A \lif !B$, then $\Gamma
\Proves !B$.}{}

\tagitem{prvIf}{\ollabel{prop:provability-lif} If $\Gamma \Proves
\lnot !A$ or $\Gamma \Proves !B$, then $\Gamma \Proves
!A \lif !B$.}{} \end{enumerate} \end{prop}

\begin{proof} 
\begin{enumerate}
\item Assume that $\Gamma_0 \cup \{!A\} \Proves \lfalse$.

\begin{tabular}{rll} 
1. & $\Gamma_0 \cup \{!A\} \Proves \lfalse$ & hyp. \\
2. & $\Gamma_1 \Proves !A$ & hyp.\\ 
3. & $\Gamma_0 \cup \Gamma_1 \Proves \lfalse$ & cut \\ 
\end{tabular}

Since $\Gamma_0 \subseteq \Gamma$ and $\Gamma_1 \subseteq \Gamma$,
$\Gamma_0 \cup \Gamma_1 \subseteq \Gamma$, hence $\Gamma \Proves \lfalse$.

\item We assume that $\Gamma_0 \cup\{\lnot !A\} \Proves \lfalse$.

\begin{tabular}{rll} 
1. & $\Gamma_0 \cup\{\lnot !A\} \Proves \lfalse$ & hyp. \\ 
2. & $\Gamma_0 \Proves \lnot !A \lif \lfalse$ & Deduction Theorem\\
3. & $\Gamma_0 \Proves (\lnot !A \lif \lfalse) \lif !A$ & Ax0\\ 
4. & $\Gamma_0 \Proves !A$ & MP 2, 3 \\ 
\end{tabular}

\item We assume that $\Gamma \cup \{ !A \} \Proves \lfalse$ and that
$\Gamma_1 \cup \{\lnot !A \} \Proves \lfalse$.

\begin{tabular}{rll} 
1. & $\Gamma_0 \cup\{!A\} \Proves \lfalse$ & hyp. \\
2. & $\Gamma_0 \Proves !A \lif \lfalse$ & Deduction Theorem \\ 
3. & $\Gamma_1 \cup \{\lnot !A \} \Proves \lfalse$ & hyp\\ 
4. & $\Gamma_0 \Proves (!A \lif \lfalse) \lif \lnot !A$ & Ax0\\ 
5. & $\Gamma_0 \Proves \lnot !A$ & MP 2, 4\\ 
6. & $\Gamma_0 \cup \Gamma_1 \Proves \lfalse$ & Cut 3, 5 \\
\end{tabular}

Since $\Gamma_0 \subseteq \Gamma$ and $\Gamma_1 \subseteq \Gamma$,
$\Gamma_0 \cup \Gamma_1 \subseteq \Gamma$. Hence $\Gamma \Proves \lfalse$.

% prop:provability-lor-left 
\tagitem{defOr}{}{
\iftag{probOr}{Exercise.}{Exercise.}}

% prop:provability-lor-right 
\tagitem{defOr}{}{ \iftag{probOr}{Exercise.}{
We assume that $\Gamma_0 \Proves !A$.

\begin{tabular}{rll} 
1. & $\Gamma_0 \Proves !A$ & hyp \\ 
2. & $\Gamma_0 \Proves !A \lif (!A \lor !B)$ & Ax0\\ 
3. & $\Gamma_0 \Proves !A \lor !B$ & MP 1, 2 \\
\end{tabular}

Therefore $\Gamma \Proves !A \lor !B$. The proof for when $\Gamma \Proves
!B$ is similar.}}

% prop:provability-land-left 
\tagitem{defAnd}{}{ \iftag{probAnd}{Exercise}{
Assume that $\Gamma_0 \Proves !A \land !B$

\begin{tabular}{rll} 
1. & $\Gamma_0 \Proves !A \land !B$ & hyp \\ 
2. & $\Gamma_0 \Proves (!A\land !B) \lif !A $ & Ax0\\ 
3. & $\Gamma_0 \Proves !A \lor !B$ & MP 1, 2 \\
\end{tabular}

Hence, $\Gamma \Proves !A$. A similar !!{derivation} starting shows that
$\Gamma \Proves !B$.}}

% prop:provability-land-right
\tagitem{defAnd}{}{\iftag{probAnd}{Exercise.}{Exercise.}}

% prop:provability-mp 
\tagitem{defIf}{}{ \iftag{probIf}{Exercise.}{ We
assume both that $\Gamma_0 \Proves !A$ and $\Gamma_0 \Proves !A \lif !B $.

\begin{tabular}{rll} 
1. & $\Gamma_0 \Proves !A$ & hyp \\ 
2. & $\Gamma_0 \Proves !A \lif !B $ & hyp\\ 
3. & $\Gamma_0 \Proves !B$ & MP 1, 2\\
\end{tabular}

Since $\Gamma_0 \cup \Gamma_1 \subseteq \Gamma$, this means that $\Gamma
\Proves !B$.}}

% prop:provability-lif 
\tagitem{defIf}{}{\iftag{probIf}{Exercise.}{First
suppose $\Gamma \Proves \lnot !A$.
    
\begin{tabular}{rll} 
1. & $\Gamma_0 \Proves \lnot !A$ & hyp \\ 
2. & $\Gamma_0 \Proves \lnot !A \lif (!A \lif !B) $ & Ax0\\ 
3. & $\Gamma_0 \Proves !A \lif !B$ & MP 1,
2\\ \end{tabular}

The proof for $\Gamma \Proves !B$ is similar:

\begin{tabular}{rll} 
1. & $\Gamma_0 \Proves !B$ & hyp \\ 
2. & $\Gamma_0 \Proves !B \lif (!A \lif !B) $ & Ax0\\ 
3. & $\Gamma_0 \Proves !A \lif !B$ & MP 1, 2\\
\end{tabular}}}

\end{enumerate} 
\end{proof}

\begin{probtag}{probOr,probAnd,probIf} Complete the proof of
\olref[fol][axd][prv]{prop:provability}. 
\end{probtag}

\begin{thm}[Generalization] \ollabel{thm:Generalization} If $\Gamma \Proves
!A$ and $x$ is not free in any !!{formula} in $\Gamma$, then $\Gamma
\Proves \lforall[x][!A]$. 
\end{thm}

\begin{proof} By induction on $\mathsf{Thm}_\Gamma$: if $!A$ is an axiom,
so is $\lforall[x][!A]$. If $!A\in \Gamma$ then $x$ is not free in $!A$, so
that $!A \lif \lforall[x][!A]$ is an axiom (\textbf{Ax3}), and $\Gamma
\Proves \lforall[x][!A]$ by MP. Now suppose $!A$ follows by MP because
$\Gamma \Proves !B$ and $\Gamma \Proves !B \lif !A$. By induction
hypothesis, $\Gamma \Proves \lforall[x][!B]$ and $\Gamma \Proves
\lforall[x][( !B \lif !A)]$. But by \textbf{Ax2} also \[ \Gamma \Proves
\lforall[x][( !B \lif !A)] \lif (\lforall[x][ !B] \lif \lforall[x][!A]), \]
and two applications of MP give $\Gamma \Proves \lforall[x][!A]$ as
desired. \end{proof}

\begin{thm}\ollabel{thm:WeakGen} (\emph{Weak Generalization on Constants})
If $\Gamma \Proves !A$ and $c$ is a !!{constant} not in occurring in
$\Gamma$, then there is a variable $x$ not in $!A$ such that $\Gamma
\Proves \lforall[x][\Subst{!A}{x}{c}]$, and the proof does not involve $c$.
\end{thm}

\begin{proof} Let $!A_1,\ldots,!A_n$ be a proof of $!A$ from $\Gamma$, so
that $!A=!A_n$. Pick a variable $x$ not in $!A_1,\ldots,!A_n$, and consider
the new sequence: $\Subst{!A_1}{x}{c},\ldots,\Subst{!A_n}{x}{c}.$ Such a
sequence is a proof of $\Subst{!A}{x}{c}$ from $\Gamma$. In fact, for each
$i=1,\ldots,n$: \begin{itemize} \item if $!A$ is an axiom, then so is
$\Subst{!A_i}{x}{c}$; \item if $!A_i \in \Gamma$, then since $c$ is not in
$\Gamma$, we have $\Subst{!A_i}{x}{c} = !A_i \in \Gamma$; \item if $!A_i$
is obtained from $!A_j$ and $!A_j \lif !A_i$ then $\Subst{!A_i}{x}{c}$
follows by MP from $\Subst{!A_j}{x}{c}$ and $\Subst{(!A_j \lif !A_i)}{x}{c}
= \Subst{!A_j}{x}{c} \lif \Subst{!A_i}{x}{c}$. \end{itemize} It is clear
that the !!{constant} $c$ no longer occurs in the new sequence. Now let
$\Gamma'$ comprise those !!{formula}s from $\Gamma$ that appear in the
!!{derivation} of $\Subst{!A}{x}{c}$; then $x$ is not free in any
!!{formula} in $\Gamma'$ and $\Gamma' \Proves \Subst{!A}{x}{c}$. By
generalization (Theorem \olref[fol][axd][prv]{thm:Generalization}) $\Gamma'
\Proves \lforall[x][\Subst{!A}{x}{c}]$, whence by monotony $\Gamma \Proves
\lforall[x][\Subst{!A}{x}{c}]$, as desired. 
\end{proof}

\begin{explain}
Our goal is to replace the requirement in Theorem
\olref[fol][axd][prv]{thm:WeakGen} that $x$ does not occur in $!A$ (at all)
by the weaker requirements that $x$ is not free in $!A$ and it is !!{free
for} $c$ in $!A$. Clearly this can be accomplished by a change of bound
!!{variable} --- so that is what we set out to prove first.
\end{explain}

\begin{lem}\ollabel{lem:8.19} If $x$ is free for $c$ in $!A$ and $y$ is not
free in $!A$, then $x$ is !!{free for} $y$ in $\Subst{!A}{y}{c}$. 
\end{lem}

\begin{proof} If $x$ is not !!{free for} $y$ in $\Subst{!A}{y}{x}$, then
some free occurrence of $y$ in $\Subst{!A}{y}{c}$ falls within the scope of
a quantifier $\lforall[x]$. But $y$ is not free in $!A$, by hypothesis, so
all such occurrences come from the substitution $\Subst{}{y}{c}$. So some
occurrence of $c$ falls within the scope of $\lforall[x]$ and $x$ is not
!!{free for} $c$ in $!A$. 
\end{proof}

\begin{lem}\ollabel{lem:ChangeBdVar} (\emph{Change of Bound Variable}) If
$x$ and $y$ are not free in $!A$ and they are both !!{free for} $c$ in
$!A$, then $\Proves \lforall[x][\Subst{!A}{x}{c}] \equiv
\lforall[y][\Subst{!A}{y}{c}]$ (and the proof does not involve $c$).
\end{lem}

\begin{proof} From the hypotheses, $y$ is !!{free for} $c$ in $!A$ and $x$
is not free in $!A$, so by the previous lemma
\olref[fol][axd][prv]{lemma:8.19}, $y$ is !!{free for} $x$ in
$\Subst{!A}{x}{c}$. It follows that $\lforall[x][\Subst{!A}{x}{c} \lif
\Subst{!A}{x}{c} \Subst{}{y}{x}]$ is an axiom (\textbf{Ax1}). Since $x$ is
not free in $!A$, we have $\Subst{!A}{x}{c} \Subst{}{y}{x} =
\Subst{!A}{y}{c}$, so that \[ \Proves \lforall[x][\Subst{!A}{x}{c}] \lif
\Subst{!A}{y}{c}, \] and by the Deduction Theorem
$\lforall[x][\Subst{!A}{x}{c}] \Proves \Subst{!A}{y}{c}$. Since $y$ is not
free in $!A$, it also not free in $\lforall[x][\Subst{!A}{x}{c}]$, so that
by Generalization $\lforall[x][\Subst{!A}{x}{c}] \Proves
\lforall[y][\Subst{!A}{y}{c}]$, and the Deduction Theorem again gives
$\Proves \lforall[x][\Subst{!A}{x}{c}] \lif \lforall[y][\Subst{!A}{y}{c}]$.
The proof of $\Proves \lforall[y][\Subst{!A}{y}{c}] \lif \lforall[x]
[\Subst{!A}{x}{c}]$ is perfectly symmetric, so that the conclusion follows
by Proposition T. 
\end{proof}

\begin{thm}\ollabel{thm:StrongGen} (\emph{Strong Generalization on
Constants}) If $\Gamma \Proves !A$, the !!{constant} $c$ does not occur in
$\Gamma$, and $x$ is not free in $!A$ but it is !!{free for} $c$ in $!A$,
then $\Gamma \Proves \lforall[x][\Subst{!A}{x}{c}]$. 
\end{thm}

\begin{proof} Since $\Gamma \Proves !A$ and $c$ is not in $\Gamma$, then by
Weak Generalization there is !!a{variable} $y$ not in $!A$ such that
$\Gamma \Proves \lforall[y][\Subst{!A}{y}{c}]$. Since $y$ is not in $!A$
(at all), it is not free in $!A$ and it is also !!{free for} $c$ in $!A$;
if moreover (by hypothesis) $x$ is not free in $!A$ and !!{free for} $c$ in
$!A$, then the requirements for a change of bound !!{variable} are met, so
$\Proves \lforall[x][\Subst{!A}{x}{c}] \equiv
\lforall[y][\Subst{!A}{y}{c}]$, whence $\Gamma \Proves
\lforall[x][\Subst{!A}{x}{c}]$. 
\end{proof}

\begin{thm} \ollabel{thm:provability-quantifiers}
\begin{tagenumerate}{prvEx,prvAll} \tagitem{prvEx}{If $\Gamma \Proves
!A(t)$ then $\Gamma \Proves \lexists[x][!A(x)]$.}{}

\tagitem{prvAll}{If $\Gamma \Proves \lforall[x][!A(x)]$ then $\Gamma
\Proves !A(t)$.}{} 
\end{tagenumerate} 
\end{thm}

\begin{proof} \begin{tagenumerate}{prvEx,prvAll} 
%%Need axioms for exists
\tagitem{prvEx}{Exercise.}{}

\tagitem{prvAll} {Assume that  $\Gamma_0 \Proves \lforall[x][!A(x)]$.

\begin{tabular}{rll} 
1. & $\Gamma_0 \Proves \lforall[x][!A(x)]$ & hyp. \\
2. & $\Gamma_0 \Proves \lforall[x][!A(x)] \lif \Subst{!A}{t}{x}$ & Ax1 \\ 
3. & $\Gamma_0 \Proves \Subst{!A}{t}{x}$ & MP 1, 2 \\ 
\end{tabular}}{}

\end{tagenumerate} 
\end{proof}
\end{document}