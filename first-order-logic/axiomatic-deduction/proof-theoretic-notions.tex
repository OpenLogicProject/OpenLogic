% Part: first-order-logic
% Chapter: axiomatic-deduction
% Section: proof-theoretic-notions

\documentclass[../../include/open-logic-section]{subfiles}

\begin{document}

\olfileid{fol}{axd}{ptn}
\olsection{Proof-Theoretic Notions}

\begin{explain}
Just as we've defined a number of important semantic notions
(validity, entailment, satisfiabilty), we now define corresponding
\emph{proof-theoretic notions}.  These are not defined by appeal to
satisfaction of !!{sentence}s in !!{structure}s, but by appeal to the
!!{derivability} or !!{nonderivability} of certain formulas.  It was
an important discovery, due to G\"odel, that these notions coincide.
That they do is the content of the \emph{completeness theorem}.

The proof-theoretic notions for propositional logic are similar.
\end{explain}

\begin{defn}[Theorems]
$\Thms{\Gamma} = \Setabs{!A}{\Gamma \Proves !A}$. 
\end{defn}

\begin{prop}
$\Thms{\Gamma}$ is the smallest set of !!{formula}s containing
$\Gamma$, the axioms, and is closed under \emph{modus
  ponens}. Accordingly, we have a principle of proof by
\emph{induction on the theorem of $\Gamma$}.
\end{prop}

\begin{proof}
We know from \olref[prp][prp][axd]{prop:contain} that $\Thms{\Gamma}$ has the
desired properties; so we need to show that it is the smallest
such. Let $A$ be any other set of !!{formula}s containing the axioms
and $\Gamma$ and closed under \emph{modus ponens}. Prove that
$\Thms{\Gamma} \subseteq A$ by induction on the length of a proof of
$!A$ from~$\Gamma$.
\end{proof}

\begin{cor}[Principle of induction on theorems]
\ollabel{cor:induction-thms} 
Any property $P$ that holds of the axioms, of !!{formula}s in
$\Gamma$, and is preserved by \emph{modus ponens} holds of every
!!{formula} in $\Thms{\Gamma}$.
\end{cor}

\begin{defn}[!!^{derivability}]
A !!{formula} $!A$ is \emph{!!{derivable}} from $\Gamma$, written
$\Gamma \Proves !A$, if there is a !!{derivation} from $\Gamma$ ending
in $!A$.
\end{defn}

\begin{defn}[Consistency]
A set $\Gamma$ of !!{formula}s is \emph{consistent} if and only if
$\Gamma\Proves/ \lfalse$; it is \emph{inconsistent} otherwise.
\end{defn}

\begin{prop} 
\ollabel{prop:prov-incons}
$\Gamma \Proves !A$ iff $\Gamma \cup \{\lnot !A\}$ is inconsistent.
\end{prop}

\begin{proof}
Exercise.
\end{proof}

\begin{prob}
Prove \olref[fol][axd][ptn]{prop:prov-incons}
\end{prob}

\begin{prop}
\ollabel{prop:incons}
$\Gamma$ is inconsistent iff $\Gamma \Proves {!A}$ for every
  sentence~$!A$.
\end{prop}

\begin{proof}
Exercise.
\end{proof}

\begin{prob}
Prove \olref[fol][axd][ptn]{prop:incons}
\end{prob}

\begin{prop}
\ollabel{prop:proves-compact}
If $\Gamma \Proves !A$ iff for some finite $\Gamma_0 \subseteq
\Gamma$, $\Gamma_0 \Proves !A$.
\end{prop}

\begin{proof}
If $\Gamma \Proves !A$, then there is a finite sequence of
!!{formula}s $!A_1$, \dots,~$!A_n$ so that $!A \ident !A_n$ and each
$!A_i$ is either a logical axiom, !!a{element} of~$\Gamma$ or follows
from previous !!{formula}s by modus ponens.  Take $\Gamma_0$ to be
those $!A_i$ which are in~$\Gamma$.  Then the !!{derivation} is
likewise a !!{derivation} from~$\Gamma_0$, and so $\Gamma_0 \Proves
!A$.  The other direction is obvious.
\end{proof}

\end{document}
