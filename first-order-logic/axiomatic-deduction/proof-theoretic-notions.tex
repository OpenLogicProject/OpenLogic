% Part: first-order-logic
% Chapter: axiomatic-deduction
% Section: proof-theoretic-notions

\documentclass[../../include/open-logic-section]{subfiles}

\begin{document}

\olfileid{fol}{axd}{ptn}
\olsection{Proof-Theoretic Notions}

\begin{explain}
Just as we've defined a number of important semantic notions
(validity, entailment, satisfiabilty), we now define corresponding
\emph{proof-theoretic notions}.  These are not defined by appeal to
satisfaction of !!{sentence}s in !!{structure}s, but by appeal to the
!!{derivability} or !!{nonderivability} of certain formulas.  It was
an important discovery, due to G\"odel, that these notions coincide.
That they do is the content of the \emph{completeness theorem}.
\end{explain}

\begin{defn}[Theorems]
  $\mathsf{Thm}_\Gamma = \{ !A :  \Gamma \Proves !A\}$. 

 $\mathsf{Thm}_\Gamma$ is the smallest set of !!{formula}s containing
 $\Gamma$, the axioms, and closed under \emph{modus
   ponens}. Accordingly, we have a principle of proof by
 \emph{induction on the theorem of $\Gamma$}.  
\end{defn}

\begin{proof}
  We know from Proposition \olref{prop:contain} that
  $\mathsf{Thm}(\Gamma)$ has the desired properties; so we need to
  show that it is the smallest such. Let $A$ be any other set of
  !!{formula}s containing the axioms and $\Gamma$ and closed under
  \emph{modus ponens}. Prove that $\mathsf{Thm}(\Gamma) \subseteq A$
  by induction on the length of a proof of $!A$ from $\Gamma$.
\end{proof}

\begin{cor}\ollabel{cor:induction-thms}
  \emph{Principle of induction on theorems}: any property $P$ that
  holds of the axioms, of !!{formula}s in $\Gamma$, and is preserved by
  \emph{modus ponens} holds of every !!{formula} in
  $\mathsf{Thm}(\Gamma)$.
\end{cor}

\begin{defn}[!!^{derivability}]
 A !!{formula}
  $!A$ is \emph{{!!derivable}} from $\Gamma$, written $\Gamma \Proves
  !A$, if there is a !!{derivation} from $\Gamma$ ending in $!A$. 
\end{defn}

\begin{defn}[Consistency]
  A set $\Gamma$ of !!{formula}s is \emph{consistent} if and only if 
  $\Gamma\Proves/ \lfalse$; it is \emph{inconsistent} otherwise.
\end{defn}

\begin{prop} \olref{prop:prov-incons}
$\Gamma \Proves[\Log{LK}] !A$ iff $\Gamma \cup \{\lnot !A\}$ is inconsistent.
\end{prop}

\begin{proof}
Exercise.
\end{proof}

\begin{prob}
Prove \olref[fol][axd][ptn]{prop:prov-incons}
\end{prob}

\begin{prop}
\ollabel{prop:incons}
$\Gamma$ is inconsistent iff $\Gamma \Proves {!A}$ for every
  sentence~$!A$.
\end{prop}

\begin{proof}
Exercise.
\end{proof}

\begin{prob}
Prove \olref[fol][axd][ptn]{prop:incons}
\end{prob}

\begin{prop}
\ollabel{prop:proves-compact}
If $\Gamma \Proves !A$ iff for some finite $\Gamma_0 \subseteq
\Gamma$, $\Gamma_0 \Proves !A$.
\end{prop}

\begin{proof}
Follows immediately from the definion of~$\Proves$.
\end{proof}

\end{document}
