% Part: first-order-logic
% Chapter: completeness
% Section: henkin-expansion

\documentclass[../../include/open-logic-section]{subfiles}

\begin{document}

\olfileid{fol}{com}{hen}
\olsection{Henkin Expansion}

\begin{explain}
Part of the challenge in proving the completeness theorem is that the
model we construct from a maximally consistent set~$\Gamma$ must make
all the quantified !p{formula} in~$\Gamma$ true.  In order to
guarantee this, we use a trick due to Leon Henkin.  In essence, the
trick consists in expanding the language by infinitely many constants
and adding, for each !!{formula} with one free !!{variable} $!A(x)$ a
formula of the form $\lexists[x][!A(x)] \lif !A(c)$, where $c$ is one
of the new !p{constant}.
\end{explain}

\begin{lem}
If $\Gamma$ is consistent in $\Lang L$ and $\Lang L'$ is obtained from
$\Lang L$ by adding countably many new constants $c_1, c_2, \dots$,
then $\Gamma$ is consistent in $\Lang L'$.
\end{lem}

\begin{defn}
  A set $\Gamma$ of !p{formula} of a language $\Lang {L}$ is
  \emph{saturated} if and only if for each !!{formula} $!A \in
  \Frm[L]$ and !!{variable} $x$ there is a constant $c$ such that
  $\lexists[x][!A] \lif !A(c) \in \Gamma$.
\end{defn}

% do we need to also add conditionals for the universally quantified 
% formulas or does max. consistency take care of those?

The following definition will be used in the proof of the next theorem.

\begin{defn}
Fix an enumeration $\langle !A_1, x_1 \rangle$, $\langle !A_2, x_2
\rangle$, \dots of all !!{formula}-!!{variable} pairs of $\Lang L'$,
and define the sentence $!D_n$ by recursion on~$n$. Assuming that
$!D_1, \dots, !D_n$ have been defined, denote by $c_{n+1}$ the first
new constant not occurring in $!D_1, \dots, !D_n$, and let $!D_{n+1}$
be the !!{formula} $\lexists[x_{n+1}] !A_{n+1} \lif
!A_{n+1}[c_{n+1}/x_{n+1}]$. This includes the case where $n = 0$,
i.e., $D_1$ is $\lexists[x_1] !A_1 \lif !A_1[c_1/x_1]$ as the new
constant $c_1$ does not occur in any~$A_i$.
\end{defn}

\begin{thm}
  Every consistent set $\Gamma$ can be extended to a saturated
  consistent set.
\end{thm}

\begin{proof}
Given a consistent $\Gamma$, expand the language by adding countably
many new constants. By the previous Lemma, $\Gamma$ is still
consistent in the richer language. Further, let $!D$ be as in the
previous definition: then $\Gamma \cup !D$ is saturated by
construction. Let
\begin{align*}
\Gamma_0 & = \Gamma \\
\Gamma_{n+1} & = \Gamma_n \cup \{!D_{n+1} \}
\end{align*}
i.e., $\Gamma_i = \Gamma \cup \{ !D_1, \dots, !D_n \}$, and let
$\Gamma^* = \bigcup_{n} \Gamma_n$.  To show that $\Gamma^*$ is
consistent it suffices to show, by induction on $n$, that each
set~$\Gamma_n$ is consistent.

The induction basis is simply the claim that $\Gamma_0 = \Gamma$ is
consistent, which is the hypothesis of the theorem.

For the induction step, suppose that $\Gamma_{n-1}$ is consistent but
$\Gamma_n = \Gamma_{n-1} \cup \{!D_n\}$ is inconsistent. Recall that
$!D_{n} = \lexists[x_n][!A_n] \lif !A_n [c_n/x_n]$.  If $\Gamma_{n-1}
\cup \{!D_n\}$ is inconsistent, then $\Gamma_{n-1} \Proves \lnot
!D_n$, end hence then both of the following hold:
\[
\Gamma_{n-1} \Proves \lexists[x_n][!A_n]
\quad
\Gamma_{n-1} \Proves \lnot !A_n[c_n/x_n]
\]
But $c_n$ does not occur in $\Gamma_{n-1}$ and the !!{variable} $x_n$
is not free in $!A_n[c_n/x_n]$. Moreover, $x_n$ is free for $c_n$ in
$!A_n [c_n/x_n]$ (since by the choice of $c_n$, the only occurrences
of $c_n$ in $!A_n [c_n/x_n]$ derive from the substitution $[c_n/x_n]$,
and therefore fall outside the scope of any quantifier binding
$x_n$). By \olref[seq][prv]{thm:strong-generalization}, from $\Gamma
\Proves \lnot !A_n [c_n/x_n]$, we obtain $\Gamma \Proves
\lforall[x_n][(\lnot !A_0 [c_n/x_n])[x_n/c_n]]$, hence $\Gamma \Proves
\lforall[x_n][\lnot !A_n]$. Thus we have that both $\Gamma_{n-1}
\Proves \lexists[x_0] !A_0$ and $\Gamma_{n-1} \Proves
\lforall[x_0][\lnot !A_0]$, so $\Gamma$ itself is
inconsistent. Contradiction: $\Gamma_{n-1}$ was supposed to be
consistent. Hence $\Gamma_n \cup \{ !D_n\}$ is consistent.
\end{proof}

\end{document}
