% Part: first-order-logic
% Chapter: completeness
% Section: henkin-expansion

\documentclass[../../include/open-logic-section]{subfiles}

\begin{document}

\olfileid{fol}{com}{hen}
\olsection{Henkin Expansion}

\begin{explain}
Part of the challenge in proving the completeness theorem is that the
model we construct from a maximally consistent set~$\Gamma$ must make
all the quantified !p{formula} in~$\Gamma$ true.  In order to guarantee
this, we use a trick due to Leon Henkin.  In essence, the trick
consists in expanding the language by infinitely many constants and
adding, for each !!{formula} with one free !!{variable} $[!A(x)]$ a formula of
the form $\lexists[x][!A(x)] \lif !A(c)$, where $c$ is one of the new
!p{constant}.  
\end{explain}

\begin{prop}[Generalization]
If $\Gamma \Proves !A$ and $x$ is not free in any !!{formula} in $\Gamma$, 
then $\Gamma \Proves \lforall[x] !A$.
\end{prop}

\begin{prop}[Monotony]
If $\Gamma \subseteq \Delta$ and $\Gamma \Proves !A$, then $\Delta \Proves !A$.
\end{prop}

\begin{thm}[Weak Generalization on Constants]
If $\Gamma \Proves !A$ and $c$ is a constant not occurring in
$\Gamma$, then there is a !!{variable} $x$ not in $!A$ such that $\Gamma
\Proves \lforall[x] !A [x/c]$ and the derivation does not involve $c$.
\end{thm}

\begin{proof}
Let $!A_1, \dots, !A_n$ be a derivation of $!A$ from $\Gamma$, so $!A
= !A_n$.  Choose a !!{variable} $x$ not in $!A_1, \dots, !A_n$, and
consider the new sequence $!A_1[x/c], \dots, !A_n [x/c]$. Such a
sequence is a derivation of $!A[x/c]$ from $\Gamma$, and the constant
$c$ no longer occurs in the new sequence.
%More detail here
Now let $\Gamma' \subseteq \Gamma$ be the !!{formula}e in $\Gamma$ that
appear in the derivation of $!A [x/c]$; then $x$ is not free in any
!!{formula} in $\Gamma'$ and $\Gamma' \Proves !A [x/c]$. By
generalization, $\Gamma' \Proves \lforall[x] !A[x/c]$, whence by
monotony, $\Gamma \Proves \lforall[x] !A [x/c]$, as desired.
\end{proof}

\begin{defn} 
A term $t$ is \emph{free for $x$ in $!A$} if $x$ does not occur in
$!A$ within the scope of a quantifier $\lforall[z]$ binding a !!{variable}
$z$ occurring in $t$.
\end{defn}

%Examples of this maybe?

\begin{prop}[Change of bound !!{variable}]
If $x$ and $y$ are not free in $!A$ and are both free for $c$ in $!A$,
then $\Proves \lforall[x] !A[x/c] \liff \lforall[y] !A[y/c]$, and the
derivation does not involve $c$.
\end{prop}

\begin{thm}[Strong Generalization on Constants]
If $\Gamma \Proves !A$, the constant $c$ does not occur in $\Gamma$,
and $x$ is not free in $!A$ but it is free for $c$ in $!A$, then
$\Gamma \Proves \lforall[x] !A[x/c]$.
\end{thm}

\begin{proof}
Since $\Gamma \Proves !A$ and $c$ is not in $\Gamma$, by Weak
Generalization, there is a !!{variable} $y$ not in $!A$ such that $\Gamma
\Proves \lforall[y] !A [y/c]$.  Since $y$ is not in $!A$, it is not
free in $!A$ and it is also free for $c$ in $!A$; if moreover (by
hypothesis) $x$ is not free in $!A$ and is free for $c$ in $!A$, then
the requirements for a change of bound !!{variable} are met, so $\Proves
\lforall[x] !A[x/c] \liff \lforall[y] !A[y/c]$, whence $\Gamma \Proves
\lforall[x] !A[x/c]$.
\end{proof}

\begin{lem}
If $\Gamma$ is consistent in $\Lang L$ and $\Lang L'$ is obtained from
$\Lang L$ by adding countably many new constants $c_0, c_1, \dots$,
then $\Gamma$ is consistent in $\Lang L'$.
\end{lem}

\begin{defn}
  A set $\Gamma$ of !p{formula} of a language $\Lang {L}$ is
  \emph{saturated} if and only if for each !!{formula} $!A \in
  \Frm {L}$ and !!{variable} $x$ there is a constant $c$ such that
  $\lexists[x][!A] \lif !A(c) \in \Gamma$.
\end{defn}

% do we need to also add !p{conditional} for the universally quantified 
% !p{formula} or does max. consistency take care of those?

The following definition will be used in the proof of the next theorem.

\begin{defn}
Fix an enumeration $\langle !A_0, x_0 \rangle, \langle !A_1, x_1
\rangle, \dots$ of all !!{formula}-!!{variable} pairs of $\Lang L'$, and
define the sentence $!D_n$ by recursion on $n$. For $!D_0$,
let $c_0$ be the first new constant that does not occur in $!A_0$ and
let $!D_0$ be the !!{formula} $\lnot \lforall[x_0] !A_0 \lif \lnot
!A_0[c_0/x_0]$. Assuming that $!D_0, \dots, !D_n$ have been
defined, denote by $c_{n+1}$ the first new constant not occurring in
$!D_0, \dots, !D_n$, and let $!D_{n+1}$ be the !!{formula}
$\lnot \lforall[x_{n+1}] !A_{n+1} \lif \lnot
!A_{n+1}[c_{n+1}/x_{n+1}]$. Finally, set $!D = \{ !D_n : n
\geq 0 \}$.
\end{defn}

\begin{thm}
  Every consistent set $\Gamma$ can be extended to a saturated
  consistent set.
\end{thm}

\begin{proof}
Given a consistent $\Gamma$, expand the language by adding countably
many new constants. By the previous Lemma, $\Gamma$ is still
consistent in the richer language. Further, let $!D$ be as in the
previous definition: then $\Gamma \cup !D$ is saturated by
construction. To show that it is also consistent, it suffices to show,
by induction on $n$, that each set of the form $\Gamma \cup \{
!D_0, \dots, !D_n \}$ is consistent.

For the basis of the induction, suppose that $\Gamma \cup \{ !D_0
\}$ is inconsistent. Recall that $!D_0 = \lnot \lforall[x_0] !A_0
\lif \lnot !A_0 [c_0/x_0]$: then both of the following hold:
\[
\Gamma \Proves \lforall[x_0] !A_0
\quad
\Gamma \Proves !A_0[c_0/x_0]
\]
But $c_0$ does not occur in $\Gamma$ and the !!{variable} $x_0$ is not
free in $!A_0 [ c_0/x_0]$. Moreover, $x_0$ is free for $c_0$ in $!A_0
[c_0/x_0]$ (since by the choice of $c_0$, the only occurrences of
$c_0$ in $!A_0 [c_0/x_0]$ derive from the substitution $[c_0/x_0]$,
and therefore fall outside the scope of any quantifier binding
$x_0$). By Strong Generalization on Constants, from $\Gamma \Proves
!A_0 [c_0/x_0]$, we obtain $\Gamma \Proves \lforall[x_0] ( !A_0
[c_0/x_0] ) [x_0/c_0]$, hence $\Gamma \Proves \lforall[x_0]
!A_0$. Thus we have that both $\Gamma \Proves \lforall[x_0] !A_0$ and
$\Gamma \Proves \lnot \lforall[x_0] !A_0$, so $\Gamma$ itself is
inconsistent. Contradiction: $\Gamma$ was supposed to be
consistent. Hence $\Gamma \cup \{ !D_0\}$ is consistent. The
demonstration of the induction step is analogous.
\end{proof}

\end{document}
