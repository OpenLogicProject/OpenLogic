% Part: first-order-logic
% Chapter: completeness
% Section: introduction

\documentclass[../../include/open-logic-section]{subfiles}

\begin{document}

\olfileid{fol}{com}{int}
\olsection{Introduction}

The completeness theorem is one of the most fundamental results about
logic.  It comes in two formulations, the equivalence of which we'll
prove. In its first formulation it says somethign fundamental about
the relationship between semantic consequence and our proof system: if
a sentences~$!A$ follows from some sentences $\Gamma$, then there is
also a proof of $!A$ from $\Gamma$.  Thus, the proof system is as
strong as it can possibly be without proving things that don't
actually follow.  In its second formulation, it can be stated as a
model existence result: every consistent set of sentences is
satisfiable.

These aren't the only reasons the completeness theorem---or rather,
its proof---is important. It has a number of importnat consequences,
some of which we'll discuss separately.  For instance, since any proof
of $!A$ from $\Gamma$ is finite and so can only use finitely many of
the sentences in~$\Gamma$, it follows by the completeness theorem that
if $!A$ is a consequence of $\Gamma$, it is a consequence of already a
finite subset.  This is called \emph{compactness}.  It also follows
from the proof of the completeness theorem that any satisfiable set of
sentences has a finite or denumerable model. This result is called the
L\"owenheim-Skolem theorem.

\end{document}
