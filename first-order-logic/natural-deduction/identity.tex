% Part: first-order-logic
% Chapter: natural-deduction
% Section: identity

\documentclass[../../include/open-logic-section]{subfiles}

\begin{document}

\olfileid{fol}{ntd}{ide}

\olsection{\usetoken{P}{derivation} with \usetoken{S}{identity}}

!!^{derivation}s with the !!{identity} require additional inference rules.

\paragraph{Rules for $\eq$:}

\[
\AxiomC{[$\eq[t][t]$]$^n$}
\UnaryInfC{}
\DisplayProof
\]

\[
\AxiomC{$\eq[t_1][t_2]$}
\AxiomC{$!A(t_1)$}
\RightLabel{$\eq Elim$}
\BinaryInfC{$!A(t_2)$}
\DisplayProof
\quad
\textrm{  and  }
\quad
\AxiomC{$\eq[t_1][t_2]$}
\AxiomC{$!A(t_2)$}
\RightLabel{$\eq Elim$}
\BinaryInfC{$!A(t_1)$}
\DisplayProof
\]
where $t_1$ and $t_2$ are closed terms. The $\eq$ Intro
rule allows us to assume an identity statement.

\begin{ex}
If $s$ and $t$ are ground terms, then $!A(s), \eq[s][t] \Proves !A(t)$:
\[
\AxiomC{[$!A(s)$]}
\AxiomC{[$\eq[s][t]$]}
\RightLabel{$\eq$ Elim}
\BinaryInfC{$!A(t)$}
\DisplayProof
\]
This may be familiar as the principle of substitutability of
identicals, or Leibniz' Law.
\end{ex}

\begin{prob}
Prove that $=$ is both transitive and symmetric
\end{prob}

\begin{prob}
Give !!{derivation}s of the following !!{formula}s:
\begin{enumerate}
\item $\lforall[x][\lforall[y][((x = y \land !A(x)) \lif !A(y))]]$
\item $\lexists[x][!A(x)] \land \lforall[y][\lforall[z][((!A(y) \land !A(z)) \lif y = z)]] \lif {}$\\
\hskip 1em$\lexists[x][(!A(x) \land \lforall[y][(!A(y) \lif y = x)])]$
\end{enumerate}
\end{prob}

\begin{prop}
Natural deduction with rules for identity is sound.
\end{prop}

\begin{proof}
Any !!{formula} of the form $\eq[t][t]$ are valid, since
for every !!{structure}~$\Struct M$, $\Sat{M}{\eq[t][t]}$. (Note that
we assume the term $t$ to be ground, i.e., it contains no variables,
so variable assignments are irrelevant).

Suppose the last inference in a !!{derivation} is $=$. Then the
premises contain $\eq[t_1][t_2]$ and $!A(t_1)$. Consider a 
!!{structure}~$\Struct M$. $\Struct M$ satisfies the two premises
by induction hypothesis. So, $\Sat{M}{\eq[t_1][t_2]}$. Therefore, 
$\Value{t_1}{M} = \Value{t_2}{M}$. Let $s$ be any variable assignment, 
and $s'$ be the $x$-variant given by $s'(x) = \Value{t_1}{M} = \Value
{t_2}{M}$. By \olref[fol][syn][ext]{prop:ext-formulas}, $\Sat{M}{!A(t_2)}
[s]$ iff $\Sat{M}{!A(x)}[s']$ iff $\Sat{M}{!A(t_1)}[s]$. Since $\Sat{M}
{!A(t_1)}$ therefore $\Sat{M}{!A(t_2)}$.
\end{proof}

\end{document}
