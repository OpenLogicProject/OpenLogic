% Part: first-order-logic
% Chapter: natural-deduction
% Section: soundness

\documentclass[../../include/open-logic-section]{subfiles}

\begin{document}

\olfileid{fol}{ntd}{sou}
\olsection{Soundness}

\begin{explain}
A !!{derivation} system, such as natural deduction, is \emph{sound}
if it cannot !!{derive} things that do not actually follow.  Soundness is
thus a kind of guaranteed safety property for !!{derivation} systems.
Depending on which proof theoretic property is in question, we would
like to know for instance, that
\begin{enumerate}
\item every !!{derivable} !!{sentence} is valid;
\item if a !!{sentence} is !!{derivable} from some others, it is also a
  consequence of them;
\item if a set of !!{sentence}s is inconsistent, it is unsatisfiable.
\end{enumerate}
These are important properties of a !!{derivation} system. If any of them
do
not hold, the !!{derivation} system is deficient---it would !!{derive} too
much.
Consequently, establishing the soundness of a !!{derivation} system is of
the
utmost importance.
\end{explain}

\begin{thm}[Soundness]
\ollabel{ntd-soundness} If $!A$ is !!{derivable} from the open assumptions
$!C_1, ... !C_n$, then if $\Sat{M}{\{!C_1, ... !C_n\}}$, then
$\Sat{M}{!A}$.
\end{thm}

\begin{proof}
\emph{Inductive Hypothesis}: The premises of an inference rule follow
fromthe open assumptions of the subproofs ending in those premises.
 
\emph{Inductive Step}: Show that $!A$ follows from the open assumptions of
the entire
proof.

Let $\Pi$ be a !!{derivation} of $!A$. We proceed by
induction on the number of inferences in~$\Pi$.

If the number of inferences is~0, then $\Pi$ consists only of an
initial !!{formula}. Every initial !!{formula} $!A$ is an open assumption,
and as such, any !!{structure} $\Struct{M}$ that satisfies all of the
open assumptions of the proof satisfies $!A$.

If the number of inferences is greater than~0, we distinguish cases
according to the type of the lowermost inference. By induction
hypothesis, we can assume that the premises of that inference are
valid.

First, we consider the possible inferences with only one premise.

\begin{enumerate}
\item Suppose that the last inference is $\lnot$~Intro: 
By inductive hypothesis, $\lfalse$ follows from the open assumptions
$!C_1,...,!C_n$ and $!A$. Consider a !!{structure}~$\Struct M$. We need to
show that, if $\Sat{M}{\{!C_1,...,!C_n\}}$,
then $\Sat{M}{\lnot !A}$. Suppose for reductio that
$\Sat{M}{\{!C_1,...,!C_n\}}$,
and $\Sat/{M}{\lnot !A}$. But this would mean that $\Sat{M}{!C_1,...,!C_n,
!A}$. But this is contrary to our inductive hypothesis. So, $\Sat{M}{\lnot !A}$.
  
\item The last inference is $\lnot$~Elim: Exercise.

\item The last inference is $\land$~Elim: There are two variants: $!A$ or
$!B$ may be inferred from the premise $!A \land !B$. Consider the first
case.
By inductive hypothesis, $!A \land !B$ follows from the open assumptions 
$!C_1,...,!C_n$. Consider a !!{structure}~$\Struct M$. We need to show
that, if $\Sat{M}{\{!C_1,...,!C_n\}}$,
then $\Sat{M}{!A}$. By our inductive hypothesis, we know that
$\Sat{M}{!A\land !B}$. So, $\Sat{M}{!A}$.
  The case where $!B$ is inferred from $!A \land !B$ is handled similarly.
  
\item The last inference is $\lor$~Intro: There are two variants: $!A \lor
!B$ may be inferred from the premise $!A$ or the premise $!B$. Consider
the first case. By inductive hypothesis, $!A$ follows from the open 
assumptions $!C_1,...,!C_n$.
Consider a !!{structure}~$\Struct M$. We need to show that, if
$\Sat{M}{\{!C_1,...,!C_n\}}$,
then $\Sat{M}{!A \lor !B}$. It must be the case that $\Sat{M}{!A}$, by
inductive hypothesis.So it must be the case that $\Sat{M}{!A \lor !B}$.
  The case where $!A \lor !B$ is inferred from $!B$ is handled similarly.
  
\item The last inference is $\lif$~Intro: $!A \lif !B$ is inferred from a
subproof with assumption $!A$ and conclusion $!B$. By inductive
hypothesis, $!B$ follows from the open assumptions $!C_1,...,!C_n$ 
and $!A$. Consider a !!{structure}~$\Struct M$.We need to show 
that, if $\Sat{M}{\{!C_1,...,!C_n\}}$, then $\Sat{M}{!A \lif !B}$.
For reductio, suppose that $!A \lif !B$ does not follow from
$!C_1,...,!C_n$. This means that some !!{structure}
$\Struct M$, $\Sat{M}{\{!C_1,...,!C_n\}}$, and $\Sat/{M}{!A \lif !B}$. So,
$\Sat{M}{!A}$ and $\Sat/{M}{!B}$. But
by hypothesis, $!B$ is a consequence of $!C_1,...,!C_n$ and $!A$. So,
$\Sat{M}{!A \lif !B}$.
  
\item The last inference is $\forall$~Intro: .
 
$!A(a)$ is a consequence of the open assumptions $!C_1,...,!C_n$.
Consider some structure, $\Struct{M}$, such that $\Sat{M}{!C_1,...!C_n}$.
By inductive hypothesis, $\Sat{M}{!A(a)}$.
 Let $\Struct{M'}$ be exactly like $\Struct M$ except
that$\Assign{a}{M} \neq \Assign{a}{M'}$. We must have $\Sat{M'}{!A(a)}$.

  We now show that $\Sat{M}{\lforall[x][!A(x)]}$.  To do this, we have
  to show that for every variable assignment~$s$,
  $\Sat{M}{\lforall[x][!A(x)]}[s]$.  This in turn means that for every
  $x$-variant $s'$ of $s$, we must have $\Sat{M}{!A(x)}[s']$.  So
  consider any variable assignment~$s$ and let $s'$ be an $x$-variant
  of~$s$.  Since $!C_1,...,!C_n$ consists entirely of sentences,
  $\Sat{M}{!E}[s]$ iff $\Sat{M}{!E}[s']$ iff $\Sat{M}{!E}$ for all
  $!E \in \{!C_1,...,!C_n\}$.  Let $\Struct M'$ be like $\Struct M$
  except that $\Assign{a}{M'} = s'(x)$.  Then $\Sat{M}{!A(x)}[s']$ iff
  $\Sat{M'}{!A(a)}$ (as $!A(x)$ does not contain~$a$).  Since we've
  already established that $\Sat{M'}{!A(a)}$ for all $\Struct M'$
  which differ from $\Struct M$ at most in what they assign to~$a$,
  this means that $\Sat{M}{!A(x)}[s']$.  Thus we've shown that
  $\Sat{M}{\lforall[x][!A(x)]}[s]$.  Since $s$ is an arbitrary variable
  assignment and $\lforall[x][!A(x)]$ is a sentence, then
  $\Sat{M}{\lforall[x][!A(x)]}$.

  
\item The last inference is $\lexists$~Intro: Exercise.

\item The last inference is $\forall$~Elim: Exercise.


\end{enumerate}
Now let's consider the possible inferences with several premises:
$\lor$~Elim, $\land$~Intro, and $\lif$~Elim.
\begin{enumerate}

\item The last inference is $\land$~Intro. $!A \land !B$ is inferred from
the premises $!A$
and $!B$. By induction hypothesis, $!A$ follows from the open assumptions
$!C_1,...,!C_n$ and $!B$ follows from the open assumptions
$!D_1,...,!D_n$.Consider a !!{structure}~$\Struct M$. We need to show that, 
if $\Sat{M}{\{!C_1,...,!C_n,!D_1,...,!D_n\}}$, then $\Sat{M}{!A \land !B}$.
It must be the case that $\Sat{M}{!A}$, and $\Sat{M}{!B}$, by inductive
hypothesis.
So by definition, it must be the case that $\Sat{M}{!A \land !B}$.
  
\item The last inference is $\lor$~Elim: Exercise.

\item The last inference is $\lif$~Elim. $!B$ is inferred from the
premises $!A \lif !B$ and $!A$. By induction hypothesis, $!A \lif !B$ 
follows from the open assumptions $!C_1,...,!C_n$
and $!A$ follows from the open assumptions $!D_1,...,!D_n$. Consider a
!!{structure}~$\Struct M$. We need to show that, if
$\Sat{M}{\{!C_1,...,!C_n,!D_1,...,!D_n\}}$, then $\Sat{M}{!B}$.
It must be the case that $\Sat{M}{!A \lif !B}$, and $\Sat{M}{!A}$, by
inductive hypothesis.
So by definition, it must be the case that $\Sat{M}{!B}$.
    
\item The last inference is $\lexists$~Elim: Exercise.
\end{enumerate}
\end{proof}

\begin{prob}
Complete the proof of \olref[fol][seq][sou]{sequent-soundness}.
\end{prob}

\begin{cor}
\ollabel{weak-soundness}
If $\Proves !A$ then $!A$ is valid.
\end{cor}

\begin{cor}
\ollabel{entailment-soundness}
If $\Gamma \Proves !A$ then $\Gamma \Entails !A$.
\end{cor}

\begin{proof}
If $\Gamma \Proves !A$ then for some finite subset $\Gamma_0 \subseteq
\Gamma$, there is !!a{derivation} of $!A$ from $\Gamma_0$.  By
\olref{ntd-soundness}, every !!{structure} $\Struct M$ either
makes some $!B \in \Gamma_0$ false or makes $!A$ true.  Hence, if
$\Sat{M}{\Gamma_0}$ then also $\Sat{M}{!A}$.
\end{proof}

\begin{cor}
\ollabel{consistency-soundness}
If $\Gamma$ is satisfiable, then it is consistent.
\end{cor}

\begin{proof}
We prove the contrapositive.  Suppose that $\Gamma$ is not
consistent.  Then $\Gamma \Proves \lfalse$, i.e., there is a finite
$\Gamma_0 \subseteq \Gamma$ and !!a{derivation} of $\lfalse$
from $\Gamma_0$. By \olref{ntd-soundness}, any !!{structure} 
$\Struct M$ that satisfies $\Gamma_0$ must satisfy $\lfalse$.  
Since $\Sat/{M}{\lfalse}$ for every !!{structure}~$\Struct M$,
there must be an $!E \in \Gamma_0$ so that $\Sat/{M}{!E}$, 
and since $\Gamma_0 \subseteq \Gamma$, that $!E$ is also
in~$\Gamma$.  In other words, no $\Struct M$ satisfies $\Gamma$, 
i.e., $\Gamma$ is not satisfiable.
\end{proof}

\end{document}
