% Part: incompleteness
% Chapter: theories-computability
% Section: extensions-of-q-not-decidable

\documentclass[../../include/open-logic-section]{subfiles}

\begin{document}

\olfileid{inc}{tcp}{cqn}

\olsection{Consistent Extensions of $\Th{Q}$ are Undecidable}

\begin{explain}
Remember that a theory is {\em consistent} if it does not prove $!A$
and $\lnot !A$ for any formula $!A$. Since anything follows from a
contradiction, an inconsistent theory is trivial: every sentence is
provable. Clearly, if a theory if $\omega$-consistent, then it is
consistent. But being consistent is a weaker requirement (i.e., there
are theories that are consistent but not $\omega$-consistent --- we
will see an example soon). We can weaken the assumption in
\olref[oqn]{thm:oconsis-q} to simple consistency to obatin a stronger
theorem.
\end{explain}

\begin{lem}
There is no ``universal computable relation.'' That is, there is no
binary computable relation $R(x,y)$, with the following property:
whenever $S(y)$ is a unary computable relation, there is some $k$ such
that for every $y$, $S(y)$ is true if and only if $R(k,y)$ is true.
\end{lem}

\begin{proof}
Suppose $R(x,y)$ is a universal computable relation. Let $S(y)$
be the relation $\lnot R(y,y)$. Since $S(y)$ is computable, for some
$k$, $S(y)$ is equivalent to $R(k,y)$. But then we have that $S(k)$ is
equivalent to both $R(k,k)$ and $\lnot R(k,k)$, which is a
contradiction.
\end{proof}


\begin{thm}
Let $\Th{T}$ be any consistent theory that includes $\Th{Q}$. Then
$\Th{T}$ is not decidable.
\end{thm}


\begin{proof}
Suppose $\Th{T}$ is a consistent, decidable
extension of $\Th{Q}$. We will obtain a contradiction by using $\Th{T}$ to
define a universal computable relation.

Let $R(x,y)$ hold if and only if
\begin{quote}
$x$ codes a formula $!D(u)$, and $\Th{T}$ proves $!D(\num y)$.
\end{quote}
Since we are assuming that $\Th{T}$ is decidable, $R$ is computable. Let us
show that $R$ is universal. If $S(y)$ is any computable relation, then
it is representable in $\Th{Q}$ (and hence $\Th{T}$) by a formula
$!D_S(u)$. Then for every $n$, we have
\begin{eqnarray*}
S(\num n) & \lif & T \vdash !D_S(\num n) \\
& \lif & R(\#(!D_S(u)), n)
\end{eqnarray*}
and
\begin{eqnarray*}
\lnot S(\num n) & \lif & T \vdash \lnot !D_S(\num n) \\
& \lif & T \not\vdash !D_S(\num n) \quad \text{(since $\Th{T}$ is
  consistent)} \\
& \lif & \lnot R(\#(!D_S(u)),n).
\end{eqnarray*}
That is, for every $y$, $S(y)$ is true if and only if
$R(\#(!D_S(u)),y)$ is. So $R$ is universal, and we have the
contradiction we were looking for.
\end{proof}

Let ``true arithmetic'' be the theory $\Setabs{!A}{ \Sat{\Nat}{!A}}$,
that is, the set of sentences in the language of arithmetic that are
true in the standard interpretation.

\begin{cor}
True arithmetic is not decidable.
\end{cor}

%In Section~\olref{undefinability:truth:section} we will state a stronger
%result, due to Tarski.

\end{document}

