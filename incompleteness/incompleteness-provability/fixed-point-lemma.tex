% Part: incompleteness
% Chapter: incompleteness-provability
% Section: fixed-point-lemma

\documentclass[../../include/open-logic-section]{subfiles}

\begin{document}

\olfileid{inc}{inp}{fix}

\olsection{The Fixed-Point Lemma}

Let $\fn{diag}(y)$ be the computable (in fact, primitive recursive)
function that does the following: if $y$ is the G\"odel number of a
formula~$!B(x)$, $\fn{diag}(y)$ returns the G\"odel number
of~$!B(\gn{!B(x)})$. ($\gn{!B(x)}$ is the standard numeral of the
G\"odel number of~$!B(x)$, i.e., $\num{\Gn{!B(x)}}$). If $\Obj{diag}$
were a function symbol in $\Th{T}$ representing the function
$\fn{diag}$, we could take $!A$ to be the formula
$!B(\Obj{diag}(\gn{!B(\Obj{diag}(x))}))$. Notice that
\begin{align*}
\fn{diag}(\Gn{!B(\Obj{diag}(x))}) & = 
\Gn{!B(\Obj{diag}(\gn{!B(\Obj{diag}(x))})} \\
& = \Gn{!A}.
\end{align*}
Assuming $\Th{T}$ can prove
\[
\Obj{diag}(\gn{!B(\Obj{diag}(x))}) = \gn{!A},
\]
it can prove $!B(\Obj{diag}(\gn{!B(\Obj{diag}(x))}))
\liff !B(\gn{!A})$. But the left hand side is, by
definition, $!A$.

In general, $\Obj{diag}$ will not be a function symbol of
$\Th{T}$. But since $\Th{T}$ extends $\Th{Q}$, the function
$\fn{diag}$ will be {\em represented} in $\Th{T}$ by some formula
$!D_{\fn{diag}}(x,y)$. So instead of writing $!B(\Obj{diag}(x))$ we
will have to write $\lexists[y][(!D_{\fn{diag}}(x,y) \land
  !B(y))]$. Otherwise, the proof sketched above goes through.

\begin{lem}
  Let $\Th{T}$ be any theory extending $\Th{Q}$, and let $!B(x)$ be any
  formula with free variable $x$. Then there is a sentence $!A$ such
  that $\Th{T}$ proves $!A \liff !B(\gn{!A})$.
\end{lem}


\begin{proof}
Given $!B(x)$, let $!E(x)$ be the formula
$\lexists[y][(!D_{\fn{diag}}(x,y) \land !B(y))]$ and let $!A$ be the
formula $!E(\gn{!E(x)})$.

Since $!D_{\fn{diag}}$ represents $\fn{diag}$, $\Th{T}$ can prove
\[
\lforall[y][(!D_{\fn{diag}}(\gn{!E(x)},y) \liff
  \eq[y][\num{\fn{diag}(\gn{!E(x)})}])].
\]
But by definition, 
$\fn{diag}(\Gn{!E(x)}) = \Gn{!E(\gn{!E(x)})} =
\Gn{!A}$, so $\Th{T}$ can prove
\[
\lforall[y][(!D_{\fn{diag}}(\gn{!E(x)},y) \liff \eq[y][\gn{!A}])].
\]
Going back to the definition of $!E(x)$, we see
$!E(\gn{!E(x)})$ is just the formula
\[
\lexists[y][(!D_{\fn{diag}}(\gn{!E(x)},y) \land !B(y))].
\]
Using the last two sentences and ordinary first-order logic, one can
then prove
\[
!E(\gn{!E(x)}) \liff !B(\gn{!A}).
\]
But the left-hand side is just $!A$. 
\end{proof}

\begin{digress}
You should compare this to the proof of the fixed-point lemma in
computability theory. The difference is that here we want to define a
{\em statement} in terms of itself, whereas there we wanted to define
a {\em function} in terms of itself; this difference aside, it is
really the same idea.
\end{digress}

\end{document}
