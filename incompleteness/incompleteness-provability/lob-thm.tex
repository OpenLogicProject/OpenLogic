% Part: incompleteness
% Chapter: incompleteness-provability
% Section: lob-thm

\documentclass[../../include/open-logic-section]{subfiles}

\begin{document}

\olfileid{inc}{inp}{lob}

\olsection{L\"ob's Theorem}

\newcommand{\pt}{\fn{Prov}_T}

In this section, we will consider a fun application of the fixed-point
lemma. We now know that any ``reasonable'' theory of arithmetic is
incomplete, which is to say, there are sentences $!A$ that are neither
provable nor refutable in the theory. One can ask whether, in general,
a theory can prove ``If I can prove $!A$, then it must be true.'' The
answer is that, in general, it can't. More precisely, we have:

\begin{thm}
  Let $\Th{T}$ be any theory extending $\Th{Q}$, and suppose
  $\OProv[T](y)$ is a formula satisfying conditions 1--3 from
  \olref[2in]{sec}. If $\Th{T}$ proves $\OProv[T](\gn{!A}) \lif !A$,
  then in fact $\Th{T}$ proves $!A$.
\end{thm}
Put differently, if $!A$ is not provable in $\Th{T}$, $\Th{T}$ can't
prove $\OProv[T](\gn{!A}) \lif !A$. This is known as L\"ob's theorem.

The heuristic for the proof of L\"ob's theorem is a clever proof that
Santa Claus exists. (If you don't like that conclusion, you are free
to substitute any other conclusion you would like.) Here it is:
\begin{enumerate}
\item Let $X$ be the sentence, ``If $X$ is true, then Santa Claus
  exists.''
\item Suppose $X$ is true.
\item Then what it says is true; i.e., if $X$ is true, then
  Santa Claus exists.
\item Since we are assuming $X$ is true, we can conclude that
  Santa Claus exists.
\item So, we have shown: ``If $X$ is true, then Santa Claus exists.''
\item But this is just the statement $X$. So we have shown that $X$ is
  true.
\item But then, by the argument above, Santa Claus exists.
\end{enumerate}
A formalization of this idea, replacing ``is true'' with ``is
provable,'' yields the proof of L\"ob's theorem. 

\begin{proof}
Suppose $!A$ is a sentence such that $\Th{T}$ proves
$\OProv[T](\gn{!A}) \lif !A$. Let $!B(y)$ be the formula $\OProv[T](y)
\lif !A$, and use the fixed-point lemma to find a sentence $!D$
such that $\Th{T}$ proves $!D \liff !B(\gn{!D})$. Then each of the
following is provable in $\Th{T}$:
\begin{align*}
& !D \lif (\OProv[T](\gn{!D}) \lif !A) \\
& \OProv[T](\gn{!D \lif (\OProv[T](\gn{!D}) \lif !A)}) && \text{by 1} \\
& \OProv[T](\gn{!D}) \lif \OProv[T](\gn{\OProv[T](\gn{!D}) \lif !A})
  && \text{using 2} \\
& \OProv[T](\gn{!D}) \lif \\
&\qquad(\OProv[T](\gn{\OProv[T](\gn{!D})}) \lif \OProv[T](\gn{!A}))
  && \text{using 2} \\
& \OProv[T](\gn{!D}) \lif \OProv[T](\gn{\OProv[T](\gn{!D})}) 
  && \text{by 3} \\
& \OProv[T](\gn{!D}) \lif \OProv[T](\gn{!A})  \\
& \OProv[T](\gn{!A}) \lif !A && \text{by assumption} \\
& \OProv[T](\gn{!D}) \lif !A \\
& !D && \text{def of $!D$} \\
& \OProv[T](\gn{!D}) && \text{by 1} \\
& !A
\end{align*}
\end{proof}

With L\"ob's theorem in hand, there is a short proof of the first
incompleteness theorem (for theories having a provability predicate
satisfying 1--3): if a theory proves $\OProv[T](\gn{\eq[0][1]}) \lif
\eq[0][1]$, it proves $\eq[0][1]$. 

% Going in the other direction, for homework I
% may ask you to work through a short proof of L\"ob's theorem, using
% the second incompleteness theorem instead of the fixed-point lemma.

\begin{prob}
Let $\Th{T}$ be a computably axiomatized theory, and
let $\OProv[T]$ be a provability predicate for $\Th{T}$. Consider the
following four statements:
\begin{enumerate}
\item If $T \Proves !A$, then $T \Proves \OProv[T](\gn{!A})$.
\item $T \Proves !A \lif \OProv[T](\gn{!A})$.
\item If $T \Proves \OProv[T](\gn{!A})$, then $T \Proves !A$.
\item $T \Proves \OProv[T](\gn{!A}) \lif !A$
\end{enumerate}
Under what conditions are each of these statements true?
\end{prob}

\end{document}
