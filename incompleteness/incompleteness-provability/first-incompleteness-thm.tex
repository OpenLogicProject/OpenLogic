% Part: incompleteness
% Chapter: incompleteness-provability
% Section: first-incompleteness-thm

\documentclass[../../include/open-logic-section]{subfiles}

\begin{document}

\olfileid{inc}{inp}{1in}

\olsection{The First Incompleteness Theorem}

We can now describe G\"odel's original proof of the first
incompleteness theorem. Let $\Th{T}$ be any computably axiomatized theory
in a language extending the language of arithmetic, such that $\Th{T}$
includes the axioms of $\Th{Q}$. This means that, in particular, $\Th{T}$
represents computable functions and relations.

We have argued that, given a reasonable coding of formulas and proofs
as numbers, the relation $\Prf[T](x,y)$ is computable, where $\Prf[T](x,y)$
holds if and only if $x$ is a proof of formula $y$ in $\Th{T}$. In fact,
for the particular theory that G\"odel had in mind, G\"odel was able
to show that this relation is primitive recursive, using the list of
45 functions and relations in his paper. The 45th relation, $x B y$,
is just $\Prf[T](x,y)$ for his particular choice of $\Th{T}$. Remember that
where G\"odel uses the word ``recursive'' in his paper, we would now
use the phrase ``primitive recursive.''

Since $\Prf[T](x,y)$ is computable, it is representable in $\Th{T}$. We
will use $\OPrf[T](x,y)$ to refer to the formula that represents
it. Let $\OProv[T](y)$ be the formula
$\lexists[x][\OPrf[T](x,y)]$. This describes the 46th relation,
$\fn{Bew}(y)$, on G\"odel's list. As G\"odel notes, this is the only
relation that ``cannot be asserted to be recursive.''  What he
probably meant is this: from the definition, it is not clear that it
is computable; and later developments, in fact, show that it isn't.

\begin{defn}
\ollabel{thm:oconsis-q}
A theory $\Th{T}$ is $\omega$-consistent if the following holds: if
$\lexists[x][!A(x)]$ is any sentence and $\Th{T}$ proves $\lnot
!A(\num 0)$, $\lnot !A(\num 1)$, $\lnot !A(\num 2)$, \dots then $\Th{T}$
does not prove $\lexists[x][!A(x)]$.
\end{defn}

We can now prove the following.
\begin{thm}
\ollabel{thm:first-incompleteness}
Let $\Th{T}$ be any $\omega$-consistent, computably axiomatized theory
extending $\Th{Q}$. Then $\Th{T}$ is not complete.
\end{thm}

\begin{proof}
Let $\Th{T}$ be any computably axiomatized theory containing $\Th{Q}$,
and let $\OProv[T](y)$ be the formula we described above. By the
fixed-point lemma, there is a formula $!G_\Th{T}$ such that $\Th{T}$ proves
\begin{equation}
\ollabel{eqn:qpf}
!G_\Th{T} \liff \lnot \OProv[T](\gn{!G_\Th{T}}).
\end{equation}
Note that $!A$ says, in essence, ``I am not provable.''

We claim that
\begin{enumerate}
\item If $\Th{T}$ is consistent, $\Th{T}$ doesn't prove $!G_\Th{T}$
\item If $\Th{T}$ is $\omega$-consistent, $\Th{T}$ doesn't prove
  $\lnot !G_\Th{T}$.
\end{enumerate}
This means that if $\Th{T}$ is $\omega$-consistent, it is incomplete,
since it proves neither $!G_\Th{T}$ nor $\lnot !G_\Th{T}$. Let us take
each claim in turn.

Suppose $\Th{T}$ proves $!G_\Th{T}$. Then there {\em is} a proof, and
so, for some number $m$, the relation $\Prf[T](m, \Gn{!G_\Th{T}})$
holds. But then $\Th{T}$ proves the sentence $\OPrf[T](\num m,
\gn{!G_\Th{T}})$. So $\Th{T}$ proves
$\lexists[x][\OPrf[T](x,\gn{!G_\Th{T}})]$, which is, by definition,
$\OProv[T](\gn{!G_\Th{T}})$. By \olref{eqn:qpf}, $\Th{T}$ proves $\lnot
!G_\Th{T}$. We have shown that if $\Th{T}$ proves $!G_\Th{T}$, then it
also proves $\lnot !G_\Th{T}$, and hence it is inconsistent.

For the second claim, let us show that if $\Th{T}$ proves $\lnot
!G_\Th{T}$, then it is $\omega$-inconsistent. Suppose $\Th{T}$ proves
$\lnot !G_\Th{T}$. If $\Th{T}$ is inconsistent, it is
$\omega$-inconsistent, and we are done. Otherwise, $\Th{T}$ is
consistent, so it does not prove $!G_\Th{T}$. Since there is no proof
of $!G_\Th{T}$ in $\Th{T}$, $\Th{T}$ proves
\[
\lnot \OPrf[T](\num 0, \gn{!G_\Th{T}}), \lnot \OPrf[T](\num 1,
\gn{!G_\Th{T}}), \lnot \OPrf[T](\num 2, \gn{!G_\Th{T}}), \dots
\]
On the other hand, by \olref{eqn:qpf}, $\lnot !G_\Th{T}$
is equivalent to $\lexists[x][\OPrf[T](x,\gn{!G_\Th{T}})]$. So $\Th{T}$
is $\omega$-inconsistent.
\end{proof}

\end{document}
