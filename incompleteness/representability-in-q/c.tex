% Part: incompleteness
% Chapter: representability-in-q
% Section: c

\documentclass[../../include/open-logic-section]{subfiles}

\begin{document}

\olfileid{inc}{req}{cee}
\olsection{The Functions $C$}

Let $C$ be the smallest set of functions containing
\begin{enumerate}
\item $0$,
\item successor,
\item addition,
\item multiplication,
\item projections, and
\item the characteristic function for equality, $\Char{=}$;
\end{enumerate}
and closed under
\begin{enumerate}
\item composition, and
\item unbounded search, applied to regular
functions.
\end{enumerate}
Remember this last restriction means simply that you can only use the
$\mu$ operation when the result is total. Compare this to the
definition of the \emph{general recursive} functions: here we have
added plus, times, and $\Char{=}$, but we have dropped primitive
recursion.

Clearly everything in $C$ is recursive, since plus, times,
and $\Char{=}$ are. We will show that the converse is also true; this
amounts to saying that with the other stuff in $C$ we can carry out
primitive recursion.

\end{document}
