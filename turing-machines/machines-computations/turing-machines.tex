% Part: computability
% Chapter: tm-computations
% Section: turing-machines

\documentclass[../../include/open-logic-section]{subfiles}

\begin{document}

\olfileid{cmp}{tur}{tur}
\olsection{Turing Machines}

\begin{explain}
The formal definition of what constitutes a Turing machine looks
abstract, but is actually simple: it merely packs into one
mathematical !!{structure} all the information needed to specify the
workings of a Turing machine. This includes (1) which states the
machine can be in, (2) which symbols are allowed to be on the tape, (3)
which state the machine should start in, and (4) what the instruction
set of the machine is.
\end{explain}

\begin{defn}
A \emph{Turing machine} $T = \langle Q, \Sigma, s, I\rangle$ consists of
\begin{enumerate}
\item a finite set of states $Q$ which includes the \emph{halting state}~$h$,
\item a finite alphabet $\Sigma$ which includes $\TMendtape$ and
  $\TMblank$,
\item an initial state $s \in Q$,
\item a finite instruction set~$I \subseteq Q \times \Sigma \times Q
  \times \Sigma \times \{L, R, N\}$.
\end{enumerate}
\end{defn}

\begin{explain}
We assume that the tape is infinite in one direction only. For this
reason it is useful to designate a special symbol~$\TMendtape$ as
a marker for the left end of the tape. This makes it easier for
Turing machne programs to tell when they're ``in danger'' of running
off the tape. Other definitions of Turing machines are possible,
including one where the tape is infinite in both directions. In that
case, marker for the left end of the tape is not necessary.
\end{explain}



\end{document}
