% Part: computability
% Chapter: tm-computations
% Section: turing-machines

\documentclass[../../include/open-logic-section]{subfiles}

\begin{document}

\olfileid{cmp}{tur}{con}
\olsection{Configurations and Computations}

\begin{explain}
The imaginary mechanism consisting of tape, read/write head, and
Turing machine program is really just in intuitive way of visualizing
what a Turing machine computation is.  Formally, we can define the
computation of a Turing machine on a given input as a sequence of
\emph{configurations}---and a configuration in turn is a sequence of
symbols (corresponding to the contents of the tape at a given point in
the computation), a number indicating the position of the read/write
head, and a state.  Using these, we can define what the Turing machine
$M$ computes on a given input.
\end{explain}

\begin{defn}
A \emph{configuration} of Turing machine $M = \langle Q, \Sigma, s,
I\rangle$ is a triple $\langle C, n, q\rangle$ where
\begin{enumerate}
\item $C \in \Sigma^*$ is a finite sequence of symbols from $\Sigma$,
\item $n \in \Nat$ is a number $\le \len{C}$, and
\item $q \in Q$
\end{enumerate}
\end{defn}

\begin{explain}
The potential input for a Turing machine is a sequence of symbols,
usually a sequence that encodes a number in some form.  The initial
configuration of the Turing machine is that configuration in which we
start the Turing machine to work on that input: the tape contains the
tape end marker immediately followed by the input written on the
squares to the right, the read/write head is scanning the leftmost
square of the tape (i.e., the left end marker), and the mechanism is
in the designated start state~$s$.
\end{explain}

\begin{defn}
The \emph{initial configuration} of $M$ for input $I \in \Sigma^*$ is
\[
\langle \TMendtape \frown I, 0, s\rangle
\]
\end{defn}

\begin{explain}
The~$\frown$ symbol is for \emph{concatenation} - we want to
ensure that there are no blanks between the left end marker and
the beginning of the input.
\end{explain}

\begin{defn}
We say that a configuration $\langle C, n, q\rangle$ \emph{yields
  $\langle C', n', q'\rangle$ in one step} (according to $M$), iff
\begin{enumerate}
\item the $n$-th symbol of $C$ is $\sigma$,
\item the instruction set of $M$ contains a tuple $\langle q, \sigma,
  q', \sigma', D\rangle$,
\item the $n$-th symbol of $C'$ is $\sigma'$,
\item
\begin{enumerate}
\item $D = L$ and $n' = n -1$, or
\item $D = R$ and $n' = n$, or
\item $D = N$ and $n' = n$,
\end{enumerate}
\item for all $i \neq n$, $C'(i) = C(i)$,
\item if $n' > \len{C}$, then $\len{C'} = \len{C} + 1$ and the $n'$-th
  symbol of $C'$ is $\TMblank$.
\end{enumerate}
\end{defn}

\begin{defn}
A \emph{run of $M$ on input~$I$} is a sequence $C_i$ of configurations
of $M$, where $C_0$ is the initial configuration of $M$ for input $I$,
and each $C_i$ yields $C_{i+1}$ in one step.

We say that $M$ \emph{halts on input $I$ after $k$ steps} if $C_k =
\langle \TMendtape \frown O, n, h\rangle$.  In that case the
\emph{output} of $M$ for input $I$ is $O$.
\end{defn}

\begin{ex}
For any machine and input, it is possible to trace through the configurations
to determine the output. we will first run the even machine on an input of $4$ 
strokes, in which case, the machine will halt. We will then run the machine on 
an input of ~$3$ strokes, where the machine will loop infinitely (and thus reject
the input). 

The machine starts in state~$q_1$, scanning the leftmost stroke.
The initial configuration of the machine is thus:
\[
\TMendtape \TMstroke_1 \TMstroke \TMstroke \TMstroke \TMblank \ldots
\]
The above configuration is straightforward. As can be seen, the machine starts
in state one, scanning the leftmost stroke. This is represented by a subscript of
the state name on the first stroke. The applicable instruction at this point is $\langle
q_1, \TMstroke, q_2, \TMstroke, \TMright \rangle$, and so the machine moves
right on the tape and changes to state $q_2$.
\[
\TMendtape \TMstroke \TMstroke_2 \TMstroke \TMstroke \TMblank \ldots
\]
Since the machine is now in state 2 scanning a stroke, we have the instruction
$\langle q_2, \TMstroke, q_2, \TMstroke, \TMright \rangle$. This results in the
configuration
\[
\TMendtape \TMstroke \TMstroke \TMstroke_1 \TMstroke \TMblank \ldots
\]
As the machine loops, the rules are applied again in the same order, resulting
in the following two configurations:
\[
\TMendtape \TMstroke \TMstroke \TMstroke \TMstroke_2 \TMblank \ldots
\]
\[
\TMendtape \TMstroke \TMstroke \TMstroke \TMstroke \TMblank_1 \ldots
\]
The machine is now in state~$q_1$ scanning a blank. Based on the flow
chart, we can easily see that there is no instruction to carry out, and thus the
machine has halted. This means that the input has been accepted.

Suppose next we start the machine with an input of three strokes. The
first few configurations are similar, as the same instructions are carried, with
only a small difference of the tape input:
\[
\TMendtape \TMstroke_1 \TMstroke \TMstroke \TMblank \ldots
\]
\[
\TMendtape \TMstroke \TMstroke_2 \TMstroke \TMblank \ldots
\]
\[
\TMendtape \TMstroke \TMstroke \TMstroke_1 \TMblank \ldots
\]
\[
\TMendtape \TMstroke \TMstroke \TMstroke \TMblank_0 \ldots
\]
The machine has now traversed past all the strokes, and is reading
a blank in state~$q_2$. As shown in the diagram, there is an instruction
of the form $\langle q_2, \TMblank, q_2, \TMblank, \TMright \rangle$.
Since the tape is inifinitely blank to the right, the machine will continue to
execute this instruction \emph{ad infinitum}. The machine will never halt,
and does not accept the input.
\end{ex}


\end{document}
