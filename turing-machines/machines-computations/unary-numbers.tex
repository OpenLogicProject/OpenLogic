% Part: computability
% Chapter: tm-computations
% Section: unary-numbers

\documentclass[../../include/open-logic-section]{subfiles}

\begin{document}

\olfileid{cmp}{tur}{una}
\olsection{Unary Representation of Numbers}

\begin{explain}
Turing machines work on sequences of symbols written on their tape.
Depending on the alphabet a Turing machine uses, these sequences of
symbols can represent various inputs and outputs.  Of particular
interest, of course, are Turing machines which compute
\emph{arithmetical} functions, i.e., functions of natural numbers.
One very simple way to represent positive integers is by coding them
as sequences of a single symbol~$\TMstroke$.
\end{explain}

\begin{defn}
A Turing machine~$M$ \emph{computes} the function $f\colon \Nat^n \to \Nat$ iff
$M$ halts on input
\[
\TMstroke^{k_1} \TMblank \TMstroke^{k_2} \TMblank \dots \TMblank \TMstroke^{k_n}
\]
with output $\TMstroke^{f(k_1, \dots, k_n)}$.
\end{defn}

% need examples!

\end{document}
