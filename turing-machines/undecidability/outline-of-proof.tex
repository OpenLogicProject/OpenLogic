% Part: turing-machines
% Chapter: undecidability
% Section: outline-of-proof

\documentclass[../../include/open-logic-section]{subfiles}

\begin{document}

\olfileid{tms}{und}{out}
\olsection{Outline of Proof}

\begin{explain}
The proof of undecidability is complex, and it will be beneficial to
outline the proof first. 

We know that first-order logic is decidable iff there is an effective mechanical
method for determining whether or not a sentence is valid. We also know, from
the Chruch-Turing thesis, that every function which is computable is 
Turing-computable. We have proved that the function~$h(m,w)$, that halts with
an output $\TMstroke$ if the Turing-machine described by $m$ halts on input $w$ 
and outputs $\TMblank$ otherwise, is not Turing-computable. Via the Church-Turing
thesis,~$h$ is not computable. 

What we need to show is that if first-order logic is decidable, then~$h$
is computable. By doing so, we show that first-order logic is not decidable.

We will proceed in five parts. First, we show that
if first-order-logic with identity and terms ($\Lang{L_T}$) is decidable,
then the halting problem is solvable (i.e., we reduce the halting problem to the
decision problem). We then show that for any sentence $!A$ of $L_T$,
$\Sat{}{\lforall[x]\lexists[y][\Atom{R}{x}{y}]\lif !A'}$ if, and only if,
$\Sat{}{!A}$. From these two things, it follows that if first-order
logic with identity ($\Lang{L_=}$) is decidable, then the halting problem
 is also solvable. We then prove that for any sentence $!A$ of $\Lang{L_=} , 
\Sat{}{(!E \land !D) \lif !A(=/!E)}$. Finally, we can then show that if firsr-order
logic without identity and terms is decidable, then the halting problem is 
solvable.

This proof proves our main claim, that first-order logic is not decidable. However,
we also get the corrolaries that both first-order logic without identity and terms,
as well as first-order logic without identity, are not decidable.


\end{explain}


\end{document}