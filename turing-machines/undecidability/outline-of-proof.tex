% Part: turing-machines
% Chapter: undecidability
% Section: outline-of-proof
\documentclass[../../include/open-logic-section]{subfiles}

\begin{document}

\olfileid{tms}{und}{out}
\olsection{Outline of the Proof}

\begin{explain}
We know that first-order logic is decidable iff there is an effective mechanical
method for determining whether or not a !!{sentence} is valid. We also know, from
the Chruch-Turing thesis, that every function which is computable is 
Turing-computable. We have proved that the function~$h(m,w)$, that halts with
an output $\TMstroke$ if the Turing-machine described by $m$ halts on input $w$ 
and outputs $\TMblank$ otherwise, is not Turing-computable. Via the Church-Turing
thesis,~$h$ is not computable. 

In order to show that first-order logic is undecidable, we prove the following.

\emph{Lemma:} Given input~$n$ and a Turing machine~$M$ we can effectively 
describe a set of !!{sentence}s $\Delta$ and a !!{sentence} H such that:

$\Delta \Entails H$ iff $M$ halts for input $n$.
\end{explain}

This gives us the theorem that if first-order logic is deciadble, then the halting problem
is solvable. However, we already know that the halting problem is not solvable. And
therefore, first-order logic is not decidable.

\end{document}