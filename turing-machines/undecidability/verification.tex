% Part: turing-machines
% Chapter: undecidability
% Section: verification

\documentclass[../../include/open-logic-section]{subfiles}

\begin{document}

\olfileid{tms}{und}{ver}
\olsection{Verifying the Representation}

\begin{explain}
In order to verify that our representation works, we first have to
make sure that if $M$ halts on input~$w$, then $T(m, w) \lif H(M, w)$
is valid.  We can do this simly by proving that $T(m, w)$ implies a
descriptin of the configuration of $M$ for each step of the execution
of $M$ on input $w$.  If $M$ halts on input $w$, then for some $n$,
$M$ will be in a halting configuration at step $n$ (and be scanning
square~$m$, for some $m$).  Hence, $T(M, w)$ implies $\Obj
Q_h(\overline m, \overline n)$.  
\end{explain}

\begin{defn}
Let $C(M, w, n)$ be the sentence
\[
\Obj Q_q(\overline m, \overline n) \land \Obj S_{\sigma_0}(\overline
0, \overline n) \land \dots \land \Obj S_{\sigma_k}(\overline k,
\overline n)
\]
where $q$ is the state of $M$ at time~$n$, $M$ is scanning sqare~$m$
at time~$n$, square $i$ contains symbol~$\sigma_i$ at time~$n$ for $0
\le i \le k$ and $k$ is the right-most non-blank square of the tape at
time~$m$.
\end{defn}

\begin{lem}
For each $n$, $T(M, w)$ implies $C(M, w, n)$.
\end{lem}

\begin{proof}
By induction on $n$.

If $n = 0$, then $C(M, w, n)$ is a conjunct of $T(M, w)$, so implied by it.

Suppose $n > 0$ and at time $n$, $M$ started on $w$ is in state~$q$
scanning square $m$, and the content of the tape is $\sigma_0$, \dots,
$\sigma_k$.

\dots to be completed

%Nicole find notes on this

\end{proof}

\begin{explain}
To complete the verification of our clam, we also have to establish
the reverse direction: if $T(M, w) \lif H(M, w)$ is valid, then $M$
does in fact halt when started on input~$m$.  
\end{explain}

\begin{lem}
If $T(M, w)$ entails $H(M, w)$, then $M$ halts on input~$w$.
\end{lem}

\begin{proof}
Consider the $\Lang L_M$-!!{structure} $\Struct M$ with domain $\Nat$
which inerprets $\Obj 0$ as $0$, $'$ as the successor function, and
$<$ as the lest-than relation, and the predicates $\Obj Q_q$ and
$\Obj S_\sigma$ as follows:
\begin{align*}
\Assign{\Obj Q_q}{M} & = \{(m, n) : \text{after $n$ steps, $M$ started
  on $w$ is in state $q$ scanning square~$m$}\} \\
\Assign{\Obj S_\sigma}{M} & = \{(m, n) : \text{after $n$ steps, $M$ started
  on $w$ has symbol~$\sigma$ on square~$m$}\} \\
\end{align*}
Clearly, $\Sat{M}{T(M, w)}$. If $T(M, w)$ implies $H(M, w)$, then
$\Sat{M}{H(M, w)}$, i.e.,
\[
\Sat{M}{\lexists[\Obj x]\lexists[\Obj y][\Obj{Q}_h(\Obj x, \Obj y)]}.
\]
As $\Domain{M} = \Nat$, there must be $m$, $n \in \Nat$ so that
$\Sat{M}{\Obj Q_h(\overline{n}, \overline{m})}$. By the definition of
$\Struct M$, this means that $M$ started on input~$w$ is in state~$h$
after $m$ steps, i.e., has halted.
\end{proof}

\end{document}
