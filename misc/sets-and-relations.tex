\documentclass[misc]{subfiles}

\begin{document}

\section{Sets and Relations}\label{sec:Sets-and-Relations}


\subsection{Sets}\label{ssec:Sets}

Sets and relations are the fundamental building blocks from which
we construct structures used to interpret logical languages. For instance,  a model~$\Mod M$
for a first-order language consists of a set~$D$ (the ``domain''
or ``universe of discourse'') and relations on~$D$ for each predicate
symbol in the language. For the semantics of modal logic, we will
make extensive use of sets and relations; in particular, certain specific
kinds of relations will play a prominent role. Hence, let's first
review what sets and relations are.

\begin{defn}
A set is a collection of objects, considered independently of the
way it is specified, of the order of its elements, or of their
multiplicity. The objects making up the set are called \emph{elements} of
the set. If $a$ is an element of a set $X$, we write $a \in X$ (otherwise,
$a \notin X$). The set which has no elements is called the \emph{empty set}
and denoted~$\emptyset$.
\end{defn}

\begin{ex}
Whenever you have a bunch of objects, you can collect them together
in a set. The set of Richard's siblings, for instance, is a set that
contains one person, and we could write it as $S=\{\textrm{Ruth}\}$.
In general, when we have some objects $a_{1}$, \dots{}, $a_{n}$,
then the set consisting of exactly those objects is written $\{a_{1},\dots,a_{n}\}$.
\end{ex}

Frequently we'll specify a set by some property that its elements
share---as we just did, for instance, by specifying $S$ as the set
of Richard's siblings. We'll use the following shorthand notation
for that: $\{x : \ldots x \ldots\}$ (pronounced ``the set of all
$x$ such that $\ldots x\ldots$''), where the $\ldots x\ldots$
is the property that $x$ has to have in order to be counted among
the elements of the set. In our example, we could have specified $S$
also as $S=\{x : x\text{ is a sibling of Richard}\}$. When we say
that sets are independent of the way they are specified, we mean
that the elements of a set are all that matters. For instance, it
so happens that $\{\text{Nicole},\text{Jacob}\}$, $\{x : x\text{ is a niece or nephew of Richard}\}$ and
$\{x :d x\text{ is a child of Ruth}\}$ are three ways of specifying
one and the same set. Saying that sets are considered independently
of the order of their elements and their multiplicity is a fancy way
of saying that $\{\text{Nicole}, \text{Jacob}\}$ and $\{\text{Jacob},\text{Nicole}\}$ are
two ways of specifying the same set; and that $\{\text{Nicole},\text{Jacob}\}$
and $\{\text{Jacob},\text{Nicole}, \text{Nicole}\}$ are two ways of
specifying the same set.

Mostly we'll be dealing with sets that have mathematical objects as
members. You will remember the various sets of numbers: $\mathbb{N}$
is the set of \emph{natural} numbers $0$, $1$, $2$, $3$, \dots{};
$\mathbb{Z}$ the set of \emph{integers} \dots{}, $-3$, $-2$,
$-1$, $0$, $1$, $2$, $3$, \dots{}; $\mathbb{Q}$ the set of
\emph{rationals} ($\mathbb{Q}=\{p/q : p,q\in\mathbb{Z}\}$); and
$\mathbb{R}$ the set of \emph{real} numbers. These are all \emph{infinite}
sets, that is, they each have infinitely many elements. As it turns
out, $\mathbb{N}$, $\mathbb{Z}$, $\mathbb{Q}$ have the same number
of elements, while $\mathbb{R}$ has a whole bunch more---$\mathbb{N}$,
$\mathbb{Z}$, $\mathbb{Q}$ are ``countably infinite'' whereas
$\mathbb{R}$ is ``uncountable''.

\begin{ex}\label{ex:strings}
Another interesting set is the set $A^{*}$ of
\emph{strings} over an alphabet $A$: any sequence of elements of
$A$ is a string over $A$. We include the \emph{empty string $\Lambda$}
among the strings over $A$, for every alphabet~$A$. For instance,
if $A = \{0,1\}$, then 
\[
\mathbb{B}=A^{*}=\{\Lambda,0,1,00,01,10,11,000,001,010,011,100,101,110,111,0000,\ldots\}.
\]
 If $x=x_{1}\ldots x_{n}\in A^{*}$ is a string consisting of $n$
``letters'' from $A$, then we say the \emph{length} of the string is~$n$
and write $\len(x)=n$.
\end{ex}

\begin{defn}
A set $X$ is a \emph{subset} of a set $Y$, $X \subseteq Y$, iff every
element of $X$ is also an element of $Y$. If $X\subseteq Y$ and
$X\neq Y$, we say that $X$ is a \emph{proper subset} of $Y$ and
write $X\subsetneq Y$. The set of all subsets of a set~$X$ is called
the \emph{power set} of~$X$, and denoted~$\wp(X)$.
\end{defn}

\begin{ex}
The empty set~$\emptyset$ is a subset of every set, and every set
is a subset of itself. If $X$ is a finite set with $n$ elements,
then there are $2^{n}$ subsets of~$X$. For instance, $X=\{1,2,3\}$has
$2^{3}=8$ subsets, namely:
\[
\wp(X)=\{\emptyset,\{1\},\{2\},\{3\},\{1,2\},\{1,3\},\{2,3\},\{1,2,3\}\}.
\]
Of our previous examples, we have $\mathbb{N}\subseteq\mathbb{Z}\subseteq\mathbb{Q}\subseteq\mathbb{R}$
(in fact, these are all proper subsets). If $A\subseteq B$, then
also $A^{*}\subseteq B^{*}$ and $\wp(A) \subseteq \wp(B)$. 
\end{ex}

\begin{defn}
If $X$and $Y$ are sets, then the \emph{intersection} $X\cap Y$
of $X$ and $Y$ is the set of all elements of $X$ which are also
elements of $Y$, i.e., $X\cap Y=\{x : x\in X\text{ and }x\in Y\}$.
If $X$ and $Y$ have no elements in common, then $X \cap Y = \emptyset$,and
we call $X$ and $Y$ \emph{disjoint.} The \emph{union}~$X \cup Y$
of $X$ and $Y$ is the set consisting of the elements of $X$ together
with all the elements of $Y$, i.e., $X \cup Y=\{x : x\in X\text{ or }x\in Y\}$.
The \emph{difference}~$X \setminus Y$ is the set of all elements
of $X$ which are not also elements of $Y$, i.e., $X \setminus Y = \{x : x\in X\text{ and }x\notin Y\}$.
\end{defn}

It will ofent be necessary to describe the unions or intersections not just of two, but of many sets.  Since sets are collections of arbitrary objects, we can form sets of sets. The power set of a set~$X$ is aset of sets, for instance, and its powerset is a set of sets of sets.

\begin{defn}
If $X$ is a set (of sets), then $\bigcap X = \{x : x \in Y \text{ for all } Y \in X\}$ and $\bigcup X = \{x : x \in Y \text{ for some } Y \in X\}$ are the intersection and union, respectively, of all sets in~$X$.
\end{defn}

If the elements of $X$ are indexed, e.g., $X = \{Y_0, Y_1, Y_2, \dots\}$, we will also write $\bigcap_{i=0}^\infty Y_i$ and $\bigcap_{i=0}^\infty Y_i$ for $\bigcap X$ and $\bigcup X$, respectively.  Often infinite sets are constructed as the unions or intersections of an infinite sequence of sets, which may be nested ($Y_0 \subseteq Y_1 \subseteq Y_2 \subseteq \dots$).  It will be important to remember in such cases that $x \in \bigcup_{i=0}^\infty$ iff $x \in Y_i$ for some~$i$.

\subsection{Relations}

Now that we've reviewed what sets are and seen some examples, you
will no doubt remember some interesting relations between objects
of some of these sets. For instance, numbers come with an \emph{order
relation}~$<$ and from the theory of whole numbers the relation
of \emph{divisibility without remainder} (usually written $n \mid m$)
may be familar. There is also the relation \emph{is identical with}
that every object bears to itself and to no other thing. But there
are many more interesting relations that we'll encounter, and even
more possible relations. Before we review them, we'll just point out
that we can look at relations as a special sort of set. For this,
we'll first need to define what a \emph{pair} is: if $a$ and $b$
are two objecta, we can combine them into the \emph{ordered pair}~$(a,b)$.
Note that for ordered pairs the order \emph{does} matter, e.g, $(a,b)\neq(b,a)$
(in contrast to unordered pairs, i.e., 2-element sets, where $\{a,b\}=\{b,a\}$).

\begin{defn}
If $X$ and $Y$ are sets, then the \emph{Cartesian product} $X \times Y$
of $X$ and $Y$ is the set of all pairs $(a,b)$ with $a\in X$ and
$b\in Y$. In particular, $X^{2}=X \times X$ is the set of all pairs
from~$X$.
\end{defn}

Now consider a relation on a set, e.g., the $<$-relation on the set
$\mathbb{N}$ of natural numbers, and consider the set of all pairs
of numbers $(n,m)$ where $n<m$, i.e.,
\[
R=\{(n,m)\mid n,m\in\mathbb{N}\text{ and }n<m\}.
\]
Then there is a close connection between the a number $n$ being less
than a number $m$ and the corresponding pair $(n,m)$ being a member
of $R$, namely, $n<m$ if and only if $(n,m) \in R$. In a sense we
can consider the set $R$ to \emph{be} the $<$-relation on the set
$\mathbb{N}$. In the same way we can construct a subset of $\mathbb{N}^{2}$ for
any relation between numbers. Conversely, given any set of pairs of
numbers $S\subseteq\mathbb{N}^{2}$, there is a corresponding relation
between numbers, namely, the relationship $n$ bears to $m$ if and
only if $(n,m) \in S$. This justifies the following definition:

\begin{defn}
A \emph{binary relation} on a set $X$ is a subset of $X^{2}$. If
$R\subseteq X^{2}$ is a binary relation on~$X$ and $x,y\in X$,
we write $Rxy$ (or $xRy$) for $(x,y)\in R$.
\end{defn}

\begin{ex}label{ex:relations}
The set $\mathbb{N}^{2}$of
pairs of natural numbers can be listed in a 2-dimensional matrix like
this:
\[
\begin{array}{ccccc}
\mathbf{(0,0)} & (0,1) & (0,2) & (0,3) & \ldots\\
(1,0) & \mathbf{(1,1)} & (1,2) & (1,3) & \ldots\\
(2,0) & (2,1) & \mathbf{(2,2)} & (2,3) & \ldots\\
(3,0) & (3,1) & (3,2) & \mathbf{(3,3)} & \ldots\\
\vdots & \vdots & \vdots & \vdots & \mathbf{\ddots}
\end{array}
\]
The subset consisting of the pairs lying on the diagonal, $I=\{(0,0),(1,1),(2,2),\ldots\}$,
is the \emph{identity relation on}~$\mathbb{N}$. (Since the identity
relation is popular, let's define $I_{X}=\{(x,x) : x\in X\}$ for any
set $X$.) The subset of all pairs lying above the diagonal, $L = \{(0,1),(0,2),\ldots,(1,2),(1,3),\ldots,(2,3),(2,4),\ldots\}$ is
the \emph{less than} relation, i.e., $Lnm$ iff $n<m$. The subset
of pairs below the diagonal, $G=\{(1,0),(2,0),(2,1),(3,0),(3,1),(3,2),\ldots\}$
is the \emph{greater than} relation, i.e., $Gnm$ iff $n>m$. The
union of $L$ with $I$, $K=L\cup I$, is the \emph{less than or equal
to} relation: $Kmn$ iff $n\le m$. Similarly, $H=G\cup I$ is the
\emph{greater than or equal to relation.} $L$, $G$, $K$, and $H$
are special kinds of relations called \emph{orders}. $L$ and $G$
have the property that no number bears $L$ or $G$ to itself (i.e.,
for all $n$, neither $Lnn$ nor $Gnn$). Relations with this property
are called \emph{antireflexive}, and, if they also happen to be orders,
they are called \emph{strict orders.}

Although orders and identity are important and natural relations,
it should be emphasized that according to our definition \emph{any}
subset of $X^{2}$ is a relation on~$X$, regardless of how unnatural
or contrived it seems. In particular, $\emptyset$ is a relation on
any set (the \emph{empty relation}, which no pair of elements bears),
and $X^{2}$ itself is a relation on $X$ as well (the universal relation, which every
pair bears). But also something like $E=\{(n,m) : d n>5\text{ or }m\times n\ge 34\}$
counts as a relation.
\end{ex}

\begin{defn}
If $R$ is a relation, then the \emph{domain}~$\dom{R}$ is the set of all elements related by~$R$, i.e., $\dom{R} = \{a : \text{for some }b, Rab\}$.
The \emph{range}~$\ran{R}$ is the set of all elements related to by~$R$, i.e., $\ran{R} = \{b : \text{for some }a, Rab\}$.
\end{defn}

\begin{defn}
A relation $R$ over a set $X$ is called \emph{reflexive} if, for
every $x\in X$, $Rxx$. A relation which has the property that whenever
$Rxy$ and $Ryz$, then also $Rxz$, is called \emph{transitive}.
$R$ is called \emph{anti-symmetric,} if, whenever both $Rxy$ and
$Ryx$, then $x=y$ (or, in other words: if $x\neq y$ then either
$\not Rxy$ or $\not Ryx$). Finally, $R$ is \emph{total} if for
all $x,y\in X$, either $Rxy$ or $Ryx$.

A relation which is both reflexive and transitive is called a \emph{preorder.}
A preorder which is also anti-symmetric is called a \emph{partial
order}. A partial order which is also total is called a \emph{total
order} or \emph{linear order.} (If we want to emphasize that the order
is reflexive, we add the adjective ``weak''---see below).
\end{defn}

\begin{ex}
Every linear order is also a partial order, and every partial order
is also a preorder, but the converses don't hold. For instance, the
identity relation and the universal relation on~$X$ are preorders, but
they are not partial orders, because they are not anti-symmetric (if
$X$ has more than one element). For a somewhat less silly example,
consider the \emph{no longer than} relation $\preccurlyeq$on~$\mathbb{B}$:
$x\preccurlyeq y$ iff $\len(x)\le\len(y)$. This is a preorder, even
a total preorder, but not a partial order. The relation of \emph{divisibility
without remainder} gives us an example of a partial order which isn't
a total order: for integers $n$, $m$, we say $n$ (evenly) divides
$m$, in symbols: $n\mid m$, if there is some $k$ so that $m=kn$.
On $\mathbb{N}$, this is a partial order, but not a linear order:
for instance, $2\nmid3$ and also $3\nmid2$. Considered as a relation
on $\mathbb{Z}$, divisibility is only a preorder since anti-symmetry
fails: $1\mid-1$ and $-1\mid1$ but $1\neq-1$. Another important
partial order is the relation $\subseteq$ on a set of sets.

Notice that the examples $L$ and $G$ from \ref{ex:relations},
although we said there that they were called ``strict orders'' are
not total orders even though they are total. But there is a close
connection, as we will see momentarily.
\end{ex}

\begin{defn}
A relation $R$ on $X$is called \emph{irreflexive} if, for all $x\in X$,
$x\not Rx$. $R$ is called \emph{asymmetric} if for no pair $x,y\in X$
we have $xRy$ and $yRx$. A \emph{strict partial order} is a relation
which is irreflexive, asymmetric, and transitive. A strict partial
order which is also linear is called a \emph{strict linear order.}
\end{defn}

A strict partial order $R$ on $X$ can be turned into a weak partial
order $R'$ by adding the identity relation on $X$.
Conversely, starting from a weak partial order, one can get a strict
partial order by removing~$I_{X}$

\begin{prop}
Let $R$ be a relation on $X$ and $R'= R \cup I{}_{X}$. Then
$R$ is a strict partial (linear) order on $X$ iff $R'$ is a weak
partial (linear) order. Moreover, $xRy$ iff $xR'y$ for all $x\neq y$.\end{prop}

\begin{ex}
$\le$ is the weak linear order corresponding to the strict linear
order $<$. $\subseteq$ is the weak partial order corresponding to
the strict partial order $\subsetneq$.
\end{ex}

\begin{prob}
Show that if $R$ is a weak partial order on $X$, then $R^{-} = R\setminus I_{X}$
is a strict partial order and $xRy$ iff $xR^{-}y$ for all $x\neq y$.\end{prob}

\subsection{Operations on Relations}

It is often useful to modify or combine relations. We've already used the union of relations above (which is just the union of two relations considered as sets of pairs). Here are some other ways:

\begin{defn} Let $R$, $S$ be relations and $X$ a set.
\begin{enumerate}
\item The \emph{inverse}~$R^{-1}$ of $R$ is $R^-1 = \{ (b, a) : (a, b) \in R\}$.
\item The \emph{relative product}~$R \mid S$ of $R$ and $S$ is 
\[
(R \mid S) = \{(a, c) : \text{for some }b, Rab \text{ and } Rbc\}
\]
\item The \emph{restriction}~$R \restrict X$ of $R$ to $X$
\item The \emph{application}~$R[X]$ of $R$ to $X$ is
\[
R[X] = \{b : \text{for some }a, Rab\}
\]
\end{enumerate}
\end{defn}

\begin{defn}
The \emph{transitive closure}~$R^+$ of a relation $R$ is $R^+ = \bigcup_{i=1}^\infty R^i$ where $R^1 = R$ and $R^{i+1} = R^i \mid R$.

The reflexive transitive closure~$R^*$ of $R$ is $R* = R^+ \cup I_{\dom{R}}$.
\end{defn}

\begin{prob}
Show that the transitive closure of $R$ is in fact transitive.
\end{prob}

\end{document}
