% Part: first-order-logic 
% Chapter: axiomatic-proofs 
% Section: soundness-completeness-prop

\documentclass[../../include/open-logic-section]{subfiles}

\begin{document}

\olfileid{fol}{axp}{psc}

\olsection{Soundness and Completeness of propositional logic}

\begin{thm}[Soundness] \ollabel{thm:soundness} If $\Gamma\Proves!A$ then
$\Gamma \Entails !A$. \end{thm}

\begin{proof} By induction on theorems. If $!A$ is an axiom then $\Entails
!A$ and hence a fortiori $\Gamma \Entails!A$. Similarly if $!A \in \Gamma$,
$\Gamma \Entails!A$. For the inductive step, suppose $!A$ is obtained by
\emph{modus ponens} from $!C$ and $!C\lif!A$; also assume as inductive
hypothesis that the theorem holds for $!C\lif !A$ and $!C$. The third item
in Exercise \ref{ex:semanticalfacts} gives the desired result. \end{proof}

\begin{cor} If $\Gamma$ is satisfiable, then $\Gamma$ is consistent. Hence,
propositional logic is consistent. \end{cor}

\begin{defn} \ollabel{def:MaxCon} A set $\Gamma$ of !!{formula}s is
\emph{maximally consistent} if it is consistent and if $\Delta$ is a
consistent set such that $\Gamma \subseteq \Delta$ then $\Gamma = \Delta$.
\end{defn}

\begin{prop} (\emph{Truth Lemma}) let $\Gamma$ be maximally consistent;
then: \begin{enumerate} \item $\Gamma\Proves!A$ if and only if
$!A\in\Gamma$; \item $!A\in\Gamma$ if and only if $\lnot!A\notin\Gamma$;
\item $!A\lif!B \in\Gamma$ if and only if either $!A\notin\Gamma$ or
$!B\in\Gamma$. \end{enumerate} \end{prop} \begin{proof} By the way of
example, let $!A\in\Gamma$; if also $\lnot!A\in\Gamma$, then $\Gamma$ is
inconsistent; and if neither $!A$ nor $\lnot!A$ is in $\Gamma$ then by
Proposition \ref{prop:phi} one of $\Gamma\cup\{!A\}$ and
$\Gamma\cup\{\lnot!A\}$ is consistent, which means that $\Gamma$ is not
maximal. \end{proof}

\noindent The left-to-right of item $(a)$ of the Truth Lemma is the
\emph{deductive closure} of $\Gamma$, i.e., if $\Gamma \Proves !A$ then $!A
\in \Gamma$.

\begin{thm} [Completeness] \ollabel{thm:completeness} if $\Gamma$ is
consistent then $\Gamma$ is satisfiable. \end{thm}

\begin{proof} Let $!A_0,!A_1,\ldots$ be an exhaustive listing of all the
!!{formula}s of the language. Recursively define an increasing sequence of
sets of !!{formula}s $\Gamma_0,\Gamma_1,\ldots$, by putting: \begin{align*}
\Gamma_0 & = \Gamma\\ \Gamma_{n+1} & = \begin{cases} \Gamma_n\cup\{!A_n\} &
\text{if $ \Gamma_n\cup\{!A_n\}$ is consistent;}\\
\Gamma_n\cup\{\lnot!A_n\} & \text{otherwise}. \end{cases} \end{align*} Then
define: \[ \Gamma^* = \bigcup_{0\le n}\Gamma_n. \] The proof now proceeds
by establishing, in turn, the following facts: \begin{enumerate} \item For
each $n$, the set $\Gamma_n$ is consistent (by induction on $n$, using
Proposition \ref{prop:phi}); \item $\Gamma^*$ is consistent; \item
$\Gamma^*$ is maximal. \end{enumerate} Then define a valuation $v$ by
putting $v(p_i) = \mathsf{t}$ if and only if $p_i \in\Gamma^*$. By
induction on $!A$ it is then shown that membership in $\Gamma^*$ coincides
with truth according to $v$ also for more complex !!{sentence}s:
$\overline{v}(!A) = \mathsf{t}$ if and only if $!A\in \Gamma^*$. In
particular, $v$ satisfies $\Gamma^*$, and since $\Gamma\subseteq \Gamma^*$,
also $\Gamma$ is satisfiable, as desired. \end{proof}

\begin{cor} If $\Gamma \Entails !A$ then $\Gamma \Proves !A$. \end{cor}
\begin{proof} If $\Gamma\Proves/ !A$ then $\Gamma\cup\{\lnot!A\}$ is
consistent, by Proposition \ref{prop:proves}. So by the theorem
$\Gamma\cup\{\lnot !A\}$ is satisfiable, so $\Gamma\Entails/ !A$.
\end{proof}

\begin{prop} [Compactness Theorem] \ollabel{prop:Compactness} $\Gamma$ is
satisfiable if and only if every \emph{finite} subset $\Gamma_0$ of
$\Gamma$ is satisfiable. \end{prop}

\begin{proof} $\Gamma$ is unsatisfiable if and only if it is inconsistent,
if and only if some finite subset $\Gamma_0$ of $\Gamma$ is inconsistent,
if and only if some finite subset $\Gamma_0$ of $\Gamma$ is unstasfiable.
\end{proof}

\begin{cor} $\Gamma \Entails !A$ if and only if for some finite subset
$\Gamma_0$ of $\Gamma$, $\Gamma_0\Entails!A$. \end{cor} \end{document}
