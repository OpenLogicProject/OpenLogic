% Part: sets-functions-relations
% Chapter: relations
% Section: operations

\documentclass[open-logic-section]{subfiles}

\begin{document}

\olfileid{sfr}{rel}{ops}
\olsection{Operations on Relations}

It is often useful to modify or combine relations. We've already used
the union of relations above (which is just the union of two relations
considered as sets of pairs). Here are some other ways:

\begin{defn} Let $R$, $S$ be relations and $X$ a set.
\begin{enumerate}
\item The \emph{inverse}~$R^{-1}$ of $R$ is $R^-1 = \{ (b, a) : (a, b)
  \in R\}$.
\item The \emph{relative product}~$R \mid S$ of $R$ and $S$ is 
\[
(R \mid S) = \{(a, c) : \text{for some }b, Rab \text{ and } Rbc\}
\]
\item The \emph{restriction}~$R \restrict X$ of $R$ to $X$
\item The \emph{application}~$R[X]$ of $R$ to $X$ is
\[
R[X] = \{b : \text{for some }a, Rab\}
\]
\end{enumerate}
\end{defn}

\begin{defn}
The \emph{transitive closure}~$R^+$ of a relation $R$ is $R^+ =
\bigcup_{i=1}^\infty R^i$ where $R^1 = R$ and $R^{i+1} = R^i \mid R$.

The reflexive transitive closure~$R^*$ of $R$ is $R* = R^+ \cup
I_{\dom{R}}$.
\end{defn}

\begin{prob}
Show that the transitive closure of $R$ is in fact transitive.
\end{prob}

\end{document}

