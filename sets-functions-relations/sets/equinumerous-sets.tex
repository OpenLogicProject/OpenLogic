% Part:sets-functions-relations
% Chapter: Sets
%Section: Equinumerous sets

\documentclass[../../include/open-logic-section]{subfiles}

\begin{document}

\olfileid{sfr}{set}{equ}
\olsection{Equinumerous Sets}

\begin{intro}
We have an intuitive notion of ``same-size''. It is easy to compare the size of finite sets, but what about infinite sets? We need a formal way of comparing the sizes of two sets of \emph{any} size. Moreover, we want the formal notion to correspond with our intuitive notion of ``same-size'', hence the formal notion ought to satisfy the following properties:

\begin{description}
\item[Reflexivity:] Every set is equinumerous with itself.
\item[Symmetry:] For any two sets $\Gamma$ and $\Delta$, if $\Gamma$ is equinumerous with $\Delta$, then $\Delta$ is equinumerous with $\Gamma$.
\item[Transitivity:] For any three sets $\Gamma, \Delta$, and $\Lambda$, if $\Gamma$ is equinumerous with $\Delta$ and $\Delta$ is equinumerous with $\Lambda$, then $\Gamma$ is equinumerous with $\Lambda$.
\end{description}

In other words, we want our formal notion of equinumerosity to be an \emph{equivalence relation}. 

\end{intro}

\begin{defn}
A set $\Gamma$ is \emph{equinumerous} with a set $\Delta$ if and only if there is a total bijection $f$  from $\Gamma$ to $\Delta$ (that is, $f: \Gamma \rightarrow \Delta$).
\end{defn}

\begin{explain} Claim: the above formal definition of equinumerosity defines an equivalence relation.

\begin{proof} Let $\Gamma, \Delta$, and $\Lambda$ be sets.

\begin{description}

\item[Reflexivity:] Using identity map $1_\Gamma : \Gamma \rightarrow \Gamma$, where $1_\Gamma (\gamma) = \gamma$ for all $\gamma \in \Gamma$, we see that $\Gamma$ is equinumerous with itself (clearly, $1_\Gamma$ is bijective).

\item[Symmetry:] Suppose that $\Gamma$ is equinumerous with $\Delta$ via the total bijection $f$; ie. $f: \Gamma \rightarrow \Delta$ is a total bijection. Then since $f$ is bijective, its inverse $f^{-1}$ is also a bijection. Since $f$ is onto, $f^{-1}$ is total. Hence, $f^{-1}:\Delta \rightarrow \Gamma$ is a total bijection, so $\Delta$ is also equinumerous with $\Gamma$. 

\item[Transitivity:] Suppose that $\Gamma$ is equinumerous with $\Delta$ via the total bijection $f$ and that $\Delta$ is equinumerous with $\Lambda$ via the total bijection $g$. Then the composition of $g \circ f: \Gamma \rightarrow \Lambda$ is a total bijection, and $\Gamma$ is thus equinumerous with $\Lambda$.
\end{description}
Therefore, equinumerosity is an equivalence relation by the given definition.
\end{proof}
\end{explain}

Since equinumerosity is an equivalence relation, when two sets $\Gamma$ and $\Delta$ are equinumerous, we write $|\Gamma| = |\Delta|$.

\begin{defn}
$|\Gamma|$ is the \emph{cardinality} (or \emph{size}) of a set $\Gamma$.
\end{defn}

\begin{thm}
Suppose $|\Gamma| = |\Delta|$. Then $\Gamma$ is countable if and only if $\Delta$ is countable.
\end{thm}

\begin{proof}
Let $|\Gamma|=|\Delta|$ and, for the forward direction of the proof, suppose that $|\Gamma|$ is countable. Then there is an onto function $f: \Nat \rightarrow \Gamma$. Since $|\Gamma| = |\Delta|$, there is a total bijection $g: \Gamma \rightarrow \Delta$. Claim: $g \circ f: \Nat \rightarrow \Delta$ is an onto function. Clearly, $g \circ f$ is a function (since functions are closed under composition). For ontoness, let $\delta \in \Delta$. Since $g$ is onto $\Delta$, there is a $\gamma \in \Gamma$ such that $g(\gamma) = \delta$. Since $f$ is onto $\Gamma$, there is an $n\in \Nat$ such that $f(n) = \gamma$. Hence,
\[ (g \circ f)(n) = g(f(n)) = g(\gamma) = \delta \]
and thus $g \circ f$ is onto $\Delta$. Since $g \circ f: \Nat \rightarrow \Delta$ is an onto function, $\Delta$ is be countable (backwards direction is symmetric).
\end{proof}

\begin{explain}
With real numbers, two real numbers can be equal to each other, or one can be lesser than the other; there is also a way of comparing sets of different cardinalities. We can formalize this size comparison in a similar way to the way we formalized ``same-size'', but this time, instead of requiring a bijection between two sets, all we need is a total one-to-one function from the first set to the second in order to establish that the second is larger than the first. This should make sense because, intuitively, a one-to-one function from one set to another means that the range of the function has at least as many elements as the domain, since no two domain elements map to the same range element.
\end{explain}

\begin{defn}
Let $\Gamma$ and $\Delta$ be sets. Then $|\Gamma| \leq |\Delta|$ if and only if there is a total one-to-one function $f$ from $\Gamma$ to $\Delta$ ($f: \Gamma\rightarrow \Delta$).
\end{defn}

\begin{thm}[Schr\"oder-Bernstein]
Let $\Gamma$ and $\Delta$ be sets. If $|\Gamma| \leq |\Delta|$ and $|\Delta| \leq |\Gamma|$, then $|\Gamma| = |\Delta|$. 
\end{thm}

\begin{explain}In other words, if there is a total one-to-one function from $\Gamma$ to $\Delta$, and if there is a total one-to-one function from $\Delta$ back to $\Gamma$, then there is a total bijection from $\Gamma$ to $\Delta$. Sometimes, it can be difficult to think of a bijection between two equinumerous sets, so the Schr\"oder-Bernstein theorem allows us to break the comparison down into cases so we only have to think of an injection from the first to the second, and vice-versa. The Schr\"oder-Bernstein theorem, apart from being convenient, justifies the act of discussing the ``sizes'' of sets, for it tells us that set cardinalities have the familiar anti-symmetric property that numbers have.
\end{explain}

\end{document}
