% Part:sets-functions-relations
% Chapter: Sets
%Section: Countability

\documentclass[../../include/open-logic-section]{subfiles}

\begin{document}

\olfileid{sfr}{set}{ctb}
\olsection{Countability}

\begin{intro}
When Georg Cantor developed set theory in the 1870's, his 
interest was in part to make palatable the idea of an infinite 
collection --- an actual infinity, as the medievals would say. 
Key to this rehabilitation of the notion of the infinite was the 
notion of countability. 
\end{intro}

\begin{defn}
Informally, an \emph{enumeration} of a set $\Gamma$ is a list 
such that every member of $\Gamma$ appears some finite 
number of places into the list. If $\Gamma$ has an enumeration, 
then $\Gamma$ is said to be \emph{countable}.
\end{defn}

\begin{explain}
A couple of points about enumerations:

\begin{itemize}
\item Order does not matter, as long as every member appears: 
$4, 1, 25, 16,9$ enumerates the first five square numbers just 
as well as $1, 4, 9, 16, 25$ does.
\item Redundant lists are stil lists: $1, 1, 2, 2, 3, 3, \dots$ 
enumerates the same set as $1, 2, 3, \dots$ does.
\item When order and redundancy do matter is in our 
understanding of an enumeration: the list $1, 2, 3, 1, \dots$ may 
enumerate the natural numbers, but the pattern is easier to see 
when listed as $1, 2, 3, 4, \dots$
\item Enumerations must have a beginning: $\dots, 3, 2, 1$ is 
not an enumeration of the natural numbers because it has no 
first member. To see how this follows from the informal definition, 
ask yourself, ``at what place in the list does the number 76 
appear?''
\item The following is not an enumeration of the natural numbers: 
$1, 3, 5, \dots, 2, 4, 6, \dots$ The problem is that the even 
numbers occur at places $\infty + 1, \infty +2, \infty+3$, rather 
than at finite positions.
\item Lists may be gappy: $2, -, 4, -, 6, -, \dots$ enumerates the 
even natural numbers.
\item The empty set is enumerable: it is enumerated by the 
empty list!
\end{itemize}

The following provides a more formal definition of an 
enumeration:
\end{explain}

\begin{defn}
An enumeration of a set $\Gamma$ is any onto function $f: 
\Nat \rightarrow \Gamma$.
\end{defn}

\begin{explain}
As we said above, a function (partial or total) from $\Nat$ 
onto a set $\Gamma$ enumerates $\Gamma$. Why? Because 
such a function determines an enumeration as defined informally 
above. Let $f$ be a function from $\Nat$ onto a set 
$\Gamma$. Then an enumeration for $\Gamma$ is the list $f(1), 
f(2), f(3), \dots$. Since $f$ is onto, every member of $\Gamma$ is 
guaranteed to be the value of $f(n)$ for some $n \in \Nat$. 
Hence, every member of $\Gamma$ appears at some finite place 
in the list. Since the function may be partial or one-to-one, the list 
may be gappy or redundant, but that is acceptable (as noted above).
\end{explain}

\begin{ex}
A function enumerating the natural numbers ($\Nat$) is 
simply the identity function: $\lforall[n] \in \Nat, f(n) = n$.
\end{ex}

\begin{ex}
The functions $f(n) = 2n$ and $f(n) = 2n+1$ (for $n\in\Nat$) 
enumerate the even natural numbers and the odd natural numbers, 
respectively. However, neither function is an enumeration for all of 
$\Nat$, since neither is onto $\Nat$.
\end{ex}

\begin{ex}
The function $f(n) = \lceil \frac{(-1)^n n}{2}\rceil$ (where $\lceil x 
\rceil$ denotes the \emph{ceiling} function, which rounds $x$ up 
to the nearest integer) enumerates the set of  integers, 
$\Int$. Notice how $f$ generates the values of $\Int$ 
by ``hopping'' back and forth across 0:
\center
\begin{tabular}{c c c c c c c}
$f(1)$ & $f(2)$ & $f(3)$ & $f(4)$ & $f(5)$ & $f(6)$ & \dots \\ \\
$\lceil - \tfrac{1}{2}\rceil$ & $\lceil \tfrac{2}{2} \rceil$ & $\lceil - 
\tfrac{3}{2} \rceil$ & $\lceil \tfrac{4}{2} \rceil $ & $\lceil -\tfrac{5}{2} 
\rceil$ & $\lceil \tfrac{6}{2} \rceil$ & \dots \\ \\
0 & 1 & -1 & 2 & -2 & 3 & \dots
\end{tabular}
\flushleft
\end{ex}

\begin{explain}
That is fine for ``easy'' sets. What about the set of, say, pairs of 
natural numbers?
\[ \Nat^2 = \Nat \times \Nat = \Setabs{(n,m)}{n,m \in \Nat}\]
Another method we can use to enumerate sets is to organize them 
in an \emph{array}. Observe the following array:

\center
\begin{tabular}{| c | c | c | c | c | c}
\hline
& \textbf 1 & \textbf 2 & \textbf 3 & \textbf 4 & \dots \\
\hline
\textbf 1 & (1,1) & (1,2) & (1,3) & (1,4) & \dots \\
\hline
\textbf 2 & (2,1) & (2,2) & (2,3) & (2,4) & \dots \\
\hline
\textbf 3 & (3,1) & (3,2) & (3,3) & (3,4) & \dots \\
\hline
\textbf 4 & (4,1) & (4,2) & (4,3) & (4,4) & \dots \\
\hline
\vdots & \vdots & \vdots & \vdots & \vdots & $\ddots$\\
\end{tabular}
\flushleft

Clearly, every ordered pair in $\Nat^2$ will appear at least 
once in the array. In particular, $(n,m)$ will appear in the $n$th 
column and $m$th row. But how do we organize the elements of an 
array into a list? The pattern in the array below demonstrates one 
way to do this:

\center
\begin{tabular}{| c | c | c | c | c | c }
\hline
& & & & & \\
\hline
& 1 & 2 & 4 & 7 & \dots \\
\hline
& 3 & 5 & 8 & \dots & \dots \\
\hline
& 6 & 9 & \dots & \dots & \dots \\
\hline
& 10 & \dots & \dots & \dots & \dots \\
\hline
& \vdots & \vdots & \vdots & \vdots & $\ddots$\\ 
\end{tabular}

\flushleft
This pattern is called \emph{Cantor's zig-zag method}. Other 
patterns are perfectly permissible, as long as they ``zig-zag'' 
through every cell of the array. By Cantor's zig-zag method, the 
enumeration for $\Nat^2$ according to this scheme would be:
\[(1,1), (1,2), (2,1), (1,3), (2,2), (3,1), (1,4), (2,3), (3,2), (4,1), \dots\]

What ought we do about enumerating, say, the set of ordered triples 
of natural numbers?
\[ \Nat^3 = \Nat \times \Nat \times \Nat = 
\{ (n,m,k) : n,m,k \in \Nat \} \]
We can think of $\Nat^3$ as the Cartesian product between 
$\Nat$ and $\Int^2$, that is, 

\[ \Nat^3 = \Nat^2 \times \Nat = \Setabs{(\vec a, 
k)}{\vec a \in \Nat^2, k \in \Nat } \]

and thus we can enumerate $\Nat^3$ with an array by 
labelling one axis with the enumeration of $\Nat$, and the 
other axis with the enumeration of $\Nat^2$:

\center
\begin{tabular}{| c | c | c | c | c | c}
\hline
& \textbf 1 & \textbf 2 & \textbf 3 & \textbf 4 & \dots \\
\hline
$\mathbf{(1,1)}$ & (1,1,1) & (1,1,2) & (1,1,3) & (1,1,4) & \dots \\
\hline
$\mathbf{(1,2)}$ & (1,2,1) & (1,2,2) & (1,2,3) & (1,2,4) & \dots \\
\hline
$\mathbf{(2,1)}$ & (2,1,1) & (2,1,2) & (2,1,3) & (2,1,4) & \dots \\
\hline
$\mathbf{(1,3)}$ & (1,3,1) & (1,3,2) & (1,3,3) & (1,3,4) & \dots\\
\hline
\vdots & \vdots & \vdots & \vdots & \vdots & $\ddots$ \\
\end{tabular}

\flushleft
Thus, by using a method like Cantor's zig-zag method, we may 
similarly obtain an enumeration of $\Nat^3$. 

\end{explain}

\end{document}