% Part: sets-functions-relations
% Chapter: sets
% Section: basics

\documentclass[../../include/open-logic-section]{subfiles}


\begin{document}

\olfileid{sfr}{set}{bas}
\olsection{Basics}

\begin{explain}
Sets are the most fundamental building blocks of mathematical
objects. In fact, almost every mathematical object can be seen as a
set of some kind. In logic, as in other parts of mathematics, sets
and set theoretical talk is ubiquitous. So it will be important to
discuss what sets are, and introduce the notations necessary to talk
about sets and operations on sets in a standard way.
\end{explain}

\begin{defn}
A \emph{set} is a collection of objects, considered independently of
the way it is specified, of the order of its elements, or of their
multiplicity. The objects making up the set are called \emph{elements}
or \emph{members} of the set. If $a$ is an element of a set $X$, we
write $a \in X$ (otherwise, $a \notin X$). The set which has no elements
is called the empty set and denoted $\emptyset$.
\end{defn}

\begin{ex}
Whenever you have a bunch of objects, you can collect them together in
a set. The set of Richard's siblings, for instance, is a set that
contains one person, and we could write it as $S=\{\textrm{Ruth}\}$.
In general, when we have some objects $a_{1}$, \dots, $a_{n}$, then
the set consisting of exactly those objects is written $\{
a_{1}, \dots, a_{n}\}$. Frequently we'll specify a set by some
property that its elements share---as we just did, for instance, by
specifying $S$ as the set of Richard's siblings. We'll use the
following shorthand notation for that: $\Setabs{x}{\ldots x\ldots}$,
where the $\ldots x\ldots$ stands for the property that $x$ has to
have in order to be counted among the elements of the set. In our
example, we could have specified $S$ also as $S = \Setabs{x}{x \text{ is
a sibling of Richard}}$.
\end{ex}

\begin{explain}
When we say that sets are independent of the way they are specified,
we mean that the elements of a set are all that matters. For instance,
it so happens that $\{\text{Nicole}, \text{Jacob}\}$,
$\Setabs{x}{\text{is a niece or nephew of Richard}}$ and
$\Setabs{x}{\text{is a child of Ruth}}$ are three ways of specifying
one and the same set.

Saying that sets are considered independently of the order of their
elements and their multiplicity is a fancy way of saying that
$\{\text{Nicole}, \text{Jacob}\}$ and $\{\text{Jacob},
\text{Nicole}\}$ are two ways of specifying the same set; and that
$\{\text{Nicole}, \text{Jacob}\}$ and $\{\text{Jacob}, \text{Nicole},
\text{Nicole}\}$ are two ways of specifying the same set.
\end{explain}

\end{document}
