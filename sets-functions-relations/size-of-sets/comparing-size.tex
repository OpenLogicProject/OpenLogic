% Part:sets-functions-relations
% Chapter: sets
% Section: comparing-sizes

\documentclass[../../include/open-logic-section]{subfiles}

\begin{document}

\olfileid{sfr}{set}{car}

\olsection{Comparing Sizes of Sets}

\begin{explain}
Just like we were able to make precise when two sets have the same
size in a way that also accounts for the size of infinite sets, we can
also compare the sizes of sets in a precise way. Our definition of
``is smaller than (or equinumerous)'' will require, instead of a
!!{bijection} between the sets, a total !!{injective} function from the first
set to the second. If such a function exists, the size of the first
set is less than or equal to the size of the second.  Intuitively, 
!!a{injective} function from one set to another guarantees that the range of
the function has at least as many elements as the domain, since no two
!!{element}s of the domain map to the same !!{element} of the range.
\end{explain}

\begin{defn}
$\card{X} \leq \card{Y}$ if and only if there is an !!{injective}
  function~$f \colon X \to Y$.
\end{defn}

\begin{thm}[Schr\"oder-Bernstein]
Let $X$ and $Y$ be sets. If $\card{X} \leq \card{Y}$ and $\card{Y}
\leq \card{X}$, then $\card{X} = \card{Y}$.
\end{thm}

\begin{explain}
In other words, if there is a total !!{injective} function from $X$ to
$Y$, and if there is a total !!{injective} function from $Y$ back to $X$,
then there is a total !!{bijection} from $X$ to $Y$. Sometimes, it can be
difficult to think of a !!{bijection} between two equinumerous sets, so
the Schr\"oder-Bernstein theorem allows us to break the comparison
down into cases so we only have to think of an !!{injection} from the
first to the second, and vice-versa. The Schr\"oder-Bernstein theorem,
apart from being convenient, justifies the act of discussing the
``sizes'' of sets, for it tells us that set cardinalities have the
familiar anti-symmetric property that numbers have.
\end{explain}

\end{document}
