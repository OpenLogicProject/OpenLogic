% Part:sets-functions-relations
% Chapter: size-of-sets
% Section: introduction

\documentclass[../../include/open-logic-section]{subfiles}

\begin{document}

\olfileid{sfr}{siz}{int}

\olsection{Introduction}

When Georg Cantor developed set theory in the 1870s, his interest was
in part to make palatable the idea of an infinite collection---an
actual infinity, as the medievals would say.  Key to this
rehabilitation of the notion of the infinite was a way to assign
sizes---``cardinalities''---to sets.  The cardinality of a finite set
is just a natural number, e.g., $\emptyset$ has cardinality~0, and a
set containing five things has cardinality~$5$.  But what about
infinite sets?  Do they all have the same cardinality, $\infty$?  It
turns out, they do not.

The first important idea here is that of an enumeration.  We can list
every finite set by listing all its elements.  For some infinite sets,
we can also list all their elements if we allow the list itself to be
infinite.  Such sets are called !!{enumerable}.  Cantor's surprising
result was that some infinite sets are not !!{enumerable}.

\end{document}
