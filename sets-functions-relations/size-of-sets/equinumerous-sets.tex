% Part:sets-functions-relations
% Chapter: sets
% Section: equinumerous-sets

\documentclass[../../include/open-logic-section]{subfiles}

\begin{document}

\olfileid{sfr}{set}{equ}

\olsection{Equinumerous Sets}

\begin{intro}
We have an intuitive notion of ``size'' of sets, which works fine for
finite sets. But what about infinite sets? If we want to come up with
a formal way of comparing the sizes of two sets of \emph{any} size, it
is a good idea to start with defining when sets are the same size.
Let's say sets of the same size are \emph{equinumerous}. We want the
formal notion of equinumerosity to correspond with our intuitive
notion of ``same size,'' hence the formal notion ought to satisfy the
following properties:

\begin{description}
\item[Reflexivity:] Every set is equinumerous with itself.
\item[Symmetry:] For any sets $X$ and $Y$, if $X$ is equinumerous with
  $Y$, then $Y$ is equinumerous with $X$.
\item[Transitivity:] For any sets $X, Y$, and $Z$, if $X$ is
  equinumerous with $Y$ and $Y$ is equinumerous with $Z$, then $X$ is
  equinumerous with $Z$.
\end{description}

In other words, we want equinumerosity to be an
\emph{equivalence relation}.
\end{intro}

\begin{defn}
A set $X$ is \emph{equinumerous} with a set $Y$ if and only if there
is a total bijection $f$ from $X$ to $Y$ (that is, $f \colon X
\to Y$).
\end{defn}

\begin{prop} 
Equinumerosity defines an equivalence relation.
\end{prop}

\begin{proof} Let $X, Y$, and $Z$ be sets.

\begin{description}

\item[Reflexivity:] Using the identity map $1_X \colon X \rightarrow
  X$, where $1_X (x) = x$ for all $x \in X$, we see
  that $X$ is equinumerous with itself (clearly, $1_X$ is bijective).

\item[Symmetry:] Suppose that $X$ is equinumerous with $Y$. Then there
  is a bijection $f\colon X \rightarrow Y$. Since $f$ is bijective,
  its inverse $f^{-1}$ is also a bijection. Since $f$ is !!{surjective},
  $f^{-1}$ is total. Hence, $f^{-1}\colon Y \rightarrow X$ is a total
  !!{bijection} from $Y$ to~$X$, so $Y$ is also equinumerous with $X$.

\item[Transitivity:] Suppose that $X$ is equinumerous with $Y$ via the
  total bijection~$f$ and that $Y$ is equinumerous with $Z$ via the
  total bijection~$g$. Then the composition of $g \circ f \colon X
  \rightarrow Z$ is a total bijection, and $X$ is thus equinumerous
  with $Z$.
\end{description}
Therefore, equinumerosity is an equivalence relation by the given definition.
\end{proof}

\begin{thm}
Suppose $X$ and $Y$ are equinumerous. Then $X$ is !!{enumerable} if
and only if $Y$ is.
\end{thm}

\begin{proof}
Let $X$ and $Y$ be equinumerous.  Suppose that $X$ is
!!{enumerable}. Then there is a possibly partial, !!{surjective}
function $f\colon \Nat \to X$. Since $X$ and $Y$ are equinumerous,
there is a total bijection $g\colon X \to Y$. Claim: $g \circ f \colon
\Nat \rightarrow Y$ is !!{surjective}. Clearly, $g \circ f$ is a
function (since functions are closed under composition). To see $g
\circ f$ is !!{surjective}, let $y \in Y$. Since $g$ is
!!{surjective}, there is an $x \in X$ such that $g(x) = y$. Since $f$
is !!{surjective}, there is an $n \in \Nat$ such that $f(n) =
x$. Hence,
\[ 
(g \circ f)(n) = g(f(n)) = g(x) = y 
\]
and thus $g \circ f$ is !!{surjective}. Since $g \circ f\colon \Nat
\to Y$ is !!{surjective}, $Y$ it is an enumeration of~$Y$, and so $Y$ is
!!{enumerable}.
\end{proof}

\begin{prob}
Show that if $X$ is equinumerous with $U$ and and $Y$ is equinumerous
with $V$, and the intersections $X \cap Y$ and $U \cap V$ are empty,
then the unions $X \cup Y$ and $U \cup V$ are equinumerous.
\end{prob}

\begin{prob}
Given an enumeration of a set $X$, show that if $X$ is not finite then
it is equinumerous with the positive integers~$\Int^+$.
\end{prob}

\end{document}
