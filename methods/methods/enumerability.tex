% Part: methods
% Chapter: methods 
% Section: enumerability

\documentclass[../../include/open-logic-section]{subfiles}

\begin{document}

\olfileid{mth}{mth}{enu}
\olsection{Enumerability}

A set can be shown to be !!{enumerable} by giving an enumeration, either in
the form of a list or a function. But how might one generate such a list?
Arrays can play an important role in generating new enumerations out of
given ones.

Consider, for example, the set $\mathbb{Q}^{+}$ of all positive rational
numbers. Construct an array as follows:

\begin{tabular}{c|c|c|c} 
1/1 & 1/2 & 1/3 & ... \\\hline
2/1 & 2/2 & 2/3 & ... \\\hline
3/1 & 3/2 & 3/3 & ... \\\hline
... & ... & ... & ... \\
\end{tabular}

Clearly every rational number appears at at least one point in the array,
since an arbitrary rational m/n will appear at the $m^{th}$ row and $n^{th}$
column. Indeed, it will also appear at the $2m^{th}$ row and $2n^{th}$ column,
the $3m^{th}$ row and $3n^{th}$ column, and so on. But of course to show that
$\mathbb{Q}^{+}$ is enumberable we must arrange the rationals in the array
into a single list. The pattern in the array below represents one way to do
this.

\begin{tabular}{c|c|c|c|c} 
1 & 2 & 4 & 7 &  ... \\\hline
3 & 5 & 8 &  ...& ... \\\hline
6 & 9 &  ... & ... & ... \\\hline
10&  ... & ...& ... & ... \\\hline
... & ... & ... & ... & ...\\
\end{tabular}

This pattern is called \emph{Cantor's zig-zag} method. Together these
arrays determine the enumeration 1/1, 1/2, 2/2, 2/1, 1/3, 2/3, 3/3, 3/2,
3/1, ...

A perfectly parallel technique can be used to enumerate the ordered pairs
of positive integers, giving the enumeration $\langle1,1\rangle,
\langle1,2\rangle, \langle2,2\rangle, \langle2,1\rangle, \langle1,3\rangle,
\langle2,3\rangle, \langle3,3\rangle,$ $\langle3,2\rangle,
\langle3,1\rangle, ...$ With that in place we can then use an array to
enumerate the triples by treating the columns as corresponding to the
enumeration of the pairs and the rows to the enumeration of the positive
integers and combining to produce the triple $\langle k,n,m\rangle$, as in
the following array:

\begin{tabular}{c|c|c|c} 
1,1,1 & 1,1,2 & 1,1,3 & ... \\\hline
2,2,1 & 2,2,2 & 2,2,3 & ... \\\hline
3,3,1 & 3,3,2 & 3,3,3 & ... \\\hline
... & ... & ... & ... \\
\end{tabular}

For any arbitrary triple $\langle k,n,m\rangle$ it will appear at row k and
column $\langle n,m\rangle$. The list is then determined by following the
same pattern through the array as before.

\end{document}