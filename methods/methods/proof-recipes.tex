% Part: methods
% Chapter: methods 
% Section: proof-recipes

\documentclass[../../include/open-logic-section]{subfiles}

\begin{document}

\olfileid{mth}{mth}{rec}
\olsection{Recipes for Proofs}

\begin{enumerate}

\item Question: Show that $\Sigma \Entails !A$.

Proof outline: Let $v$ be a truth assignment satisfying all of $\Sigma$. Wewill show that $Val(!A) = T$.

    [...]

Thus $Val(!A) = T$. So any truth assignment satisfying all of $\Sigma$ alsosatisfies $!A$, which implies that $\Sigma \Entails !A$.

\item Question: Show that $\Sigma \Entails/ !A$.

Answer: Specify a truth assignment $v$ satisfying all of $\Sigma$ such that$Val(!A) = F$. Just give the truth values $v$ assigns to all relevant
!!{sentence} symbols.

\item Question: Show that $S \subseteq T$.

Proof outline: Let $a \in S$. We will show that $a \in T$.

[...]

Therefore, $a \in T$. Since every element of $S$ is also an element of $T$,this means $S \subseteq T$.

\item Question: Show that $S = T$ (where $S$ and $T$ are sets).

Answer: First show that $S \subseteq T$ and then show that $T \subseteq S$.
\item Question (Conditional Proof): Show that if $X$, then $Y$.

Proof: Suppose that $X$.

[...]

Therefore, $Y$. So if $X$ is true, then $Y$ must also be true.

\item Question (Biconditional Proof): Show that $X$ if and only if (iff)
$Y$.

Proof: Suppose that $X$.

[...]

Therefore, $Y$.

Now instead suppose $Y$.

[...]

Therefore, $X$.

So we conclude that $X$ if and only if $Y$.

\item Question (Proof by Cases.): We know that either $X$ or $Y$ is true.
Prove that $Z$.

Proof: Suppose $X$.

[...]

Therefore, $Z$.

Now instead suppose $Y$.

[...]

Therefore, $Z$.

Since $Z$ follows from both cases, $Z$ must be true.

\item Question (Proof by Contradiction): Show that $X$ is not true.

Proof: Suppose that $X$ is true.

[...]

But this is a contradiction. Therefore $X$ must be false.

\item Question (Universal Intro): Show that for every $a \in S$, it is $P$.
Take an arbitrary $a \in S$.

[...]

Therefore, $a$ is $P$. Since any arbitrary object in $S$ is $P$, then all
of $S$ is $P$.

\end{enumerate}

\end{document}