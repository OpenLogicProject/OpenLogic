% Part: non-classical-logics
% Chapter: trivalent-logics
% Section: syntax

\documentclass[../../../include/open-logic-section]{subfiles}

\begin{document}

\olfileid{ncl}{trv}{syn}

\olsection{Syntax}

The syntax of tri-valued logics is the language $\Lang L_{\Log PL}$ of propositional logic. In some extensions we consider the language $\Lang L^{\Delta}$ that supplements $\Lang L$ with a one-place determinacy operator $\Delta$:

\begin{defn}[Language $\Lang L^{\Delta}$]
\ollabel{defn:delta-language}
The language $\Lang L^{\Delta}$ is defined as follows:

\begin{enumerate}
  \item All the !!{formula}s of $\Lang L_{\Log PL}$ are formulas of $\Lang L^{\Delta}$.
  \item If $!A$ is a !!{formula} of $\Lang L^{\Delta}$, then $\Delta !A$ is a !!{formula} of $\Lang L^{\Delta}$.
  \tagitem{limitClause}{Nothing else is a !!{formula}.}{}
\end{enumerate}

\end{defn}

We can also add the indeterminacy operator $\nabla$. This can be added separately or instead defined as follows: $\nabla !A$ $\lnot \Delta !A \land \lnot \Delta \lnot !A$. 

\end{document}
