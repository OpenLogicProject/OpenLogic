% Part: normal-modal-logic
% Chapter: syntax-and-semantics
% Section: 

\documentclass[../../../include/open-logic-section]{subfiles}

\begin{document}

\olfileid{ncl}{trv}{trb}

\olsection{Semantics: basic connectives}

For the semantics, we now have three truth-values: $\True$, $\False$ and $\Indet$. We call $\True$ and $\False$ the \emph{classical truth-value}. Our !!{valuation}s will now assign one of these three values to atomic formulas: 

\begin{defn}[Trivalent !!^{valuation}s] 
Let $\{\True, \False, \Indet \}$ be the set of truth values. A trivalent \emph{!!{valuation}} for is a function~$\pAssign{v}$ assigning either $\True$ or $\False$ or $\Indet$ to the
!!{propositional variable}s of the language, i.e., $\pAssign{v} \colon
\PVar \to \{\True, \False, \Indet \}$.
\end{defn}

We then extend the !!{valuation} to complex formulas. This is done truth-functionally: each connective is associated with a function such that the truth-value of a complex formula is a function of the truth-value of its component. The functions obey the following constraint:

\begin{prop}[Preservation of classical values]
\ollabel{prop:preservation-classical-values}
  \item \emph{Preservation of classical !!{valuation}}. If $!A$ and $!B$ have classical truth-values, $\lnot !A$, $!A \land !B$, $!A \lif !B$ etc. have their classical !!{valuation}.
\end{prop}

What remains to be settled is what happens when component formulas have the truth-value $\Indet$.

\iftag{prvNot}{For negation the standard rule is given by:
\begin{center}
  \begin{tabular}{|c||c|} \hline 
    $!A$ & $ \lnot !A$ \\ 
    \hline \hline 
    $\True$ & $\False$ \\ 
    $\Indet$ & $\Indet$ \\
    $\False$ & $\True$ \\ 
    \hline 
  \end{tabular}
\end{center}
}{}

For disjunction and conjunction there are two options. The first is to say that $\Indet$ `spreads': the complex has value $\Indet$ as soon as one component has. The result is a \emph{weak Kleene} !!{valuation}, and is the basis of the weak Kleene logic. Most trivalent logics adopt instead the following tables:

\begin{center}
\iftag{prvAnd}{
  \begin{tabular}{|cc|ccc|} \hline 
     $!A \land !B$ & & & $!B$ &  \\
     & & $\True$ & $\Indet$ & $\False$ \\
    \hline \hline 
    & $\True$ & $\True$ & $\Indet$ & $\False$ \\ 
    $!A$ & $\Indet$ & $\Indet$ & $\Indet$ & $\False$ \\ 
    & $\False$ & $\False$ & $\False$ & $\False$ \\
    \hline 
  \end{tabular}
}{}
\iftag{prvOr}{
  \begin{tabular}{|cc|ccc|} \hline 
     $!A \lor !B$ & & & $!B$ &  \\
     & & $\True$ & $\Indet$ & $\False$ \\
    \hline \hline 
    & $\True$ & $\True$ & $\True$ & $\True$ \\ 
    $!A$ & $\Indet$ & $\True$ & $\Indet$ & $\Indet$ \\ 
    & $\False$ & $\True$ & $\Indet$ & $\False$ \\
    \hline 
  \end{tabular}
}{}%
\end{center} 

\begin{explain}
These tables preserve the usual interactions between $\lnot$, $\lor$ and $\land$: 

\begin{itemize}
	\item Double negation. $\lnot \lnot !A$ is equivalent to $!A$.
	\item De Morgan laws. $!A \lor !B$ is equivalent to $\lnot (!A \land !B)$. And similarly swapping $\land$ for $\lor$.
	\item Distribution. $!A \lor (!B \land !C)$ is equivalent to $(!A \land !C) \lor (!A \lor !C)$. And similarly swapping $\land$ for $\lor$.
\end{itemize}

\end{explain}

Finally we need to settle the conditional. But let's leave it aside for the moment and look at validity with the connectives we already have. 

\end{document}
