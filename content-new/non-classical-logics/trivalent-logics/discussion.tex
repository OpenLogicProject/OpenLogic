% Part: non-classical-logics
% Chapter: trivalent-logics
% Section: discussion

\documentclass[../../../include/open-logic-section]{subfiles}

\begin{document}

\olfileid{ncl}{trv}{dis}

\olsection{Discussion}

\subsection{On bivalence}

\emph{Bivalence} is the idea that every (meaningful) claim is true, or false, and not both. But seemingly meaningful claims that cannot comfortably be ascribed truth or falsity simpliciter: future contingents, failure of denotation, vagueness, liar paradoxes. Future contingents, vagueness denotation failures suggest that there are ``gaps'': claims that fail to be true but also fail to be false. The liar paradox has also been taken to suggest the existence of ``gaps'', but others argue that it suggests instead the existence of ``gluts'': claims that are both true and false. 

% These motivations raise questions about logic. For instance, a common thought about logic is that it is ``a priori''; its truth are discoverable independently of experience. One consequence of that idea is that if something is a logical truth, we should in principle be able to recognize by reflection alone that it is a logical truth.  

% we can motivate non-classical logics with the thought that one should be able to recognize non-logical truths:

% \begin{itemize}
% 	\item If $!A$ is not a logical truth, one can \emph{a priori} recognize that $!A$ isn't a logical truth.
% 	\item If classical logic is correct, a non-denoting claim of the form $!A \lif !A$ isn't a logical truth. 
% 	\item We can't \emph{a priori} recognize that a non-denoting claim of the form $!A \lif !A$ isn't a logical truth.
% 	\item So, classical logic isn't correct.
% \end{itemize}

% Similarly, if we think that logic is (at least in part) meant to give rules for reasoning, and we accept that we sometimes have to reason with non-denoting or ill-defined notions, this may require us to allow non-bivalent logics.

\paragraph{A naïve argument for bivalence} One might argue for bivalence as follows:

\begin{quote} ``false'' simply means ``not true''. Take any claim. Is it true? If yes, it's true. If not, then it's false, because being false just is not being true.	
\end{quote} 

The argument uses the Law of Excluded middle (the claim is true, or it is not true). You might think that this is somewhat question-begging. But the major flaw in the argument is that ``false'' doesn't simply mean ``not true''. Take anything that isn't a claim or sentence or assertion: a dog, London, the second-world war, the flu, etc. The flu or London aren't \emph{true}. They aren't claims at all; they don't say anything, so they aren't true. But they aren't false either. 

Note that there is a sense of ``true'' to mean ``real'': for instance, we may (try to) say that London is ``true'' while the mythical land of Eldorado is ``false'' or ``fake''. But that's a different sense of the word. You can see that by focusing on something that is real but says something false. Take a sign that states, wrongly, ``Sharks here'': the sign is real (hence ``true'' in the real-sense) but it is false (it claims something that isn't so). The second sense of true / false is what is relevant to logic.

Taking this into account, we can patch the argument. We say that something is true iff \emph{it says that something is so} and that thing is so, and false iff \emph{it says that something is so} and that is \emph{not} so. The argument goes thus:

\begin{enumerate}
	\item X says that p. (Assumption)
	\item Either p or not p. (Law of excluded middle)
	\item If X says that p and p, X is true. (Def truth)
	\item If X says that p and not-p, X is false. (Def falsity)
	\item So, either X is true or X is false.
	\item Generalizing: everything that says that something is so is either true or false.
\end{enumerate}

We can add a parallel argument using the Law of Non-contradiction to argue that everything that says that something is so is not both true and false.

\subsection{On expressive power}

Classical propositional logic is \emph{truth-functionally complete for bivalence}. If there are just two truth-values, the propositional language allows us to express any truth function. 

Standard trivalent logics are \emph{truth-functionally incomplete}. If there are three truth-values, $\True$, $\False$ and $\Indet$.  

\end{document}
