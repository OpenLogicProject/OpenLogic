% Part: propositional-logic
% Chapter: syntax-and-semantics
% Section: problem-set-truth-functions

\documentclass[../../../include/open-logic-section]{subfiles}

\begin{document}

\olfileid{pl}{syn}{pco}

% These problems will appear in the automated Problems section
% (either at the end of chapter or end of book)
% after all the problems that are included within the section

\begin{prob}
Fill in the truth-tables for negation, conjunction, disjunction, material
implication and biconditional.

\noindent
\begin{tabular}{|cc|c|}
\hline 
\multicolumn{2}{|c|}{$\lnot p$} & \tabularnewline
\hline 
\multirow{2}{*}{$p$} & 1 & $\ldots$\tabularnewline
 & 0 & $\ldots$\tabularnewline
\hline 
\end{tabular}\qquad{}%
\begin{tabular}{|cc|cc|}
\hline 
\multicolumn{2}{|c|}{} & \multicolumn{2}{c|}{$q$}\tabularnewline
\multicolumn{2}{|c|}{$p\land q$} & 1 & 0\tabularnewline
\hline 
\multirow{2}{*}{$p$} & 1 & $\ldots$ & $\ldots$\tabularnewline
 & 0 & $\ldots$ & $\ldots$\tabularnewline
\hline 
\end{tabular}\qquad{}%
\begin{tabular}{|cc|cc|}
\hline 
\multicolumn{2}{|c|}{} & \multicolumn{2}{c|}{$q$}\tabularnewline
\multicolumn{2}{|c|}{$p\lor q$} & 1 & 0\tabularnewline
\hline 
\multirow{2}{*}{$p$} & 1 & $\ldots$ & $\ldots$\tabularnewline
 & 0 & $\ldots$ & $\ldots$\tabularnewline
\hline 
\end{tabular}
\begin{tabular}{|cc|cc|}
\hline 
\multicolumn{2}{|c|}{} & \multicolumn{2}{c|}{$q$}\tabularnewline
\multicolumn{2}{|c|}{$p\lif q$} & 1 & 0\tabularnewline
\hline 
\multirow{2}{*}{$p$} & 1 & $\ldots$ & $\ldots$\tabularnewline
 & 0 & $\ldots$ & $\ldots$\tabularnewline
\hline 
\end{tabular}\qquad{}%
\begin{tabular}{|cc|cc|}
\hline 
\multicolumn{2}{|c|}{} & \multicolumn{2}{c|}{$q$}\tabularnewline
\multicolumn{2}{|c|}{$p\liff q$} & 1 & 0\tabularnewline
\hline 
\multirow{2}{*}{$p$} & 1 & $\ldots$ & $\ldots$\tabularnewline
 & 0 & $\ldots$ & $\ldots$\tabularnewline
\hline 
\end{tabular}
\end{prob}

\begin{prob}
For each formula below, give a valuation that makes it true and one
that makes it false.
\begin{enumerate}
\item  $\lnot P\lor Q$.
\item $\lnot(P\lor Q)$.
\item $\lnot(P\lif Q)$.
\item $\lnot P\lif Q$.
\item $P\lif\lnot P$.
\item $(P\liff Q)\land(\lnot Q\lif R)$.
\item $(P\land Q)\lif R.$
\item $\lnot(P\land(\lnot Q\lor R))$.
\end{enumerate}
\emph{Note: a valuation is an assignement truth-value for each of
$P$, $Q$, $R$, $\ldots$}.
\end{prob}

\begin{prob}
Three\emph{ }equally good poker players, Patricia, Quinn and Roy,
are playing a game. Only one of them can be the winner.

\begin{enumerate}
\item Consider the following sentences. Does it seem to you that they must
be true, must be false, or could be either dependening on how the
game plays out?
\begin{enumerate}
\item If Patricia loses then she doesn't lose.
\item Either if Patricia loses then Quinn loses too, or if Quinn loses then
Patricia loses too. 
\end{enumerate}
\item Consider the formulas below. Check whether they can be true or false
depending on whether $P$ and $Q$ are.
\begin{enumerate}
\item $P\lif\lnot P$.
\item $(P\lif Q)\lor(Q\lif P)$.
\end{enumerate}
\item When, if at all, 1(a) and 2(a) come apart?
\end{enumerate}
\end{prob}

\begin{prob}
Answer the following.
\begin{enumerate}
\item  Give a formula equivalent to $P\lif Q$ that only uses $\lor$
and $\lnot$.
\item Give a formula equivalent to $(P\land Q)\lif(Q\lor R)$ that
only uses $\land,\lor$ and $\lnot$.
\item Give a formula equivalent to $P\land Q$ that only uses $\lor$ and
$\lnot$.
\item Give a formula equivalent to $P\lor Q$ that only uses $\land$ and
$\lnot$.
\item Give a formula equivalent to $P\lor Q$ that only uses $\lif$
and $\lnot$.
\item Give a formula equivalent to $P\land Q$ that only uses $\lif$
and $\lnot$.
\item Give a formula equivalent to $P\liff Q$ that only uses
$\lif$ and $\lnot$.
\end{enumerate}

\emph{Note: an equivalent formula is one that is true if the original
is true, false if the original is false, no matter what the values
of $P,Q,R,\ldots$ are.}
\end{prob}

\begin{prob}\ollabel{exer:possible-connectives}
Each clause below describes a possible connective $\bullet$. For each
of them, (a) write its truth-table, and (b) give an equivalence using
only $\lnot,\land,\lor,\lif$. 
\begin{enumerate}
\item \emph{Sheffer stroke, nand}. $P\bullet Q$ is true iff one of $P$ or $Q$
is false. (Normally noted `$|$')
\item \emph{Quine's dagger, nor}. $P\bullet Q$ is true iff neither $P$ nor $Q$ are true.
(Often noted `$\downarrow$'.)
\item $P\bullet Q$ is true just if $P$ is true.
\item $\bullet P$ is true just if $P$ is true.
\item $\bullet P$ is always false.
\item $P\bullet Q$ is some connective of your invention, distinct from
any of the connectives we have seen so far. 
\end{enumerate}
\end{prob}

\begin{prob}
Check that the pairs of formulas below are equivalent.
\begin{enumerate}
\item \emph{De Morgan law.} $\lnot(P\land Q)$ and $\lnot P\lor\lnot Q$.
\item \emph{De Morgan law.} $\lnot(P\lor Q)$ and $\lnot P\land\lnot Q$.
\item \emph{Contraposition.} $P\lif Q$ and $\lnot Q\lif\lnot P$. 
\item \emph{Definition of negation.} $\lnot P$ and $P\lif\bot$.
\item \emph{Import / Export.} $P\lif(Q\lif R)$ and $(P\land Q)\lif R$. 
\end{enumerate}

\emph{Notes: two formulas are equivalent iff they have the same truth
value no matter what the values of $P,Q,R,\ldots$ are.}
\emph{$\bot$, the falsum, is a symbol that is always false.}
\end{prob}

\begin{prob}
Check that the formulas below are logically valid. 
\begin{enumerate}
\item \emph{Law of non-contradiction. }$\lnot(P\land\lnot P)$.
\item \emph{Law of excluded middle}. $P\lor\lnot P$.
\item $P\lif(Q\lif P)$.
\item $\bot\lif P$.
\item $\lnot(Q\lif P)\lif(P\lif Q)$.
\item $\lnot(Q\land P)\lif(P\lif\lnot Q)$.
\end{enumerate}
\emph{Notes: a formula is logically valid iff it is true no matter
whether $P$, $Q$, $\ldots$ are true or false.}
\emph{$\bot$, the falsum, is a symbol that is always false.}
\end{prob}

\begin{prob}
Say whether the formulas are logically valid. If not, give a counterexample.
\begin{enumerate}
\item $\lnot(P\lif Q)\lif(\lnot P\lif Q)$.
\item $((P\lif Q)\lif Q)\lif Q$.
\item $(P\lif Q)\lif(P\land R\lif Q)$. 
\item $(P\lif Q)\lif\lnot(P\lif\lnot Q)$.
\item $(\lnot(P\land Q)\lif(P\land Q))\lif Q$.
\end{enumerate}
\end{prob}

\begin{prob}
There are 16 possible two-places connectives $P\bullet Q$. Why?

The following 14 are (simple variations of) the ones we've seen so
far. What are the missing two? (One of the missing ones could be the
one you invented in exercise \olref[pl][syn][pco]{exer:possible-connectives}.
\begin{itemize}
\item always true; always false;
\item true if $P$ is true; true if $P$ is false; true if $Q$ is true;
true if $Q$ is false;
\item conjunction; disjunction; conditional; biconditional;
\item Sheffer's stroke; nor.
\end{itemize}
\end{prob}

\end{document}
