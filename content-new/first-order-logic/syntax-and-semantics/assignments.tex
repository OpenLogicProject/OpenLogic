% Part: first-order-logic
% Chapter: syntax-and-semantics
% Section: assignments

\documentclass[../../../include/open-logic-section]{subfiles}

\begin{document}

\olfileid{fol}{syn}{ass}

\olsection{Variable Assignments}

\begin{explain}
A !!{variable} assignment~$s$ provides a value for \emph{every}
variable---and there are infinitely many of them. This is of course
not necessary. We require !!{variable} assignments to assign values to
all !!{variable}s simply because it makes things a lot easier.  The
value of a term~$t$, and whether or not a !!{formula}~$!A$ is
satisfied in a !!{structure} with respect to~$s$, only depend on the
assignments~$s$ makes to the !!{variable}s in~$t$ and the free
!!{variable}s of~$!A$.  This is the content of the next two
propositions.  To make the idea of ``depends on'' precise, we show
that any two variable assignments that agree on all the variables
in~$t$ give the same value, and that $!A$ is satisfied relative to one
iff it is satisfied relative to the other if two variable assignments
agree on all free variables of~$!A$.
\end{explain}

\begin{prop}\ollabel{prop:valindep}
If the !!{variable}s in a term~$t$ are among $x_1$, \dots,~$x_n$, and
$s_1(x_i) = s_2(x_i)$ for $i = 1$, \dots,~$n$, then $\Value{t}{M}[s_1]
= \Value{t}{M}[s_2]$.
\end{prop}

\begin{proof}
By induction on the complexity of~$t$. For the base case, $t$ can be
!!a{constant} or one of the variables~$x_1$, \dots,~$x_n$.  If $t
= c$, then $\Value{t}{M}[s_1] = \Assign{c}{M} = \Value{t}{M}[s_2]$. If
$t = x_i$, $s_1(x_i) = s_2(x_i)$ by the hypothesis of the proposition,
and so $\Value{t}{M}[s_1] = s_1(x_i) = s_2(x_i) = \Value{t}{M}[s_2]$.

For the inductive step, assume that $t = \Atom{f}{t_1, \dots, t_k}$
and that the claim holds for $t_1$, \dots, $t_k$. Then
\begin{align*}
  \Value{t}{M}[s_1] & = \Value{\Atom{f}{t_1, \dots, t_k}}{M}[s_1] =\\
  & = \Assign{f}{M}(\Value{t_1}{M}[s_1], \dots, \Value{t_k}{M}[s_1])
\intertext{For $j = 1$, \dots,~$k$, the !!{variable}s of~$t_j$ are
    among $x_1$, \dots,~$x_n$. So by induction hypothesis,
    $\Value{t_j}{M}[s_1] = \Value{t_j}{M}[s_2]$. So,}
\Value{t}{M}[s_1] & = \Value{\Atom{f}{t_1, \dots, t_k}}{M}[s_2] =\\
  & = \Assign{f}{M}(\Value{t_1}{M}[s_1], \dots, \Value{t_k}{M}[s_1]) =\\
  & = \Assign{f}{M}(\Value{t_1}{M}[s_2], \dots, \Value{t_k}{M}[s_2]) =\\
  & = \Value{\Atom{f}{t_1, \dots, t_k}}{M}[s_2] = \Value{t}{M}[s_2].
\end{align*}
\end{proof}

\begin{prop}\ollabel{prop:satindep}
If the free !!{variable}s in $!A$ are among $x_1$, \dots,~$x_n$, and
$s_1(x_i) = s_2(x_i)$ for $i = 1$, \dots,~$n$, then $\Sat{M}{!A}[s_1]$
iff $\Sat{M}{!A}[s_2]$.
\end{prop}

\begin{proof}
We use induction on the complexity of $!A$. For the base case, where
$!A$ is atomic, $!A$ can be:
\iftag{prvTrue}{$\ltrue$,}{}
\iftag{prvFalse}{$\lfalse$,}{}
$\Atom{R}{t_1, \dots, t_k}$ for a $k$-place predicate $R$ and terms
$t_1$, \dots,~$t_k$, or $\eq[t_1][t_2]$ for terms $t_1$ and~$t_2$.

\begin{enumerate}
\tagitem{prvTrue}{%
  \indcase{!A}{\ltrue}{both $\Sat{M}{\indfrm}[s_1]$ and
    $\Sat{M}{\indfrm}[s_2]$.}}{}

\tagitem{prvFalse}{%
  \indcase{!A}{\lfalse}{both $\Sat/{M}{!A}[s_1]$ and
    $\Sat/{M}{!A}[s_2]$.}}{}

\item
  \indcase{!A}{\Atom{R}{t_1, \ldots, t_k}}{let
    $\Sat{M}{\indfrm}[s_1]$. Then
    \[
    \langle \Value{t_1}{M}[s_1], \ldots, \Value{t_k}{M}[s_1] \rangle
    \in \Assign{R}{M}.
    \]
    For $i = 1$, \dots,~$k$, $\Value{t_i}{M}[s_1] =
    \Value{t_i}{M}[s_2]$ by \olref{prop:valindep}. So we also have
    $\langle \Value{t_i}{M}[s_2], \ldots, \Value{t_k}{M}[s_2] \rangle
    \in \Assign{R}{M}$.}
\item
  \indcase{!A}{\eq[t_1][t_2]}{suppose $\Sat{M}{\indfrm}[s_1]$.
    Then $\Value{t_1}{M}[s_1] = \Value{t_2}{M}[s_1]$. So,
    \begin{align*}
      \Value{t_1}{M}[s_2] & = \Value{t_1}{M}[s_1]
      & \text{(by \olref{prop:valindep})} \\
      & = \Value{t_2}{M}[s_1] 
      & \text{(since $\Sat{M}{\eq[t_1][t_2]}[s_1]$)}\\
      &= \Value{t_2}{M}[s_2]
      & \text{(by \olref{prop:valindep}),}
    \end{align*}
    so $\Sat{M}{\eq[t_1][t_2]}[s_2]$.}
\end{enumerate}

Now assume $\Sat{M}{!B}[s_1]$ iff $\Sat{M}{!B}[s_2]$ for all
!!{formula}s $!B$ less complex than~$!A$. The induction step proceeds
by cases determined by the main operator of~$!A$. In each case, we
only demonstrate the forward direction of the !!{biconditional}; the
proof of the reverse direction is symmetrical. In all cases except
those for the quantifiers, we apply the induction hypothesis to
sub-!!{formula}s~$!B$ of~$!A$.  The free variables of~$!B$ are among
those of~$!A$. Thus, if $s_1$ and $s_2$ agree on the free variables
of~$!A$, they also agree on those of~$!B$, and the induction
hypothesis applies to~$!B$.

\begin{enumerate}
\tagitem{defNot}{}{%
  \iftag{probNot}{%
    \indcase!{!A}{\lnot !B}{}}{%
    \indcase{!A}{\lnot !B}{if $\Sat{M}{\indfrm}[s_1]$, then
      $\Sat/{M}{!B}[s_1]$, so by the induction hypothesis,
      $\Sat/{M}{!B}[s_2]$, hence $\Sat{M}{\indfrm}[s_2]$.}}}

\tagitem{defAnd}{}{%
  \iftag{probAnd}{%
    \indcase!{!A}{!B \land !C}{}}{%
    \indcase{!A}{!B \land !C}{if $\Sat{M}{\indfrm}[s_1]$, then
      $\Sat{M}{!B}[s_1]$ and $\Sat{M}{!C}[s_1]$, so by induction
      hypothesis, $\Sat{M}{!B}[s_2]$ and $\Sat{M}{!C}[s_2]$. Hence,
      $\Sat{M}{\indfrm}[s_2]$.}}}

\tagitem{defOr}{}{%
  \iftag{probOr}{%
    \indcase!{!A}{!B \lor !C}{}}{%
    \indcase{!A}{!B \lor !C}{if $\Sat{M}{\indfrm}[s_1]$, then
      $\Sat{M}{!B}[s_1]$ or $\Sat{M}{!C}[s_1]$. By induction hypothesis,
      $\Sat{M}{!B}[s_2]$ or $\Sat{M}{!C}[s_2]$, so $\Sat{M}{\indfrm}[s_2]$.}}}

\tagitem{defIf}{}{%
  \iftag{probIf}{%
    \indcase!{!A}{!B \lif !C}{}}{%
    \indcase{!A}{!B \lif !C}{if $\Sat{M}{\indfrm}[s_1]$, then
    $\Sat/{M}{!B}[s_1]$ or $\Sat{M}{!C}[s_1]$. By the induction hypothesis,
    $\Sat/{M}{!B}[s_2]$ or $\Sat{M}{!C}[s_2]$, so $\Sat{M}{\indfrm}[s_2]$.}}}

\tagitem{defIff}{}{%
  \iftag{probIff}{%
    \indcase!{!A}{!B \liff !C}{}}{%
    \indcase{!A}{!B \liff !C}{if $\Sat{M}{\indfrm}[s_1]$, then either
      $\Sat{M}{!B}[s_1]$ and $\Sat{M}{!C}[s_1]$, or $\Sat/{M}{!B}[s_1]$ and
      $\Sat/{M}{!C}[s_1]$. By the induction hypothesis, either
      $\Sat{M}{!B}[s_2]$ and $\Sat{M}{!C}[s_2]$ or $\Sat/{M}{!B}[s_2]$
      and $\Sat/{M}{!C}[s_2]$. In either case, $\Sat{M}{\indfrm}[s_2]$.}}}

\tagitem{defEx}{}{%
  \iftag{probEx}{%
    \indcase!{!A}{\lexists[x][!B]}{}}{%
    \indcase{!A}{\lexists[x][!B]}{if $\Sat{M}{\indfrm}[s_1]$, there is
      an $x$-variant $s_1'$ of~$s_1$ so that $\Sat{M}{!B}[s_1']$. Let
      $s_2'$ be the $x$-variant of $s_2$ that assigns the same thing
      to~$x$ as does $s_1'$. The free variables of $!B$ are among
      $x_1$, \dots, $x_n$, and $x$. $s_1'(x_i) = s_2'(x_i)$, since
      $s_1'$ and $s_2'$ are $x$-variants of $s_1$ and~$s_2$,
      respectively, and by hypothesis $s_1(x_i) = s_2(x_i)$. $s_1'(x)
      = s_2'(x)$ by the way we have defined $s_2'$. Then the induction
      hypothesis applies to $!B$ and $s_1'$, $s_2'$, so
      $\Sat{M}{!B}[s_2']$. Hence, there is an $x$-variant of $s_2$
      that satisfies~$!B$, and so $\Sat{M}{\indfrm}[s_2]$.}}}

\tagitem{defAll}{}{%
  \iftag{probAll}{%
    \indcase!{!A}{\lforall[x][!B]}{}}{%
    \indcase{!A}{\lforall[x][!B]}{if $\Sat{M}{\indfrm}[s_1]$, then for
      every $x$-variant $s_1'$ of $s_1$, $\Sat{M}{!B}[s_1']$. Take an
      arbitrary $x$-variant~$s_2'$ of $s_2$, let $s_1'$ be the
      $x$-variant of $s_1$ which assigns the same thing to~$x$ as
      does~$s_2'$. The free variables of~$!B$ are among $x_1$, \dots,
      $x_n$, and $x$. $s_1'(x_i) = s_2'(x_i)$, since $s_1'$ and $s_2'$
      are $x$-variants of $s_1$ and~$s_2$, respectively, and by
      hypothesis $s_1(x_i) = s_2(x_i)$. $s_1'(x) = s_2'(x)$ by the way
      we have defined $s_1'$. Then the induction hypothesis applies
      to~$!B$ and~$s_1'$, $s_2'$, and we have $\Sat{M}{!B}[s_2']$.
      Since $s_2'$ is an arbitrary $x$-variant of~$s_2$, every
      $x$-variant of $s_2$ satisfies~$!B$, and so
      $\Sat{M}{\indfrm}[s_2]$.}}}
\end{enumerate}
By induction, we get that $\Sat{M}{!A}[s_1]$ iff $\Sat{M}{!A}[s_2]$
whenever the free !!{variable}s in $!A$ are among $x_1$, \dots, $x_n$
and $s_1(x_i)=s_2(x_i)$ for $i = 1$, \dots,~$n$.
\end{proof}

\begin{probtag}{probNot,probOr,probAnd,probIf,probIff,probEx,probAll}
Complete the proof of \olref[fol][syn][ass]{prop:satindep}.
\end{probtag}

\begin{explain}
!!^{sentence}s have no free variables, so any two variable assignments  
assign the same things to all the (zero) free variables of any
sentence. The proposition just proved then means that whether or not
!!a{sentence} is satisfied in a structure relative to a variable
assignment is completely independent of the assignment. We'll record
this fact. It justifies the definition of satisfaction of
!!a{sentence} in !!a{structure} (without mentioning a variable
assignment) that follows.
\end{explain}

\begin{cor}
\ollabel{cor:sat-sentence}
If $!A$ is !!a{sentence} and $s$ a variable assignment, then
$\Sat{M}{!A}[s]$ iff $\Sat{M}{!A}[s']$ for every variable
assignment~$s'$.
\end{cor}

\begin{proof}
  Let $s'$ be any variable assignment. Since $!A$ is !!a{sentence}, it
  has no free variables, and so every variable assignment~$s'$
  trivially assigns the same things to all free variables of~$!A$ as
  does~$s$. So the condition of \olref{prop:satindep} is satisfied,
  and we have $\Sat{M}{!A}[s]$ iff $\Sat{M}{!A}[s']$.
\end{proof}

\begin{defn}
\ollabel{defn:satisfaction}
If $!A$ is !!a{sentence}, we say that !!a{structure}~$\Struct M$
\emph{satisfies}~$!A$, $\Sat{M}{!A}$, iff $\Sat{M}{!A}[s]$ for all
variable assignments~$s$.
\end{defn}

If $\Sat{M}{!A}$, we also simply say that \emph{$!A$ is true
  in~$\Struct{M}$.}

\begin{prop}\ollabel{prop:sentence-sat-true}
  Let $\Struct{M}$ be !!a{structure}, $!A$ be a sentence, and $s$ a
  variable assignment.  $\Sat{M}{!A}$ iff $\Sat{M}{!A}[s]$.
\end{prop}

\begin{proof}
Exercise.
\end{proof}

\begin{prob}
Prove \olref[fol][syn][ass]{prop:sentence-sat-true}
\end{prob}

\begin{prop}\ollabel{prop:sat-quant}
  Suppose $!A(x)$ only contains $x$ free, and $\Struct M$ is
  !!a{structure}. Then:
  \begin{tagenumerate}{prvEx,prvAll}
    \tagitem{prvEx}{$\Sat{M}{\lexists[x][!A(x)]}$ iff $\Sat{M}{!A(x)}[s]$
      for at least one variable assignment~$s$.}{}
    \tagitem{prvAll}{$\Sat{M}{\lforall[x][!A(x)]}$ iff $\Sat{M}{!A(x)}[s]$
  for all variable assignments~$s$.}{}
\end{tagenumerate}
\end{prop}

\begin{proof}
Exercise.
\end{proof}

\begin{prob}
Prove \olref[fol][syn][ass]{prop:sat-quant}.
\end{prob}

\begin{prob}
\DeclareRobustCommand{\VDash}{\mathrel{||}\joinrel\Relbar}  
Suppose $\Lang L$ is a language without !!{function}s. Given a
!!{structure}~$\Struct M$, $c$ a !!{constant} and $a \in \Domain M$,
define $\Struct M[a/c]$ to be the !!{structure} that is just
like~$\Struct M$, except that $\Assign{c}{M[a/c]} = a$. Define
$\Struct M \VDash !A$ for !!{sentence}s~$!A$ by:
\begin{enumerate}
\tagitem{prvFalse}{%
  \indcase{!A}{\lfalse}{not $\Struct{M} \VDash \indfrm$.}}{}

\tagitem{prvTrue}{%
  \indcase{!A}{\ltrue}{$\Struct{M} \VDash \indfrm$.}}{}

\item \indcase{!A}{\Atom{R}{d_1, \dots, d_n}}{$\Struct M \VDash \indfrm$
  iff $\langle \Assign{d_1}{M}, \dots, \Assign{d_n}{M} \rangle \in
  \Assign{R}{M}$.}

\item \indcase{!A}{\eq[d_1][d_2]}{$\Struct M \VDash \indfrm$ iff
  $\Assign{d_1}{M} = \Assign{d_2}{M}$.}

\tagitem{prvNot}{%
  \indcase{!A}{\lnot !B}{$\Struct{M} \VDash \indfrm$ iff
    not $\Struct{M} \VDash {!B}$.}}{}

\tagitem{prvAnd}{%
  \indcase{!A}{(!B \land !C)}{$\Struct{M} \VDash
    \indfrm$ iff $\Struct{M} \VDash!B$ and $\Struct{M} \VDash !C$.}}{}

\tagitem{prvOr}{%
  \indcase{!A}{(!B \lor !C)}{$\Struct{M} \VDash \indfrm$ iff
    $\Struct{M} \VDash !B$ or $\Struct{M} \VDash !C$ (or both).}}{}

\tagitem{prvIf}{%
  \indcase{!A}{(!B \lif !C)}{$\Struct{M} \VDash \indfrm$ iff
    not $\Struct {M} \VDash !B$ or $\Struct M \VDash !C$ (or both).}}{}

\tagitem{prvIff}{%
  \indcase{!A}{(!B \liff !C)}{$\Struct{M} \VDash {\indfrm}$ iff either
    both $\Struct{M} \VDash {!B}$ and $\Struct{M} \VDash {!C}$, or
    neither $\Struct{M} \VDash {!B}$ nor $\Struct{M} \VDash {!C}$.}}{}

\tagitem{prvAll}{%
  \indcase{!A}{\lforall[x][!B]}{$\Struct{M} \VDash {\indfrm}$ iff for
    all $a \in \Domain{M}$, $\Struct{M[a/c]} \VDash \Subst{!B}{c}{x}$,
    if $c$ does not occur in~$!B$.}}{}

\tagitem{prvEx}{%
  \indcase{!A}{\lexists[x][!B]}{$\Struct{M} \VDash
  {\indfrm}$ iff there is an $a \in \Domain M$ such that
  $\Struct{M[a/c]} \VDash \Subst{!B}{c}{x}$, if $c$ does not occur
  in~$!B$.}}{}
\end{enumerate}
Let $x_1$, \dots, $x_n$ be all free !!{variable}s in~$!A$,
$c_1$, \dots, $c_n$ constant symbols not in~$!A$,
$a_1$, \dots, $a_n \in \Domain M$, and $s(x_i) = a_i$.

Show that $\Sat{M}{!A}[s]$ iff $\Struct M[a_1/c_1,\dots,a_n/c_n]
\VDash \Subst{\Subst{!A}{c_1}{x_1}\dots}{c_n}{x_n}$.

(This problem shows that it is possible to give a semantics for
first-order logic that makes do without variable assignments.)
\end{prob}

\begin{prob}
Suppose that $f$ is a function symbol not in~$!A(x,y)$. Show that
there is !!a{structure}~$\Struct{M}$ such that
$\Sat{M}{\lforall[x][\lexists[y][!A(x,y)]]}$ iff there is an~$\Struct
M'$ such that $\Sat{M'}{\lforall[x][!A(x,f(x))]}$.

(This problem is a special case of what's known as Skolem's Theorem;
$\lforall[x][!A(x,f(x))]$ is called a \emph{Skolem normal form} of
$\lforall[x][\lexists[y][!A(x,y)]]$.)
\end{prob}



\end{document}
