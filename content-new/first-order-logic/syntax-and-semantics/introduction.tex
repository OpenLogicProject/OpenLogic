% Part: first-order-logic
% Chapter: syntax-and-semantics
% Section: introduction

\documentclass[../../../include/open-logic-section]{subfiles}

\begin{document}

\olfileid{fol}{syn}{int}

\olsection{Introduction}

In order to develop the theory and metatheory of first-order logic, we
must first define the syntax and semantics of its expressions.  The
expressions of first-order logic are terms and !!{formula}s.  Terms
are formed from !!{variable}s, !!{constant}s, and !!{function}s.
!!^{formula}s, in turn, are formed from !!{predicate}s together with
terms (these form the smallest, ``atomic'' !!{formula}s), and then
from atomic !!{formula}s we can form more complex ones using logical
connectives and quantifiers.  There are many different ways to set
down the formation rules; we give just one possible one. Other systems
will chose different symbols, will select different sets of
connectives as primitive, will use parentheses differently (or even not
at all, as in the case of so-called Polish notation).  What all
approaches have in common, though, is that the formation rules define
the set of terms and !!{formula}s \emph{inductively}. If done
properly, every expression can result essentially in only one way
according to the formation rules.  The inductive definition resulting
in expressions that are \emph{uniquely readable} means we can give
meanings to these expressions using the same method---inductive
definition.

Giving the meaning of expressions is the domain of semantics.  The
central concept in semantics is that of satisfaction in
!!a{structure}. !!^a{structure} gives meaning to the building blocks
of the language: !!a{domain} is a non-empty set of objects. The
quantifiers are interpreted as ranging over this domain, !!{constant}s
are assigned elements in the domain, !!{function}s are assigned
functions from the !!{domain} to itself, and !!{predicate}s are
assigned relations on the !!{domain}.  The !!{domain} together with
assignments to the basic vocabulary constitutes
!!a{structure}. !!^{variable}s may appear in !!{formula}s, and in
order to give a semantics, we also have to assign !!{element}s of the
!!{domain} to them---this is a variable assignment. The satisfaction
relation, finally, brings these together. !!^a{formula} may be satisfied
in !!a{structure}~$\Struct{M}$ relative to a variable assignment~$s$,
written as $\Sat{M}{!A}[s]$. This relation is also defined by
induction on the structure of~$!A$, using the truth tables for the
logical connectives to define, say, satisfaction of $!A \land !B$ in
terms of satisfaction (or not) of $!A$ and ~$!B$.  It then turns out
that the variable assignment is irrelevant if the !!{formula}~$!A$ is
!!a{sentence}, i.e., has no free variables, and so we can talk of
!!{sentence}s being simply satisfied (or not) in !!{structure}s.

On the basis of the satisfaction relation $\Sat{M}{!A}$ for sentences
we can then define the basic semantic notions of validity, entailment,
and satisfiability.  A sentence is valid, $\Entails !A$, if every
structure satisfies it. It is entailed by a set of !!{sentence}s,
$\Gamma \Entails !A$, if every !!{structure} that satisfies all the
!!{sentence}s in~$\Gamma$ also satisfies~$!A$. And a set of sentences
is satisfiable if some !!{structure} satisfies all !!{sentence}s in it
at the same time.  Because !!{formula}s are inductively defined, and
satisfaction is in turn defined by induction on the structure of
!!{formula}s, we can use induction to prove properties of our
semantics and to relate the semantic notions defined.




\end{document}
