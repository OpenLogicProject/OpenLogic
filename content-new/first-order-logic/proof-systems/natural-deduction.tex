% Part: first-order-logic
% Chapter: proof-systems
% Section: natural-deduction

\documentclass[../../../include/open-logic-section]{subfiles}

\begin{document}

\iftag{FOL}
      {\olfileid{fol}{prf}{ntd}}
      {\olfileid{pl}{prf}{ntd}}

\olsection{Natural Deduction}

Natural deduction is !!a{derivation} system intended to mirror actual
reasoning (especially the kind of regimented reasoning employed by
mathematicians).  Actual reasoning proceeds by a number of ``natural''
patterns. For instance, proof by cases allows us to establish a
conclusion on the basis of a disjunctive premise, by establishing that
the conclusion follows from either of the disjuncts. Indirect proof
allows us to establish a conclusion by showing that its negation leads
to a contradiction. Conditional proof establishes a conditional claim
``if \dots then \dots'' by showing that the consequent follows from
the antecedent.  Natural deduction is a formalization of some of these
natural inferences.  Each of the logical connectives and quantifiers
comes with two rules, an introduction and an elimination rule, and
they each correspond to one such natural inference pattern. For
instance, $\Intro{\lif}$ corresponds to conditional proof, and
$\Elim{\lor}$ to proof by cases.  A particularly simple rule is
$\Elim{\land}$ which allows the inference from $!A \land !B$ to~$!A$
(or $!B$).

One feature that distinguishes natural deduction from other
!!{derivation} systems is its use of assumptions. !!^a{derivation} in
natural deduction is a tree of !!{formula}s.  A single !!{formula}
stands at the root of the tree of !!{formula}s, and the ``leaves'' of
the tree are !!{formula}s from which the conclusion is derived.  In
natural deduction, some leaf !!{formula}s play a role inside the
!!{derivation} but are ``used up'' by the time the !!{derivation}
reaches the conclusion. This corresponds to the practice, in actual
reasoning, of introducing hypotheses which only remain in effect for a
short while.  For instance, in a proof by cases, we assume the truth
of each of the disjuncts; in conditional proof, we assume the truth of
the antecedent; in indirect proof, we assume the truth of the negation
of the conclusion.  This way of introducing hypothetical assumptions
and then doing away with them in the service of establishing an
intermediate step is a hallmark of natural deduction. The formulas at
the leaves of a natural deduction !!{derivation} are called
assumptions, and some of the rules of inference may ``!!{discharge}''
them.  For instance, if we have !!a{derivation} of~$!B$ from some
assumptions which include~$!A$, then the $\Intro{\lif}$ rule allows us
to infer~$!A \lif !B$ and discharge any assumption of the form~$!A$.
(To keep track of which assumptions are discharged at which
inferences, we label the inference and the assumptions it discharges
with a number.)  The assumptions that remain !!{undischarged} at the
end of the !!{derivation} are together sufficient for the truth of the
conclusion, and so !!a{derivation} establishes that its
!!{undischarged} assumptions entail its conclusion.

The relation $\Gamma \Proves !A$ based on natural deduction holds iff
there is !!a{derivation} in which $!A$~is the last !!{sentence} in the
tree, and every leaf which is !!{undischarged} is in~$\Gamma$. $!A$~is
a theorem in natural deduction iff there is !!a{derivation} in which
$!A$~is the last !!{sentence} and all assumptions are !!{discharged}.
For instance, here is !!a{derivation} that shows that $\Proves (!A
\land !B) \lif !A$:
\begin{prooftree}
  \AxiomC{$\Discharge{!A \land !B}{1}$}
  \RightLabel{\Elim{\land}}
  \UnaryInfC{$!A$}
  \DischargeRule{\Intro{\lif}}{1}
  \UnaryInfC{$(!A \land !B) \lif !A$}
\end{prooftree}
The label~$1$ indicates that the assumption $!A \land !B$ is
!!{discharged} at the \Intro{\lif} inference.
  
A set~$\Gamma$ is inconsistent iff $\Gamma \Proves \lfalse$ in natural
deduction.  The rule \FalseInt{} makes it so that from an inconsistent
set, any !!{sentence} can be !!{derive}d.

Natural deduction systems were developed by Gerhard Gentzen and
Stanis\l{}aw Ja\'skowski in the 1930s, and later developed by Dag
Prawitz and Frederic Fitch. Because its inferences mirror natural
methods of proof, it is favored by philosophers. The versions
developed by Fitch are often used in introductory logic textbooks. In
the philosophy of logic, the rules of natural deduction have sometimes
been taken to give the meanings of the logical operators
(``proof-theoretic semantics'').

\end{document}
