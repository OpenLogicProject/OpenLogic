% Part: first-order-logic
% Chapter: sequent-calculus
% Section: soundness-identity

\documentclass[../../../include/open-logic-section]{subfiles}

\begin{document}

\olfileid{fol}{seq}{sid}

\olsection{Soundness with \usetoken{S}{identity}}

\begin{prop}
$\Log{LK}$ with initial sequents and rules for identity is sound.
\end{prop}

\begin{proof}
Initial sequents of the form ${} \Sequent \eq[t][t]$ are valid, since
for every !!{structure}~$\Struct M$, $\Sat{M}{\eq[t][t]}$. (Note that
we assume the term $t$ to be closed, i.e., it contains no variables,
so variable assignments are irrelevant).

Suppose the last inference in !!a{derivation} is $=$. Then the premise
is $\eq[t_1][t_2], \Gamma \Sequent \Delta, !A(t_1)$ and the conclusion
is $\eq[t_1][t_2], \Gamma \Sequent \Delta, !A(t_2)$. Consider
!!a{structure}~$\Struct M$. We need to show that the conclusion is
valid, i.e., if $\Sat{M}{\eq[t_1][t_2]}$ and $\Sat{M}{\Gamma}$, then
either $\Sat{M}{!C}$ for some $!C \in \Delta$ or $\Sat{M}{!A(t_2)}$.

By induction hypothesis, the premise is valid. This means that if
$\Sat{M}{\eq[t_1][t_2]}$ and $\Sat{M}{\Gamma}$ either (a) for some $!C
\in \Delta$, $\Sat{M}{!C}$ or (b) $\Sat{M}{!A(t_1)}$. In case (a) we
are done. Consider case (b).  Let $s$ be a variable assignment with
$s(x) = \Value{t_1}{M}$.  By \olref[syn][ass]{prop:sentence-sat-true},
$\Sat{M}{!A(t_1)}[s]$. Since $s \sim_x s$, by
\olref[syn][ext]{prop:ext-formulas}, $\Sat{M}{!A(x)}[s]$. since
$\Sat{M}{\eq[t_1][t_2]}$, we have $\Value{t_1}{M} = \Value{t_2}{M}$,
and hence $s(x) = \Value{t_2}{M}$.  By applying
\olref[syn][ext]{prop:ext-formulas} again, we also have
$\Sat{M}{!A(t_2)}[s]$.  By \olref[syn][ass]{prop:sentence-sat-true},
$\Sat{M}{!A(t_2)}$.
\end{proof}

\end{document}
