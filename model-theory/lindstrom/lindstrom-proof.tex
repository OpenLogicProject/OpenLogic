% Part: first-order-logic
% Chapter: lindstrom
% Section: lindstrom-proof

\documentclass[../../include/open-logic-section]{subfiles}

\begin{document}

\olfileid{mod}{lin}{prf}

\olsection{Lindstr\"om's Theorem}

\begin{lem}
\ollabel{lem:lindstrom}
Suppose $!E \in L(\Lang{L})$, with $\Lang{L}$ finite, and assume
also that there is an $n \in \Nat$ such that for any two
!!{structure}s $\Struct{M}$ and~$\Struct{N}$, if $\Struct{M} \equiv_n
\Struct{N}$ and $\Struct{M} \models_L !E$ then also $\Struct{N}
\models_L !E$. Then $!E$ is equivalent to a first-order
!!{sentence}, i.e., there is a first-order $!D$ such that
$\Mod(L){!E} = \Mod(L){!D}$.
\end{lem}

\begin{proof} 
Let $n$ be such that any two $n$-equivalent !!{structure}s
$\Struct{M}$ and $\Struct{N}$ agree on the value assigned to~$!E$.
Recall \olref[bas][pis]{prop:qr-finite}: there are only finitely many
first-order !!{sentence}s in a finite !!{language} that have
quantifier rank no greater than~$n$, up to logical equivalence. Now,
for each fixed !!{structure} $\Struct{M}$ let $!D_{\Struct{M}}$ be the
conjunction of all first-order !!{sentence}s~$!E$ true in~$\Struct{M}$
with $\QuantRank{!E} \le n$ (this conjunction is finite), so that
$\Struct{N} \models !D_{\Struct{M}}$ if and only if $\Struct{N}
\equiv_n \Struct{M}$. Then put $!D = \textstyle\bigvee
\Setabs{!D_{\Struct{M}}}{\Struct{M} \models_L !E}$; this disjunction
is also finite (up to logical equivalence).

The conclusion $\Mod(L){!E} = \Mod(L){!D}$ follows. In fact, if
$\Struct{N} \models_L !D$ then for some $\Struct{M} \models_L
!E$ we have $\Struct{N} \models !D_{\Struct{M}}$, whence also
$\Struct{N} \models_L !E$ (by the hypothesis of the
lemma). Conversely, if $\Struct{N} \models_L !E$ then
$!D_\Struct{N}$ is a disjunct in $!D$, and since $\Struct{N}
\models !D_\Struct{N}$, also $\Struct{N} \models_L !D$.
\end{proof}

\begin{thm}[Lindstr\"om's Theorem]
  \ollabel{thm:lindstrom} Suppose $\tuple{L, \models_L}$ has the
  Compactness and the L\"owenheim-Skolem Properties. Then
  $\tuple{L, \models_L} \le \tuple{F, \models}$ (so
  $\tuple{L, \models_L}$ is equivalent to first-order logic).
\end{thm}

\begin{proof}
By \olref{lem:lindstrom}, it suffices to show that for any $!E
\in L(\Lang{L})$, with $\Lang{L}$ finite, there is $n \in \Nat$
such that for any two !!{structure}s $\Struct{M}$ and~$\Struct{N}$: if
$\Struct{M} \equiv_n \Struct{N}$ then $\Struct{M}$ and $\Struct{N}$
agree on~$!E$. For then $!E$ is equivalent to a first-order
!!{sentence}, from which $\tuple{L, \models_L} \le \tuple{F, \models}$
follows. Since we are working in a finite, purely relational
!!{language}, by \olref[bas][pis]{thm:b-n-f} we can replace the statement
that $\Struct{M} \equiv_n \Struct{N}$ by the corresponding algebraic
statement that $I_n(\emptyset,\emptyset)$.

Given $!E$, suppose towards a contradiction that for each $n$ there
are !!{structure}s $\Struct{M}_n$ and $\Struct{N}_n$ such that
$I_n(\emptyset, \emptyset)$, but (say) $\Struct{M}_n \models_L !E$
whereas $\Struct{N}_n \not\models_L !E$. By the Isomorphism Property
we can assume that all the $\Struct{M}_n$'s interpret the constants of
the language by the same objects; furthermore, since there are only
finitely many atomic !!{sentence}s in the language, we may also assume
that they satisfy the same atomic !!{sentence}s (we can take a
subsequence of the $\Struct{M}$'s otherwise). Let $\Struct{M}$ be the
union of all the $\Struct{M}_n$'s, i.e., the unique minimal
!!{structure} having each $\Struct{M}_n$ as a substructure.  As in the
proof of \olref[lsp]{thm:abstract-p-isom}, let $\Struct{M}^*$ be the
extension of $\Struct{M}$ with !!{domain} $\Domain{M} \cup
\Domain{M}^{<\omega}$, in the expanded !!{language} comprising the
concatenation predicates $P$ and~$Q$.

Similarly, define $\Struct{N}_n$, $\Struct{N}$ and $\Struct{N}^*$. Now
let $\Struct{M}$ be the !!{structure} whose !!{domain} comprises the
!!{domain}s of $\Struct{M}^*$ and $\Struct{N}^*$ as well as the natural
numbers~$\Nat$ along with their natural ordering~$\le$, in the
!!{language} with extra predicates representing the !!{domain}s
$\Domain{M}$, $\Domain{N}$, $\Domain{M}^{<\omega}$ and
$\Domain{N}^{<\omega}$ as well as predicates coding the domains of
$\Struct{M}_n$ and $\Struct{N}_n$ in the sense that:
\begin{align*}
  \Domain{M_n} & = \Setabs{a \in \Domain{M}}{R(a, n)}; & 
  \Domain{N_n} & = \Setabs{a \in \Domain{N}}{S(a,n)}; \\
  \Domain{M}^{<\omega}_n & = \Setabs{a \in \Domain{M}^{<\omega}}{R(a,n)}; &
  \Domain{N}^{<\omega}_n & = \Setabs{a \in \Domain{N}^{<\omega}}{S(a,n)}. 
\end{align*}
The !!{structure} $\Struct{M}$ also has a ternary relation $J$ such
that $J(n, \mathbf{a}, \mathbf{b})$ holds if and only if
$I_n(\mathbf{a}, \mathbf{b})$.

Now there is a !!{sentence}~$!D$ in the !!{language}~$\Lang{L}$ augmented
by $R$, $S$, $J$, etc., saying that $\le$ is a discrete linear ordering
with first but no last element and such that $\Struct{M}_n \models
!E$, $\Struct{N}_n \not\models !E$, and for each $n$ in the
ordering, $J(n, \mathbf{a}, \mathbf{b})$ holds if and only if
$I_n(\mathbf{a}, \mathbf{b})$.

Using the Compactness Property, we can find a model $\Struct{M}^*$ of
$!D$ in which the ordering contains a non-standard element~$n^*$. In
particular then $\Struct{M^*}$ will contain sub!!{structure}s
$\Struct{M_{n^*}}$ and $\Struct{N_{n^*}}$ such that $\Struct{M_{n^*}}
\models_L !E$ and $\Struct{N_{n^*}} \not\models_L !E$. But now we can
define a set $\mathcal{I}$ of pairs of $k$-tuples from
$\Domain{M_{n^*}}$ and $\Domain{N_{n^*}}$ by putting
$\tuple{\mathbf{a}, \mathbf{b}} \in \mathcal{I}$ if and only if
$J(n^*-k, \mathbf{a}, \mathbf{b})$, where $k$ is the length of
$\mathbf{a}$ and $\mathbf{b}$. Since $n^*$ is non-standard, for each
standard $k$ we have that $n^* - k >0$, and the set $\mathcal{I}$
witnesses the fact that $\Struct{M_{n^*}} \simeq_p
\Struct{N_{n^*}}$. But by \olref[lsp]{thm:abstract-p-isom},
$\Struct{M_{n^*}}$ is $L$-equivalent to $\Struct{N_{n^*}}$, a
contradiction.
\end{proof}

\end{document}
