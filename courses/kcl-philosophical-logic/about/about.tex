\chapter*{About this handbook}

\paragraph{Content}

This is a draft of lectures notes for a course on philosophical logic.

\textbf{Warning}. Work in progress. Some parts are incomplete and you will encounter occasional mistakes and typos.

The course covers:

\begin{itemize}
\item \emph{Classical logic}, consisting of \emph{propositional} and 
\emph{predicate} (or \emph{first-order}) logic.
\item \emph{Modal logic}, an extension of classical logic. We only cover its 
propositional part.
\item \emph{Non-classical logics}, namely \emph{many-valued logics} (propositional) and \emph{free logic} (predicate).
\end{itemize}

\paragraph{Structure}

Each chapter roughly corresponds to a lecture. Each is divided in
four parts.
\begin{itemize}
\item \emph{Overview}. Each chapter starts with an overview of the key points
to remember. Review it to see whether you can recall the chapter's
content.
\item \emph{Content}. The chapter sections cover the points you should know. 
\item \emph{Extras}. These are points that go beyond what you have to know.
These contains further topics of interest that go beyond what your
are assessed on. Read them if you want to learn more and think.
\item \emph{Exercises}. Exercises to practice the chapter material and prepare
the exam. Most have a correction in the appendices.
\end{itemize}

The best way to learn logic is to do exercises. Do not spend too much
time reading the books or notes again and again. Read, try to do the
exercises, and go back to the texts whenever you are stuck in the
exercises.

\paragraph{Textbooks}

The course is supported by two textbooks:
\begin{itemize}
\item \citet{Sider2010-SIDLFP} \emph{Logic for Philosophy}, Oxford University
Press\emph{. }The course covers chapters 1 to 6.
\item \citet{PriestINCL} \emph{An Introduction to Non-Classical Logic}, 2nd 
edition, Cambridge University Press, 
\end{itemize}

\paragraph{About the Open Logic project}

This hanbook uses material (code and some text) from the Open Logic project. 

The \textit{Open Logic Text} is an open-source, collaborative textbook
of formal meta-logic and formal methods, starting at an intermediate level
(i.e., after an introductory formal logic course). Though aimed at a
non-mathematical audience (in particular, students of philosophy and
computer science), it is rigorous.

The project operates in the spirit of open source. Not only is the
text freely available, we provide the LaTeX source under the
Creative Commons Attribution license, which gives anyone the right to
download, use, modify, re-arrange, convert, and re-distribute our
work, as long as they give appropriate credit.
Please see the Open Logic Project website at
\href{http://openlogicproject.org/}{openlogicproject.org} for
additional information.

