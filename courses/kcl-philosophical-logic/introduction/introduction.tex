% Part: none

\documentclass[../../../include/open-logic-part]{subfiles}

\begin{document}

\olpart{pre}{Introduction}

\olchapter{pre}{int}{Introduction}

\begin{overview}

\section*{Overview}

\begin{itemize} 
	\item Logical validity: logical consequence
(arguments) and logical truth (propositions). 
	\item The model-theoretic conception of validity. 
	\item Formal logic.
\end{itemize}

\end{overview}

\section{Philosophical logic}

\section{Logical validity}

Logic is concerned with how certain claims \emph{follow} or \emph{are
entailed} by others. More precisely, it is concerned with
\emph{patterns} that ensure that some claims are true if others are.
This is called \emph{logical consequence}. It is also concerned with
patterns that ensure that some claims are true, no matter what. This
is called \emph{logical truth}. Logical truth can be considered a
limiting case of logical consequence: claims that follow from any
claims whatsoever, and even `follow from' no claim at all. Together
logical truth and logical consequence are called \emph{logical
validity}.

A slogan often used is: \begin{quote} ``Logical truth is \emph{truth
in virtue of form}. Logical consequence is \emph{truth-preservation in
virtue of form. }''  \end{quote}

The idea behind the slogan is that from the patterns alone we can say
that certain claims must be true (logical truths) or that certain
arguments are such that if the premises are true the conclusion is
true too (logical consequence). This raises some questions: what's the
``form'' or ``pattern''? What does it mean that we can ``see''
something ``from the form alone''? There are various conceptions that
try to say something more precise. We'll see three of them below.  

The dominant one can be characterized thus: 

\begin{description} \item
[{Model-theoretic~conception~of~validity}] An argument is valid just
if: \emph{no matter what its non-logical terms mean, and no matter how
things are}, its conclusion sentence is true or some of his premises
aren't true. \\ A sentence is valid just if: \emph{no matter what its
non-logical terms mean, and no matter how things are}, the sentence is
true.  \end{description}

\section{Three conceptions of validity}

To understand the model-theoretic conception we can contrast it with
two alternatives, the \emph{modal conception} and what I'll call the
\emph{re-interpretation} conception. The modal conception is the idea
that logical validity is truth(-preservation) \emph{no matter how
things are}. We call it ``modal'' because ``modal'' is the
philosophers' term for matters of necessity and possibility, and on
this view, what makes a proposition logically valid is that it
couldn't possibly be false.

\begin{description} \item [{Modal-conception~of~validity}] An argument
is valid just if: \emph{and no matter how things are}, its conclusion
sentence is true or some of his premises aren't true. \\ A sentence is
valid just if: \emph{no matter how things are}, the sentence is true. 
\end{description}

The main objection to this conception is that it qualifies too many
things as `logical' validities. 

\begin{itemize} 
	\item If Greg is a bachelor then Greg is not married.
	\item Jay-Z is Shaw Carter. 
	\item The glass contains water. Therefore, the glass contains 
	hydrogen atoms. \end{itemize}

Another conception is the re-interpretation one.\footnote{According to
some this is Tarski's conception. Others think Tarski adopted the
model-theoretic one.} This is the idea that logical validity is
truth(-preservation) \emph{no matter what the non-logical terms mean}.
For instance, the sentence ``if there are crabs, there are crabs''
would express something true no matter what ``crabs'' means: whether
it refer to crabs, trees, planes or parties. 

\begin{description} \item [{Re-interpretation~conception~of~validity}]
An argument is valid just if: \emph{no matter what its non-logical
terms mean}, its conclusion sentence is true or some of his premises
aren't true. \\ A sentence is valid just if: \emph{no matter what its
non-logical terms mean, and no matter how things are}, the sentence is
true.  \end{description}

One issue with that conception is to say which terms are ``logical''
ones and which aren't --- but set it aside. The main problem for the
conception is that sentences expressing \emph{how many things} there
are can be expressed in purely logical terms. For instance, we can
have a sentence using only logical terms that states: 

\begin{itemize} \item There are at least five things. \end{itemize}

As it happens, the sentence is true. Since it doesn't contain any
non-logical terms, it is \emph{a fortiori} true no matter what its
non-logical terms mean. So, on the re-interpretation conception, it is
a logical truth. And so is any true claim about how many things there
are. But most reject the idea that these are \emph{logical} truths.
One influential idea here is that logical truths are \emph{a priori}
--- they are things that can be figured out by reasoning alone ---
while truths about how many things there are --- truths that can only
be discovered by doing some observation, not reasoning alone. 

On the modal conception logical validity is about preserving truth no
matter how the facts are. On the re-interepretation it is about
preserving truth no matter what the meaning of 'non-logical' (or
`substantial') terms are. The model-theoretic conception combines
both: it says that logical validity is a matter of preserving truth no
matter how the facts are and no matter what the meaning of non-logical
terms are. 

The model-theoretic conception doesn't count ``If Greg is a bachelor
then Greg is unmarried'' as a logical truth because it wouldn't be
true no matter what the non-logical terms ``bachelor'' and
``unmarried'' mean. If ``bachelor'' meant ``dog'' and ``is unmarried''
meant ``has wings'' then the sentence would not be true. The
model-theoretic conception doesn't count ``There are at least five
things'' as a logical truth because if the facts were different--- if
there were only three things, for instance, then it would not be true.

\section{What is a ``logical'' term?}

The standard, model-theoretic conception of validity invokes the
notion of ``non-logical terms''. (So does the re-interpretation
conception.) It opposes ``logical terms'' (like ``or'', ``and'', ``if
\ldots{} then'', ``something'', \ldots ) \emph{vs.}~non-logical ones
(``hippopotamus'', ``atom'', ``song'', \ldots{} ). ``Logical terms''
are also called ``logical constants'' \textemdash{} preferably because
when ask whether a sentence is valid we hold ``contstant'' the meaning
of those terms. 

Thus what the standard conception counts as logically valid depends on
what we count as ``logical terms''. What sets these terms apart? Can
we give a principled definition of what's a ``logical term''? There
are deep and live philosophical issues here. 

\section{Formal logic}

\paragraph{Formalization}

We could study logical validity in English, or in any other ``natural
language'' (Hindi, Chinese, etc.). That is how logic was studied
from Aristotle up until the XIXth century. Unfortunately natural languages
are highly complex, ambiguous and leave many things implicit.

Contemporary logic proceeds indirectly instead. Rather than using
natural language, we create:
\begin{itemize}
\item \textbf{a formal language}: an artificial language whose rules are
stated in a mathematically precise way
\begin{itemize}
\item \emph{syntax} or \emph{grammar}: specification of what sentences are,
\item \emph{semantics}: specification of what each sentence mean.
\end{itemize}
\item \textbf{a formal proof system}: an artificial procedure to derive
some sentences from others, stated in a mathematically precise way.
\end{itemize}
That is, we create a ``toy'' language, and a ``toy'' reasoning
procedure. These allow us to study logical validity in a very simple
setting. That is like physicists studying the movement of planets
by treating them as points or perfect balls: studying a simplified,
``toy'' universe allows them to better understand the real one. 

So paradoxically, while many people are 'scared' by formal logic because
the use of formal logic make it look complicated, it actually uses
extremely simple languages\textemdash languages so simple that even
a dumb machine like a computer can understand it. In fact, the main
difficulty you face when learning logic is \emph{not to think too
fast or too much}. To think logically, you have to get used to think
in a 'dumb' way: a step-by-step, fully explicit, simple-minded thinking
of formal languages. 

Why would anyone do this\textemdash adopt simpleminded, stripped-down
artificial languages? First, the languages are a powerful tool. They
allow you to clarify and simplify arguments. They allow you to process
complex ideas that you would struggle with otherwise (e.g. infinities,
complex 'quantification', 'binding' ambiguities - see predicate logic).
Second, the point isn't so much to \emph{use} the languages, but to
\emph{think about }them. The point is to study how logical validity
works in these languages, so as to understand logical validity in
general. This is a way, for instance, to answer questions about what
follows - or doesn't follow - from certain theories in mathematics,
physics or philosophy. For that purposes simple, fully explicit languages
are perfect. 

Summing up, a contemporary logic has three components:
\begin{quote}
1. Language: a grammar.\\
2. Language: a semantics.\\
3. A proof system.
\end{quote}
For each of the logics we study: propositional, predicate and modal,
we'll have the three components. 

\paragraph{Object language and meta-language}

Once we create an artificial language we have two languages in play.
We call them thus:
\begin{enumerate}
\item \emph{The object language}: the language we study. That is, the formal
language of propositional logic, predicate logic, or modal logic.
\item \emph{The meta-language}: the language we use to study the object
language. This is English supplemented with some technical terms and
some symbols.
\end{enumerate}
Compare with studying a foreign language. If you're an English linguist
studying ancient Sumerian, your object language is ancient Sumerian
(that's what you study) but your meta-language is English (that's
the language that you us to talk about ancient Sumerian).

In you meta-language we'll sometimes use symbols for convenience.
For instance, we'll often say that some sentence in our formal language
``can be proved on the basis'' of some other sentences. For brievity
we'll write that with the symbol ``$\Proves$'': 
\begin{quote}
$P\land P,P\lif Q\Proves[PL] Q$ means ``In the system PL there
is a proof of $P$ that uses the premises $P\land P$ and $P\lif Q$''.
\end{quote}
Similarly we write ``$\Entails$'' to abbreviate ``is a logical
consequence of'': $P\land P\Entails[PL]P$ means that $P$ is a logical
consequence of $P\land P$ in the system PL. 

One of the most common metalanguage symbols are $!A$, $!B$,
which we use to talk about unspecific formulas of our language. 

\paragraph{Logic vs. logics}

Logic is a discipline, like philosophy or mathematics. 

\emph{A }logic is a theory or system of that discipline. For instance,
propositional logic, modal logic, etc. Some of these systems deal
with different objects and are thus compatible with each other: propositional
logic deals with logical validities that arises for ``truth-functional''
connectives like ``and'' and ``or'', while predicate logic deals
with logical validity that arises from ``quantifiers'' like ``all''
and ``some''. They are compatible with each other\textemdash in
fact, predicate logic is an \emph{extension }of propositional logic. 

Some of these systems are competitors of each other: for instance
``classical'' logic (which includes propositional and predicate
logic studied here) and ``intuitionistic'' logic (which rejects
some of the principles of classical propositional and predicate logic)
are rival systems. 

Metalogic is the subpart of logic that contains theories \emph{about
}logics. For instance, when you show that a certain system of logic
(say, propositional logic) doesn't 'prove' contradictions, you're
doing metalogic. 


\OLEndChapterHook

\end{document}
