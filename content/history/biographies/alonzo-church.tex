% Part: history
% Chapter: biographies 
% Section: alonzo-church

\documentclass[../../../include/open-logic-section]{subfiles}

\begin{document}

\olfileid{his}{bio}{chu}

\olsection{Alonzo Church}

\begin{figure}[h!]
\centering 
\includegraphics[scale=1]{alonzo-church.jpg} 
\caption{Alonzo
Church. Photo Credit: Wikimedia.} 
\end{figure}

Alonzo Church was born in Washington, DC on June 14, 1903. He is perhaps
best known for the Church-Turing thesis and Church's theorem, although he
published also in set theory and philosophy.
 
In early childhood, an air gun incident left Church blind in one eye
\citep[2]{EndertonND}. He finished prepratory school in Connecticut in
1920 and began his university education at Princeton that same year. He
completed his doctoral studies in 1927. After a couple years abroad, Church
began his academic career back at Princeton. Throughout his career he was
known as a polite and careful man. His blackboard writing was immaculate,
and he would preserve important papers by carefully covering them in Duco
cement \citep[4]{EndertonND}. Outside of his academic pursuits, he
enjoyed reading science fiction magazines and was not afraid to write to
the editors if he spotted any inaccuracies in the writing
\citep[5]{EndertonND}.

Church's achievments were great. He developed a theory of effective
calculability, the lambda calculus, independently of Alan Turing's
development of the Turing machine. The two definitions of computabiliy are
equivalent, and give rise to what is now known as the \emph{Church-Turing
Thesis}, that a function of the natural numbers is effectively computable
if and only if it is computable via Turing machine (or lambda calculus). He
also proved what is now known as \emph{Church's Theorem}: The decision
problem for the validity of first-order formulas is unsolvable.

Church continuted his work into old age. In 1967 he left Princeton for UCLA
in order to pursue his work with the \emph{Journal of Symbolic Logic}. He
was professor there until his retirement in 1990. Church passed away on
August 1, 1995 at the age of 92.


\begin{reading} 
\begin{itemize} 
\item Church was interviewed in 1984 about
the Princeton mathematics community in the 1930s. To read the interview,
see \citet{Aspray1984}.

\item To read Church's writings on the lambda calculus, and the
Entscheidungsproblem (Church's Thesis), see \citet{Church1936,Church1936a}.

\item Church wrote a series of book reviews of the \emph{Journal of
Symbolic Logic} from 1936 until 1979. They are all archived on John
MacFarlane's website, see \citet{MacFarlane2015}.

\item For a brief biography of Church, see \citet{EndertonND}.

\end{itemize} 
\end{reading} 
\end{document}
