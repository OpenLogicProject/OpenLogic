% Part: history 
% Chapter: biographies 
% Section: rozsa-Peter
\documentclass[../../../include/open-logic-section]{subfiles}

\begin{document}

\olfileid{his}{bio}{pet} 

\olsection{R\'{o}zsa P\'{e}ter}

\olphoto{rozsa-Peter}{Rozsa Peter} 
R\'{o}zsa P\'{e}ter was born in Budapest,
Hungary on February 17, 1905. She is best known for her work on recursive
functions, and for creating the field of recursion theory. Despite the
harsh political climate during her youth, P\'{e}ter was able to attend the affluent
Maria Terezia Girls' School in Budapest, graduating in 1922 before heading
to university.

P\'{e}ter attended P\'{a}zm\'{a}ny P\'{e}ter University (later renamed
 Lor\'{a}nd E\"{o}tv\"{o}s University) in Budapest. She began studying 
 chemistry (at the insistence of her father), but later swtiched to mathematics. 
 She graduated in 1927. Although she had the credentials to teach mathematics,
  the economic situation was dire as the Great Depression affected 
  the world economy.  During this time, P\'{e}ter
took odd jobs as a tutor and private teacher of mathematics. She returned
to university, taking up studies at the University of Budapest, and acheiving
her Ph.D. in 1935. During this time, P\'{e}ter wrote her first papers on
recursion theory (inspired by David Hilbert's foundational program), which
would end up being essentiall in the creation of recursion theory as its own
field of mathematical research. In 1939 she became an editor for the
\emph{Journal of Symbolic Logic}.

Despite the importance and influence of her work, P\'{e}ter did not gain a
full-time teaching position until 1945. During Nazi occupation of Hungary
during World War II, P\'{e}ter was not allowed to teach due to anti-Semitic
laws. In 1944 the government decided that a Jewish ghetto should be
erected in Budapest; it was cut off from the rest of the city and attended
by armed guards. P\'{e}ter was forced to live in the ghetto until 1945 when it
was liberated. She went on to teach at the Budapest Teachers Training
College, and E\"{o}tv\"{o}s Lor\'{a}nd University. She was the first female hungarian
mathematician to become an Academic Doctor of Mathematics.

P\'{e}ter was known as a passionate teacher of mathematics, who preferred to
explore the nature and beauty of mathematical problems with her students
rather than teach. As a result, she was affectionately called "Aunt Rosa"
by her students. P\'{e}ter passed away in 1977 at the age of 71.

\begin{reading} 
For more biographical reading, see \citet{Oconnor2014} and \citet{Andrasfai1986}.
See \citet{Tamassy1994} for a brief interview with R\'{o}zsa. For a fun read
about mathematics, see P\'{e}ter's book \emph{Playing With Infinity} 
\citep{Peter2010}.
\end{reading}

\end{document}