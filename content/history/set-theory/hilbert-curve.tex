\documentclass[../../../include/open-logic-section]{subfiles}

\begin{document}

\olfileid{his}{set}{hilbertcurve}

\olsection{Hilbert's Space-filling Curves}

In chapter \olref[pathology]{sec}, we mentioned that Cantor's proof that
a line and a square have exactly the same number of points
(\olref[his][set][cantorplane]{thm:cantorplane}) prompted Peano to
ask whether there might be a space-filling \emph{curve}. He obtained a
positive answer in 1890. In this section, we explain (in a
hand-wavy way) how to construct Hilbert's space-filling curve (with a
tiny tweak).\footnote{For a more rigorous explanation, see
\cite{Rose2010}. The tweak amounts to the inclusion of the red
parts of the curves below. This makes it slightly easier to check that
the curve is continuous.}

We must define a function, $h$, as the limit of a sequence of functions $h_1$, $h_2$, $h_3$, \dots\@ We first describe the construction. Then we show it is space-filling. Then we show it is a curve. 

We will take $h$'s range to be the unit square, $\unitsquare$. Here is our first approximation to $h$, i.e., $h_1$:
\begin{center}
	\begin{tikzpicture}
	\draw[gray] (0pt, 0pt) rectangle (80pt, 80pt);
	\draw[step = 40pt, gray, very thin] (0pt, 0pt) grid (80pt, 80pt);
	\draw[oldiagcolorC, thick] (20pt, 0)--(20pt, 20pt);
	\draw[oldiagcolorC, thick] (60pt, 0)--(60pt, 20pt);		
	\begin{scope}[xshift=20pt, yshift=20pt]
	\draw [black, thick, l-system={Hilbert curve, step=40pt, angle=90, axiom=L, order=1}]  lindenmayer system;
	\end{scope}
	\end{tikzpicture}
\end{center}
To keep track of things, we have imposed a $2 \times 2$ grid on the square. We can think of the curve starting in the bottom left quarter, moving to the top left, then to the top right, then finally to the bottom right. Here is the second stage in the construction, i.e., $h_2$:
\begin{center}
	\begin{tikzpicture}
	\draw[gray] (0pt, 0pt) rectangle (80pt, 80pt);
	\draw[step = 20pt, gray, very thin] (0pt, 0pt) grid (80pt, 80pt);
	\draw[oldiagcolorC, thick] (0pt, 10pt)--(10pt, 10pt);
	\draw[oldiagcolorC, thick] (70pt, 10pt)--(80pt, 10pt);
	\draw[thick, oldiagcolorE] (10pt, 30pt)--(10pt, 50pt); 
	\draw[thick, oldiagcolorE] (30pt, 50pt)--(50pt, 50pt); 
	\draw[thick, oldiagcolorE] (70pt, 50pt)--(70pt, 30pt); \begin{scope}[xshift=10pt, yshift=30pt]
	\draw [thick, rotate=270,black, l-system={Hilbert curve, step=20pt, angle=90, axiom=L, order=1}]  lindenmayer system;
	\end{scope}
	\begin{scope}[xshift=10pt, yshift=50pt]
	\draw [thick, black, l-system={Hilbert curve, step=20pt, angle=90, axiom=L, order=1}]  lindenmayer system;
	\end{scope}
	\begin{scope}[xshift=50pt, yshift=50pt]
	\draw [thick, black, l-system={Hilbert curve, step=20pt, angle=90, axiom=L, order=1}]  lindenmayer system;
	\end{scope}
	\begin{scope}[xshift=70pt, yshift=10pt]
	\draw [thick, rotate=90,black, l-system={Hilbert curve, step=20pt, angle=90, axiom=L, order=1}]  lindenmayer system;
	\end{scope}
	\end{tikzpicture}
\end{center}
The different colours will help explain how $h_2$ was constructed. We first place scaled-down copies of the non-!!{colorC} bit of $h_1$ into the bottom left, top left, top right, and bottom right of our square (drawn in black). We then connect these four figures (with !!{colorE} lines). Finally, we connect our figure to the boundary of the square (with !!{colorC} lines).

Now to $h_3$. Just as $h_2$ was made from four connected, scaled-down copies of the non-red bit of $h_1$, so $h_3$ is made up of four scaled-down copies of the non-red bit of $h_2$ (drawn in black), which are then joined together (with !!{colorE} lines) and finally connected to the boundary of the square (with !!{colorC} lines).
\begin{center}
	\begin{tikzpicture}
	\draw[gray] (0pt, 0pt) rectangle (80pt, 80pt);
	\draw[step = 10pt, gray, very thin] (0pt, 0pt) grid (80pt, 80pt);
	\draw[thick, oldiagcolorC](5pt, 0pt)--(5pt, 5pt); 
	\draw[thick, oldiagcolorC] (75pt, 0pt)--(75pt, 5pt); 
	\draw[thick, oldiagcolorE] (75pt, 45pt)--(75pt, 35pt);
	\draw[thick, oldiagcolorE] (5pt, 35pt)--(5pt, 45pt); 
	\draw[thick, oldiagcolorE] (35pt, 45pt)--(45pt, 45pt); 
	\draw[thick, oldiagcolorE] (75pt, 45pt)--(75pt, 35pt); \begin{scope}[xshift=5pt, yshift=35pt]
	\draw [rotate=270, thick, black, l-system={Hilbert curve, step=10pt, angle=90, axiom=L, order=2}]  lindenmayer system;
	\end{scope}
	\begin{scope}[xshift=5pt, yshift=45pt]
	\draw [thick, black, l-system={Hilbert curve, step=10pt, angle=90, axiom=L, order=2}]  lindenmayer system;
	\end{scope}
	\begin{scope}[xshift=45pt, yshift=45pt]
	\draw [thick, black, l-system={Hilbert curve, step=10pt, angle=90, axiom=L, order=2}]  lindenmayer system;
	\end{scope}
	\begin{scope}[xshift=75pt, yshift=5pt]
	\draw [thick, rotate=90,black, l-system={Hilbert curve, step=10pt, angle=90, axiom=L, order=2}]  lindenmayer system;
	\end{scope}
	\end{tikzpicture}
\end{center}
And now we see the general pattern for defining $h_{n+1}$ from $h_n$.
At last we define the curve $h$ \emph{itself} by considering the
point-by-point limit of these successive functions $h_1$, $h_2$,
\dots\@ That is, for each $x \in \unitsquare$:
\begin{align*}
	h(x) &= \lim_{n \rightarrow \infty} h_n(x)
\end{align*} 
We now show that this curve fills space. When we draw the curve $h_n$,
we impose a $2^n \times 2^n$ grid onto $\unitsquare$. By Pythagoras's
Theorem, the diagonal of each grid-location is of length:
\[
\sqrt{\left(\nicefrac{1}{2^{n}}\right)^2+\left(\nicefrac{1}{2^{n}}\right)^2} = 2^{(\frac{1}{2}-n)}
\]
and evidently $h_n$ passes through every grid-location. So each point
in $\unitsquare$ is \emph{at most} $2^{(\frac{1}{2}-n)}$ distance away
from some point on $h_n$. Now, $h$ is defined as the limit of the
functions $h_1$, $h_2$, $h_3$, \dots\@ So the maximum distance of any
point from $h$ is given by:
\[
\lim_{n \rightarrow \infty} 2^{(\frac{1}{2}-n)} = 0.
\]
That is: every point in $\unitsquare$ is $0$ distance from~$h$. In
other words, every point of $\unitsquare$ lies \emph{on} the curve. So $h$
fills space!{}

It remains to show that $h$ is, indeed, a \emph{curve}. To show this,
we must define the notion. The modern definition builds on one given
by Jordan in 1887 (i.e., only a few years before the first
space-filling curve was provided): 

\begin{defn}
A curve is a continuous map from $\unitline$ to $\Real^2$. 
\end{defn}

This is fairly intuitive: a curve is, intuitively, a ``smooth'' map
which takes a canonical line onto the plane $\Real^2$. Our function,
$h$, is indeed a map from $\unitline$ to $\Real^2$. So, we just need
to show that $h$ is continuous. We defined continuity in
\olref[limits]{sec} using $\epsilon$/$\delta$ notation. In the
vernacular, we want to establish the following: \emph{If you specify a
point $p$ in $\unitsquare$, together with any desired level of
precision $\epsilon$, we can find an open section of $\unitline$ such
that, given any $x$ in that open section, $h(x)$ is within $\epsilon$
of $p$.}

So: assume that you have specified $p$ and $\epsilon$. This is, in
effect, to draw a circle with centre $p$ and radius $\epsilon$ on
$\unitsquare$. (The circle might spill off the edge of $\unitsquare$,
but that doesn't matter.) Now, recall that, when describing the
function $h_n$, we drew a $2^n \times 2^n$ grid upon $\unitsquare$. It
is obvious that, no matter how small $\epsilon$ is, there is some $n$
such that some individual grid-location of the $2^n \times 2^n$ grid
on $\unitsquare$ lies wholly within the circle with centre $p$ and
radius $\epsilon$. 

So, take that $n$, and let $I$ be the largest open part of $\unitline$
which $h_n$ maps wholly into the relevant grid location. (It is clear
that $(a,b)$ exists, since we already noted that $h_n$ passes through
every grid-location in the $2^n\times 2^n$ grid.) It now suffices to
show to show that, whenever $x \in I$ the point $h(x)$ lies in that
same grid-location. And to do \emph{this}, it suffices to show that
$h_m(x)$ lies in that same grid location, for any $m > n$. But this is
obvious. If we consider what happens with $h_m$ for $m > n$, we see
that that exactly the ``same part'' of the unit interval is mapped
into the same grid-location; we just map it into that region in an
increasingly stretched-out, wiggly fashion. 

\end{document}