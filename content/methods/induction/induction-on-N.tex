% Part: methods
% Chapter: induction
% Section: induction-on-N

\documentclass[../../../include/open-logic-section]{subfiles}

\begin{document}

\olfileid{mth}{ind}{inN}

\olsection{Induction on~$\Nat$}

In its simplest form, induction is a technique used to prove results
for all natural numbers. It uses the fact that by starting from $0$
and repeatedly adding~$1$ we eventually reach every natural number.  So
to prove that something is true for every number, we can (1) establish
that it is true for $0$ and (2) show that whenever a number has it,
the next number has it too.  If we abbreviate ``number~$n$ has
property $P$'' by $P(n)$, then a proof by induction that $P(n)$ for
all $n \in \Nat$ consists of:
\begin{enumerate}
\item a proof of $P(0)$, and
\item a proof that, for any~$n$, if $P(n)$ then $P(n+1)$.
\end{enumerate}
To make this crystal clear, suppose we have both (1) and (2).  Then
(1) tells us that $P(0)$ is true.  If we also have (2), we know in
particular that if $P(0)$ then $P(0+1)$, i.e., $P(1)$. (This follows
from the general statement ``for any~$n$, if $P(n)$ then $P(n+1)$'' by
putting~$0$ for~$n$.  So by modus ponens, we have that $P(1)$.  From
(2) again, now taking $1$ for~$n$, we have: if $P(1)$ then $P(2)$.
Since we've just established $P(1)$, by modus ponens, we
have~$P(2)$. And so on.  For any number $k$, after doing this $k$
steps, we eventually arrive at~$P(k)$.  So (1) and (2) together
established~$P(k)$ for any $k \in \Nat$.

Let's look at an example.  Suppose we want to find out how many
different sums we can throw with $n$ dice.  Although it might seem
silly, let's start with $0$ dice.  If you have no dice there's only one
possible sum you can ``throw'': no dots at all, which sums to~$0$. So
the number of different possible throws is~$1$. If you have only one
die, i.e., $n=1$, there are six possible values, $1$ through~$6$. With
two dice, we can throw any sum from $2$ through $12$, that's $11$
possibilities.  With three dice, we can throw any number from $3$ to
$18$, i.e., $16$ different possibilities.  $1$, $6$, $11$, $16$: looks
like a pattern: maybe the answer is $5n+1$?  Of course, $5n+1$ is the
maximum possible, because there are only $5n+1$ numbers between $n$,
the lowest value you can throw with $n$ dice (all $1$'s) and $6n$, the
highest you can throw (all $6$'s).

\begin{thm}
  With $n$ dice one can throw all $5n+1$ possible values between $n$
  and $6n$.
\end{thm}

\begin{proof}
To prove that this holds for any number of dice, we use induction: we
prove, (1) that the number of different throws with $0$ dice
is~$5\cdot0 + 1 = 1$, and (2) that if we can throw all $5n+1$ values
between $n$ and $6n$ with $n$ dice, then with $n+1$ dice we can
throw all $5(n+1)+1 = 5n+6$ possible values between $n+1$ and
$6(n+1)$.  We've already established (1): If you have no dice,
there's only one possible outcome: you throw all $0$ dice, and the
number of dots visible is~$0$.

Now let's see how many possible total values you can throw with $n+1$
dice: we're going to show that it's all $5n+6$ different sums between
$n+1$ and $6n+6$, assuming that you can throw all $5n+1$ different
sums between $n$ and $6n$ with $n$~dice. This assumption is called the
induction hypothesis.  So: assume that you can throw all $5n+1$
different sums between $n$ and $6n$ with $n$~dice

Suppose you have $n$ dice.  Each possible throw sums to some number
between $n$ (all $n$ dice show $1$'s) and $6n$ (all $6$'s).  Now throw
another die.  You get another number between $1$ and $6$.  If your
total is less than $6n$ you could have thrown the same total with your
original $n$ dice, by induction hypothesis.  Only the possible totals
$6n+1$ through $6n+6$ are totals you couldn't have thrown with just
$n$ dice---and these are new possible totals, which you'd get, e.g.,
by throwing $n$ $6$'s plus whatever your new die shows. So there are
$6$ new possible totals.  You can still throw any of the old totals,
except one. With $n+1$ dice you can't roll just~$n$. But any other
value you can throw with $n$ dice you can also throw with $n+1$ dice:
just imagine one of your $n$ dice showed one eye fewer (and at least
one of them must show at least a $2$) and your $(n+1)$st die shows
a~$1$. With an additional die we lose only one possible
total ($n$) and gain $6$ new possible totals ($6n+1$ through $6n+6$).
So the number of possible totals with $n+1$ dice is $5n+1$, minus~$1$,
plus~$6$, i.e., $5n+6$.  So we have proved that, assuming the number
of totals we can throw with $n$ dice is $5n+1$, the number of totals
we can throw with $n+1$ dice is $5n+6 = 5(n+1) + 1$.
\end{proof}

Very often we use induction when we want to prove something about a
series of objects (numbers, sets, etc.) that is itself defined
``inductively,'' i.e., by defining the $(n+1)$-st object in terms of
the $n$-th.  For instance, we can define the sum~$s_n$ of the natural
numbers up to~$n$ by
\begin{align*}
  s_0 = 0\\
  s_{n+1} & = s_n + (n+1)
\end{align*}
This definition gives:
\begin{align*}
  s_0 & = 0,\\
  s_1 & = s_0 + 1 & = 1,\\
  s_2 & = s_1 + 2 & = 1 + 2 = 3\\
  s_3 & = s_s + 3 & = 1 + 2 + 3 = 6, \text{ etc.}
\end{align*}
Now we can prove, by induction, that $s_n = n(n+1)/2$.

\begin{prop}
  $s_n = n(n+1)/2$.
\end{prop}

\begin{proof}
  We have to prove (1) that $s_0 = 0\cdot(0 + 1)/2$ and (2) if $s_n =
  n(n+1)/2$ then $s_{n+1} = (n+1)(n+2)/2$.  (1) is obvious. To prove
  (2), we assume the inductive hypothesis: $s_n = n(n+1)/2$. Using it,
  we have to show that $s_{n+1} = (n+1)(n+2)/2$.

  What is $s_{n+1}$?  By the definition, $s_{n+1} = s_n + (n+1)$.  By
  inductive hypothesis, $s_n = n(n+1)/2$. We can substitute this into
  the previous equation, and then just need a bit of arithmetic of
  fractions:
  \begin{align*}
    s_{n+1} & = \frac{n(n+1)}{2} + (n+1) = {}\\
    & = \frac{n(n+1)}{2} + \frac{2(n+1)}{2} = {}\\
    & = \frac{n(n+1) + 2(n+1)}{2} = {}\\
    & = \frac{(n+2)(n+1)}{2}.
  \end{align*}
\end{proof}

The important lesson here is that if you're proving something about
some inductively defined sequence $a_n$, induction is the obvious way
to go. And even if it isn't (as in the case of the possibilities of
dice throws), you can use induction if you can somehow relate the case
for~$n+1$ to the case for~$n$.

\end{document}
