% Part: methods
% Chapter: induction
% Section: structural-induction

\documentclass[../../../include/open-logic-section]{subfiles}

\begin{document}

\olfileid{mth}{ind}{sti}

\olsection{Structural Induction}

So far we have used induction to establish results about all natural
numbers. But a corresponding principle can be used directly to prove
results about all !!{element}s of an inductively defined set.  This
often called \emph{structural} induction, because it depends on the
structure of the inductively defined objects.

Generally, an inductive definition is given by (a) a list of
``initial'' !!{element}s of the set and (b) a list of operations which
produce new !!{element}s of the set from old ones. In the case of
parexpressions, for instance, the initial object is~$\emptyseq$ and
the operations are
\begin{align*}
  o_1(p) = & (p) \\
  o_2(q, q') = & qq'
\end{align*}
You can even think of the natural numbers~$\Nat$ themselves as being
given be an inductive definition: the initial object is~$0$, and the
operation is the successor function~$x + 1$.

In order to prove something about all elements of an inductively
defined set, i.e., that every !!{element} of the set has a
property~$P$, we must:
\begin{enumerate}
\item Prove that the initial objects have~$P$
\item Prove that for each operation~$o$, if the arguments have~$P$,
  so does the result.
\end{enumerate}
For instance, in order to prove something about all parexpressions, we
would prove that it is true about $\emptyseq$, that it is true of
$(p)$ provided it is true of $p$, and that it is true about $qq'$
provided it is true of $q$ and $q'$ individually.

\begin{prop}
  The number of $($ equals the number of $)$ in any parexpression $p$.
\end{prop}

\begin{proof}
We use structural induction.  Parexpressions are inductively defined,
with initial object $\emptyseq$ and the operations $o_1$ and $o_2$.
\begin{enumerate}
\item The claim is true for $\emptyset$, since the number of $($ in
  $\emptyseq = 0$ and the number of $)$ in $\emptyseq$ also $= 0$.
\item Suppose the number of $($ in $p$ equals the number of $)$ in
  $p$. We have to show that this is also true for $(p)$, i.e.,
  $o_1(p)$.  But the number of $($ in $(p)$ is $1 + {}$ the number of
  $($ in $p$.  And the number of $)$ in $(p)$ is $1 + {}$ the number
  of $)$ in $p$, so the claim also holds for $(p)$.
\item Suppose the number of $($ in $q$ equals the number of $)$, and
  the same is true for $q'$. The number of $($ in $o_2(p, p')$, i.e.,
  in $pp'$, is the sum of the number $($ in $p$ and $p'$. The number
  of $)$ in $o_2(p, p')$, i.e., in $pp'$, is the sum of the number of
  $)$ in $p$ and $p'$. The number of $($ in $o_2(p, p')$ equals the
  number of $)$ in $o_2(p,p')$.
\end{enumerate}
The result follows by structural induction.
\end{proof}

\end{document}
