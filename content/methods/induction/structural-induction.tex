% Part: methods
% Chapter: induction
% Section: structural-induction

\documentclass[../../../include/open-logic-section]{subfiles}

\begin{document}

\olfileid{mth}{ind}{sti}

\olsection{Structural Induction}

So far we have used induction to establish results about all natural
numbers. But a corresponding principle can be used directly to prove
results about all !!{element}s of an inductively defined set.  This
often called \emph{structural} induction, because it depends on the
structure of the inductively defined objects.

Generally, an inductive definition is given by (a)~a list of
``initial'' !!{element}s of the set and (b)~a list of operations which
produce new !!{element}s of the set from old ones. In the case of nice
terms, for instance, the initial objects are the letters. We only have
one operation: the operations are
\begin{align*}
  o(s_1, s_2) = & [s_1 \circ s_2]
\end{align*}
You can even think of the natural numbers~$\Nat$ themselves as being
given by an inductive definition: the initial object is~$0$, and the
operation is the successor function~$x + 1$.

In order to prove something about all elements of an inductively
defined set, i.e., that every !!{element} of the set has a
property~$P$, we must:
\begin{enumerate}
\item Prove that the initial objects have~$P$
\item Prove that for each operation~$o$, if the arguments have~$P$,
  so does the result.
\end{enumerate}
For instance, in order to prove something about all nice terms, we
would prove that it is true about all letters, and that it is true
about $[s_1 \circ s_2]$ provided it is true of $s_1$ and $s_2$
individually.

\begin{prop}
  The number of $[$ equals the number of $]$ in any nice term~$t$.
\end{prop}

\begin{proof}
We use structural induction.  Nice terms are inductively defined, with
letters as initial objects and the operation $o$ for constructing new
nice terms out of old ones.
\begin{enumerate}
\item The claim is true for every letter, since the number of $[$ in a
  letter by itself is~$0$ and the number of $]$ in it is also~$0$.
\item Suppose the number of $[$ in $s_1$ equals the number of $]$, and
  the same is true for $s_2$. The number of $[$ in $o(s_1, s_2)$, i.e., in
    $[s_1 \circ s_2]$, is the sum of the number of $[$ in $s_1$ and
      $s_2$ plus one. The number of $]$ in $o(s_1, s_2)$ is the sum of the number
    of $]$ in $s_1$ and $s_2$ plus one. Thus, the number of $[$ in $o(s_1, s_2)$
    equals the number of $]$ in $o(s_1,s_2)$.
\end{enumerate}
\end{proof}

\begin{prob}
  Prove by structural induction that no nice term starts with~$]$.
\end{prob}

Let's give another proof by structural induction: a proper initial
segment of a string~$t$ of symbols is any string~$s$ that agrees with
$t$ symbol by symbol, read from the left, but $t$ is longer. So,
e.g., $[a \circ {}$ is a proper initial segment of $[a \circ b]$, but
neither are $[b \circ {}$ (they disagree at the second symbol) nor $[a
\circ b]$ (they are the same length).

\begin{prop}\ollabel{prop:initial}
  Every proper initial segment of a nice term~$t$ has more $[$'s than
    $]$'s.
\end{prop}

\begin{proof}
  By induction on~$t$:
  \begin{enumerate}
  \item $t$ is a letter by itself: Then $t$ has no proper initial segments.
  \item $t = [s_1 \circ s_2]$ for some nice terms $s_1$ and~$s_2$. If $r$ is
    a proper initial segment of $t$, there are a number of
    possibilities:
    \begin{enumerate}
    \item $r$ is just $[$: Then $r$ has one more $[$ than it does~$]$.
    \item $r$ is $[r_1$ where $r_1$ is a proper initial segment
      of~$s_1$: Since $s_1$ is a nice term, by induction hypothesis, $r_1$
      has more $[$ than $]$ and the same is true for $[r_1$.
    \item $r$ is $[s_1$ or $[s_1 \circ {}$: By the previous result, the
      number of $[$ and $]$ in~$s_1$ are equal; so the number of $[$
      in $[s_1$ or $[s_1 \circ {}$ is one more than the number of~$]$.
    \item $r$ is $[s_1 \circ r_2$ where $r_2$ is a proper initial segment
      of~$s_2$:  By induction hypothesis, $r_2$ contains more $[$ than
      $]$. By the previous result, the number of $[$ and of
      $]$ in~$s_1$ are equal. So the number of $[$ in $[s_1 \circ r_2$
      is greater than the number of~$]$.
    \item $r$ is $[s_1 \circ s_2$: By the previous result, the number of
      $[$ and $]$ in $s_1$ are equal, and the same for~$s_2$. So there
      is one more $[$ in $[s_1 \circ s_2$ than there are~$]$.
    \end{enumerate}
  \end{enumerate}
\end{proof}

\end{document}
