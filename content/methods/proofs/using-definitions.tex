% Part: methods
% Chapter: proofs
% Section: using-definitions

\documentclass[../../../include/open-logic-section]{subfiles}

\begin{document}

\olfileid{mth}{prf}{def}

\olsection{Using Definitions}

We mentioned that you must be familiar with all definitions that may
be used in the proof, and that you can properly apply them. This is a
really important point, and it is worth looking at in a bit more
detail. Definitions are used to abbreviate properties and relations so
we can talk about them more succinctly. The introduced abbreviation is
called the \emph{definiendum}, and what it abbreviates is the
\emph{definiens}.  In proofs, we often have to go back to how the
definiendum was introduced, because we have to exploit the logical
structure of the definiens (the long version of which the defined term
is the abbreviation) to get through our proof.  By unpacking
definitions, you're ensuring that you're getting to the heart of where
the logical action is.

Later on we will prove that $X \cup (Y \cap Z) = (X \cup Y) \cap (X
\cup Z)$. In order to even start the proof, we need to know what it
means for two sets to be identical; i.e., we need to know what the
``$=$'' in that equation means for sets.  (Later on, we'll also have
to use the definitions of $\cup$ and $\cap$, of course).  Sets are
defined to be identical whenever they have the same !!{element}s.  So
the definition we have to unpack is:

\begin{defn}
Sets $X$ and $Y$ are \emph{identical}, $X = Y$, if every !!{element}
of~$X$ is an !!{element} of~$Y$, and vice versa.
\end{defn}

This definition uses $X$ and~$Y$ as placeholders for arbitrary
sets. What it defines---the definiendum---is the expression ``$X =
Y$'' by giving the condition under which $X = Y$ is true.  This
condition---``every !!{element} of~$X$ is an !!{element} of~$Y$, and
vice versa''---is the definiens.\footnote{In this particular
  case---and very confusingly!---when $X = Y$, the sets $X$ and $Y$
  are just one and the same set, even though we use different letters
  for it on the left and the right side.  But the ways in which that
  set is picked out may be different.}

When you apply the definition, you have to match the $X$ and $Y$ in
the definition to the case you're dealing with.  So, say, if you're
asked to show that $U = W$, the definition tells you that in order to
do so, you have to show that every !!{element} of~$U$ is an
!!{element} of~$W$, and vice versa.  In our case, it means that order
for $X \cup (Y \cap Z) = (X \cup Y) \cap (X \cup Z)$, each $z \in X
\cup (Y \cap Z)$ must also be in $(X \cup Y) \cap (X \cup Z)$, and
vice versa.  The expression $X \cup (Y \cap Z)$ plays the role of~$X$
in the definition, and $(X \cup Y) \cap (X \cup Z)$ that of~$Y$. Since
$X$ is used both in the definition and in the statement of the theorem
to be proved, but in different uses, you have to be careful to make
sure you don't mix up the two.  For instance, it would be a mistake to
think that you could prove the claim by showing that every !!{element}
of~$X$ is an !!{element} of~$Y$, and vice versa---that would show that
$X = Y$, not that $X \cup (Y \cap Z) = (X \cup Y) \cap (X \cup Z)$.

Within the proof we are dealing with set-theoretic notions like union
and intersection, and so we must also know the meanings of the symbols
$\cup$ and $\cap$ in order to understand how the proof should
proceed. And sometimes, unpacking the definition gives rise to further
definitions to unpack. For instance, $X \cup Y$ is defined as
$\Setabs{z}{z \in X \text{ or } z \in Y}$. So if you want to prove
that $x \in X \cup Y$, unpacking the definition of $\cup$ tells you
that you have to prove $x \in \Setabs{z}{z \in X \text{ or } z \in
  Y}$.  Now you also have to remember that $x \in \Setabs{z}{\dots
  z\dots}$ iff $\dots x\dots$.  So, further unpacking the definition
of the $\Setabs{z}{\dots z \dots}$ notation, what you have to show is:
$x \in X$ or $x \in Y$.

In order to be successful, you must know what the question is asking
and what all the terms used in the question mean---you will often need
to unpack more than one definition.  In simple proofs such as the ones
below, the solution follows almost immediately from the definitions
themselves. Of course, it won't always be this simple.

\begin{prob}
Suppose you are asked to prove that $X \cap Y \neq \emptyset$. Unpack
all the definitions occuring here, i.e., restate this in a way that
does not mention ``$\cap$'', ``='', or ``$\emptyset$.
\end{prob}

\end{document}
