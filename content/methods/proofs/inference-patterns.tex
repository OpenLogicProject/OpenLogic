% Part: methods
% Chapter: proofs
% Section: inference-patterns

\documentclass[../../../include/open-logic-section]{subfiles}

\begin{document}

\olfileid{mth}{prf}{pat}

\olsection{Inference Patterns}

Proofs are composed of individual inferences. When we make an
inference, we typically indicate that by using a word like ``so,''
``thus,'' or ``therefore.''  The inference often relies on one or two
facts we already have available in our proof---it may be something we
have assumed, or something that we've concluded by an inference
already.  To be clear, we may label these things, and in the inference
we indicate what other statements we're using in the inference.  An
inference will often also contain an explanation of \emph{why} our new
conclusion follows from the things that come before it.  There are
some common patterns of inference that are used very often in proofs;
we'll go through some below. Some patterns of inference, like proofs
by induction, are more involved (and will be discussed later).

We've already discussed one pattern of inference: unpacking, or
applying, a definition. When we unpack a definition, we just restate
something that involves the definiendum by using the definiens. For
instance, suppose that we have already established in the course of a
proof that $U = V$ (a). Then we may apply the definition of $=$ for sets
and infer: ``Thus, by definition from (a), every !!{element}
of~$U$ is !!a{element} of~$V$ and vice versa.''

Somewhat confusingly, we often do not write the justification of an
inference when we actually make it, but before.  Suppose we haven't
already proved that $U = V$, but we want to.  If $U = V$ is the
conclusion we aim for, then we can restate this aim also by applying
the definition: to prove $U = V$ we have to prove that every
!!{element} of~$U$ is !!a{element} of~$V$ and vice versa.  So our
proof will have the form: (a) prove that every !!{element} of~$U$ is
!!a{element} of~$V$; (b) every !!{element} of~$V$ is !!a{element}
of~$U$; (c) therefore, from (a) and (b) by definition of $=$, $U = V$.
But we would usually not write it this way. Instead we might write
something like,
\begin{quote}
We want to show $U = V$. By definition of~$=$, this amounts to showing
that every !!{element} of~$U$ is !!a{element} of~$V$ and vice
versa.

(a) \dots (a proof that every !!{element} of~$U$ is !!a{element}
of~$V$) \dots

(b) \dots (a proof that every !!{element}
of~$V$ is !!a{element} of~$U$) \dots
\end{quote}

\subsection{Using a Conjunction}

Perhaps the simplest inference pattern is that of drawing as
conclusion one of the conjuncts of a conjunction. In other words: if
we have assumed or already proved that $p$ and~$q$, then we're
entitled to infer that~$p$ (and also that~$q$).  This is such a basic
inference that it is often not mentioned.  For instance, once we've
unpacked the definition of $U = V$ we've established that every
!!{element} of~$U$ is !!a{element} of~$V$ and vice versa. From this
we can conclude that every !!{element} of~$V$ is !!a{element} of~$U$
(that's the ``vice versa'' part).  

\subsection{Proving a Conjunction}

Sometimes what you'll be asked to prove will have the form of a
conjunction; you will be asked to ``prove $p$ and $q$.'' In this case,
you simply have to do two things: prove $p$, and then prove $q$. You
could divide your proof into two sections, and for clarity, label
them. When you're making your first notes, you might write ``(1) Prove
$p$'' at the top of the page, and ``(2) Prove $q$'' in the middle of
the page. (Of course, you might not be explicitly asked to prove a
conjunction but find that your proof requires that you prove a
conjunction. For instance, if you're asked to prove that $U = V$ you
will find that, after unpacking the definition of~$=$, you have to
prove: every !!{element} of~$U$ is !!a{element} of~$V$
\emph{and} every !!{element} of~$V$ is !!a{element} of~$U$).

\subsection{Conditional Proof}

Many theorems you will encounter are in conditional form (i.e., show
that if $p$ holds, then $q$ is also true). These cases are nice and
easy to set up---simply assume the antecedent of the conditional (in
this case, $p$) and prove the conclusion~$q$ from it.  So if your
theorem reads, ``If $p$ then $q$,'' you start your proof with ``assume
$p$'' and at the end you should have proved~$q$.

Recall that a biconditional ($p$ iff $q$) is really two conditionals
put together: if $p$ then $q$, and if $q$ then~$p$. All you have to
do, then, is two instances of conditional proof: one for the first
instance and one for the second. Sometimes, however, it is possible to
prove an ``iff'' statement by chaining together a bunch of other
``iff'' statements so that you start with ``$p$'' an end with
``$q$''---but in that case you have to make sure that each step really
is an ``iff.''

\subsection{Universal Claims}

Using a universal claim is simple: if something is true for anything,
it's true for each particular thing.  So if, say, the hypothesis of
your proof is $X \subseteq Y$, that means (unpacking the definition
of~$\subseteq$), that, for every $x \in X$, $x \in Y$. Thus, if you
already know that $z \in X$, you can conclude $z \in Y$.

Proving a universal claim may seem a little bit tricky. Usually these
statements take the following form: ``If $x$ has~$P$, then it
has~$Q$'' or ``All $P$s are $Q$s.'' Of course, it might not fit this
form perfectly, and it takes a bit of practice to figure out what
you're asked to prove exactly. But: we often have to prove that all objects
with some property have a certain other property.

The way to prove a universal claim is to introduce names or variables,
for the things that have the one property and then show that they also
have the other property.  We might put this by saying that to prove
something for \emph{all}~$P$s you have to prove it for an
\emph{arbitrary}~$P$. And the name introduced is a name for an
arbitrary~$P$.  We typically use single letters as these names for
arbitrary things, and the letters usually follow conventions: e.g., we
use $n$ for natural numbers, $!A$ for !!{formula}s, $X$ for sets, $f$
for functions, etc.

The trick is to maintain generality throughout the proof. You start by
assuming that an arbitrary object (``$x$'') has the property~$P$, and
show (based only on definitions or what you are allowed to assume)
that $x$ has the property~$Q$. Because you have not stipulated what
$x$ is specifically, other that it has the property $P$, then you can
assert that all every $P$ has the property~$Q$. In short, $x$ is a
stand-in for \emph{all} things with property~$P$.

\subsection{Proving a Disjunction}

When what you are proving takes the form of a disjunction (i.e., it is
an statement of the form ``$p$ or $q$''), it is enough to show that
one of the disjuncts is true.  However, it basically never happens
that either disjunct just follows from the assumptions of your
theorem. More often, the assumptions of your theorem are themselves
disjunctive, or you're showing that all things of a certain kind have
one of two properties, but some of the things have the one and others
have the other property.  This is where proof by cases is
useful.

\subsection{Proof by Cases}

Suppose you have a disjunction as an assumption or as an already
established conclusion---you have assumed or proved that $p$ or $q$ is
true.  You want to prove $r$.  You do this in two steps: first you
assume that $p$ is true, and prove~$r$, then you assume that $q$ is
true and prove~$r$ again.  This works because we assume or know that
one of the two alternatives holds. The two steps establish that either
one is sufficient for the truth of~$r$.  (If both are true, we have
not one but two reasons for why $r$~is true. It is not necessary to
separately prove that $r$~is true assuming both $p$ and~$q$.)  To
indicate what we're doing, we announce that we ``distinguish cases.''
For instance, suppose we know that $x \in Y \cup Z$.  $Y \cup Z$ is
defined as $\Setabs{x}{x \in Y \text{ or } x \in Z}$. In other words,
by definition, $x \in Y$ or $x \in Z$. We would prove that $x \in X$
from this by first assuming that $x \in Y$, and proving $x \in X$ from
this assumption, and then assume $x \in Z$, and again prove $x \in X$
from this.  You would write ``We distinguish cases'' under the
assumption, then ``Case (1): $x \in Y$'' underneath, and ``Case (2):
$x \in Z$ halfway down the page. Then you'd proceed to fill in the top
half and the bottom half of the page.

Proof by cases is especially useful if what you're proving is itself
disjunctive. Here's a simple example:

\begin{prop}
Suppose $Y \subseteq U$ and $Z \subseteq V$. Then $Y \cup Z \subseteq
U \cup V$.
\end{prop}

\begin{proof}
  Assume (a) that $Y \subseteq U$ and (b) $Z \subseteq V$. By
  definition, any $x \in Y$ is also $\in U$ (c) and any $x \in Z$ is
  also $\in V$ (d).  To show that $Y \cup Z \subseteq U \cup V$, we
  have to show that if $x \in Y \cup Z$ then $x \in U \cup V$ (by
  definition of $\subseteq$). $x \in Y \cup Z$ iff $x \in Y$ or $x \in
  Z$ (by definition of~$\cup$). Similarly, $x \in U \cup V$ iff $x \in
  U$ or $x \in V$. So, we have to show: for any $x$, if $x \in Y$ or
  $x \in Z$, then $x \in U$ or $x \in V$.

  (So far we've only unpacked definitions!{} We've reformulated our
  proposition without $\subseteq$ and $\cup$ and are left with trying
  to prove a universal conditional claim. By what we've discussed
  above, this is done by assuming that $x$ is something about which we
  assume the ``if'' part is true, and we'll go on to show that the
  ``then'' part is true as well. In other words, we'll assume that $x
  \in Y$ or $x \in Z$ and show that $x \in U$ or $x \in V$.)

  Suppose that $x \in Y$ or $x \in Z$. We have to show that $x \in U$
  or $x \in V$. We distinguish cases.

  Case 1: $x \in Y$. By (c), $x \in U$. Thus, $x \in U$ or $x \in
  V$. (Here we've made the inference discussed in the preceding
  subsection!)
      
  Case 2: $x \in Z$. By (d), $x \in V$. Thus, $x \in U$ or $x \in V$.
 \end{proof}


\subsection{Proving an Existence Claim}

When asked to prove an existence claim, the question will usually be
of the form ``prove that there is an~$x$ such that $\dots x \dots$'',
i.e., that some object that has the property described by ``$\dots x
\dots$''. In this case you'll have to identify a suitable object show
that is has the required property.  This sounds straightforward, but a
proof of this kind can be tricky. Typically it involves
\emph{constructing} or \emph{defining} an object and proving that the
object so defined has the required property. Finding the right object
may be hard, proving that it has the required property may be hard,
and sometimes it's even tricky to show that you've succeeded in
defining an object at all!{}

Generally, you'd write this out by specifying the object, e.g., ``let
$x$ be \dots'' (where \dots{} specifies which object you have in
mind), possibly proving that $\dots$ in fact describes an object that
exists, and then go on to show that $x$ has the property~$Q$. Here's a
simple example.

\begin{prop}
  Suppose that $x \in Y$. Then there is an~$X$ such that $X \subseteq
  Y$ and $X \neq \emptyset$.
\end{prop}

\begin{proof}
  Assume $x \in Y$. Let $X = \{x\}$. (Here we've defined the set~$X$
  by enumerating its !!{element}s. Since we assume that $x$ is an
  object, and we can always for the set containing any number of
  objects by enumeration, we don't have to show that we've succeeded
  in defining a set~$X$ here.  However, we still have to show that $X$
  has the properties required by the proposition. The proof isn't
  complete without that!) Since $x \in X$, $X \neq \emptyset$. (This
  relies on the definition of $X$ as $\{x\}$ and the obvious facts
  that $x \in \{x\}$ and $x \notin \emptyset$.) Since $x$ is the only
  !!{element} of~$\{x\}$, and $x \in Y$, every !!{element} of~$X$ is
  also !!a{element} of~$Y$. By definition of~$\subseteq$, $X \subseteq
  Y$.
\end{proof}

\subsection{Using Existence Claims}

Suppose you know that some existence claim is true (you've proved it,
or it's a hypothesis you can use), say, ``for some~$x$, $x \in X$'' or
``there is an $x \in X$.''  If you want to use it in your proof, you
can just pretend that you have a name for one of the things in your
hypothesis says exit. Since $X$ contains at least one thing, there are
things to which that name might refer. You might of course not be able
to pick one or describe it further (other than that $x \in X$). But
for the purpose of the proof, you can pretend that you have picked it
out and give a name to it. (It's important to pick a name that you
haven't already used (or that appears in your hypotheses, otherwise
things can go wrong.) You might go from ``for some $x$, $x \in X$''
to ``Let $a \in X$.''  Now you reason about~$a$, use some other
hypotheses, etc., and come to a conclusion, $p$. If $p$ no longer
mentions~$a$, $p$ is independent of the asusmption that $a \in X$, and
you've shown that it follows just from the assumption ``for some $x$,
$x \in X$.''

\begin{prop}
If $X \neq \emptyset$, then $X \cup Y \neq \emptyset$.
\end{prop}

\begin{proof}
Here the hypothesis that $X \neq \emptyset$ hides an existential
claim, which you get to only by unpacking a few definitions. The
definition of $=$ tells us that $X = \emptyset$ iff every $x \in X$ is
also in $\emptyset$ and every $x \in \emptyset$ is also $\in
X$. Negating both sides, we get: $X \neq \emptyset$ iff either some $x
\in X$ is $\notin \emptyset$ or some $x \in \emptyset$ is $\notin
X$. Since nothing is $\in \emptyset$, the second disjunct can never be
true, and ``$x \in X$ and $x \notin \emptyset$'' reduces to just $x
\in X$. So $x \neq \emptyset$ iff for some $x$, $x \in X$. That's an
existence claim.
\begin{quote}
Suppose $X \neq \emptyset$, i.e., for some $x$, $x \in X$. Let $a \in
X$.
\end{quote}
Now we've introduced a name for one of the things~$\in X$. We'll use
it, only assuming that $a \in X$:
\begin{quote}
Since $a \in X$, $a \in X \cup Y$, by definition of~$\cup$. So for
some $x$, $x \in X \cup Y$, i.e., $X \cup Y \neq \emptyset$.
\end{quote}
In that last step, we went from ``$a \in X \cup Y$'' to ``for some
$x$, $x \in X \cup Y$.'' That didn't mention $a$ anymore, so we know
that ``for some $x$, $x \in X \cup Y$'' follows from ``for some $x$, $x
\in X$ alone.''
\end{proof}

It's maybe good practice to keep bound variables like ``$x$'' separate
from hypothtical names like $a$, like we did. In practice, however, we
often don't and just use $x$, like so:
\begin{quote}
Suppose $X \neq \emptyset$, i.e., there is an $x \in X$. By definition
of $\cup$, $x \in X \cup Y$. So $X \cup Y \neq \emptyset$.
\end{quote}
However, when you do this, you have to be extra careful that you use
different $x$'s and $y$'s for different existential claims. For
instance, the following is \emph{not} a correct proof of ``If $X \neq
\emptyset$ and $Y \neq \emptyset$ then $X \cap Y \neq \emptyset$''
(which is not true).
\begin{quote}
Suppose $X \neq \emptyset$ and $Y \neq \emptyset$. So for some $x$, $x
\in X$ and also for some $x$, $x \in Y$. Since $x \in X$ and $x \in
Y$, $x \in X \cap Y$, by definition of~$\cap$. So $X \cap Y \neq
\emptyset$.
\end{quote}
Can you spot where the incorrect step occurs and explain why the
result does not hold?

\end{document}
