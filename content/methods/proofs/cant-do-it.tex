% Part: methods
% Chapter: proofs
% Section: cant-do-it

\documentclass[../../../include/open-logic-section]{subfiles}

\begin{document}

\olfileid{mth}{prf}{cnt}

\olsection{I can't do it!{}}

We all get to a point where we feel like giving up. But you \emph{can}
do it. Your instructor and teaching assistant, as well as your fellow
students, can help. Ask them for help!{}  Here are a few tips to help
you avoid a crisis, and what to do if you feel like giving up.

To make sure you can solve problems successfully, do the
following:
\begin{enumerate}
\item \emph{Start as far in advance as possible.} We get busy
  throughout the semester and many of us struggle with
  procrastination, one of the best things you can do is to start your
  homework assignments early. That way, if you're stuck, you have time
  to look for a solution (that isn't crying).
\item \emph{Talk to your classmates}.  You are not alone. Others in
  the class may also struggle---but the may struggle with different
  things. Talking it out with your peers can give you a different
  perspective on the problem that might lead to a breakthrough.  Of
  course, don't just copy their solution: ask them for a hint, or
  explain where you get stuck and ask them for the next step. And when
  you do get it, reciprocate. Helping someone else along, and
  explaining things will help you understand better, too.
\item \emph{Ask for help.} You have many resources available to
  you---your instructor and teaching assistant are there for you and
  \emph{want} you to succeed. They should be able to help you work out
  a problem and identify where in the process you're struggling.
\item \emph{Take a break.} If you're stuck, it \emph{might} be because
  you've been staring at the problem for too long. Take a short break,
  have a cup of tea, or work on a different problem for a while, then
  return to the problem with a fresh mind. Sleep on it.
\end{enumerate}

Notice how these strategies require that you've started to work on the
proof well in advance? If you've started the proof at 2am the day
before it's due, these might not be so helpful.

This might sound like doom and gloom, but solving a proof is a
challenge that pays off in the end. Some people do this as a
career---so there must be something to enjoy about it. Like basically
everything, solving problems and doing proofs is something that
requires practice.  You might see classmates who find this easy:
they've probably just had lots of practice already.  Try not to give
in too easily.

If you do run out of time (or patience) on a particular problem:
that's ok. It doesn't mean you're stupid or that you will never get
it. Find out (from your instructor or another student) how it is done,
and identify where you went wrong or got stuck, so you can avoid doing
that the next time you encounter a similar issue.  Then try to do it
without looking at the solution.  And next time, start (and ask for
help) earlier.

\end{document}
