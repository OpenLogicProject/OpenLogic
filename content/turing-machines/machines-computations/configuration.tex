% Part: computability
% Chapter: machines-computations
% Section: configuration

\documentclass[../../../include/open-logic-section]{subfiles}

\begin{document}

\olfileid{cmp}{tur}{con}
\olsection{Configurations and Computations}

\begin{explain}
Recall tracing through the configurations of the even machine earlier.
The imaginary mechanism consisting of tape, read/write head, and
Turing machine program is really just an intuitive way of visualizing
what a Turing machine computation is.  Formally, we can define the
computation of a Turing machine on a given input as a sequence of
\emph{configurations}---and a configuration in turn is a sequence of
symbols (corresponding to the contents of the tape at a given point in
the computation), a number indicating the position of the read/write
head, and a state. Using these, we can define what the Turing
machine~$M$ computes on a given input.
\end{explain}

\begin{defn}[Configuration]
A \emph{configuration} of Turing machine $M = \tuple{Q, \Sigma, q_0,
\delta}$ is a triple $\tuple{C, m, q}$ where
\begin{enumerate}
\item $C \in \Sigma^*$ is a finite sequence of symbols from $\Sigma$,
\item $m \in \Nat$ is a number $< \len{C}$, and
\item $q \in Q$
\end{enumerate}
Intuitively, the sequence~$C$ is the content of the tape (symbols of
all squares from the leftmost square to the last non-blank or
previously visited square), $m$~is the number of the square the
read/write head is scanning (beginning with $0$ being the number of
the leftmost square), and $q$ is the current state of the machine.
\end{defn}

\begin{explain}
The potential input for a Turing machine is a sequence of symbols,
usually a sequence that encodes a number in some form.  The initial
configuration of the Turing machine is that configuration in which we
start the Turing machine to work on that input: the tape contains the
tape end marker immediately followed by the input written on the
squares to the right, the read/write head is scanning the leftmost
square of the input (i.e., the square to the right of the left end
marker), and the mechanism is in the designated start state~$q_0$.
\end{explain}

\begin{defn}[Initial configuration]
The \emph{initial configuration} of $M$ for input $I \in \Sigma^*$ is
\[
\tuple{\TMendtape \frown I, 1, q_0}.
\]
\end{defn}

\begin{explain}
The~$\frown$ symbol is for \emph{concatenation}---the input string
begins immediately to the left end marker.
\end{explain}

\begin{defn}
We say that a configuration $\tuple{C, m, q}$ \emph{yields the
configuration $\tuple{C', m', q'}$ in one step} (according to~$M$),
iff
\begin{enumerate}
\item the $m$-th symbol of $C$ is $\sigma$,
\item the instruction set of $M$ specifies $\delta(q, \sigma) =
  \tuple{q', \sigma', D}$,
\item the $m$-th symbol of $C'$ is $\sigma'$, and 
\item
\begin{enumerate}
\item $D = L$ and $m' = m - 1$ if $m>0$, otherwise $m'=0$, or
\item $D = R$ and $m' = m + 1$, or
\item $D = N$ and $m' = m$,
\end{enumerate}
\item if $m' = \len{C}$, then $\len{C'} = \len{C} + 1$ and the $m'$-th
  symbol of $C'$ is~$\TMblank$. Otherwise $\len{C'}=\len{C}$.
\item for all $i$ such that $i < \len{C}$ and $i \neq m$, $C'(i) = C(i)$,
\end{enumerate}
\end{defn}

\begin{defn}\ollabel{defn:run-output}
A \emph{run of $M$ on input~$I$} is a sequence $C_i$ of configurations
of $M$, where $C_0$ is the initial configuration of $M$ for input~$I$,
and each $C_i$ yields $C_{i+1}$ in one step.

We say that $M$ \emph{halts on input $I$ after $k$ steps} if $C_k =
\tuple{C, m, q}$, the $m$th symbol of~$C$ is~$\sigma$, and $\delta(q,
\sigma)$ is undefined.  In that case, the \emph{output} of~$M$ for
input~$I$ is~$O$, where $O$ is a string of symbols not ending
in~$\TMblank$ such that $C = \TMendtape \concat O \concat \TMblank^j$
for some~$i, j \in \Nat$.
\end{defn}

\begin{explain}
According to this definition, the output~$O$ of~$M$ always 
ends in a symbol other than~$\TMblank$, or, if at time~$k$ the entire
tape is filled with~$\TMblank$ (except for the leftmost~$\TMendtape$),
$O$~is the empty string.
\end{explain}

\end{document}
