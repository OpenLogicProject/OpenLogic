% Part: computability
% Chapter: tm-computations
% Section: halting-problem

\documentclass[../../../include/open-logic-section]{subfiles}

\begin{document}

\olfileid{cmp}{tur}{hal}
\olsection{The Halting Problem}

\begin{explain}
We know, via a simple proof, that there are some functions that are not
Turing-computable (i.e. computable by Turing machine).
We can show that the set of all Turing-machines is !!{enumerable}. This
follows from the fact that each Turing machine is described by a sequence
of quadruples. Take the !!{denumerable} vocabulary: $\TMendtape$,
$\TMblank$, $\TMstroke$, $\TMright$, $\TMleft$, $\TMstay$, $q_0$,
$q_1$, \dots. Any Turing machine
can be specified by some finite string of symbols from this alphabet
(though not every finite string of symbols specifies a Turing machine). We
know that the set of finite strings of symbols from !!a{denumerable}
alphabet is !!{enumerable}. This gives us an enumeration of the Turing
machines (a subset of the finite strings): $M_1$, $M_2$,~$M_3$.

The set of all functions is not !!{enumerable}. This follows immediately from
the fact that not even the set of all functions of one argument from the
positive integers to the set $\{0,1\}$ is !!{enumerable}.
If all functions were computable by some Turing machine we could enumerate
the set of all functions. So there are some functions that are not
Turing-computable. One such function is the halting function.
\end{explain}

\begin{defn} The function~$h$ is defined as
\[
h(m,w) =
\begin{cases}
  \text{0} & \text{if machine~$M_m$ does not halt for input w} \\
  \text{1} & \text{if machine~$M_m$ halts for input w}
\end{cases}
\]
\end{defn}

\begin{defn}
The \emph{Halting Problem} is the problem of determining (for any m, w)
whether the Turing machine~$M_m$ halts for an input of~$w$ $\TMstroke$s.
\end{defn}

\begin{explain}
We show that~$h$ is not Turing-computable by showing that a related
function,~$s$, is not Turing-computable. This proof relies on the fact
thatanything that can be computed by a Turing machine can be computed
using just two symbols: $\TMblank$ and $\TMstroke$, and the fact that two
Turing machines can be hooked together to create a single machine.
\end{explain}

\begin{defn} The function~$s$ is defined as
\[
s(w) =
\begin{cases}
  \text{0} & \text{if machine~$M_w$ does not halt for input w} \\
  \text{1} & \text{if machine~$M_w$ halts for input w}
\end{cases}
\]
\end{defn}

\begin{proof}
We suppose, for contradiction, that the function~$s$ is Turing-computable.
Then there would be a Turing machine that computes~$s$. Call it $S$. This
machine can be hooked up to another machine $J$, where $J$ halts iff it is
started on a blank tape. $S-J$, the machine created from hooking $S$ to
$J$, is a
Turing machine, so it is $M_w$ for some~$w$ (i.e., it appears in the
enumeration). Start $S-J$ on an input of~$w \TMstroke$s. There are two
possibilities: either $S-J$ (i.e. $M_w$) halts or it does not halt.
\begin{enumerate}
\item Suppose $S-J$/$M_w$ halts for an input of w $\TMstroke$s. Then
  $s(w) = 1$. So $S$ halts with a $\TMstroke$ on the tape.  $J$ starts
  with a $\TMstroke$ on the tape. So $J$ does not halt. So $S-J$/$M_w$
  does not halt for an input of w $\TMstroke$'s.

\item Suppose $S-J$/$M_w$ does not halt for an input of $w$
  $\TMstroke$s.  Then $s(w) = 0$, and $S$ halts with a blank tape. So
  $J$ starts with a blank tape.  So $J$ halts. So $S-J$/$M_w$ does
  halt for an input of $w$ $\TMstroke$'s.
\end{enumerate}

Since any Turing machine which computed~$h$ would \emph{ipso facto}
compute~$s$, it follows immediately that there can be no Turing machine
which computes~$h$ either, and thus that no Turing machine can serve as a
solution to the halting problem.
\end{proof}


\end{document}
