% Part: turing-machines 
% Chapter: undecidability 
% Section: unsolvability-decision-problem

\documentclass[../../../include/open-logic-section]{subfiles}

\begin{document}

\olfileid{tur}{und}{uns} 
\olsection{The Decision Problem is Unsolvable}

\begin{thm}
\ollabel{thm:decision-prob}
The decision problem is unsolvable.  
\end{thm}

\begin{proof}
Suppose the decision problem were solvable, i.e., suppose there were a
Turing machine~$D$ of the following sort. Whenever $D$ is started on a
tape that contains a !!{sentence}~$!B$ of first-order logic as input,
$D$ eventually halts, and outputs $1$ iff $!B$ is valid and $0$
otherwise.  Then we could solve the halting problem as follows. We
construct a Turing machine~$E$ that, given as input the number~$e$ of
Turing machine~$M_e$ and input~$w$, computes the corresponding
!!{sentence}~$!T(M_e, w) \lif !E(M_e, w)$ and halts, scanning the
leftmost square on the tape.  The machine $E \concat D$ would then,
given input $e$ and $w$, first compute~$!T(M_e, w) \lif !E(M_e, w)$
and then run the decision problem machine~$D$ on that input.  $D$
halts with output~$1$ iff $!T(M_e, w) \lif !E(M_e, w)$ is valid and
outputs~$0$ otherwise. By \olref[ver]{lem:halt-if-valid} and
\olref[ver]{lem:valid-if-halt}, $!T(M_e, w) \lif !E(M_e, w)$ is valid
iff $M_e$ halts on input~$w$. Thus, $E\concat D$, given input $e$ and
$w$ halts with output~$1$ iff $M_e$ halts on input~$w$ and halts with
output~$0$ otherwise. In other words, $E \concat D$ would solve the
halting problem.  But we know, by \olref[hal]{thm:halting-problem},
that no such Turing machine can exist.
\end{proof}

\end{document}
