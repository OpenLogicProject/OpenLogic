% Part: turing-machines 
% Chapter: undecidability 
% Section: unsolvability-decision-problem

\documentclass[../../../include/open-logic-section]{subfiles}

\begin{document}

\olfileid{tur}{und}{uns} 
\olsection{The Decision Problem is Unsolvable}

\begin{thm}
\ollabel{thm:decision-prob}
The decision problem is unsolvable: There is no Turing machine~$D$,
which when started on a tape that contains !!a{sentence}~$!B$ of
first-order logic as input, $D$~eventually halts, and outputs~$1$ iff
$!B$ is valid and $0$ otherwise.
\end{thm}

\begin{proof}
Suppose the decision problem were solvable, i.e., suppose there were a
Turing machine~$D$. Then we could solve the halting problem as
follows. We construct a Turing machine~$E$ that, given as input the
number~$e$ of Turing machine~$M_e$ and input~$w$, computes the
corresponding !!{sentence}~$!T(M_e, w) \lif !E(M_e, w)$ and halts,
scanning the leftmost square on the tape.  The machine $E \concat D$
would then, given input $e$ and $w$, first compute~$!T(M_e, w) \lif
!E(M_e, w)$ and then run the decision problem machine~$D$ on that
input.  $D$ halts with output~$1$ iff $!T(M_e, w) \lif !E(M_e, w)$ is
valid and outputs~$0$ otherwise. By \olref[ver]{lem:halt-if-valid} and
\olref[ver]{lem:valid-if-halt}, $!T(M_e, w) \lif !E(M_e, w)$ is valid
iff $M_e$ halts on input~$w$. Thus, $E\concat D$, given input $e$ and
$w$ halts with output~$1$ iff $M_e$ halts on input~$w$ and halts with
output~$0$ otherwise. In other words, $E \concat D$ would solve the
halting problem.  But we know, by \olref[hal]{thm:halting-problem},
that no such Turing machine can exist.
\end{proof}

\begin{cor}\ollabel{cor:undecidable-sat}%
It is undecidable if an arbitrary !!{sentence} of first-order logic is satisfiable.
\end{cor}

\begin{proof}
  Suppose satisfiability were decidable by a Turing machine~$S$. Then
  we could solve the decision problem as follows: Given
  !!a{sentence}~$B$ as input, move $!B$ to the right one square.
  Return to square~$1$ and write the symbol~$\lnot$.

  Now run the Turing machine~$S$. It eventually halts with output
  either $1$ (if $\lnot !B$ is satisfiable) or~$0$ (if $\lnot !B$ is
  unsatisfiable) on the tape. If there is a~$\TMstroke$ on square~$1$,
  erase it; if square~$1$ is empty, write a~$\TMstroke$, then halt.

  This Turing machine always halts, and its output is~$1$ iff $\lnot
  !B$ is unsatisfiable and $0$~otherwise. Since $!B$ is valid iff
  $\lnot !B$~is unsatisfiable, the machine outputs~$1$ iff $!B$ is
  valid, and $0$~otherwise, i.e., it would solve the decision problem.
\end{proof}

\begin{explain}
So there is no Turing machine which always gives a
correct ``yes'' or ``no'' answer to the question ``Is $!B$ a valid
!!{sentence} of first-order logic?'' However, there \emph{is} a Turing
machine that always gives a correct ``yes'' answer---but simply does
not halt if the answer is ``no.'' This follows from the soundness and
completeness theorem of first-order logic, and the fact that
!!{derivation}s can be effectively enumerated.
\end{explain}

\begin{thm}
  \ollabel{thm:valid-ce}%
  Validity of first-order !!{sentence}s is semi-decidable: There is a
  Turing machine~$E$, which when started on a tape that contains
  !!a{sentence}~$!B$ of first-order logic as input, $E$~eventually
  halts and outputs~$1$ iff $!B$ is valid, but does not halt
  otherwise.
\end{thm}

\begin{proof}
  All possible !!{derivation}s of first-order logic can be generated,
  one after another, by an effective algorithm.  The machine~$E$ does
  this, and when it finds !!a{derivation} that shows that $\Proves
  !B$, it halts with output~$1$. By the soundness theorem, if $E$
  halts with output~$1$, it's because~$\Entails !B$. By the
  completeness theorem, if $\Entails !B$ there is !!a{derivation} that
  shows that~$\Proves !B$. Since $E$ systematically generates all
  possible !!{derivation}s, it will eventually find one that
  shows~$\Proves !B$, so will eventually halt with output~$1$.
\end{proof}

\end{document}
