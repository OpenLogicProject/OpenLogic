% Part: turing-machines 
% Chapter: undecidability 
% Section: verification

\documentclass[../../../include/open-logic-section]{subfiles}

\begin{document}

\olfileid{tur}{und}{ver} 
\olsection{Verifying the Representation}

\begin{explain}
In order to verify that our representation works, we have to prove two
things. First, we have to show that if $M$ halts on input~$w$, then
$!T(M, w) \lif !E(M, w)$ is valid. Then, we have to show the converse,
i.e., that if $!T(M, w) \lif !E(M, w)$ is valid, then $M$ does in fact
eventually halt when run on input~$w$.

The strategy for proving these is very different. For the first
result, we have to show that !!a{sentence} of first-order logic
(namely, $!T(M, w) \lif !E(M, w)$) is valid. The easiest way to do
this is to give !!a{derivation}. Our proof is supposed to work for all
$M$ and $w$, though, so there isn't really a single !!{sentence} for
which we have to give !!a{derivation}, but infinitely many.  So the best
we can do is to prove by induction that, whatever $M$ and~$w$ look
like, and however many steps it takes $M$ to halt on input~$w$, there
will be !!a{derivation} of $!T(M, w) \lif !E(M, w)$.

Naturally, our induction will proceed on the number of steps $M$ takes
before it reaches a halting configuration. In our inductive proof,
we'll establish that for each step~$n$ of the run of $M$ on input~$w$,
$!T(M, w) \Entails !C(M, w, n)$, where $!C(M, w, n)$ correctly
describes the configuration of $M$ run on~$w$ after~$n$ steps. Now if
$M$ halts on input~$w$ after, say, $n$ steps, $!C(M, w, n)$ will
describe a halting configuration. We'll also show that $!C(M, w, n)
\Entails !E(M, w)$, whenever $!C(M, w, n)$ describes a halting
configuration.  So, if $M$ halts on input~$w$, then for some~$n$, $M$
will be in a halting configuration after $n$~steps. Hence, $!T(M, w)
\Entails !C(M, w, n)$ where $!C(M, w, n)$ describes a halting
configuration, and since in that case $!C(M, w, n) \Entails !E(M, w)$,
we get that $T(M, w) \Entails !E(M, w)$, i.e., that $\Entails !T(M, w)
\lif !E(M, w)$.

The strategy for the converse is very different. Here we assume that
$\Entails !T(M, w) \lif !E(M, w)$ and have to prove that $M$ halts on
input~$w$.  From the hypothesis we get that $!T(M, w) \Entails !E(M,
w)$, i.e., $!E(M, w)$ is true in every !!{structure} in which $!T(M,
w)$ is true. So we'll describe !!a{structure}~$\Struct{M}$ in which
$!T(M, w)$ is true: its domain will be $\Nat$, and the interpretation
of all the $\Obj Q_q$ and $\Obj S_\sigma$ will be given by the
configurations of~$M$ during a run on input~$w$.  So, e.g.,
$\Sat{M}{\Obj Q_q(\num{m}, \num{n})}$ iff $T$, when run on input~$w$
for $n$ steps, is in state~$q$ and scanning square~$m$.  Now since
$!T(M, w) \Entails !E(M, w)$ by hypothesis, and since $\Sat{M}{!T(M,
  w)}$ by construction, $\Sat{M}{!E(M, w)}$.  But $\Sat{M}{!E(M, w)}$
iff there is some $n \in \Domain{M} = \Nat$ so that $M$, run on
input~$w$, is in a halting configuration after~$n$ steps.
\end{explain}

\begin{defn} 
Let $!C(M, w, n)$ be the !!{sentence}
\[ 
\Obj Q_q(\num{m}, \num{n}) \land \Obj S_{\sigma_0}(\num{0}, \num{n})
\land \dots \land \Obj S_{\sigma_k}(\num{k}, \num{n}) \land
\lforall[x][(\num{k} < x \lif \Obj S_\TMblank(x, \num{n}))]
\] 
where $q$ is the state of $M$ at time~$n$, $M$ is scanning square~$m$
at time~$n$, square~$i$ contains symbol~$\sigma_i$ at time~$n$ for $0
\le i \le k$ and $k$ is the right-most non-blank square of the tape at
time~$0$, or the right-most square the tape head has visited after $n$
steps, whichever is greater.
\end{defn}

\begin{lem}\ollabel{lem:halt-config-implies-halt}
If $M$ run on input~$w$ is in a halting configuration after $n$ steps,
then $!C(M, w, n) \Entails !E(M, w)$.
\end{lem}

\begin{proof}
Suppose that $M$ halts for input~$w$ after $n$ steps.
There is some state~$q$, square~$m$, and symbol~$\sigma$ such that:
\begin{enumerate} 
\item After $n$ steps, $M$ is in state~$q$ scanning square~$m$ on
  which~$\sigma$ appears.
\item The transition function $\delta(q, \sigma)$ is undefined.
\end{enumerate}
$!C(M, w, n)$ is the description of this configuration and will
include the clauses $\Obj Q_{q}(\num{m}, \num{n})$ and $\Obj
S_{\sigma}(\num{m}, \num{n})$. These clauses together imply $!E(M,
w)$:
\[
\lexists[x][\lexists[y][(\bigvee_{\tuple{q, \sigma} \in
      X}(\Obj Q_q(x, y) \land \Obj S_\sigma(x, y)))]]
\]
since $\Obj Q_{q'}(\num{m}, \num{n}) \land S_{\sigma'}(\num{m},
\num{n}) \Entails \bigvee_{\tuple{q, \sigma} \in X} (\Obj Q_q(\num{m},
\num{n}) \land \Obj S_{\sigma}(\num{m}, \num{n}))$, as
$\tuple{q',\sigma'} \in X$. 
\end{proof}

\begin{explain}
So if $M$ halts for input $w$, then there is some~$n$ such that $!C(M,
w, n) \Entails !E(M,w)$.  We will now show that for any time~$n$,
$!T(M, w) \Entails !C(M, w, n)$.
\end{explain}

\begin{lem}
\ollabel{lem:config}
For each $n$, if $M$ has not halted after $n$ steps, $!T(M, w)
\Entails !C(M, w, n)$.
\end{lem}

\begin{proof}
Induction basis: If $n = 0$, then the conjuncts of $!C(M, w, 0)$ are
also conjuncts of $!T(M, w)$, so entailed by it.

Inductive hypothesis: If $M$ has not halted before the $n$th step,
then $!T(M,w) \Entails !C(M, w, n)$. We have to show that (unless
$!C(M, w, n)$ describes a halting configuration), $!T(M, w) \Entails
!C(M, w, n+1)$.

Suppose $n > 0$ and after $n$ steps, $M$ started on $w$ is in
state~$q$ scanning square~$m$. Since $M$ does not halt after~$n$
steps, there must be an instruction of one of the following three
forms in the program of~$M$:

\begin{enumerate} 
\item \ollabel{right} $\delta(q, \sigma) = \tuple{q', \sigma', \TMright}$

\item \ollabel{left} $\delta(q, \sigma) = \tuple{q', \sigma', \TMleft}$

\item \ollabel{stay} $\delta(q, \sigma) = \tuple{q', \sigma', \TMstay}$
\end{enumerate}

We will consider each of these three cases in turn. 

\begin{enumerate} 
\item Suppose there is an instruction of the form~\olref{right}.
  By \olref[rep]{defn:tm-descr}\olref[rep]{rep-right}, this means that
\begin{align*} 
& \lforall[x][\lforall[y][((\Obj Q_{q}(x,
  y) \land \Obj S_\sigma(x, y)) \lif {}]]\\
& \qquad (\Obj Q_{q'}(x',
  y') \land \Obj S_{\sigma'}(x, y') \land !A(x, y)))  
\intertext{is a conjunct of $!T(M,w)$. This entails the following
  !!{sentence} (universal instantiation, $\num{m}$ for~$x$ and
  $\num{n}$ for~$y$):}
& (\Obj Q_{q}(\num{m}, \num{n}) \land \Obj S_{\sigma}(\num{m},
\num{n})) \lif {}\\
& \qquad (\Obj Q_{q'}(\num{m}', \num{n}') \land
\Obj S_{\sigma'}(\num{m}, \num{n}') \land !A(\num{m}, \num{n})).
\intertext{By induction hypothesis, $!T(M, w) \Entails !C(M, w, n)$,
  i.e.,}
& \Obj Q_q(\num{m}, \num{n}) \land \Obj S_{\sigma_0}(\num{0}, \num{n})
\land \dots \land \Obj S_{\sigma_k}(\num{k}, \num{n}) \land \\
& \qquad\lforall[x][(\num{k} < x \lif \Obj S_\TMblank(x, \num{n}))]\\
\intertext{Since after $n$ steps, tape square~$m$ contains~$\sigma$,
  the corresponding conjunct is~$\Obj S_\sigma(\num{m}, \num{n})$,
  so this entails:}
& \Obj Q_{q}(\num{m}, \num{n}) \land \Obj S_{\sigma}(\num{m},
   \num{n})
\intertext{We now get}
& \Obj Q_{q'}(\num{m}', \num{n}') \land \Obj S_{\sigma'}(\num{m},
  \num{n}') \land {}\\
& \qquad\Obj S_{\sigma_0}(\num{0}, \num{n}') \land \dots \land
  \Obj S_{\sigma_k}(\num{k}, \num{n}') \land {}\\
& \qquad \lforall[x][(\num{k} < x \lif \Obj S_\TMblank(x, \num{n}'))]
\end{align*}
as follows: The first line comes directly from the consequent of the
preceding conditional, by modus ponens. Each conjunct in the middle
line---which excludes $S_{\sigma_m}(\num{m},\num{n}')$---follows from
the corresponding conjunct in~$!C(M, w, n)$ together with $!A(\num{m},
\num{n})$.

If $m < k$, $!T(M,w) \Proves \num{m} < \num {k}$
(\olref[rep]{prop:mlessk}) and by transitivity of~$<$, we have
$\lforall[x][(\num{k} < x \lif \num{m} < x)]$.  If $m = k$, then
$\lforall[x][(\num{k} < x \lif \num{m} < x)]$ by logic alone.  The
last line then follows from the corresponding conjunct in $!C(M, w,
n)$, $\lforall[x][(\num{k} < x \lif \num{m} < x)]$, and $!A(\num{m},
\num{n})$.  If $m<k$, this already is $!C(M, w, n+1)$.

Now suppose $m=k$. In that case, after $n+1$ steps, the tape head has
also visited square~$k+1$, which now is the right-most square
visited.  So $!C(M, w, n+1)$ has a new conjunct, $\Obj
S_\TMblank(\num{k}',\num{n}')$, and the last conjunct is
$\lforall[x][(\num{k}' < x \lif \Obj S_\TMblank(x, \num{n}'))]$. We
have to verify that these two !!{sentence}s are also implied.

We already have $\lforall[x][(\num{k} < x \lif \Obj S_\TMblank(x,
  \num{n}'))]$. In particular, this gives us $\num{k} < \num{k}' \lif
\Obj S_\TMblank(\num{k}', \num{n}')$. From the axiom $\lforall[x][x <
  x']$ we get $\num{k} < \num{k}'$. By modus ponens, $\Obj
S_\TMblank(\num{k}',\num{n}')$ follows.

Also, since $!T(M,w) \Proves \num{k} < \num{k}'$, the axiom for
transitivity of~$<$ gives us $\lforall[x][(\num{k}' < x \lif \Obj
  S_\TMblank(x, \num{n}'))]$. (We leave the verification of this as an
exercise.)

\item Suppose there is an instruction of the form~\olref{left}.
  Then, by \olref[rep]{defn:tm-descr}\olref[rep]{rep-left},
\begin{align*} 
& \lforall[x][\lforall[y][((\Obj Q_{q}(x', y) \land \Obj
    S_{\sigma}(x', y)) \lif {}]]\\
& \qquad (\Obj Q_{q'}(x, y') \land \Obj
  S_{\sigma'}(x', y') \land !A(x, y))) \land {}\\
& \lforall[y][((\Obj Q_{q_i}(\num{0}, y) \land \Obj S_{\sigma}(\num{0},
    y)) \lif {}]\\
& \qquad (\Obj Q_{q_j}(\num{0}, y') \land \Obj S_{\sigma'}(\num{0}, y')
  \land !A(\num{0}, y)))
\intertext{is a conjunct of $!T(M,w)$. If $m>0$, then let $l = m - 1$
  (i.e., $m = l+1$). The first conjunct of the above !!{sentence}
  entails the following:}
& (\Obj Q_{q}(\num{l}', \num{n}) \land \Obj S_{\sigma}(\num{l}', \num{n}))
\lif {} \\
& \qquad (\Obj Q_{q'}(\num{l}, \num{n}') \land \Obj S_{\sigma'}(\num{l}',
\num{n}') \land !A(\num{l}, \num{n}))
\intertext{Otherwise, let $l = m = 0$ and consider the following !!{sentence}
  entailed by the second conjunct:}
& ((\Obj Q_{q_i}(\num{0}, \num{n}) \land \Obj S_{\sigma}(\num{0}, \num{n})) \lif {}\\
& \qquad   (\Obj Q_{q_j}(\num{0}, \num{n}') \land \Obj S_{\sigma'}(\num{0}, \num{n}') \land
!A(\num{0}, \num{n}))) 
\intertext{Either sentence implies}
&  \Obj Q_{q'}(\num{l}, \num{n}') \land \Obj S_{\sigma'}(\num{m},
  \num{n}') \land {}\\
&\qquad  \Obj S_{\sigma_0}(\num{0}, \num{n}')\land \dots \land
  \Obj S_{\sigma_k}(\num{k}, \num{n}')  \land {} \\
& \qquad  \lforall[x][(\num{k} < x
  \lif \Obj S_\TMblank(x, \num{n}'))]
\end{align*}
as before. (Note that in the first case, $\num{l}' \ident \num{l+1}
\ident \num{m}$ and in the second case $\num{l} \ident \num{0}$.) But
this just is $!C(M, w, n+1)$.

\item Case \olref{stay} is left as an exercise.
\end{enumerate}
We have shown that for any~$n$, $!T(M, w) \Entails !C(M, w, n)$.
\end{proof}

\begin{prob}
Complete case~\olref[tur][und][ver]{stay} of the proof of
\olref[tur][und][ver]{lem:config}.
\end{prob}

\begin{prob}
Give !!a{derivation} of $\Obj S_{\sigma_i}(\num{i}, \num{n}')$ from
$\Obj S_{\sigma_i}(\num{i}, \num{n})$ and $!A(m, n)$ (assuming $i \neq
m$, i.e., either $i < m$ or $m < i$).
\end{prob}

\begin{prob}
Give !!a{derivation} of $\lforall[x][(\num{k}' < x \lif \Obj
  S_\TMblank(x, \num{n}'))]$ from $\lforall[x][(\num{k} < x \lif \Obj
  S_\TMblank(x, \num{n}'))]$, $\lforall[x][x < x']$, and
$\lforall[x][\lforall[y][\lforall[z][ ((x < y \land y < z) \lif x <
      z)]]]$.)
\end{prob}


\begin{lem}
\ollabel{lem:valid-if-halt}
If $M$ halts on input~$w$, then $!T(M, w) \lif
!E(M, w)$ is valid.
\end{lem}

\begin{proof}
By \olref{lem:config}, we know that, for any time~$n$, the
description~$!C(M, w, n)$ of the configuration of $M$ at time~$n$ is
entailed by~$!T(M, w)$.  Suppose $M$ halts after $k$ steps. At that
point, it will be scanning square~$m$, for some~$m \in \Nat$. Then
$!C(M, w, k)$ describes a halting configuration of~$M$, i.e., it
contains as conjuncts both $\Obj Q_q(\num{m}, \num{k})$ and $\Obj
S_\sigma(\num{m}, \num{k})$ with $\delta(q,\sigma)$ undefined.  Thus,
by \olref{lem:halt-config-implies-halt}, $!C(M, w, k) \Entails !E(M,
w)$. But since $!T(M, w) \Entails !C(M, w, k)$, we have $!T(M, w)
\Entails !E(M, w)$ and therefore $!T(M, w) \lif !E(M, w)$ is valid.
\end{proof}


\begin{explain} 
To complete the verification of our claim, we also have to
establish the reverse direction: if $!T(M, w) \lif !E(M, w)$ is valid, then
$M$ does in fact halt when started on input~$w$. 
\end{explain}

\begin{lem}
\ollabel{lem:halt-if-valid}
If $\Entails !T(M, w) \lif !E(M, w)$, then $M$ halts on input~$w$.
\end{lem}

\begin{proof}
Consider the $\Lang L_M$-!!{structure}~$\Struct M$ with
domain~$\Nat$ which interprets $\Obj 0$ as~$0$, $\prime$~as the successor
function, and $<$~as the less-than relation, and the predicates $\Obj Q_q$
and~$\Obj S_\sigma$ as follows:
\begin{align*}
  \Assign{\Obj Q_q}{M} & =
\Setabs{\tuple{m, n}}{\begin{array}{ll}\text{started on~$w$, after~$n$ steps,}\\ \text{$M$ is in state $q$
  scanning square~$m$}\end{array}} \\
\Assign{\Obj S_\sigma}{M} & = \Setabs{\tuple{m, n}}{\begin{array}{ll}
\text{started on~$w$, after $n$ steps,}\\ \text{square~$m$ of $M$ contains
  symbol~$\sigma$}\end{array}}
\end{align*}
In other words, we construct the !!{structure}~$\Struct{M}$ so that it
describes what $M$ started on input~$w$ actually does, step by step.
Clearly, $\Sat{M}{!T(M, w)}$. If $\Entails !T(M, w) \lif !E(M, w)$,
then also $\Sat{M}{!E(M, w)}$, i.e.,
\[
\Sat{M}{\lexists[x][\lexists[y][(\bigvee_{\tuple{q, \sigma} \in
      X}(\Obj Q_q(x, y) \land \Obj S_\sigma(x, y)))]]}.
\]
As $\Domain{M} = \Nat$, there must be $m$, $n \in \Nat$ so that
$\Sat{M}{\Obj Q_q(\num{m}, \num{n}) \land \Obj S_\sigma(\num{m},
\num{n})}$ for some~$q$ and~$\sigma$ such that $\delta(q, \sigma)$ is
undefined. By the definition of~$\Struct M$, this means that $M$
started on input~$w$ after~$n$ steps is in state~$q$ and reading
symbol~$\sigma$, and the transition function is undefined, i.e.,
$M$~has halted.
\end{proof}

\end{document}
