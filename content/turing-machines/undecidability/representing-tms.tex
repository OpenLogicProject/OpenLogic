% Part: turing-machines
% Chapter: undecidability
% Section: representing-tms

\documentclass[../../../include/open-logic-section]{subfiles}

\begin{document}

\olfileid{tms}{und}{rep}
\olsection{Representing Turing Machines}

\begin{explain}
In order to represent Turing machines and their behavior by !!a{sentence}
of first-order logic, we have to define a suitable language. The
language consists of two parts: !!{predicate}s for describing
configurations of the machine, and expressions for numbering execution
steps (``moments'') and positions on the tape.

We introduce two kinds of !!{predicate}s, both of them 2-place: For
each state~$q$, !!a{predicate}~$\Obj Q_q$, and for each tape
symbol~$\sigma$, !!a{predicate}~$\Obj S_\sigma$. The former allow us
to describe the state of~$M$ and the position of its tape head, the
latter allow us to describe the contents of the tape.  

In order to express the positions of the tape head and the number of
steps executed, we need a way to express numbers. This is done using
!!a{constant}~$\Obj 0$, and a $1$-place function~$\prime$, the
successor function. By convention it is written \emph{after} its
argument (and we leave out the parentheses). So $\Obj 0$ names the
leftmost position on the tape as well as the time before the first
execution step (the initial configuration), $\Obj 0'$ names the square
to the right of the leftmost square, and the time after the first
execution step, and so on. We also introduce !!a{predicate}~$<$ to
express both the ordering of tape positions (when it means ``to the
left of'') and execution steps (then it means ``before'').

Once we have the language in place, we list the ``axioms'' of $!T(M,
w)$, i.e., the !!{sentence}s which, taken together, describe the
behavior of~$M$ when run on input~$w$.  There will be !!{sentence}s
which lay down conditions on $\Obj 0$, $\prime$, and $<$,
!!{sentence}s that describes the input configuration, and
!!{sentence}s that describe what the configuration of $M$ is after it
executes a particular instruction.
\end{explain}

\begin{defn}
  \ollabel{defn:tm-descr}
Given a Turing machine $M = \tuple{Q, \Sigma, q_0, \delta}$, the
language~$\Lang L_M$ consists of:
\begin{enumerate}
\item A two-place !!{predicate} $\Obj Q_q(x, y)$ for every state~$q \in
  Q$.  Intuitively, $\Obj Q_q(\num{m}, \num{n})$ expresses ``after $n$
  steps, $M$ is in state~$q$ scanning the $m$th square.''
\item A two-place !!{predicate} $\Obj S_\sigma(x, y)$ for every
  symbol~$\sigma\in \Sigma$.  Intuitively, $\Obj S_\sigma(\num{m},
  \num{n})$ expresses ``after $n$ steps, the $m$th square contains
  symbol~$\sigma$.''
\item A !!{constant} $\Obj 0$
\item A one-place !!{function} $\prime$
\item A two-place !!{predicate} $<$
\end{enumerate}
\end{defn}

For each number $n$ there is a canonical term $\num{n}$, the
\emph{numeral} for~$n$, which represents it in~$\Lang L_M$. $\num{0}$
is $\Obj 0$, $\num{1}$ is $\Obj 0'$, $\num{2}$ is $\Obj 0''$, and so
on. More formally:
\begin{align*}
\num{0} & = \Obj 0 \\
\num{n+1} &= \num{n}'
\end{align*}

The !!{sentence}s describing the operation of the Turing machine~$M$ on
input $w = \sigma_{i_1}\dots\sigma_{i_k}$ are the following:
\begin{enumerate}
\item Axioms describing numbers:
\begin{enumerate}
\item \ollabel{tm-rep-a} !!^a{sentence} that says that the successor function is injective:
\[
\lforall[x][\lforall[y][
    (\eq[x'][y'] \lif \eq[x][y])]]
\]
\item !!^a{sentence} that says that every number is less than its successor:
\[
\lforall[x][x < x']
\]
\item !!^a{sentence} that ensures that $<$ is transitive:
\[
\lforall[x][\lforall[y][\lforall[z][
      ((x < y \land y < z) \lif x < z)]]]
\]
\item !!^a{sentence} that connects $<$ and~$=$:
\[
\lforall[x][\lforall[y][(x < y \lif \eq/[x][y])]]
\]
\end{enumerate}
\item Axioms describing the input configuration:
\begin{enumerate}
\item After after $0$~steps---before the machine starts---$M$ is in
  the inital state~$q_0$, scanning square~$1$:
\[
\Obj Q_{q_0}(\num{1}, \num{0})
\]
\item The first $k+1$ squares contain the symbols $\TMendtape$,
  $\sigma_{i_1}$, \dots, $\sigma_{i_k}$:
\[
\Obj S_\TMendtape(\num{0}, \num{0}) \land
\Obj S_{\sigma_{i_1}}(\num{1}, \num{0}) \land
\dots \land
\Obj S_{\sigma_{i_k}}(\num{n}, \num{0})
\]
\item Otherwise, the tape is empty:
\[
\lforall[x][(\num{k} < x \lif \Obj S_\TMblank(x, \num{0}))]
\]
\end{enumerate}
\item Axioms describing the transition from one configuration to
  the next:

For the following, let $!A(x, y)$ be the conjunction of all !!{sentence}s
of the form
\[
\lforall[z][
  (((z < x \lor x < z) \land \Obj S_\sigma(z, y))
  \lif \Obj S_\sigma(z, y'))]
\]
where $\sigma \in \Sigma$.  We use $!A(\num{m},\num{n})$ to express
``other than at square~$m$, the tape after $n+1$ steps is the same as
after $n$ steps.''
\begin{enumerate}
\item \ollabel{rep-right} For every instruction $\delta(q_i, \sigma) =
  \tuple{q_j, \sigma', \TMright}$, the !!{sentence}:
\begin{align*}
& \lforall[x][\lforall[y][(
   (\Obj Q_{q_i}(x, y) \land \Obj S_{\sigma}(x, y)) \lif {}]] \\
&\qquad   (\Obj Q_{q_j}(x', y') \land \Obj S_{\sigma'}(x, y') \land
!A(x, y)))
\end{align*}
This says that if, after~$y$ steps, the machine is in state~$q_i$
scanning square~$x$ which contains symbol~$\sigma$, then after $y+1$
steps it is scanning square~$x+1$, is in state~$q_j$, square~$x$ now
contains~$\sigma'$, and every square other than~$x$ contains the
same symbol as it did after~$y$ steps.

\item \ollabel{rep-left} For every instruction $\delta(q_i, \sigma) =
  \tuple{q_j, \sigma', \TMleft}$, the !!{sentence}:
\begin{align*}
& \lforall[x][\lforall[y][
    ((\Obj Q_{q_i}(x', y) \land \Obj S_{\sigma}(x', y)) \lif {}]]\\
& \qquad   (\Obj Q_{q_j}(x, y') \land \Obj S_{\sigma'}(x', y') \land
!A(x, y))) \land {}\\
& \lforall[y][((\Obj Q_{q_i}(\Obj 0, y) \land \Obj S_{\sigma}(\Obj 0,
    y)) \lif {}]\\
& \qquad (\Obj Q_{q_j}(\Obj 0, y') \land \Obj S_{\sigma'}(\Obj 0,
  y') \land !A(\Obj 0, y)))
\end{align*}
Take a moment to think about how this works: now we don't start with
``if scanning square~$x$ \dots'' but: ``if scanning square $x+1$
\dots'' A move to the left means that in the next step the machine is
scanning square~$x$.  But the square that is written on is~$x+1$.  We
do it this way since we don't have subtraction or a predecessor
function.

Note that numbers of the form $x+1$ are $1$, $2$, \dots, i.e., this
doesn't cover the case where the machine is scanning square~$0$ and is
supposed to move left (which of course it can't---it just stays
put). That special case is covered by the second conjunction: it says
that if, after $y$ steps, the machine is scanning square~$0$ in state
$q_i$ and square~$0$ contains symbol~$\sigma$, then after $y+1$ steps
it's still scanning square~$0$, is now in state~$q_j$, the symbol on
square~$0$ is $\sigma'$, and the squares other than square~$0$ contain
the same symbols they contained ofter $y$~steps.
\item \ollabel{rep-stay} For every instruction $\delta(q_i, \sigma) =
  \tuple{q_j, \sigma', \TMstay}$, the !!{sentence}:
\begin{align*}
& \lforall[x][\lforall[y][(
   (\Obj Q_{q_i}(x, y) \land \Obj S_{\sigma}(x, y)) \lif {}]] \\
&\qquad   (\Obj Q_{q_j}(x, y') \land \Obj S_{\sigma'}(x, y') \land
!A(x, y)))
\end{align*}
\end{enumerate}
\end{enumerate}
Let $!T(M, w)$ be the conjunction of all the above !!{sentence}s for Turing
machine~$M$ and input~$w$

In order to express that $M$ eventually halts, we have to find a
!!{sentence} that says ``after some number of steps, the transition
function will be undefined.''  Let $X$ be the set of all pairs
$\tuple{q, \sigma}$ such that $\delta(q, \sigma)$ is undefined.  Let
$!E(M, w)$ then be the !!{sentence}
\[
\lexists[x][\lexists[y][(\bigvee_{\tuple{q, \sigma} \in
      X}(\Obj Q_q(x, y) \land \Obj S_\sigma(x, y)))]]
\]

If we use a Turing machine with a designated halting state~$h$, it
is even easier: then the !!{sentence}~$!E(M, w)$
\[
\lexists[x][\lexists[y][\Obj Q_h(x, y)]]
\]
expresses that the machine eventually halts.

\begin{prop}
\ollabel{prop:mlessk}
If $m < k$, then $!T(M, w) \Entails \num{m} < \num{k}$
\end{prop}

\begin{proof}
Exercise.
\end{proof}

\begin{prob}
Prove \olref[tms][und][rep]{prop:mlessk}.
(Hint: use induction on $k-m$).
\end{prob}

\end{document}
