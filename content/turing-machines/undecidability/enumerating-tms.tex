% Part: turing-machines
% Chapter: undecidability
% Section: enumerating-tms

\documentclass[../../../include/open-logic-section]{subfiles}

\begin{document}

\olfileid{tur}{und}{enu}
\olsection{Enumerating Turing Machines}

\begin{explain}
We can show that the set of all Turing machines is !!{enumerable}.
This follows from the fact that each Turing machine can be finitely
described.  The set of states and the tape vocabulary are finite sets.
The transition function is a partial function from $Q \times \Sigma$
to $Q \times \Sigma \times \{\TMleft, \TMright, \TMstay\}$, and so
likewise can be specified by listing its values for the finitely many
argument pairs for which it is defined.

This is true as far as it goes, but there is a subtle difference. The
definition of Turing machines made no resriction on what !!{element}s
the set of states and tape alphabet can have. So, e.g., for every real
number, there technically is a Turing machine that uses that number as
a state. However, the \emph{behavior} of the Turing machine is
independent of which objects serve as states and vocabulary. Consider
the two Turing machines in \olref{fig:variants}.
\begin{figure}
\begin{center}
\begin{tikzpicture}[->,>=stealth',shorten >=1pt,auto,node distance=2.8cm,
                    semithick]
  \tikzstyle{every state}=[fill=none,draw=black,text=black]

  \node[initial,state]         (A)              {$q_0$};
  \node[state]         (B) [right of=A] {$q_1$};

  \path (A) edge [bend left] node {\TMtrans{\TMstroke}{\TMstroke}{\TMright}} (B)
        (B) edge [loop above] node {\TMtrans{\TMblank}{\TMblank}{\TMright}} (B)
            edge [bend left] node {\TMtrans{\TMstroke}{\TMstroke}{\TMright}} (A);
\end{tikzpicture}\\
\begin{tikzpicture}[->,>=stealth',shorten >=1pt,auto,node distance=2.8cm,
  semithick]
\tikzstyle{every state}=[fill=none,draw=black,text=black]

\node[initial,state]         (A)              {$s$};
\node[state]         (B) [right of=A] {$h$};

\path (A) edge [bend left] node {\TMtrans{A}{A}{\TMright}} (B)
(B) edge [loop above] node {\TMtrans{\TMblank}{\TMblank}{\TMright}} (B)
edge [bend left] node {\TMtrans{A}{A}{\TMright}} (A);
\end{tikzpicture}
\end{center}
\caption{Variants of the \emph{Even} machine}
\ollabel{fig:variants}
\end{figure}
These two diagrams correspond to two machines, $M$ with the tape
alphabet $\Sigma = \{\TMendtape,\TMblank,\TMstroke\}$ and set of
states $\{q_0,q_1\}$, and $M'$ with alphabet $\Sigma' =
\{\TMendtape,\TMblank,A\}$ and states $\{s,h\}$. But their
instructions are otherwise the same: $M$ will halt on a sequence of
$n$ $\TMstroke$'s iff $n$ is even, and $M'$ will halt on a sequence of
$n$ $A$'s iff $n$ is even. All we've done is rename $\TMstroke$
to~$A$, $q_0$ to~$s$, and $q_1$ to~$h$. This example generalizes: we
can think of Turing machines as the same as long as one results from
the other by such a renaming of symbols and states.  In fact, we can
simply think of the symbols and states of a Turing machine as positive
integers: instead of $\sigma_0$ think~$1$, instead of $\sigma_1$
think~$2$, etc.; $\TMendtape$ is~$1$, $\TMblank$ is~$2$, etc. In this
way, the \emph{Even} machine becomes the machine depicted in
\olref{fig:standard-even}.
\begin{figure}
\[\begin{tikzpicture}[->,>=stealth',shorten >=1pt,auto,node distance=2.8cm,
  semithick]
\tikzstyle{every state}=[fill=none,draw=black,text=black]

\node[initial,state]         (A)              {$1$};
\node[state]         (B) [right of=A] {$2$};

\path (A) edge [bend left] node {\TMtrans{3}{3}{\TMright}} (B)
(B) edge [loop above] node {\TMtrans{2}{2}{\TMright}} (B)
edge [bend left] node {\TMtrans{3}{3}{\TMright}} (A);
\end{tikzpicture}
\]
\caption{A standard \emph{Even} machine}
\ollabel{fig:standard-even}
\end{figure}
We might call a Turing machine with states and symbols that are
positive integers a \emph{standard} machine, and only consider
standard machines from now on.\footnote{The terminology ``standard
machine'' is not standard.}

We wanted to show that the set of Turing machines is !!{enumerable},
and with the above considerations in mind, it is enough to show that
the set of standard Turing machines is !!{enumerable}. Suppose we are
given a standard Turing machine $M = \tuple{Q, \Sigma, q_0, \delta}$.
How could we describe it using a finite string of positive integers?
We'll first list the number of states, the states themselves, the
number of symbols, the symbols themselves, and the starting state.
(Remember, all of these are positive integers, since $M$ is a standard
machine.)  What about~$\delta$? The set of possible arguments, i.e.,
pairs $\tuple{q,\sigma}$, is finite, since $Q$ and~$\Sigma$ are
finite. So the information in~$\delta$ is simply the finite list of
all $5$-tuples $\tuple{q, \sigma, q', \sigma', d}$ where
$\delta(q,\sigma) = \tuple{q', \sigma', D}$, and $d$ is a number that
codes the direction~$D$ (say, $1$ for~$\TMleft$, $2$ for~$\TMright$,
and $3$ for~$\TMstay$).

In this way, every standard Turing machine can be described by a
finite list of positive integers, i.e., as a sequence $s_M \in
(\PosInt)^*$. For instance, the standard \emph{Even} machine is coded
by the sequence
\[
2, \underbrace{1, 2}_Q, 3, \overbrace{1, 2, 3}^\Sigma, 1, \underbrace{1, 3, 2, 3, 2}_{\delta(1,3) = \tuple{2,3,R}}, 
\overbrace{2, 2, 2, 2, 2}^{\delta(2,2) = \tuple{2,2,R}},
\underbrace{2, 3, 1, 3, 2}_{\delta(2,3) = \tuple{1,3,R}}.
\]
\end{explain}

\begin{thm}
There are functions from $\Nat$ to~$\Nat$ which are not Turing
computable.
\end{thm}

\begin{proof}
We know that the set of finite sequences of positive
integers~$(\PosInt)^*$ is !!{enumerable}
(\cref{sfr:siz:zigzag:prob:posint-star}). This gives us that the set
of descriptions of standard Turing machines, as a subset
of~$(\PosInt)^*$, is itself enumerable.  Every Turing computable
function $\Nat$ to~$\Nat$ is computed by some (in fact, many) Turing
machines. By renaming its states and symbols to positive integers (in
particular, $\TMendtape$ as~$1$, $\TMblank$ as~$2$, and $\TMstroke$
as~$3$) we can see that every Turing computable function is computed
by a standard Turing machine. This means that the set of all Turing
computable functions from $\Nat$ to~$\Nat$ is also enumerable.

On the other hand, the set of all functions from $\Nat$ to~$\Nat$ is
not !!{enumerable} (\cref{sfr:siz:red:prob:nat-nat}). If all functions
were computable by some Turing machine, we could enumerate the set of
all functions by listing all the descriptions of Turing machines that
compute them. So there are some functions that are not Turing
computable. 
\end{proof}

\begin{prob}
  Can you think of a way to describe Turing machines that does not
  require that the states and alphabet symbols are explicitly listed?
  You may define your own notion of ``standard'' machine, but say
  something about why every Turing machine can be computed by a
  ``standard'' machine in your new sense.
\end{prob}

\end{document}
