% Part: counterfactuals
% Chapter: minimal-change-semantics
% Section: transitivity

\documentclass[../../../include/open-logic-section]{subfiles}

\begin{document}

\olfileid{cnt}{min}{cpo}

\olsection{Contraposition}

Material and strict conditionals are equivalent to their
contrapositives. Counterfactuals are not.  Here is an example due to
Kratzer:
\begin{quote}
  If Goethe hadn't died in 1832, he would (still) be dead now.

  If Goethe weren't dead now, he would have died in 1832.
\end{quote}
The first sentence is true: humans don't live hundreds of years.  The
second is clearly false: if Goethe weren't dead now, he would be still
alive, and so couldn't have died in 1832.

\begin{ex}\ollabel{ex:contraposition-counterex}
  The sphere semantics invalidates contraposition, i.e., we have $p
  \cif q \Entails/ \lnot q \cif \lnot p$. Consider the model
  $\mModel{M} = \tuple{W, O, V}$ where $W = \{w, w_1, w_2\}$, $O_w =
  \{\{w\}, \{w, w_1\}, \{w, w_1, w_2\}\}$, $V(p) = \{w_1, w_2\}$ and
  $V(q) = \{w, w_1\}$. There is a $p$-admitting sphere $S = \{w,
  w_1\}$ and $p \lif q$ is true at all worlds in it, so $\mSat{M}{p
    \cif q}[w]$. However, the $\lnot q$-admitting sphere $\{w, w_1,
  w_2\}$ contains a world, namely~$w_2$, where $q$ is false and $p$ is
  true, so $\mSat/{M}{\lnot q \lif \lnot p}[w_2]$.
\end{ex}

\begin{prob}
Draw the sphere diagram corresponding to the counterexample in
\olref[cnt][min][cpo]{ex:contraposition-counterex}.
\end{prob}

\end{document}
