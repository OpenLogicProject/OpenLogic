% Part: counterfactuals
% Chapter: minimal-change-semantics
% Section: agglomeration

\documentclass[../../../include/open-logic-section]{subfiles}

\begin{document}

\olfileid{cnt}{min}{agg}

\olsection{Agglomeration}

Agglomeration, or strengthening the antecedent, refers to the
inference $!A \lif !C \Entails (!A \land !B) \lif !C$.  It is valid
for the material conditional, but invalid for counterfactuals. Suppose
if is true that if I were to strike this match, it would light. (That
means, there is nothing wrong with the match or the matchbook surface,
I will not break the match, etc.) But it is not true that if I were to
light this match in outer space, it would light. So the following
inference is invalid:
\begin{quote}
  I the match were struck, it would light.

  Therefore, if the match were struck in outer space, it would light.
\end{quote}

The Lewis-Stalnaker account of conditionals explains this: the closest
world where I light the match and I do so in outer space is much
further removed from the actual world than the closes world where I
light the match is. So although it's true that the match lights in the
latter, it is not in the former. And that is as it schould be.

\begin{ex}
  The sphere semantics invalidates the inference, i.e., we have $p
  \cif r \Entails/ (p \land q) \cif r$. Consider the model $\mModel{M}
  = \tuple{W, O, V}$ where $W = \{w, w_1, w_2\}$, $O_w = \{\{w\}, \{w,
  w_1\}, \{w, w_1, w_2\}\}$, $V(p) = \{w_1, w_2\}$, $V(q) = \{w_2\}$,
  and $V(r) = \{w_1\}$. There is a $p$-admitting sphere $S = \{w,
  w_1\}$ and $p \lif r$ is true at all worlds in it, so $\mSat{M}{p
    \cif r}[w]$. There is also a $(p \land q)$-admitting sphere $S' =
  \{w, w_1, w_2\}$ but $\mSat/{M}{(p \land q) \lif r}[w_2]$, so
  $\mSat/{M}{(p \land q) \cif r}[w]$ (see \olref{fig:agglomeration}).

\begin{figure}
\begin{center}
\begin{tikzpicture}[modal]
  \draw[dotted] (0,0) circle (.8)  circle (2.2) circle (3.6);
  \path (0,0) node[world] {$w$};
  \begin{scope}
    \draw[clip] plot [smooth,tension=1.5] coordinates { (5,.5) (2.2,-1) (5,-2.9)};
    \fill[fill=lightgray] (0,0) circle (3.6);
  \end{scope}
  \begin{scope}
    \draw[clip] plot [smooth,tension=1.5] coordinates { (1.5,3.6) (1,0) (5,-3.6)};
    \fill[fill=lightgray] (0,0) circle (2.2);
  \end{scope}
  \draw plot [smooth,tension=1] coordinates { (4,3.6) (.9,.5) (5,1.2)};
  \path (2.5,3.6) node[below] {$p$};
  \path (5,2.5) node[left] {$r$};
  \path (5,-.8) node[left] {$q$};
  \path (5,-1.3) node[left] {$p \land q$};
  \path (1.3,.8) node[world] {$w_1$};
  \path (2.7,-1) node[world] {$w_2$};
\end{tikzpicture}
\caption{Counterexample to agglomeration}
\ollabel{fig:agglomeration}
\end{center}
\end{figure}
\end{ex}

\end{document}
