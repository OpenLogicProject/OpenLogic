% Part: counterfactuals
% Chapter: introduction
% Section: paradoxes-material

\documentclass[../../../include/open-logic-section]{subfiles}

\begin{document}

\olfileid{cnt}{int}{par}

\olsection{Paradoxes of the Material Conditional}

One of the first to criticize the use of $!A \lif !B$ as a way to
symbolize ``if \dots then \dots'' statements of English was
C.~I. Lewis. Lewis was criticizing the use of the material condition
in Whitehead and Russell's \emph{Principia Mathematica}, who
pronounced $\lif$ as ``implies.''  Lewis rightly complained that if
$\lif$ meant ``implies,'' then any false proposition~$p$ implies that
$p$ implies~$q$, since $p \lif (p \lif q)$ is true if~$p$ is false,
and that any true proposition~$q$ implies that $p$ implies~$q$, since
$q \lif (p \lif q)$ is true if $q$~is true.

Logicians of course know that \emph{implication}, i.e., logical
entailment, is not a connective but a relation between !!{formula}s or
statements. So we should just not read $\lif$ as ``implies'' to avoid
confusion.\footnote{Reading ``$\lif$'' as ``implies'' is still widely
  practised by mathematicians and computer scientists, although
  philosophers try to avoid the confusions Lewis highlighted by
  pronouncing it as ``only if.''} As long as we don't, the particular
worry that Lewis had simply does not arise: $p$ does not ``imply'' $q$
even if we think of $p$ as standing for a false English sentence. To
determine if $p \Entails q$ we must consider \emph{all}
!!{valuation}s, and $p \Entails/ q$ even when we use $p$ to symbolize
a sentence which happens to be false.

But there is still something odd about ``if \dots then\dots''
statements such as Lewis's
\begin{quote}
If the moon is made of green cheese, then $2+2=4$.
\end{quote}
and about the inferences
\begin{quote}
  The moon is not made of green cheese. Therefore, if the moon is made
  of green cheese, then $2+2=4$.

  $2+2 = 4$. Therefore, if the moon is made
  of green cheese, then $2+2=4$.
\end{quote}
Yet, if ``if \dots then \dots'' were just $\lif$, the sentence would
be unproblematically true, and the inferences unproblematically valid.

Another example of concerns the tautology $(!A \lif !B) \lor (!B \lif
!A)$.  This would suggest that if you take two indicative
sentences~$S$ and $T$ from the newspaper at random, the sentence ``If
$S$ then $T$, or if $T$ then~$S$'' should be true.

\end{document}
