% Part: counterfactuals
% Chapter: introduction
% Section: counterfactuals

\documentclass[../../../include/open-logic-section]{subfiles}

\begin{document}

\olfileid{cnt}{int}{cnt}

\olsection{Counterfactuals}

A very common and important form of ``if \dots then \dots''
constructions in English are built using the past subjunctive form of
\emph{to be}: ``if it were the case that \dots then it would be the
case that \dots'' Because usually the antecedent of such a conditional
is false, i.e., counter to fact, they are called \emph{counterfactual
  conditionals} (and because they use the subjunctive for of \emph{to
  be}, also \emph{subjunctive conditionals}. They are distinguished
from \emph{indicative} conditionals which take the form of ``if it is
the case that \dots then it is the case that \dots'' Counterfactual and
indicative conditionals differ in truth conditions. Consider Adams's
famous example:
\begin{quote}
  If Oswald didn't kill Kennedy, someone else did.
  
  If Oswald hadn't killed Kennedy, someone else would have.
\end{quote}
The first is indicative, the second counterfactual. The first is
clearly true: we know JFK was killed by someone, and if that someone
wasn't (contrary to the Warren Report) Lee Harvey Oswald, then someone
else killed JFK. The second one says something different. It claims
that if Oswald hadn't killed Kennedy, i.e., if the Dallas shooting had
been avoided or had been unsuccessful, history would have subsequently
unfolded in such a way that another assassination would have been
successful. In order for it to be true, it would have to be the case
that powerful forces had conspired to ensure JFK's death (as many JFK
conspiracy theorists believe).

It is a live debate whether the \emph{indicative} conditional is
correctly captured by the material conditional, in particular, whether
the paradoxes of the material conditional can be ``explained'' in a
way that is compatible with it giving the truth conditions for English
indicative conditionals. By contrast, it is uncontroversial that
counterfactual conditionals cannot be symbolized correctly by the
material conditionals. That is clear because, even though generally
the antecedents of counterfactuals are false, not all counterfactuals
with false antecedents are true---for instance, if you believe the
Warren Report, and there was no conspiracy to assassinate JFK, then
Adams's counterfactual conditional is an example.

Counterfactual conditionals play an important role in causal
reasoning: a prime example of use of counterfactuals is to express
causal relationships. E.g., striking a match causes it to light, and
you can express this by saying ``if this match were struck, it would
light.''  Material, and generally indicative conditionals, cannot be
used to express this: ``the match is struck $\lif$ the match lights''
is true if the match is never struck, regardless of what would happen
if it were. Even worse, ``the match is struck $\lif$ the match turns
into a bouquet of flowers'' is also true if it is never struck, but
the match would certainly not turn into a bouquet of flowers if it
were struck.

What exactly the correct logic of counterfactuals is, is still
debated. An influential analysis of counterfactuals was given by
Stalnaker and Lewis.  According to them, a counterfactual ``if it were
the case that~$S$ then it would be the case that~$T$'' is true iff $T$
is true in the counterfactual situation (``possible world'') that is
closest to the way the actual world is and where~$S$ is true. This is
called an ``ontic'' analysis, sine it makes reference to an ontology
of possible worlds. Other analyses make use of conditional
probabilities or theories of belief revision.  There is a
proliferation of different proposed logics of counterfactuals. There
isn't even a single Lewis-Stalnaker logic of counterfactuals: even
though Stalnaker and Lewis proposed accounts along similar lines with
reference to closest possible worlds, the assumptions they made result
in different valid inferences.



\end{document}
