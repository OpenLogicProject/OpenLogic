% Part: counterfactuals
% Chapter: introduction
% Section: material-conditional

\documentclass[../../../include/open-logic-section]{subfiles}

\begin{document}

\olfileid{cnt}{int}{mat}

\olsection{The Material Conditional}

In its simplest form in English, a conditional is a sentence of the
form ``If \dots then \dots,'' where the \dots{} are themselves
sentences, such as ``If the butler did it, then the gardener is
innocent.'' In introductory logic courses, we earn to symbolize
conditionals using the $\lif$ connective: symbolize the parts
indicated by \dots, e.g., by !!{formula}s $!A$ and~$!B$,
and the entire conditional is symbolized by $!A \lif !B$.

The connective $\lif$ is \emph{truth-functional}, i.e., the truth
value---$\True$ or $\False$---of $!A \lif !B$ is determined by the truth
values of $!A$ and~$!B$: $!A \lif !B$ is true iff $!A$~is false or
$!B$~is true, and false otherwise. Relative to a truth value
assignment~$\pAssign v$, we define $\pSat{v}{!A \lif !B}$ iff
$\pSat/{v}{!A}$ or $\pSat{v}{!B}$. The connective $\lif$ with this
semantics is called the \emph{material conditional}.

This definition results in a number of elementary logical facts. First
of all, the deduction theorem holds for the material conditional:
\begin{align}
  \text{If } \Gamma, !A \Entails !B \text{ then } \Gamma \Entails !A \lif !B
\end{align}
It is truth-functional: $!A \lif !B$ and $\lnot !A \lor !B$ are equivalent:
\begin{align}
  !A \lif !B & \Entails \lnot !A \lor !B\\
  \lnot !A \lor !B & \Entails !A \lif !B
  \intertext{A material conditional is entailed by its consequent and
    by the negation of its antecedent:}
  !B & \Entails !A \lif !B\\
  \lnot !A & \Entails !A \lif !B
  \intertext{A false material conditional is equivalent to the
    conjunction of its antecedent and the negation of its consequent:
    if $!A \lif !B$ is false, $!A \land \lnot !B$ is true, and vice versa:}
  \lnot(!A \lif !B) & \Entails !A \land \lnot !B\\
  !A \land \lnot !B & \Entails \lnot(!A \lif !B)
  \intertext{The material conditional supports modus ponens:}
  !A, !A \lif !B & \Entails !B
  \intertext{The material conditional agglomerates:}
  !A \lif !B,  !A \lif !C & \Entails !A \lif (!B \land !C)
  \intertext{We can always strengthen the antecedent, i.e., the
    conditional is \emph{monotonic}:}
  !A \lif !B & \Entails (!A \land !C) \lif !B
  \intertext{The material conditional is transitive, i.e., the chain
    rule is valid:}
  !A \lif !B, !B \lif !C & \Entails !A \lif !C
  \intertext{The material conditional is equivalent to its
    contrapositive:}
  !A \lif !B & \Entails \lnot !B \lif \lnot !A\\
  \lnot !B \lif \lnot !A & \Entails !A \lif !B
\end{align}

These are all useful and unproblematic inferences in mathematical
reasoning. However, the philosophical and linguistic literature is
replete with purported counterexamples to the equivalent inferences in
non-mathematical contexts. These suggest that the material
conditional~$\lif$ is not---or at least not always---the appropriate
connective to use when symbolizing English ``if \dots then \dots''
statements.

\end{document}
