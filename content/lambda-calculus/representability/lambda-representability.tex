% Part: lambda-calculus
% Chapter: representability
% Section: lambda-representability

\documentclass[../../../include/open-logic-section]{subfiles}

\begin{document}

\olfileid{lc}{rep}{lr}
\olsection{Lambda representability}

We introduce the concept of lambda representability(or
lambda definability) as folllws:
\begin{defn}
Let $f(x_0, \dots, x_{n-1})$ be an $n$-ary partial function from $\Nat$
to $\Nat$. We say a lambda term $M$ \emph{represents} $f$ if for every
sequence of natural numbers $m_0$, \dots,~$m_{n-1}$,
\[
M ~ \num{m_0} ~ \num{m_1} \dots \num{m_{n-1}} \red \num{f(m_0, m_1, \dots,
  m_{n-1})}
\]
if $f(m_0, \dots, m_{n-1})$ is defined, and $M ~ \num{m_0} ~ \num{m_1}
\dots \num{m_{n-1}}$ has no normal form otherwise. 
\end{defn}

As we use $\num{n}$ for the term representing the number $n$, we will
use $\num{f(x_0, \dots, x_{n-1})}$ for terms representing this
function $f$.

Note that there is more than one definition of representing functions in lambda
calculus, among which the above one is the earliest and most natural,
due to Kleene and Church. However, as we will see, this definition has
some limitations, leading to other definitions, such as
Barendregt (undefined being unsolvable) and Statman (undefined belonging
to a co-Viser set). But we adopt this definition anyway as it requires
minimal prerequisite.

\end{document}
