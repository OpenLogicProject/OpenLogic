% Part: lambda-calculus
% Chapter: representability
% Section: partial-recursive-functions

\documentclass[../../../include/open-logic-section]{subfiles}

\begin{document}

\olfileid{lc}{rep}{par}
\olsection{Partial recursive functions}

In this section we will discuss the representation of partial
recursive functions. As in the last section of general recursive
functions, we can again use Kleene's Normal Theorem to simplify our
reasoning. The only difference is that the primitive recursive
function $T(e,x,s)$ that the unbounded search applied on is no longer
necessary regular, bringing us troubles. For example, we may want to
reuse the representation for composition of primitive recursive
functions after unbounded search:
\begin{align*}
  \num{U(\mu s \; T(e,x,s))} &= \lambd[x][\num{U} \num{\mu s \; T(e,x,s)}]
\end{align*}
however this is problematic: consider when $\mu s \; T(e,x,s)$ is
undefined, meaning $\num{\mu s \; T(e,x,s)}$ has no normal
form (assume our previous representation of unbounded search works
here); In most cases this makes $\num{U} \num{\mu s \; T(e,x,s)}$ has
no normal forms either, which is what we want; but consider when
$\num{U}$ is $\lambd[x][\lambd[y][y]]$, in which case $\num{U}
\num{\mu s \; T(e,x,s)}$ has normal form of $\lambd[y][y]$, not
satisfying our definition of representability.

This problem has been studied and some workarounds has be invented, 
which are, however, somewhat more complicated and less intuitive than the
approach we have taken. As this is a introductory text, we only hope to
give the reader a taste of how lambda calculus represents things, thus
these workarounds are omitted here. Interested reader can find other
reference texts for the actual working method.
\end{document}

