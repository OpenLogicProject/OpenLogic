% Part: lambda-calculus % Chapter: introduction % Section: currying

\documentclass[../../../include/open-logic-section]{subfiles}

\begin{document}

\olfileid{lc}{rep}{cur} 
\olsection{Currying}

Before we move on to the representation for natural numbers, pairs,
functions, we first have to resolve a practical problem:
the reader may have noticed that functions given by \olref[syn]{def:2} seem to only accept one argument, which
is quite a limitation when we want to define function accepting multiple
arguments.

Recall that in set theory, we use $A \times B$ (the Cartesian product
of $A$ and $B$, \olref[sfr][set][pai]) to denote the domain of functions accepting two
arguments from sets $A$ and $B$ respectively, so the functions accepts
a pair containing two arguments. While in lambda calculus we can mimic
this method by encoding pairs (as shown in \olref[lc][rep][pai]{sec}),
it is more a convention to do this by \emph{currying}.

Let's see some examples. If we want to define a function that accepts
two arguments and returns the first, we write
$\lambd[x][\lambd[y][x]]$, which literally is a function that accepts
an argument and returns a function that, accepts another argument and 
returns the first argument while drops the second. Let's see what
happens when we apply two arguments to it:
\begin{align*}
  &(\lambd[x][\lambd[y][x]])MN \\
  \bredone&(\lambd[y][M])N \\
  \bredone&M
\end{align*}

In general, to write a function with parameters $x_1, \ldots, x_n$
(assuming all distinct to each other)
and function body $N$, we can write
$\lambd[x_1][\lambd[\ldots\lambd[x_n][N]]]$. If we apply $n$ arguments
to it:

\begin{align*}
  &(\lambd[x_1][\lambd[\ldots\lambd[x_n][N]]]) M_1 \dots M_n \\
  \bredone&(\Subst{(\lambd[x_2][\ldots\lambd[x_n][N]])}{M_1}{x_1}) M_2
  \dots M_n\\
  \eqs&(\lambd[x_2][\ldots\lambd[x_n][\Subst{N}{M_1}{x_1}]]) M_2
            \dots M_n \\
  \ldots \\
  \bredone&\Subst{\Subst{P}{M_1}{x_1}\ldots}{M_n}{x_n}
\end{align*}
the last line literally means substituting $M_i$ for $x_i$ in the body
of the function definition, which is exactly what we want when
applying a multiple arguments function.
\end{document}

