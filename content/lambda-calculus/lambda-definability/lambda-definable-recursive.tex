% Part: lambda-calculus
% Chapter: lambda-definability
% Section: lambda-definable-recursive

\documentclass[../../../include/open-logic-section]{subfiles}

\begin{document}

\olfileid{lam}{dfl}{ldr}
\olsection{\usetoken{S}{lambda definable} Functions are Recursive}

Not only are all partial recursive functions !!{lambda definable}, the
converse is true, too. That is, all !!{lambda definable} functions are
partial recursive.

\begin{thm}
  \ollabel{thm:lambda-computable} If a partial function $f$ is
  !!{lambda definable}, it is partial recursive.
\end{thm}

\begin{proof}
  We only sketch the proof. First, we arithmetize $\lambd$-terms,
  i.e., systematially assign G\"odel numbers to $\lambd$-terms, using
  the usual power-of-primes coding of sequences. Then we define a
  partial recursive function $\fn{normalize}(t)$ operating on the
  G\"odel number~$t$ of a lambda term as argument, and which returns
  the G\"odel number of the normal form if it has one, or is undefined
  otherwise.  Then define two partial recursive functions
  $\fn{toChurch}$ and $\fn{fromChurch}$ that maps natural numbers to
  and from the G\"odel numbers of the corresponding Church numeral.

  Using these recursive functions, we can define the function~$f$ as a
  partial recursive function. There is a $\lambd$-term~$F$ that
  !!{lambda define}s~$f$. To compute $f(n_1, \dots, n_k)$, first obtain
  the G\"odel numbers of the corresponding Church numerals using
  $\fn{toChurch}(n_i)$, append these to $\Gn{F}$ to obtain the G\"odel
  number of the term $F \num{n_1}\dots\num{n_k}$. Now use
  $\fn{normalize}$ on this G\"odel number. If $f(n_1, \dots, n_k)$ is
  defined, $F \num{n_1}\dots\num{n_k}$ has a normal form (which must
  be a Church numeral), and otherwise it has no normal form (and so
  \[\fn{normalize}(\Gn{F\num{n_1}\dots\num{n_k}})\] is undefined).
  Finally, use $\fn{fromChurch}$ on the G\"odel number of the
  normalized term.
\end{proof}

\end{document}
