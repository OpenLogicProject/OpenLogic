% Part: lambda-calculus
% Chapter: lambda-definability
% Section: lists

\documentclass[../../../include/open-logic-section]{subfiles}

\begin{document}

\olfileid{lam}{ldf}{lst}
\olsection{Lists}

The list $[M_1, M_2, \ldots, M_n]$ can be defined to be:
\[
  [M_1, M_2, \ldots, M_n] \ident \lambd[sz][s M_1 (s M_2 (\ldots (s M_n z)))]
\]
Informally speaking, the term representing a list is the
``accumulator'' function of the list: a function that accepts two
terms; the first is a function that receives accumulated value along
with a new element and returns the new accumulated value; the second
is the initial accumulated value; it accumulates elements one by one
and finally returns the result.

\begin{digress}
  If you are familiar with functional programming you will find the
  definition similar to the \emph{right fold}: a list is defined as
  the right fold function of the elements it contains.
\end{digress}

We can easily define some useful functions based on this encoding; 
\begin{align*}
  \fn{Sum} & \ident
  \lambd[l][l \, (\lambd[xa][\fn{Add} \, x \, a])\,  \num{0}]\\
  \fn{Len} & \ident
  \lambd[l][l \, (\lambd[xa][\fn{Add} \, a \, \num{1}]) \num{0}]
\end{align*}

$\fn{Sum}$ calculates the sum of a list of church numerals. It works
by doing an accumulation on the list, where the initial value is
$\num{0}$ and for each element, return $\fn{Add} \, x\, a$ as the new
value (where $x$ is the current element and $a$ is the accumulated
value). The result is the sum of all elements.

\begin{prob}
  What does $\fn{Len}$ do? Explain.
\end{prob}

\begin{prob}
  Define a function that reverses a list.
\end{prob}
\end{document}

