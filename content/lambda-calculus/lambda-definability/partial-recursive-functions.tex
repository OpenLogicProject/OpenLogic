% Part: lambda-calculus
% Chapter: lambda-definability
% Section: partial-recursive-functions

\documentclass[../../../include/open-logic-section]{subfiles}

\begin{document}

\olfileid{lam}{ldf}{par}
\olsection{Partial Recursive Functions are \usetoken{S}{lambda definable}}

Partial recursive functions are those obtained from the basic
functions by composition, primitive recursion, and unbounded
minimization. They differ from general recursive function in that the
functions used in unbounded search are not required to be regular.
Not requiring regularity means that functions defined by minimization
may sometimes not be defined. 

At first glance it might seem that the same methods used to show that
the (total) general recursive functions are all !!{lambda definable} can
be used to prove that all partial recursive functions are
!!{lambda definable}.  For instance, the composition of $f$ with $g$ is
!!{lambda define}d by $\lambd[x][F (G x)]$ if $f$ and $g$ are
!!{lambda define}d by terms $F$ and~$G$, respectively. However, when the
functions are partial, this is problematic. When $g(x)$ is undefined,
meaning $G x$ has no normal form. In most cases this means that $F (G
x)$ has no normal forms either, which is what we want.  But consider
when $F$ is $\lambd[x][\lambd[y][y]]$, in which case $F (G x)$ does
have a normal form ($\lambd[y][y]$). 

This problem is not insurmountable, and there are ways to
!!{lambda define} all partial recursive functions in such a way that
undefined values are represented by terms without a normal form.
These ways are, however, somewhat more complicated and less intuitive
than the approach we have taken for general recursive functions. We
record the theorem here without proof:

\begin{thm}
  All partial recursive functions are !!{lambda definable}.
\end{thm}

\end{document}

