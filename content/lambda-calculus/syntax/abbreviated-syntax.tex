% Part: lambda-calculus
% Chapter: syntax
% Section: abbreviated-syntax

\documentclass[../../../include/open-logic-section]{subfiles}

\begin{document}

\olfileid{lam}{syn}{abb}
\olsection{Abbreviated Syntax}

Terms as defined in \olref[trm]{defn:term} are sometimes cumbersome to
write, so it is useful to introduce a more concise syntax. We must of
course be careful to make sure that the terms in the concise notation
also are uniquely readable.  One widely used version called
\emph{abbreviated terms} is as follows.

\begin{enumerate}
\item When parentheses are left out, application takes place from left
  to right. For example, if $M$, $N$, $P$, and~$Q$ are terms, then
  $MNPQ$ abbreviates $(((MN)P)Q)$.
\item Again, when parentheses are left out, lambda abstraction is
  given the widest scope possible. From example, $\lambd[x][MNP]$ is
  read as $(\lambd[x][MNP])$.
\item A lambda can be used to abstract multiple variables. For
  example, $\lambd[xyz][M]$ is short for
  $\lambd[x][\lambd[y][\lambd[z][M]]]$.
\end{enumerate}

For example,
\[
\lambd[xy][xxyx \lambd[z][xz]]
\]
abbreviates
\[
(\lambd[x][(\lambd[y][((((xx)y)x)(\lambd[z][(xz)]))])]).
\]

\begin{prob}
Expand the abbreviated term $\lambd[g][(\lambd[x][g (x x)])
  \lambd[x][g (x x)]]$.
\end{prob}

\end{document}
