% Part: lambda-calculus
% Chapter: introduction
% Section: free-variables

\documentclass[../../../include/open-logic-section]{subfiles}

\begin{document}

\olfileid{lam}{syn}{fv}
\olsection{Free Variables}

Lambda calculus is about functions, and lambda abstraction is how
functions arise. Intuitively, $\lambd[x][M]$ is the function with
values given by~$M$ when the argument to the function is assigned
to~$x$. But not every occurrence of~$x$ in~$M$ is relevant: if $M$
contains another abstract $\lambd[x][N]$ then the occurrences of $x$
in~$N$ are relevant to $\lambd[x][N]$ but not to $\lambd[x][M]$.  So,
a lambda abstract $\lambd[x]$ inside $\lambd[x][M]$ \emph{binds} those
occurrences of $x$ in~$M$ that are not already bound by another lambda
abstract---the \emph{free} occurrences of~$x$ in~$M$.

\begin{defn}[Scope]
If $\lambd[x][M]$ occurs inside a term~$N$, then the corresponding
occurrence of~$N$ is the \emph{scope} of the~$\lambd[x]$.
\end{defn}


\begin{defn}[Free and bound occurrence]
  An occurrence of variable $x$ in a term~$M$ is \emph{free} if it is
  not in the scope of a $\lambd[x]$, and \emph{bound} otherwise. An
  occurrence of a variable~$x$ in $\lambd[x][M]$ is bound by the initial
  $\lambd[x]$ iff the occurrence of $x$ in~$M$ is free.
\end{defn}

\begin{ex}
  In $\lambd[x][x y]$, both $x$ and $y$ are in the scope of $\lambd[x]$,
  so $x$ is bound by $\lambd[x]$. Since $y$ is not in the scope of
  any $\lambd[y]$, it is free.  In $\lambd[x][x x]$, both occurrences
  of $x$ are bound by~$\lambd[x]$, since both are free in $xx$.  In
  $((\lambd[x][xx])x)$, the last occurrence of~$x$ is free, since it
  is not in the scope of a $\lambd[x]$. In
  $\lambd[x][(\lambd[x][x])x]$, the scope of the first $\lambd[x]$ is
  $(\lambd[x][x])x$ and the scope of the second $\lambd[x]$ is the
  second-to-last occurrence of~$x$. In $(\lambd[x][x])x$, the last
  occurrence of $x$ is free, and the second-to-last is bound.  Thus,
  the second-to-last occurrence of $x$ in $\lambd[x][(\lambd[x][x])x]$
  is bound by the second $\lambd[x]$, and the last occurrence by the
  first $\lambd[x]$.
\end{ex}

For a term $P$, we can check all variable occurrences in it and get a
set of free variables. This set is denoted by $\FV{P}$ with a natural
definition as follows:

\begin{defn}[Free variables of a term] \ollabel{def:fv}
  The set of \emph{free variables} of a term is defined inductively by:
  \begin{enumerate}
    \item $\FV{x} = \{x\}$ \ollabel{def:fv1} 
    \item $\FV{\lambd[x][N]} = \FV{N} \setminus \{x\}$    \ollabel{def:fv2}
    \item $\FV{PQ} = \FV{P} \cup \FV{Q}$ \ollabel{def:fv3}
  \end{enumerate}
\end{defn}

\begin{prob}
  \begin{enumerate}
  \item Identify the scopes of $\lambd[g]$ and the two $\lambd[x]$ in
    this term: $\lambd[g][(\lambd[x][g (x x)]) \lambd[x][g (x x)]]$.
  \item In $\lambd[g][(\lambd[x][g (x x)]) \lambd[x][g (x x)]]$, are
    all occurrences of variables bound? By which abstractions are they
    bound respectively?
    \item Give $\FV{\lambd[x][(\lambd[y][(\lambd[z][x y]) z]) y]}$
  \end{enumerate}
\end{prob}

\begin{explain}
A free variable is like a reference to the outside world (the
\emph{environment}), and a term containing free variables can be seen
as a partially specified term, since its behaviour depends on how we
set up the environment. For example, in the term $\lambd[x][f x]$, which
accepts an argument~$x$ and returns $f$ of that argument, the
variable~$f$ is free. This value of the term is dependent on the
environment it is in, in particular the value of $f$ in that
environment.

If we apply abstraction to this term, we get $\lambd[f][\lambd[x][f
    x]]$. This term is no longer dependent on the environment variable
$f$, because it now designates a function that accepts two arguments
and returns the result of applying the first to the second. Changing
$f$ in the environment won't have any effect on the behavior of this
term, as the term will only use whatever is passed as an argument, and
not the value of $f$ in the environment.
\end{explain}

\begin{defn}[Closed term, combinator]
  A term with no free variables is called a \emph{closed term}, or a
  \emph{combinator}.
\end{defn}

\begin{lem}\ollabel{lem:fv}
  \begin{enumerate}
    \item \ollabel{lem:fv-abs} If $y \neq x$, then $y \in
      \FV{\lambd[x][N]}$ iff $y \in \FV{N}$.
    \item \ollabel{lem:fv-app} $y \in \FV{PQ}$ iff $y \in \FV{P}$ or $y
      \in \FV{Q}$.
    \end{enumerate}
\end{lem}

\begin{proof}
  Exercise.
\end{proof}

\begin{prob}
  Prove \olref[lam][syn][fv]{lem:fv}.
\end{prob}

\end{document}
