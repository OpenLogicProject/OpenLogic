% Part: lambda-calculus
% Chapter: syntax
% Section: terms

\documentclass[../../../include/open-logic-section]{subfiles}

\begin{document}

\olfileid{lam}{syn}{trm}
\olsection{Terms}

The terms of the lambda calculus are built up inductively from an
infinite supply of variables $\Obj{v_0}$, $\Obj{v_1}$, \dots, the
symbol~``$\lambd$'', and parentheses. We will use $x$, $y$, $z$, \dots{}
to designate variables, and $M$, $N$, $P$, \dots{} to desginate terms.

\begin{defn}[Terms] \ollabel{defn:term}
The set of \emph{terms} of the lambda calculus is defined inductively by:
\begin{enumerate}
  \item \ollabel{defn:term-var} If $x$ is a variable, then $x$ is a
    term.
  \item \ollabel{defn:term-abs} If $x$ is a variable and $M$ is a
    term, then $(\lambd[x][M])$ is a term.
  \item \ollabel{defn:term-app} If both $M$ and $N$ are terms, then
    $(MN)$ is a term.
\end{enumerate}
\end{defn}

If a term $(\lambd[x][M])$ is formed according to
\olref{defn:term-abs} we say it is the result of an
\emph{abstraction}, and the $x$ in $\lambd[x]$ is called a
\emph{!!{parameter}}.  A term $(MN)$ formed according to
\olref{defn:term-app} is the result of an \emph{application}.

The terms defined above are fully parenthesized. This can get rather
cumbersome, as the term
$(\lambd[x][((\lambd[x][x])(\lambd[x][(xx)]))])$ demnostrates. We will
introduce conventions for avoiding parentheses.  However, the official
definition makes it easy to determine how a term is constructed
according to \olref{defn:term}. For example, the last step of forming
the term $(\lambd[x][((\lambd[x][x])(\lambd[x][(xx)]))])$ must be
abstraction where the !!{parameter} is~$x$. It results by abstraction
from the term $((\lambd[x][x])(\lambd[x][(xx)]))$, which is an
application of two terms. Each of these two terms is the result of an
abstraction, and so on.

\begin{prob}
Describe the formation of $(\lambd[g][(\lambd[x][(g (x x))])
  (\lambd[x][(g (x x))])])$.
\end{prob}

\end{document}
