% Part: lambda-calculus
% Chapter: representability
% Section: currying

\documentclass[../../../include/open-logic-section]{subfiles}

\begin{document}

\olfileid{lam}{rep}{cur} 
\olsection{Currying}

A $\lambd$-abstract $\lambd[x][M]$ represents a function of one argument,
which is quite a limitation when we want to define function accepting
multiple arguments.  One way to do this would be by extending the
$\lambd$-calculus to allow the formation of pairs, triples, etc., in
which case, say, a three-place function $\lambd[x][M]$ would expect
its argument to be a triple.  However, it is more convenient to do
this by \emph{Currying}.

Let's consider an example. If we want to define a function that
accepts two arguments and returns the first, we write
$\lambd[x][\lambd[y][x]]$, which literally is a function that accepts
an argument and returns a function that accepts another argument and
returns the first argument while it drops the second. Let's see what
happens when we apply it to two arguments:
\begin{align*}
  (\lambd[x][\lambd[y][x]])MN 
  \bredone & (\lambd[y][M])N \\
  \bredone & M
\end{align*}

In general, to write a function with parameters $x_1$, \dots,~$x_n$
defined by some term~$N$, we can write
$\lambd[x_1][\lambd[x_2][\ldots\lambd[x_n][N]]]$. If we apply $n$ arguments
to it we get:
\begin{multline*}
  (\lambd[x_1][\lambd[x_2][\ldots\lambd[x_n][N]]]) M_1 \dots M_n \bredone\\
  \begin{aligned}
  \bredone {} & (\Subst{(\lambd[x_2][\ldots\lambd[x_n][N]])}{M_1}{x_1}) M_2
  \dots M_n\\
   \eqs {} & (\lambd[x_2][\ldots\lambd[x_n][\Subst{N}{M_1}{x_1}]]) M_2
            \dots M_n \\
  \vdots & \\
  \bredone {} &\Subst{\Subst{P}{M_1}{x_1}\ldots}{M_n}{x_n}
  \end{aligned}
\end{multline*}
The last line literally means substituting $M_i$ for $x_i$ in the body
of the function definition, which is exactly what we want when
applying multiple arguments to a function.
\end{document}

