% Part: lambda-calculus
% Chapter: introduction
% Section: computable-lambda

\documentclass[../../../include/open-logic-section]{subfiles}

\begin{document}

\olfileid{lam}{int}{lrp}
\olsection{Computable Functions are \usetoken{S}{lambda definable}}

\begin{thm}
\ollabel{thm:computable-lambda}
Every computable partial function is !!{lambda definable}.
\end{thm}

\begin{proof}
We need to show that every partial computable function~$f$ is
!!{lambda defined} by a lambda term~$F$. By Kleene's normal form
theorem, it suffices to show that every primitive recursive function
is !!{lambda defined} by a lambda term, and then that the functions
!!{lambda definable} are closed under suitable compositions and
unbounded search. To show that every primitive recursive function is
!!{lambda defined} by a lambda term, it suffices to show that the
initial functions are !!{lambda definable}, and that the partial
functions that are !!{lambda definable} are closed under
composition, primitive recursion, and unbounded search.
\end{proof}

We will use a more conventional notation to make the rest of the proof
more readable. For example, we will write $M(x, y, z)$ instead of
$Mxyz$. While this is suggestive, you should remember that terms in
the untyped lambda calculus do not have associated arities; so, for
the same term~$M$, it makes just as much sense to write $M(x,y)$ and
$M(x,y,z,w)$. But using this notation indicates that we are treating
$M$ as a function of three variables, and helps make the intentions
behind the definitions clearer. In a similar way, we will say ``define
$M$ by $M(x,y,z) = \dots$'' instead of ``define $M$ by $M =
\lambd[x][\lambd[y][\lambd[z][\dots]]]$.''

\end{document}
