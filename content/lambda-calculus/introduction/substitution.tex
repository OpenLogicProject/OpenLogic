% Part: lambda-calculus
% Chapter: introduction
% Section: pre-substitution

\documentclass[../../../include/open-logic-section]{subfiles}

\begin{document}

\olfileid{lc}{int}{sub}
\olsection{Substitution}

As mentioned earlier, free variables are references to environment variables
, thus it makes sense to actually use a specific value in the place of free
variables. For example, we may want to replace $f$ in
$\lambd[x][f x]$ with a specific term like the identity
$\lambd[y][y]$ resulting in $\lambd[x][(\lambd[y][y]) x]$. This is
called substitution.

We define the subtitution as follows:

\begin{defn}[Substitution] \ollabel{def}
  The \emph{substitution} of term $N$ for a variable $x$ in a term $M$, or
  $\Subst{M}{N}{x}$,  is defined inductively by:
  \begin{align}
    \Subst{x}{N}{x}       &= N \ollabel{def:1} \\
    \Subst{y}{N}{x}       &= y, \text{if} x \neq y \ollabel{def:2} \\
    \Subst{PQ}{N}{x} &= (\Subst{P}{N}{x}) (\Subst{Q}{N}{x}) \ollabel{def:3} \\
    \Subst{(\lambd[y][P])}{N}{x}  &= \lambd[y][\Subst{P}{N}{x}], \text{if} x \neq y
                                        \text{and} y \notin FV(N) \ollabel{def:4}
  \end{align}
  otherwise it is undefined.
\end{defn}

\olref{def:1} through \olref{def:3} are routine. The last line,
especially the conditions, is the meat. 

$x \neq y$ because we don't want to replace reference to parameters.
For example, if we force the substitution
$\Subst{\lambd[x][x]}{y}{x}$ we finally get $\lambd[x][y]$ which
is not what we want.

$y \notin FV(N)$? Because in this case $N$ contains references to the environment variable $y$, which,
after $N$ replaces $x$ in $P$, will be bound by the abstraction over
$y$. That will result in a term with different intuitive meaning, as
these $y$ no longer refer to the environment variables but instead
the parameter of that abstraction. 

For example, we cannot do $\Subst{\lambd[y][x]}{y}{x}$ because
otherwise it results in $\lambd[y][y]$, a term accepts an argument and
returns it directly. But the substitute $y$ is not likely refering
to the $y$ bound by $\lambd[y]$, but instead, more likely another
environment variable named $y$! So the result we actually want is a
function that accepts an argument, drop it, and returns the
environment variable $y$ anyway.

\begin{prob}
  Do the following substitution.
  \begin{enumerate}
  \item $\Subst{\lambd[y][x(\lambd[w][vwx])]}{(uv)}{x}$
  \item $\Subst{\lambd[y][x(\lambd[x][x])]}{(\lambd[y][xy])}{x}$
  \item $\Subst{y(\lambd[v][xv])}{(\lambd[y][vy])}{x}$
  \end{enumerate}
\end{prob}

\begin{thm} \ollabel{thm:notinfv}
  If $x \notin FV(M)$, then $FV(\Subst{M}{Q}{x}) = FV(M)$, if
  the left hand is defined.
\end{thm}
\begin{proof}
  By induction on the formation of $M$.
  \begin{enumerate}
    \item $M$ is of the form $\lambd[y][N]$, and
      since $\Subst{\lambd[y][N]}{Q}{x}$ is defined, it has to be
      $\lambd[y][\Subst{N}{Q}{x}]$; then $\Subst{N}{Q}{x}$ has
      to be defined; also, $x \neq y$ and $x \notin FV(Q)$. Then:
      \begin{align*}
        &FV(\Subst{\lambd[y][N]}{Q}{x}) \\
        =&FV(\lambd[y][\Subst{N}{Q}{x}]) && \text{by \olref{def:4}}\\
        =&FV(\Subst{N}{Q}{x}) \setminus \{y\} && \text{by
                                                     \olref[fv]{def:2}}
        \\
        =&FV(N) \setminus \{y\} && \text{by inductive hypothesis} \\
        =&FV(\lambd[y][N]) && \text{by \olref[fv]{def:2}}
      \end{align*}
    \item The other two cases are easy and left as exercises. \ollabel{thm:notinfv:2}
  \end{enumerate}
\end{proof}

\begin{prob}
  Prove \olref{thm:notinfv:2}.
\end{prob}

\begin{thm} \ollabel{thm:infv}
  If $y \in FV(M)$, then $FV(\Subst{M}{R}{y}) = (FV(M) \setminus
  {y}) \cup FV(R)$, provided the left hand is defined.
\end{thm}
\begin{proof}
  By induction on the formation of $M$.
  \begin{enumerate}
  \item If it is a varialbe, left as exercise. \ollabel{thm:infv:1}
  \item $M$ is of the form $\lambd[x][N]$. Since
    $\Subst{\lambd[x][N]}{R}{y}$ is defined, it has to be
    $\lambd[x][\Subst{N}{R}{y}]$, with $\Subst{N}{R}{y}$
    defined, $x \neq y$ and $x \notin FV(N)$; also, since $y \in
    FV(\lambd[x][N])$, we have $y \in FV(N)$ too; then:
    \begin{align*}
      &FV(\Subst{(\lambd[x][N])}{R}{y}) \\
      =&FV(\lambd[x][\Subst{N}{R}{y}]) \\
      =&FV(\Subst{N}{R}{y}) \setminus \{x\} \\
      =&((FV(N) \setminus \{y\}) \cup FV(R)) \setminus \{x\}
       && \text{by inductive hypothesis} \\
      =&(FV(N) \setminus \{y\} \setminus \{x\}) \cup FV(R)
       && x \notin FV(R) \\
      =&(FV(N) \setminus \{x\} \setminus \{y\}) \cup FV(R) \\
      =& (FV(\lambd[x][N]) \setminus \{y\}) \cup FV(R)
    \end{align*}
  \item If $M$ is of the form $PQ$. Since
    $\Subst{(PQ)}{R}{y}$ is defined, it has to be
    $(\Subst{P}{R}{y})(\Subst{Q}{R}{y})$ with both substitution
    defined; also, since $y \in FV(PQ)$, either $y \in FV(P)$ or
    $y \in FV(Q)$ or both. We have to examine all three cases, which
    we will left to the reader.
  \end{enumerate}
\end{proof}

\begin{prob}
  Prove \olref{thm:infv:1}.
\end{prob}

\begin{thm}\ollabel{thm:clr}
  $x \notin FV(\Subst{M}{N}{x})$, if right hand is
  defined and $x \notin FV(N)$.
\end{thm}
\begin{proof}
  Left as exercise.
\end{proof}

\begin{prob}
  Prove \ollabel{thm:clr}.
\end{prob}


\begin{thm}\ollabel{thm:inv}
  If $\Subst{M}{y}{x}$ is defined and $y \notin FV(M)$, then $\Subst{\Subst{M}{y}{x}}{x}{y} \eq M$.
\end{thm}
\begin{proof}
  By induction on formation of $M$.
  \begin{enumerate}
  \item $M$ is just a variable $z$. Left as exercise. \ollabel{thm:inv:var}
  \item $M$ is of the form $\lambd[z][N]$. Because
    $\Subst{\lambd[z][N]}{y}{x}$ is defined, we are pretty sure
    that $z \neq x$ and $z \neq y$. So:
    \begin{align*}
      &\Subst{\Subst{(\lambd[z][N])}{y}{x}{y}{x}}{x}{y}\\
      =&\Subst{(\lambd[z][\Subst{N}{y}{x}])}{x}{y}
       && \text{by \olref{def:4}} \\
      =&\lambd[z][\Subst{\Subst{N}{y}{x}}{x}{y}]
       && \text{by \olref{def:4}} \\
      =&\lambd[z][N] && \text{by inductive hypothesis} 
    \end{align*}
  \item $M$ is of the form $PQ$. Then:
    \begin{align*}
      &\Subst{\Subst{(PQ)}{y}{x}}{x}{y} \\
      =&\Subst{((\Subst{P}{y}{x})(\Subst{Q}{y}{x}))}{x}{y}
         && \text{by \olref{def:3}} \\
      =&(\Subst{\Subst{P}{y}{x}}{x}{y})(\Subst{\Subst{Q}{y}{x}}{x}{y})
         && \text{by \olref{def:3}} \\
      =&PQ &&\text{by inductive hypothesis}
    \end{align*}
  \end{enumerate}
\end{proof}
\begin{prob}
  Prove \olref{thm:inv:var}.
\end{prob}
\end{document}
