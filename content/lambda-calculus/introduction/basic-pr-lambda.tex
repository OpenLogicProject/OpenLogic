% Part: lambda-calculus
% Chapter: introduction
% Section: basic-pr-lambda

\documentclass[../../../include/open-logic-section]{subfiles}

\begin{document}

\olfileid{lam}{int}{bas}
\olsection{The Basic Primitive Recursive Functions are \usetoken{S}{lambda definable}}

\begin{lem}
The functions $\Zero$, $\Succ$, and $\Proj{n}{i}$ are !!{lambda definable}.
\end{lem}

\begin{proof}
$\Zero$ is just
$\lambd[x][\lambd[y][y]]$.

The successor function~$\Succ$, is
defined by $\fn{Succ}(u) = \lambd[x][\lambd[y][x(uxy)]]$. You should
think about why this works; for each numeral $\num{n}$, thought of as
an iterator, and each function~$f$, $\fn{Succ}(\num{n},f)$ is a function that,
on input~$y$, applies $f$ $n$ times starting with~$y$, and then
applies it once more.

There is nothing to say about projections: $\fn{Proj}^n_i(x_0, \dots,
x_{n-1}) = x_i$. In other words, by our conventions, $\fn{Proj}^n_i$ is
the lambda term $\lambd[x_0][\dots \lambd[x_{n-1}][x_i]]$.
\end{proof}

\end{document}
