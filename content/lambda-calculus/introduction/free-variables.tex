% Part: lambda-calculus
% Chapter: introduction
% Section: free-variable

\documentclass[../../../include/open-logic-section]{subfiles}

\begin{document}

\olfileid{lc}{int}{fv}
\olsection{Free Variables}

As mentioned earlier, $\lambda$-calculus is about functions, and
lambda abstraction is where functins arises. When we
write $\lambd[@x][@M]$, we can refer to the~!!{parameter} $@x$ in $\lambd[@x]$ by
occurrences of variable $@x$ in $@M$. $@M$ is called the
scope of $\lambd[@x]$, and these occurrences are said to be \emph{bound} by this
$\lambd[@x]$. 

\begin{defn}[Scope]
For a particular occurrence of $\lambd[@x][@M]$ in a term $@P$, the
occurrence of $@M$ is called the scope of the occurrence of $\lambd[@x]$
on the left.
\end{defn}

\begin{prob}
  Point out scopes in this term: $\lambd[g][(\lambd[x][g (x x)]) \lambd[x][g (x x)]]$.
\end{prob}

\begin{defn}[Bound occurrences of a variable]
  An occurrence of variable $@x$ is bound if it's in the scope of a $\lambd[@x]$.
\end{defn}

\begin{ex}
  In $\lambd[x][x x]$, both occurrences of $x$ is bound by $\lambd[x]$.
\end{ex}

\begin{prob}
  In $\lambd[g][(\lambd[x][g (x x)]) \lambd[x][g (x x)]]$, are all
  occurrences of variables bound? By which abstractions are they
  bound recpectively?
\end{prob}

Note that not all occurrences of $@x$ in the subterm necessarily refers to this particular argument,
since there can be another but inner $\lambd[@x]$, as shown in the
following example. 
\begin{ex}
In $\lambd[x][\lambd[x][x]]$, while the variable $x$ is in the
scopes of both $\lambd[x]$, it only refers to the inner one, as one
might expect. We may say the inner abstraction shadows the outer
abstraction.
\end{ex}

We have only seen examples only using variable $@x$ in the scope of $\lambd[@x]$, 
what happens when we break it?
\begin{ex}
  $\lambd[x][x y]$, while the occurrence of $x$ refers to the~!!{argument} of $\lambd[x]$, $y$ is not in a scope
  of any $\lambd[y]$. 
\end{ex}

In the last example, $y$ is called \emph{free}. 
\begin{defn}[Free occurrences of a variable]
  An occurrence of a variable $@x$ in a term $@P$ is free if it's not bound, i.e., not in the scope of any
  $\lambd[@x]$ in $@P$.
\end{defn}


For a term $@P$, we can check all variable occurrences in it and get a set of free
variables. This set is denoted by $FV(@P)$ with a natural definition
as follows:

\begin{defn}[Free variables]
  The set of free variables of a term is defined inductively by:
  \begin{align} 
    FV(@x) &= \{@x\} \tag{\rule{VAR}} \ollabel{eq:1} \\
    FV(\lambd[@x][@N]) &= FV(@N) \setminus \{@x\}   \tag{\rule{ABS}}  \ollabel{eq:2} \\
    FV(@P@Q) &= FV(P) \cup FV(Q) \tag{\rule{APP}} \ollabel{eq:3}
  \end{align}
\end{defn}

\begin{prob}
  Run the above definition on $\lambd[x][(\lambd[y][(\lambd[z][x y]) z]) y]$
\end{prob}

\begin{explain}
A free variable is like a reference to outside world (\emph{environment}), and a term
containing free variables can be seen as a partially specified term,
since its behaviour is kind of dependent on how we setup the
environment. For example, the term $\lambd[m][\times g m]$ which
accepts a variable (hopefully a real) and returns the gravity applied
on a object with that mass, refers to an environment variable $g$. If we
setup $g$ to be $9.8$  then this function is for use on earth,
while on mars we may want a different factor.
So informally speaking this term is dependent on the environment
it's in, in particular the setting of $g$ in that environment.

Continuing the last example, if we wrap this term by an abstraction, we get
$\lambd[g][\lambd[m][\times g m]]$. Now it's no longer dependent on
the outside variable $g$, because essentially it's a function that
accepts a gravity factor and a mass, and returns the gravity applied
on that mass; Our changing $g$ in the environment won't have any
effect on the behavior of this term, as the term will only use the $g$
passed by~!!{argument} anyway and not the one in environment. The term
in this case is called a \emph{closed term}.
\end{explain}

\begin{defn}[Closed term]
  A term with no free variable is a closed term.
\end{defn}

\begin{ex}
  Reader may find the term $\lambd[m][\times g m]$ exotic as it's not
  a term at all: we can't have the $\times$ symbol in terms, and also,
  how can we have variables representing real numbers? But we will see
  in \olref[lc][rep] that both multiplication function and real
  numbers can be defined in the forms of terms. Thus it's not a big
  problem and reader should pretend that they are replaced by very
  long closed terms.
\end{ex}

\begin{lem}
  \begin{enumerate}
    \item \ollabel{lem:abs} If $@y \neq @x$, then $@y \in FV(\lambd[@x][@N])$ iff $@y \in
    FV(@N)$.
    \item \ollabel{lem:app} $@y \in FV(@P@Q)$ iff $@y \in FV(@P)$ or
      $@y \in FV(@Q)$.
    \end{enumerate}
\end{lem}
\begin{proof}
  Left as exercise.
\end{proof}

\end{document}