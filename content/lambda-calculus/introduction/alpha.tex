% Part: lambda-calculus
% Chapter: introduction
% Section: alpha

\documentclass[../../../include/open-logic-section]{subfiles}

\begin{document}

\olfileid{lc}{int}{alp}
\olsection{$\alpha$-conversion}

What's the relation between $\lambd[x][x]$ and $\lambd[y][y]$? They
both seem to be the identity function. Here is the $\alpha$-conversion:

\begin{defn}[$\alpha$-conversion]
  If a term contains an occurrences of $\lambd[@x][@M]$, $@y \notin
  FV(@M)$, and $@M[@y/@x]$ is defined, then replacing this occurrences
  by 
  \begin{equation*}
    \lambd[@y][@M[@y/@x]]
  \end{equation*}
  is called a \emph{change of bound
    variable}. If $@P$ can be changed to $@Q$ by a finite (possibly
  empty) series of such steps, then we say $@P ~\alpha\text{-converts to}~ @Q$,
  or $@P \aconv @Q$.
\end{defn}

\begin{ex}
  $\lambd[m][\times m g]$ $\alpha$-converts to $\lambd[mass][\times mass
  g]$, and the other way around too. Informally
  speaking, they are both functions that accpets an argument and
  returns the gravity applied on the object with mass of that argument,
  refering to the free variable $g$.
\end{ex}
\begin{ex}
  $\lambd[m][\times m g]$ cannot $\alpha$-converts $\lambd[m][\times m
  G]$. Informally speaking, they refers to the environment variables $g$ and $G$ respectively,
  making them different functions: they behave differently in
  environment where $g$ and $G$ is different.
\end{ex}

\begin{prob}
  Tell if the following pairs of terms are $\alpha$-convertable.
  \begin{enumerate}
  \item $\lambd[x][\lambd[y][x]]$ and $\lambd[y][\lambd[x][y]]$
  \item $\lambd[x][\lambd[y][x]]$ and $\lambd[c][\lambd[b][a]]$
  \item $\lambd[x][\lambd[y][x]]$ and $\lambd[c][\lambd[b][a]]$
  \end{enumerate}
\end{prob}

\begin{lem}{lem:cobv-inv}
  If by change of a bound variable $@P$ can be changed to $@Q$, then $@Q$
  can be changed to $@P$ too.
\end{lem}
\begin{proof}
  Suppose the change from $@P$ to $@Q$ is done by replacing an
  occurrence of $\lambd[@x][@M]$ in $@P$ by
  $\lambd[@y][\Subst{@M}{@y}{@x}]$. Then clearly $\Subst{@M}{@y}{@x}$
  is defined and $@y \notin FV(@M)$ by the requirement of change of
  bound variables. Then by \olref[sub]{thm:inv}
  $\Subst{\Subst{@M}{@y}{@x}}{@x}{@y}$ is defined and
  equals to $@M$; Also, $@x \notin FV(\Subst{@M}{@y}{@x})$ by
  \olref[fv]{thm:clr}. So we can replace $\lambd[@y][\Subst{@M}{@y}{@x}]$ in
  $@Q$ by $\lambd[@x][\Subst{\Subst{@M}{@y}{@x}}{@x}{@y}]$ which equals
  $\lambd[@x][@M]$, resulting in $@P$.
\end{proof}

\begin{thm}
  $\alpha$-conversion is a equivalence relation (\olref[sfr][rel][prp]{sec}) on terms.
\end{thm}
\begin{proof}
  
  \begin{enumerate}
    \item[reflexive] For each term $@M$, $@M$ can be changed to $M$ by
      \emph{zero} steps of change of bound variables.
    \item[symmetric] If $@P$ is changed to $@Q$ by a series of change
      of bound variables, then from $@Q$ we can just inverse these changes(by \olref{lem:cobv-inv}) in
      opposite order and we should get $@P$.
    \item[transitive] If $@P$ is changed to $@Q$ by a series of
      change, and $@Q$ to $@R$ by another series, then we can change
      $@P$ to $@R$ by first applying the first series and then the
      second series.
  \end{enumerate}
\end{proof}

\begin{digress}
  Because the $\alpha$-conversion comes so naturally, some syntax has
  been invented that doesn't distinguish terms that can be
  $\alpha$-converted to each other, among which the most famous is the
  \emph{De Bruijn index}.
  
  When we write $\lamd[@x][@M]$, we explicitly state that $@x$ is the
  parameter of the function, so that we can use $@x$ in $@M$ to refer
  to this parameter. In De Bruijn index, however, parameters have no
  name and reference to them in function body is denoted by the number of levels of
  abstraction between them. For example, instead of $\lambd[x][lambd[y][y
  x]]$, we write $\lambd[][lambd[][0 1]]$, since there is zero
  abstraction between the variable $y$ and the abstraction where the
  parameter $y$ is stated, while $x$ is $1$ since there is one level
  of abstraction (namely the abstraction of $y$) between.

  The De Bruijn index is almost unreadable to human race, but it's
  much more convenient when we try to implement lambda calculus in
  computer. 
\end{digress}
\end{document}

