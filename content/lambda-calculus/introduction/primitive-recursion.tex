% Part: lambda-calculus
% Chapter: introduction
% Section: primitive-recursion

\documentclass[../../../include/open-logic-section]{subfiles}

\begin{document}

\olfileid{lam}{int}{pr}
\olsection{\usetoken{S}{lambda definable} Functions are Closed under Primitive Recursion}

When it comes to primitive recursion, we finally need to do some
work. We will have to proceed in stages. As before, on the assumption
that we already have terms $G$ and $H$ that !!{lambda define} functions
$g$ and~$h$, respectively, we want a term $H$ that !!{lambda define}s the
function~$f$ defined by
\begin{align*}
f(0, \vec z) & = g(\vec z) \\
f(x+1, \vec z) & = h(z, f(x,\vec z), \vec z).
\end{align*}
So, in general, given lambda terms $G'$ and~$H'$, it suffices to find
a term~$F$ such that
\begin{align*}
F(\num{0}, \vec z) & \equiv G(\vec z) \\
F(\overline{n+1}, \vec z) & \equiv H(\num{n}, F(\num{n}, \vec z), \vec z)
\end{align*}
for every natural number~$n$; the fact that $G'$ and~$H'$ !!{lambda define}
$g$ and~$h$ means that whenever we plug in numerals $\num{\vec m}$
for $\vec z$, $F(\num{n+1}, \num{\vec m})$ will normalize to the
right answer.

But for this, it suffices to find a term~$F$ satisfying
\begin{align*}
F(\num 0) & \equiv G \\
F(\overline {n+1}) & \equiv H(\num{n},F(\num{n}))
\intertext{for every natural number $n$,  where}
G & = \lambd[\vec z][G'(\vec z)] \text{ and}\\
H(u,v) & = \lambd[\vec z][H'(u,v(u,\vec z),\vec z)].
\end{align*}
In other words, with lambda trickery, we can avoid having to worry
about the extra parameters $\vec z$---they just get absorbed in the
lambda notation.

Before we define the term $F$, we need a mechanism for handling
ordered pairs. This is provided by the next lemma.

\begin{lem}
There is a lambda term $D$ such that for each pair of lambda terms $M$
and~$N$, $D(M,N)(\num{0}) \red M$ and $D(M,N)(\num{1}) \red N$.
\end{lem}

\begin{proof}
First, define the lambda term $K$ by
\[
K(y) = \lambd[x][y].
\]
In other words, $K$ is the term $\lambd[y][\lambd[x][y]]$. Looking at it
differently, for every $M$, $K(M)$ is a constant function that
returns~$M$ on any input.

Now define $D(x,y,z)$ by $D(x,y,z) = z (K(y))x$. Then we have
\begin{align*}
D(M,N,\num 0) & \red \num 0 (K(N)) M \red M \text{ and}\\
D(M,N,\num 1) & \red \num 1 (K(N)) M \red K(N) M \red N,
\end{align*}
as required.
\end{proof}

The idea is that $D(M,N)$ represents the pair $\tuple{M, N}$, and if
$P$ is assumed to represent such a pair, $P(\num 0)$ and $P(\num 1)$
represent the left and right projections, $(P)_0$ and $(P)_1$. We will
use the latter notations.

\begin{lem}
The !!{lambda definable} functions are closed under primitive recursion.
\end{lem}

\begin{proof}
We need to show that given any terms, $G$ and $H$, we can find a term
$F$ such that
\begin{align*}
F(\num{0}) & \equiv G \\
F(\num{n+1}) & \equiv H(\num{n}, F(\num{n}))
\end{align*}
for every natural number~$n$. The idea is roughly to compute sequences
of \emph{pairs}
\[
\tuple{\num 0, F(\num 0)}, \tuple{\num 1, F(\num 1)}, \dots,
\]
using numerals as iterators. Notice that the first pair is just
$\tuple{\num 0, G}$. Given a pair $\tuple{\num{n}, F(\num{n})}$, the
next pair, $\tuple{\num{n+1}, F(\num{n+1})}$ is supposed to
be equivalent to $\tuple{\num{n+1}, H(\num{n}, F(\num{n}))}$. We
will design a lambda term~$T$ that makes this one-step transition.

The details are as follows. Define $T(u)$ by
\[
T(u) = \tuple{S((u)_0), H((u)_0,(u)_1)}.
\]
Now it is easy to verify that for any number~$n$,
\[
T(\tuple{\num{n}, M}) \red \tuple{\num{n+1}, H(\num{n}, M)}.
\]
As suggested above, given $G$ and $H$, define~$F(u)$ by
\[
F(u) = (u(T,\tuple{\num 0, G}))_1.
\]
In other words, on input~$\num{n}$, $F$ iterates $T$ $n$ times on
$\tuple{\num 0, G}$, and then returns the second component. To start
with, we have
\begin{enumerate}
\item $\num{0} (T, \tuple{\num 0, G}) \equiv \tuple{\num{0}, G}$
\item $F(\num{0}) \equiv G$
\end{enumerate}
By induction on~$n$, we can show that for each natural number one has
the following:
\begin{enumerate}
\item $\num{n+1}(T, \tuple{\num{0}, G}) \equiv \tuple{\num{n+1}, F(\num{n+1})}$
\item $F(\num{n+1}) \equiv H(\num{n}, F(\num{n}))$
\end{enumerate}
For the second clause, we have
\begin{align*}
F(\num{n+1}) & \red (\num{n+1}(T, \tuple{\num 0, G}))_1 \\
& \equiv (T(\num{n} (T, \tuple{\num 0, G})))_1 \\
& \equiv (T(\tuple{\num{n}, F(\num{n})}))_1 \\
& \equiv (\tuple{\num{n+1}, H(\num{n}, F(\num{n}))})_1 \\
& \equiv H(\num{n}, F(\num{n})).
\end{align*}
Here we have used the induction hypothesis on the second-to-last
line. For the first clause, we have
\begin{align*}
\num{n+1} (T, \tuple{\num 0, G}) &
\equiv T(\num{n} (T, \tuple{\num 0, G})) \\
& \equiv T( \tuple{\num{n}, F(\num{n})}) \\
& \equiv \tuple{\num{n+1}, H(\num{n}, F(\num{n}))} \\
& \equiv \tuple{\num{n+1}, F(\num{n+1})}.
\end{align*}
Here we have used the second clause in the last line. So we have shown
$F(\num 0) \equiv G$ and, for every $n$, $F(\num {n+1}) \equiv H(\num
n, F(\num{n}))$, which is exactly what we needed.
\end{proof}

\end{document}

