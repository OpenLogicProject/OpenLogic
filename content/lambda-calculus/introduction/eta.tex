% Part: lambda-calculus
% Chapter: introduction
% Section: eta

\documentclass[../../../include/open-logic-section]{subfiles}

\begin{document}

\olfileid{lc}{int}{eta}
\olsection{$\eta$-conversion}

There is another relation on $\lambda$ terms. In
\olref[fv]{sec} we used the example $[\lambd[x][fx]]$, which
accepts an argument and apply $f$ to it. 
It seems to be another $f$ if we ``look from outside''.
Indeed, applied with any term $N$, both will reduce to $fN$.
We use $\eta$-reduction (and $\eta$-extension) to
capture this idea.

\begin{defn}[$\eta$-contraction, $\eredone$]
  \emph{$\eta$-contraction} ($\eredone$) is the smallest compatible relation
  on terms satisfying the following condition:
  \begin{align*}
    &\lambd[x][M x] \eredone M && \text{provided} x \notin FV(M)
  \end{align*}
\end{defn}

\begin{defn}[$\beta\eta$-reduction, $\bered$] \ollabel{def:bered}
  \emph{$\beta\eta$-reduction} ($\bered$) is the smallest reflexive,
  transitive relation on terms containing $\bredone$ and $\eredone$,
  i.e., the rules of reflexivity and transitive plus the following two
  rules:
  \begin{enumerate}
  \item If $M \bredone N$ then $M \bered N$. \ollabel{def:bered:3}
  \item If $M \eredone N$ then $M \bered N$. \ollabel{def:bered:4}
  \end{enumerate}
    
\end{defn}

\begin{defn}
  We extend the equivalence relation $\equal$ with $\eta$-conversion rule:
  \begin{enumerate}
  \item $\lambd[x][f x] \equal f$
  \end{enumerate}
  and denote the extended relation as $\equal[\eta]$.
\end{defn}

$\eta$-equivalence is important because it's related to extensionality
of lambda terms:
\begin{defn}[Extensionality]
  We extend the equivalence relation $\equal$ with $ext$ rule:
  \begin{enumerate}
  \item If $Mx \equal Nx$ then $M \equal N$, provided $x \notin FV(MN)$.
  \end{enumerate}
  and denote the extended relation as $\equal[ext]$.
\end{defn}
roughly speaking, the rule stating that two terms, viewed as functions, should be
considered equal if they behave the same for same argument.

We now prove that the $\eta$ rule provides exactly the extensionality,
and nothing else.
\begin{thm}
  $M \equal[ext] N$ if and only if $M \equal[\eta] N$.
\end{thm}
\begin{proof}
  First we prove that $\equal[\eta]$ is closed under the 
  extensionality rule. That is, $ext$ rule doesn't add any 
  to $\equal[\eta]$, which means $\equal[\eta]$ is identical
  to $\equal[\eta,ext]$, which contains
  $\equal[ext]$. So we have $\equal[\eta]$ contains
  $\equal[ext]$, so if $M \equal[ext] N$, then $M
  \equal[\eta] N$.

  To prove the closeness, we note that for any $M \equal
  N$ derived by the extensionality rule, we have $Mx \equal[\eta]
  Nx$ as premise, then we have $\lambd[x][Mx] \equal[\eta]
  \lambd[x][Nx]$ by a rule of $\equal$, applying $\eta$ on both side gives us $M
  \equal[\eta] N$.


  Similarly we prove that the $\eta$ rule is contained in
  $\equal[ext]$. For any $\lambd[x][Mx]$ and $M$ with $x \notin FV(M)$, we always
  observe that $(\lambd[x][Mx])x \equal[ext] Mx$, giving us
  $\lambd[x][Mx] \equiv[ext] M$ by $ext$ rule.
\end{proof}

\end{document}
