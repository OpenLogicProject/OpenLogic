% Part: lambda-calculus
% Chapter: church-rosser
% Section: beta-eta-reduction

\documentclass[../../../include/open-logic-section]{subfiles}

\begin{document}

\olfileid{lam}{cr}{be}

\olsection{$\beta\eta$-reduction}

The Church--Rosser property holds for
$\beta\eta$-reduction ($\bered$).

\begin{lem}\ollabel{lem:one-par}
  If $M \beredone M'$, then $M \beredpar M'$.
\end{lem}

\begin{proof} 
  By induction on the !!{derivation} of $M \beredone M'$. If $M \bredone
  M'$ by $\eta$-conversion (i.e., \olref[syn][eta]{defn:beredone}), we
  use \olref[pbe]{thm:refl}. The other cases are as in
  \olref[b]{lem:one-par}.
\end{proof}


\begin{lem}\ollabel{lem:par-red}
  If $M \beredpar M'$, then $M \bered M'$.
\end{lem}

\begin{proof} Induction on the !!{derivation} of $M \beredpar M'$.

  If the last rule is \olref[pbe]{defn:beredpar5}, then $M$ is
  $\lambd[x][Nx]$ and $M'$ is $N'$ for some $x$, $N$, $N'$ where $x
  \notin FV(N)$ and $N \beredpar N'$. Thus we can first reduce
  $\lambd[x][Nx]$ to $N$ by $\eta$-conversion, followed
  by the series of $\beredone$ steps that show that $N \bered N'$,
  which  holds by induction hypothesis.
\end{proof}


\begin{lem}\ollabel{lem:str}
  $\bered$ is the smallest transitive relation containing $\beredpar$.
\end{lem}

\begin{proof}
  As in \olref[b]{lem:str}
\end{proof}

\begin{thm}\ollabel{thm:cr}
  $\bered$ satisfies Church--Rosser property.
\end{thm}

\begin{proof}
  By \olref[dap]{thm:str}, \olref[pbe]{thm:cr} and \olref{lem:str}.
\end{proof}
\end{document}