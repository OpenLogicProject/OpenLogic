% Part: normal-modal-logic
% Chapter: syntax-and-semantics
% Section: axiom-systems

\documentclass[../../../include/open-logic-section]{subfiles}

\begin{document}

\olfileid{mod}{syn}{axs}

\olsection{Axiom Systems for Normal Modal Logics}

One way of specifying normal modal logics as the set of !!{formula}s
!!{derivable} in certain !!{derivation} systems.  The simplest and
historically oldest !!{derivation} systems are so-called Hilbert-type
or axiomatic !!{derivation} systems.  Hilbert-type !!{derivation} systems
for many normal modal logics are relatively easy to construct: they
are simple as objects of metatheoretical study (e.g., to prove
soundness and completeness), but they are much harder to use to prove
!!{formula}s in than, say, natural deduction systems.

In Hilbert-type !!{derivation} systems, a derivation of !!a{formula}
is a sequence of !!{formula}s leading from certain axioms, via a
handful of inference rules, to the !!{formula} in question.  For
normal modal logics, there are only two inference rules that need to
be assumed: modus ponens and necessitation.  As axioms we take all
(substitution instances) of tautologies, and, depending on the modal
logic we deal with, a number of modal axioms. In order to generate a
normal modal logic, all substitution instances of $\Ax{K}$ and $\Ax{Dual}$
count as axioms. This alone generates the minimal normal modal
logic~$\Log K$.  Additional axioms generate other normal modal logics.

\begin{defn}
The rule of \emph{modus ponens} is the inference schema
\begin{prooftree}
\AxiomC{$!A$}
\AxiomC{$!A \lif !B$}
\RightLabel{\MP}
\BinaryInfC{$!B$}
\end{prooftree}
We say !!a{formula}~$!B$ \emph{follows from}~!!{formula}s $!A$, $!C$
by modus ponens iff $!C \ident !A \lif !B$.
\end{defn}

\begin{defn}
The rule of \emph{necessitation} is the inference schema
\begin{prooftree}
\AxiomC{$!A$}
\RightLabel{\Nec}
\UnaryInfC{$\Box !A$}
\end{prooftree}
We say the !!{formula} $!B$ follows from the !!{formula}s $!A$ by
necessitation iff $!B \ident \Box !A$.
\end{defn}

\begin{defn}
A \emph{!!{derivation}} from a set of axioms~$\Sigma$ is a sequence of
!!{formula}s $!B_1$, $!B_2$, \dots, $!B_n$, where each $!B_i$ is
either
\begin{enumerate}
\item a substitution instance of a tautology, or
\item a substitution instance of !!a{formula} in~$\Sigma$, or
\item follows from two !!{formula}s $!B_j$, $!B_k$ with $j$, $k < i$
  by modus ponens, or
\item follows from !!a{formula} $!B_j$ with $j < i$ by necessitation.
\end{enumerate}
If there is such !!a{derivation} with $!B_n \ident !A$, we say that
$!A$ is \emph{!!{derivable} from $\Sigma$}, in symbols $\Sigma \Proves !A$.
\end{defn}

\end{document}
