% Part: normal-modal-logic
% Chapter: syntax-and-semantics
% Section: introduction

\documentclass[../../../include/open-logic-section]{subfiles}

\begin{document}

\olfileid{mod}{syn}{int}

\olsection{Introduction}

Modal Logic deals with \emph{modal propositions} and the entailment
relations among them. Examples of modal propositions are the
following:
\begin{enumerate}
\item It is necessary that $2+2=4$.
\item It is necessarily possible that it will rain tomorrow.
\item If it is necessarily possible that~$!A$ then it is possible
  that~$!A$.
\end{enumerate}
Possibility and necessity are not the only modalities: other unary
connectives are also classified as modalities, for instance, ``it
ought to be the case that~$!A$,'' ``It will be the case that~$!A$,''
``Dana knows that~$!A$,'' or ``Dana believes that~$!A$.''

Modal logic makes its first appearance in Aristotle's \emph{De
  Interpretatione}: he was the first to notice that necessity implies
possibility, but not vice versa; that possibility and necessity are
inter-definable; that If $!A \land !B$ is possibly true then
$!A$ is possibly true and $!B$ is possibly true, but not
conversely; and that if $!A \to !B$ is necessary, then if
$!A$ is necessary, so is~$!B$.

The first modern approach to modal logic was the work of C.~I. Lewis,
culminating with Lewis and Langford, \emph{Symbolic Logic}
(1932). Lewis \& Langford were unhappy with the representation of
implication by means of the material conditional: $!A \lif !B$
is a poor substitute for ``$!A$ implies $!B$.'' Instead, they
proposed to characterize implication as ``Necessarily, if $!A$
then $!B$,'' symbolized as $!A \strictif !B$. In trying to
sort out the different properties, Lewis indentified five different
modal systems, \Log{S1}, \ldots, \Log{S4}, \Log{S5}, the last
two of which are still in use.

The approach of Lewis and Langford was purely \emph{syntactical}: they
identified reasonable axioms and rules and investigated wat was
provable with those means. A semantic approach remained elusive for a
long time, until a first attempt was made by Rudolf Carnap in
\emph{Meaning and Necessity} (1947) using the notion of a \emph{state
  description}, i.e., a collection of atomic sentences (those that are
``true'' in that state description). After lifting the truth
definition to arbitrary sentences $!A$, Carnap defines $!A$
to be \emph{necessarily true} if it is true in all state
descriptions. Carnap's approach could not handle \emph{iterated}
modalities, in that sentences of the form ``Possibly necessarily
\ldots possibly $!A$'' always reduce to the innermost modality.

The major breakthrough in modal semantics came with Saul Kripke's
article ``A Completeness Theorem in Modal Logic'' (JSL 1959). Kripke
based his work on Leibniz's idea that a statement is necessarily true
if it is true ``at all possible worlds.'' This idea, though, suffers
from the same drawbacks as Carnap's, in that the truth of statement at
a world $w$ (or a state description $s$) does not depend on $w$ at
all. So Kripke assumed that worlds are related by an
\emph{accessibility relation} $R$, and that a statement of the form
``Necessarily $!A$'' is true at a world $w$ if and only if
$!A$ is true at all worlds $w'$ \emph{accessible from}
$w$. Semantics that provide some version of this approach are called
Kripke semantics and made possible the tumultuous development of modal
logics (in the plural).

When interpreted by the Kripke semantics, modal logic shows us what
\emph{relational structures} look like ``from the inside.'' A
relational structure is just a set equipped with a binary relation
(for instance, the set of students in the class ordered by their
social security number is a relational structure). But in fact
relational structures come in all sorts of domains: besides relative
possibility of states of the world, we can have epistemic states of
some agent related by epistemic possibility, or states of a dynamical
system with their state transitions, etc. Modal logic can be used to
model all of these: the first give us ordinary, alethic, modal logic;
the others give us epistemic logic, dynamic logic, etc.

We focus on one particular angle, known to modal logicians as
``correspondence theory.'' One of the most significant early
discoveries of Kripke's is that many properties of the accessibility
relation~$R$ (whether it is transitive, symmetric, etc.)  can be
characterized \emph{in the modal language} itself by means of
appropriate ``modal schemas.'' Modal logicians say, for instance, that
the reflexivity of $R$ ``corresponds'' to the schema ``If
necessarily~$!A$, then~$!A$''. We explore mainly the correspondence
theory of a number of classical systems of modal logic (e.g., \Log{S4}
and \Log{S5}) obtained by a combination of the schemas \Ax{D}, \Ax{T},
\Ax{B}, \Ax{4}, and~\Ax{5}.

\end{document}
