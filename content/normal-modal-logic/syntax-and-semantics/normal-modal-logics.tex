% Part: normal-modal-logic
% Chapter: syntax-and-semantics
% Section: normal-modal-logics

\documentclass[../../../include/open-logic-section]{subfiles}

\begin{document}

\olfileid{nml}{syn}{nor}

\olsection{Normal Modal Logics}

A modal logic can be seen as any suitably well-behaved set of
!!{formula}s.  There are some modal logics which are of particular
interest, both because they have interesting interpretations and
applications, but mainly because they are theoretically well
understood.  Those are the so-called normal modal logics.

\begin{defn}\ollabel{defn:modal-logic}
A \emph{modal logic}~$\Log L$ is a set of !!{formula}s of the basic
modal language which 
\begin{enumerate}
\item contains all propositional tautologies, 
\item is closed under uniform substitution, and 
\item is closed under modus ponens, i.e., whenever $!A$ and $(!A \lif
  !B) \in \Log L$ then also $!B \in \Log L$.
\end{enumerate}
\end{defn}

\begin{ex}
Some admittedly not very interesting sets of !!{formula}s which
satisfy this definition and therefore count as modal logics are:
\begin{enumerate}
\item the set of all tautologies
\item the set of all !!{formula}s (the inconsistent modal logic)
\end{enumerate}
\end{ex}

\begin{defn}\label{defn:normal-modal-logic}
A modal logic~$\Log L$ is \emph{normal} iff 
\begin{enumerate}
\item it contains both
\begin{align}
\tag{K} \Box(p \lif q) & \lif (\Box p \lif \Box q) \\
\tag{Dual} \Diamond p & \liff \lnot \Box \lnot p
\end{align}
\item and it is closed under necessitation, i.e., whenever $!A \in
  \Log L$, then also $\Box !A \in \Log L$.
\end{enumerate}
\end{defn}

We will encounter two main ways of defining modal logics.  One is
proof-theoretically, as the set of !!{formula}s that can be derived in
certain !!{derivation} systems. The other is model-theoretically, as
sets of !!{formula}s which are valid in certain classes of
models. This mirrors the situation in ordinary propositional logic (or
first-order logic, for that matter).  Here we can also single out a
set of !!{formula}s in two different ways, e.g., as the set of
tautologies which are true in all truth-value assignments and as the
set of all !!{derivable} !!{formula}s in some !!{derivation} system.
For normal modal logics, this two-fold way of specifying modal logics
is possible using relational models and axiomatic (Hilbert-type) proof
systems.
\end{document}
