% Part: normal-modal-logic
% Chapter: completeness
% Section: lindenbaums-lemma

\documentclass[../../../include/open-logic-section]{subfiles}

\begin{document}

\olfileid{nml}{com}{lin}

\olsection{Lindenbaum's Lemma}

Lindenbaum's Lemma establishes that every $\Sigma$-consistent set of
!!{formula}s is contained in at least one \emph{complete}
$\Sigma$-consistent set. Our construction of the canonical model will
show that for each complete $\Sigma$-consistent set~$\Delta$, there is
a world in the canonical model where all and only the !!{formula}s
in~$\Delta$ are true. So Lindenbaum's Lemma guarantees that every
$\Sigma$-consistent set is true at some world in the canonical model.

\begin{thm}[Lindenbaum's Lemma]\ollabel{thm:lindenbaum}
  If $\Gamma$ is $\Sigma$-consistent then there is a complete
  $\Sigma$-consistent set $\Delta$ extending~$\Gamma$.
\end{thm}

\begin{proof}
Let $!A_0$, $!A_1$, \dots{} be an exhaustive listing of all formulas
of the language (repetitions are allowed). For instance, start by
listing $\Obj p_0$, and at each stage $n \ge 1$ list the finitely many
formulas of length~$n$ using only variables among $\Obj p_0$,
\dots,~$\Obj p_n$. We define sets of !!{formula}s $\Delta_n$ by
induction on~$n$, and we then set $\Delta = \bigcup_n \Delta_n$. We
first put $\Delta_0 = \Gamma$. Supposing that $\Delta_n$ has been
defined, we define $\Delta_{n+1}$ by:
\[
\Delta_{n+1} =
\begin{cases}
  \Delta_n \cup \{!A_n\}, & \text{if $\Delta_n \cup \{ !A_n\}$
    is $\Sigma$-consistent;} \\
  \Delta_n \cup \{ \lnot !A_n\}, & \text{otherwise.}
\end{cases}
\]
Now let $\Delta = \bigcup_{n=0}^\infty \Delta_n$.

We have to show that this definition actually yields a set $\Delta$
with the required properties, i.e., $\Gamma \subseteq \Delta$ and
$\Delta$ is complete $\Sigma$-consistent.

It's obvious that $\Gamma \subseteq \Delta$, since $\Delta_0 \subseteq
\Delta$ by construction, and $\Delta_0 = \Gamma$. In fact, $\Delta_n
\subseteq \Delta$ for all~$n$, since $\Delta$ is the union of
all~$\Delta_n$. (Since in each step of the construction, we add
!!a{formula} to the set already constructed, $\Delta_n \subseteq
\Delta_{n+1}$, so since $\subseteq$ is transitive, $\Delta_n \subseteq
\Delta_{m}$ whenever $n \le m$.)  At each stage of the construction, we
either add $!A_n$ or $\lnot !A_n$, and every !!{formula} appears (at
least once) in the list of all~$!A_n$. So, for every $!A$ either $!A
\in \Delta$ or $\lnot !A \in \Delta$, so $\Delta$ is complete by
definition.

Finally, we have to show, that $\Delta$ is $\Sigma$-consistent.  To do
this, we show that (a) if $\Delta$ were $\Sigma$-inconsistent, then
some $\Delta_n$ would be $\Sigma$-inconsistent, and (b) all $\Delta_n$
are $\Sigma$-consistent.

So suppose $\Delta$ were $\Sigma$-inconsistent. Then $\Delta
\Proves[\Sigma] \lfalse$, i.e., there are $!A_1$, \dots,~$!A_k \in
\Delta$ such that $\Sigma \Proves !A_1 \lif (!A_2 \lif \cdots (!A_k
\lif \lfalse)\dots)$. Since $\Delta = \bigcup_{n=0}^\infty \Delta_n$, each
$!A_i \in \Delta_{n_i}$ for some~$n_i$. Let $n$ be the largest of
these. Since $n_i \le n$, $\Delta_{n_i} \subseteq \Delta_n$. So, all
$!A_i$ are in some~$\Delta_n$. This would mean $\Delta_n
\Proves[\Sigma] \lfalse$, i.e., $\Delta_n$ is $\Sigma$-inconsistent.

To show that each $\Delta_n$ is $\Sigma$-consistent, we use a simple
induction on~$n$. $\Delta_0 = \Gamma$, and we assumed $\Gamma$ was
$\Sigma$-consistent. So the claim holds for $n = 0$. Now suppose it
holds for $n$, i.e., $\Delta_n$ is $\Sigma$-consistent. $\Delta_{n+1}$
is either $\Delta_n \cup \{!A_n\}$ if that is $\Sigma$-consistent,
otherwise it is $\Delta_n \cup \{\lnot!A_n\}$. In the first case,
$\Delta_{n+1}$ is clearly $\Sigma$-consistent. However, by
\olref[prf][con]{prop:consistencyfacts}\olref[prf][con]{prop:consistencyfacts-c},
either $\Delta_n \cup \{!A_n\}$ or $\Delta_n \cup \{\lnot!A_n\}$ is
consistent, so $\Delta_{n+1}$ is consistent in the other case as well.
\end{proof}

\begin{cor}\ollabel{cor:provability-characterization}
  $\Gamma \Proves[\Sigma] !A$ if and only if $!A \in \Delta$ for
  each  complete $\Sigma$-consistent set $\Delta$ extending $\Gamma$
  (including when $\Gamma = \emptyset$, in which case we get another
  characterization of the modal system $\Sigma$.)
\end{cor}

\begin{proof}
  Suppose $\Gamma \Proves[\Sigma] !A$, and let $\Delta$ be any
  complete $\Sigma$-consistent set extending $\Gamma$. If $!A
  \notin \Delta$ then by maximality $\lnot!A \in \Delta$ and so
  $\Delta \Proves[\Sigma] !A$ (by monotonicity) and $\Delta
  \Proves[\Sigma] \lnot!A$ (by reflexivity), and so $\Delta$ is
  inconsistent. Conversely if $\Gamma \Proves/[\Sigma] !A$, then
  $\Gamma \cup \{ \lnot!A\}$ is $\Sigma$-consistent, and by
  Lindenbaum's Lemma there is a complete consistent set $\Delta$
  extending $\Gamma \cup \{ \lnot!A \}$. By consistency, $!A
  \notin \Delta$.
\end{proof}

\end{document}
