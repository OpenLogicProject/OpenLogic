% Part: normal-modal-logic
% Chapter: completeness
% Section: modalities-ccs

\documentclass[../../../include/open-logic-section]{subfiles}

\begin{document}

\olfileid{mod}{com}{mod}

\olsection{Modalities and Complete Consistent Sets}

\begin{explain}
When we construct a model~$\mModel{M^\Sigma}$ whose set of worlds is
given by the complete $\Sigma$-consistent sets~$\Delta$ in some normal
modal logic~$\Sigma$, we will also need to define an accessibility
relation~$R^\Sigma$ between such ``worlds.'' We want it to be the case
that the accessibility relation (and the assignment $V^\Sigma)$ are
defined in such a way that $\mSat{M^\Sigma}{!A}[\Delta]$ iff $!A \in
\Delta$. How should we do this?

Once the accessibility relation is defined, the definition of truth at
a world ensures that \iftag{prvBox}{$\mSat{M^\Sigma}{\Box!A}[\Delta]$
  iff $\mSat{M^\Sigma}{!A}[\Delta']$ for all $\Delta'$ such that
  $R^\Sigma\Delta\Delta'$}{$\mSat{M^\Sigma}{\Diamond!A}[\Delta]$ iff
  $\mSat{M^\Sigma}{!A}[\Delta']$ for at least one $\Delta'$ such that
  $R^\Sigma\Delta\Delta'$}.  The proof that
$\mSat{M^\Sigma}{!A}[\Delta]$ iff $!A \in \Delta$ requires that this
is true in particular for !!{formula}s starting with a modal operator,
i.e., \iftag{prvBox}{$\mSat{M^\Sigma}{\Box!A}[\Delta]$ iff $\Box !A
  \in \Delta$}{$\mSat{M^\Sigma}{\Diamond!A}[\Delta]$ iff $\Diamond !A
  \in \Delta$}. Combining this requirement with the definition of
truth at a world for \iftag{prvBox}{$\Box !A$}{$\Diamond !A$} yields:
\[
\iftag{prvBox}{%
  \Box!A \in \Delta \text{ iff } !A \in \Delta' \text{
    for all $\Delta'$ with } R^\Sigma\Delta\Delta'}{%
  \Diamond!A \in \Delta \text{ iff } !A \in \Delta' \text{ for at
    least one } \Delta' \text{ with } R^\Sigma\Delta\Delta'}
\]
\iftag{prvBox}{Consider the left-to-right direction: it says that if
  $\Box !A \in \Delta$, then $!A \in \Delta'$ for any $!A$ and any
  $\Delta'$ with $R^\Sigma\Delta\Delta'$. If we stipulate that
  $R^\Sigma\Delta\Delta'$ iff $!A \in \Delta'$ for all $\Box!A \in
  \Delta$, then this holds. We can write the condition on the right of
  the ``iff'' more compactly as: $\Setabs{!A}{\Box!A \in \Delta}
  \subseteq \Delta'$.}{Consider the right-to-left direction: it says
  that if $!A \in \Delta'$ then $\Diamond!A \in \Delta$, for any $!A$
  and $\Delta'$ with $R^\Sigma\Delta\Delta'$. If we stipulate that
  $R^\Sigma\Delta\Delta'$ holds whenever $\Diamond!A \in \Delta$ for
  all $!A \in \Delta'$, then this holds.  We can write the condition
  on the right of the ``iff'' more compactly as:
  $\Setabs{\Diamond!A}{!A \in \Delta'} \subseteq \Delta$.}

So the question is: does this definition of $R^\Sigma$ in fact
guarantee that \iftag{prvBox}{$\Box !A \in \Delta$ iff
  $\mSat{M^\Sigma}{\Box !A}[\Delta]$? \iftag{prvDiamond}{Does it also
    guarantee that }{}}{}\iftag{prvDiamond}{$\Diamond !A \in \Delta$
  iff $\mSat{M^\Sigma}{\Diamond !A}[\Delta]$?}{} The next few results
will establish this.
\end{explain}

\begin{defn}
  If $\Gamma$ is a set of !!{formula}s, let
  \begin{align*}
    \Box\Gamma & = \Setabs{\Box !B}{!B \in \Gamma}\\
    \Diamond\Gamma & = \Setabs{\Diamond !B}{!B \in \Gamma}\\
    \intertext{and}
    \Box^{-1}\Gamma & = \Setabs{!B}{\Box !B \in \Gamma}\\
    \Diamond^{-1}\Gamma & = \Setabs{!B}{\Diamond !B \in \Gamma}\\
  \end{align*}
\end{defn}

In other words, $\Box\Gamma$ is $\Gamma$ with $\Box$ in front of every
!!{formula} in~$\Gamma$; $\Box^{-1}\Gamma$ is all the $\Box$'ed
!!{formula}s of~$\Gamma$ with the initial $\Box$'s removed.  This
definition is not terribly important on its own, but will simplify the
notation considerably.

Note that $\Box\Box^{-1}\Gamma \subseteq \Gamma$:
\[
\Box\Box^{-1}\Gamma = \Setabs{\Box!B}{\Box!B \in \Gamma}
\]
i.e., it's just the set of all those !!{formula}s of~$\Gamma$ that
start with~$\Box$.

\begin{lem}\ollabel{lem:box1}
  If $\Gamma \Proves[\Sigma] !A$ then $\Box\Gamma \Proves[\Sigma] \Box
  !A$.
\end{lem}

\begin{proof}
  If $\Gamma \Proves[\Sigma] !A$ then there are $!B_1$, \dots, $!B_k
  \in \Gamma$ such that $\Sigma \Proves !B_1 \lif (!B_2 \lif \cdots
  (!B_n \lif !A)\cdots)$. Since $\Sigma$ is normal, by rule \RK{},
  $\Sigma \Proves \Box!B_1 \lif (\Box!B_2 \lif \cdots (\Box!B_n \lif
  \Box!A)\cdots)$, where obviously $\Box!B_1$, \dots, $\Box!B_k \in
  \Box\Gamma$. Hence, by definition, $\Box\Gamma \Proves[\Sigma] \Box
  !A$.
\end{proof}

\begin{lem}\ollabel{lem:box2}
  If $\Box^{-1} \Gamma \Proves[\Sigma] !A$ then $\Gamma
  \Proves[\Sigma] \Box!A$.
\end{lem}

\begin{proof}
  Suppose $\Box^{-1}\Gamma \Proves[\Sigma] !A$; then by
  \olref{lem:box1}, $\Box\Box^{-1}\Gamma \Proves \Box!A$. But since
  $\Box\Box^{-1}\Gamma \subseteq \Gamma$, also $\Gamma \Proves[\Sigma]
  \Box!A$ by Monotony.
\end{proof}

\iftag{prvBox}{%
% Box is primitive
  
\begin{prop}\ollabel{prop:box}
  If $\Gamma$ is complete $\Sigma$-consistent, then $\Box !A \in
  \Gamma$ if and only if for every complete $\Sigma$-consistent
  $\Delta$ such that $\Box^{-1} \Gamma \subseteq \Delta$, it holds
  that $!A \in \Delta$.
\end{prop}

\begin{proof}
  Suppose $\Gamma$ is complete $\Sigma$-consistent. The ``only if''
  direction is easy: Suppose $\Box !A \in \Gamma$ and that
  $\Box^{-1}\Gamma \subseteq \Delta$. Since $\Box!A \in \Gamma$, $!A
  \in \Box^{-1}\Gamma \subseteq \Delta$, so $!A \in \Delta$.

  For the ``if'' direction, we prove the contrapositive: Suppose
  $\Box!A \notin \Gamma$. Since $\Gamma$ is complete
  $\Sigma$-consistent, it is deductively closed, and hence $\Gamma
  \Proves/[\Sigma] \Box !A$. By \olref{lem:box2},
  $\Box^{-1}\Gamma \Proves/[\Sigma] !A$. By
  \olref[prf][con]{prop:consistencyfacts}\olref[prf][con]{prop:consistencyfacts-b},
  $\Box^{-1}\Gamma \cup \{ \lnot!A \}$ is $\Sigma$-consistent. By
  Lindenbaum's Lemma, there is a complete $\Sigma$-consistent
  set~$\Delta$ such that $\Box^{-1}\Gamma \cup \{ \lnot!A \} \subseteq
  \Delta$. By consistency, $!A \notin \Delta$.
\end{proof}
}{%
% Box is not primitive, only Diamond is

\begin{prop}\ollabel{prop:diamond}
  If $\Gamma$ is complete $\Sigma$-consistent, then $\Diamond !A \in
  \Gamma$ if and only if for some complete $\Sigma$-consistent
  $\Delta$ such that $\Diamond\Delta \subseteq
  \Gamma$, it holds that $!A \in \Delta$.
\end{prop}

\begin{proof}
  Suppose $\Gamma$ is complete $\Sigma$-consistent. The right-to-left
  part is easy: Let $\Delta$ be complete $\Sigma$-consistent such that
  $\Diamond\Delta \subseteq \Gamma$ and $!A \in \Delta$.  Then
  $\Diamond!A \in \Diamond\Delta \subseteq \Gamma$, i.e., $\Diamond!A
  \in \Gamma$.

  For the left-to-right direction, assume $\Diamond !A \in \Gamma$. We
  have to show that there is a complete $\Sigma$-consistent $\Delta$
  with $!A \in \Delta$ and $\Diamond\Delta \subseteq \Gamma$.  Since
  $\Diamond !A \in \Gamma$, and $\Gamma$ is $\Sigma$-consistent,
  $\Box\lnot !A \notin \Gamma$. Since $\Gamma$ is complete
  $\Sigma$-consistent, it is deductively closed, so $\Gamma
  \Proves/[\Sigma] \Box\lnot !A$.  By \olref{lem:box2},
  $\Box^{-1}\Gamma \Proves/[\Sigma] \lnot !A$. By
  \olref[prf][con]{prop:consistencyfacts}\olref[prf][con]{prop:consistencyfacts-b},
  $\Box^{-1}\Gamma \cup \{ !A \}$ is $\Sigma$-consistent. By
  Lindenbaum's Lemma, there is a complete $\Sigma$-consistent
  set~$\Delta$ such that $\Box^{-1}\Gamma \cup \{ !A \} \subseteq
  \Delta$. Clearly, $!A \in \Delta$.  To see that $\Diamond\Delta
  \subseteq \Gamma$, assume it weren't: For some $!B \in \Delta$,
  $\Diamond !B \notin \Gamma$. Then, since $\Gamma$ is complete
  $\Sigma$-consistent, $\Box\lnot !B \in \Gamma$. But then also $\lnot
  !B \in \Box^{-1}\Gamma \subseteq \Delta$, which would make $\Delta$
  inconsistent.
\end{proof}
}

\begin{lem}\ollabel{lem:box-iff-diamond}
  Suppose $\Gamma$ and $\Delta$ are complete
  $\Sigma$-consistent. Then: $\Box^{-1}\Gamma \subseteq \Delta$ if and
  only if $\Diamond\Delta \subseteq \Gamma$.
\end{lem}

\begin{proof}
  ``Only if'' direction: Assume $\Box^{-1}\Gamma \subseteq \Delta$ and
  suppose $\Diamond!A \in \Diamond\Delta$ (i.e., $!A \in \Delta$).  In
  order to show $\Diamond!A \in \Gamma$ it suffices to show
  $\Box\lnot!A \notin \Gamma$ for then by maximality $\lnot\Box\lnot
  !A \in \Gamma$. Now, if $\Box\lnot!A \in \Gamma$ then by hypothesis
  $\lnot!A \in \Delta$, against the consistency of $\Delta$ (since $!A
  \in \Delta$). Hence $\Box\lnot!A \notin \Gamma$, as required.

  ``If'' direction: Assume $\Diamond\Delta
  \subseteq \Gamma$. We argue contrapositively: suppose $!A
  \notin \Delta$ in order to show $\Box!A \notin \Gamma$. If
  $!A \notin \Delta$ then by maximality $\lnot!A \in \Delta$
  and so by hypothesis $\Diamond\lnot!A \in \Gamma$. But in a
  normal modal logic $\Diamond\lnot!A$ is equivalent to
  $\lnot\Box !A$, and if the latter is in $\Gamma$, by consistency
  $\Box!A \notin\Gamma$, as required.
\end{proof}

\iftag{prvBox}{%
  \iftag{prvDiamond}{%
% Box and Diamond are both primitive; prove
% prop:diamond using lem:box-iff-diamond
  
\begin{prop}\ollabel{prop:diamond}
  If $\Gamma$ is complete $\Sigma$-consistent, then $\Diamond !A \in
  \Gamma$ if and only if for some complete $\Sigma$-consistent
  $\Delta$ such that $\Diamond\Delta \subseteq
  \Gamma$, it holds that $!A \in \Delta$.
\end{prop}

\begin{proof}
Suppose $\Gamma$ is complete $\Sigma$-consistent.  $\Diamond!A \in
\Gamma$ iff $\lnot\Box\lnot !A \in \Gamma$ by \Dual{} and closure.
$\lnot\Box\lnot !A \in \Gamma$ iff $\Box\lnot !A \notin \Gamma$ by
\olref[ccs]{prop:ccs-properties}\olref[ccs]{prop:ccs-lnot} since
$\Gamma$ is complete $\Sigma$-consistent.  By \olref{prop:box},
$\Box\lnot !A \notin \Gamma$ iff, for some complete
$\Sigma$-consistent $\Delta$ with $\Box^{-1}\Gamma \subseteq \Delta$,
$\lnot!A \notin \Delta$.  Now consider any such~$\Delta$. By
\olref{lem:box-iff-diamond}, $\Box^{-1}\Gamma \subseteq \Delta$ iff
$\Diamond\Delta \subseteq \Gamma$. Also, $\lnot !A \notin \Delta$ iff
$!A \in \Delta$ by
\olref[ccs]{prop:ccs-properties}\olref[ccs]{prop:ccs-lnot}.  So
$\Diamond!A \in \Gamma$ iff, for some complete $\Sigma$-consistent
$\Delta$ with $\Diamond\Delta \subseteq \Gamma$, $!A \in \Delta$.
\end{proof}

}{}}{}
  
\begin{prob}
  Show that if $\Gamma$ is complete $\Sigma$-consistent, then
  \iftag{prvBox}{$\Diamond !A \in \Gamma$ if and only if for some
    complete $\Sigma$-consistent $\Delta$ such that $\Box^{-1}\Delta'
    \subseteq \Gamma$, it holds that $!A \in \Delta$.}{$\Box !A \in
    \Gamma$ if and only if for every complete $\Sigma$-consistent
    $\Delta$ such that $\Box^{-1} \Gamma \subseteq \Delta$, it holds
    that $!A \in \Delta$.}  \emph{Do this without using
    \olref[mod][com][mod]{lem:box-iff-diamond}}.
\end{prob}

\end{document}
