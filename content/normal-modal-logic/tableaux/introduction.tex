% Part: normal-modal-logic
% Chapter: tableaux
% Section: introduction

\documentclass[../../../include/open-logic-section]{subfiles}

\begin{document}

\olfileid{mod}{tab}{int}

\olsection{Introduction}

!!^{tableau}s are certain (downward-branching) trees of !!{signed
  formula}s, i.e., pairs consisting of a truth value sign ($\True$ or
$\False$) and a !!{sentence}
\[
\sFmla{\True}{!A} \text{ or } \sFmla{\False}{!A}.
\]
!!^a{tableau} begins with a number of \emph{assumptions}. Each further
!!{signed formula} is generated by applying one of the inference
rules. Some inference rules add one or more !!{signed formula}s to a
tip of the tree; others add two new tips, resulting in two branches.
Rules result in !!{signed formula}s where the !!{formula} is
less complex than that of the !!{signed formula} to which it was
applied. When a branch contains both $\sFmla{\True}{!A}$ and
$\sFmla{\False}{!A}$, we say the branch is \emph{closed}. If every
branch in !!a{tableau} is closed, the entire !!{tableau} is closed. A
closed !!{tableau} consititues !!a{derivation} that shows that the set
of !!{signed formula}s which were used to begin the !!{tableau} are
unsatisfiable.  This can be used to define a $\Proves$ relation:
$\Gamma \Proves !A$ iff there is some finite set~$\Gamma_0 = \{!B_1,
\dots, !B_n\} \subseteq \Gamma$ such that there is a closed
!!{tableau} for the assumptions
\[
\{\sFmla{\False}{!A}, \sFmla{\True}{!B_1}, \dots, \sFmla{\True}{!B_n}\}.
\]

For modal logics, we have to both extend the notion of !!{signed
formula} and add rules that
cover~\iftag{prvBox}{$\Box$\iftag{prvDiamond}{ and
    $\Diamond$}}{$\Diamond$}. In addition to a sign($\True$ or
$\False$), !!{formula}s in modal !!{tableau}s also have
\emph{prefixes}~$\sigma$. The prefixes are non-empty sequences of
positive integers, i.e., $\sigma \in (\PosInt)^* \setminus
\{\emptyseq\}$. When we write such prefixes without the surrounding
$\tuple{\ }$, and separate the individual !!{element}s by~$.$'s
instead of $,$'s. If $\sigma$ is a prefix, then $\sigma.n$ is $\sigma
\concat \tuple{n}$; e.g., if $\sigma = 1.2.1$, then $\sigma.3$ is
$1.2.1.3$. So for instance,
\[
\sFmla{\True}{\Box !A \lif !A}[1.2]
\]
is a \emph{prefixed !!{signed formula}} (or just a \emph{prefixed
  !!{formula}} for short).

Intuitively, the prefix names a world in a model that might satisfy
the !!{formula}s on a branch of !!a{tableau}, and if $\sigma$ names
some world, then $\sigma.n$ names a world accessible from (the world
named by)~$\sigma$.

\end{document}
