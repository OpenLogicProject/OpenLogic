% Part: normal-modal-logic
% Chapter: axioms-systems
% Section: derived-rules

\documentclass[../../../include/open-logic-section]{subfiles}

\begin{document}

\olfileid{nml}{prf}{der}

\olsection{Derived Rules}

Finding and writing !!{derivation}s is obviously difficult,
cumbersome, and repetitive. For instance, very often we want to pass
from $!A \lif !B$ to $\Box !A \lif \Box !B$, i.e., apply
rule~\RK. That requires an application of \Nec, then recording the
proper instance of~\Ax{K}, then applying~\MP. Passing from $!A \lif
!B$ and $!B \lif !C$ to $!A \lif !C$ requires recording the (long)
tautological instance
\[
(!A \lif !B) \lif ((!B \lif !C) \lif (!A \lif !C))
\]
and applying \MP{} twice. Often we want to replace a sub-!!{formula}
by a formula we know to be equivalent, e.g., \iftag{prvDiamond}{$\Diamond
  !A$ by $\lnot\Box\lnot !A$}{$\Box !A$ by $\lnot\Diamond\lnot !A$}, or
$\lnot\lnot !A$ by $!A$. So rather than write out the actual
!!{derivation}, it is more convenient to simply record why the
intermediate steps are !!{derivable}. For this purpose, let us collect
some facts about !!{derivability}.

\begin{prop}
  If $\Log{K} \Proves !A_1$, \dots, $\Log{K} \Proves !A_n$, and $!B$
  follows from $!A_1$, \dots, $!A_n$ by propositional logic, then
  $\Log{K} \Proves !B$.
\end{prop}

\begin{proof}
  If $!B$ follows from $!A_1$, \dots,~$!A_n$ by propositional logic, then
  \[
  !A_1 \lif (!A_2 \lif \cdots (!A_n \lif !B)\dots)
  \]
  is a tautological instance. Applying \MP{} $n$ times gives a
  !!{derivation} of~$!B$.
\end{proof}

We will indicate use of this proposition by~\PL.

\begin{prop}
  If $\Log{K} \Proves !A_1 \lif (!A_2 \lif \cdots (!A_{n-1} \lif
  !A_n)\dots)$ then $\Log{K} \Proves \Box !A_1 \lif (\Box !A_2 \lif
  \cdots (\Box !A_{n-1} \lif \Box !A_n)\dots)$.
\end{prop}

\begin{proof}
  By induction on $n$, just as in the proof of \olref[nor]{prop:rk}.
\end{proof}

We will indicate use of this proposition by~\RK. Let's illustrate how
these results help establishing !!{derivability} results more easily.

\begin{prop}
  $\Log{K} \Proves (\Box!A \land \Box!B) \lif \Box (!A \land !B)$
\end{prop}

\begin{proof}
  \begin{derivation}
    1. & $\Log{K} \Proves !A \lif (!B \lif (!A \land !B))$ & \Taut \\
    2. & $\Log{K} \Proves \Box!A \lif (\Box !B \lif \Box(!A \land !B)))$ & \RK, 1\\
    3. & $\Log{K} \Proves (\Box!A \land \Box!B) \lif \Box(!A \land !B)$ & \PL, 2
  \end{derivation}
\end{proof}

\begin{prop}\ollabel{prop:rewriting}
  If $\Log{K} \Proves !A \liff !B$ and $\Log{K} \Proves
  \Subst{!C}{!A}{q}$ then $\Log{K} \Proves \Subst{!C}{B}{q}$
\end{prop}

\begin{proof}
  Exercise.
\end{proof}

\begin{prob}
  Prove \olref[nml][prf][der]{prop:rewriting} by proving, by induction
  on the complexity of~$!C$, that if $\Log{K} \Proves !A \liff !B$
  then $\Log{K} \Proves \Subst{!C}{!A}{q} \liff \Subst{!C}{!B}{q}$.
\end{prob}

This proposition comes in handy especially when we want to convert
$\Diamond$ into $\Box$ (or vice versa), or remove double negations
inside !!a{formula}.  In what follows, we will mark applications of
\olref{prop:rewriting} by ``$!A$ for $!B$'' whenever we re-write
!!a{formula}~$!C(!B)$ for~$!C(!A)$. In other words, ``$!A$ for $!B$''
abbreviates:
  \begin{derivation}
    & $\Proves !C(!A)$ \\
    & $\Proves !A \liff !B$\\
    & $\Proves !C(!B)$ & by \olref{prop:rewriting}
  \end{derivation}
For instance:

\begin{prop}
  $\Log{K} \Proves \lnot\Box p \lif \Diamond \lnot p$
\end{prop}

\begin{proof}
  \iftag{prvDiamond}{% Diamond is primitive
  \begin{derivation}
    1. & $\Log{K} \Proves \Diamond \lnot p \liff \lnot\Box\lnot\lnot p$ &
    \Dual\\
    2. & $\Log{K} \Proves \lnot \Box \lnot\lnot p \lif \Diamond \lnot p$ & \PL, 1\\
    3. & $\Log{K} \Proves \lnot\Box p \lif \Diamond \lnot p$ & $p$ for $\lnot\lnot p$\\
  \end{derivation}
  }{% Diamond is defined
  \begin{derivation}
    1. & $\Log{K} \Proves \lnot \Box \lnot\lnot p \lif \Diamond \lnot p$ & \Taut\\
    2. & $\Log{K} \Proves \lnot\Box p \lif \Diamond \lnot p$ & $p$ for $\lnot\lnot p$\\
  \end{derivation}
  The formula on line~$1$ is an instance of $\lnot q \lif \lnot q$,
  substituting $\Box\lnot\lnot p$ for~$q$. $\Diamond\lnot p$ and
  $\lnot\Box\lnot\lnot p$ are identical, since $\Diamond$ is defined
  as~$\lnot\Box\lnot$.}
\end{proof}

In the above !!{derivation}, the final step ``$p$ for $\lnot\lnot p$''
is short for
  \begin{derivation}
    & $\Log{K} \Proves \lnot \Box \lnot\lnot p \lif \Diamond \lnot
    p$\\
    & $\Log{K} \Proves \lnot\lnot p \liff p$ & \Taut\\
    & $\Log{K} \Proves \lnot\Box p \lif \Diamond \lnot p$ & by \olref{prop:rewriting}
  \end{derivation}
The roles of $!C(q)$, $!A$, and $!B$ in \olref{prop:rewriting} are
played here, respectively, by $\lnot \Box q \lif \Diamond \lnot p$,
$\lnot\lnot p$, and~$p$.

When !!a{formula} contains a sub-!!{formula} $\lnot\Diamond !A$,
we can replace it by $\Box\lnot !A$ using \olref{prop:rewriting}, since
$\Log{K} \Proves \lnot\Diamond !A \liff \Box\lnot !A$. We'll indicate
this and similar replacements simply by ``$\Box\lnot$ for
$\lnot\Diamond$.''

The following proposition justifies that we can establish
!!{derivability} results schematically. E.g., the previous proposition
does not just establish that $\Log{K} \Proves \lnot\Box p \lif
\Diamond \lnot p$, but $\Log{K} \Proves \lnot\Box !A \lif \Diamond
\lnot !A$ for arbitrary~$!A$.

\begin{prop}
  If $!A$ is a substitution instance of $!B$ and $\Log{K} \Proves !B$,
  then $\Log{K} \Proves !A$.
\end{prop}

\begin{proof}
  It is tedious but routine to verify (by induction on the length of
  the !!{derivation} of~$!B$) that applying a substitution to an
  entire !!{derivation} also results in a correct
  !!{derivation}. Specifically, substitution instances of tautological
  instances are themselves tautological instances, substitution
  instances of instances of~\iftag{prvDiamond}{\Dual{} and~}{}\Ax{K}
  are themselves instances of~\iftag{prvDiamond}{\Dual{}
    and~}{}\Ax{K}, and applications of \MP{} and~\Nec{} remain correct
  when substituting !!{formula}s for !!{propositional variable}s in
  both premise(s) and conclusion.
\end{proof}


\end{document}
