% Part: normal-modal-logic
% Chapter: axioms-systems
% Section: soundness

\documentclass[../../../include/open-logic-section]{subfiles}

\begin{document}

\olfileid{nml}{prf}{snd}

\olsection{Soundness}

!!^a{derivation} system is called sound if everything that can be
!!{derive}d is valid. When considering modal systems, i.e.,
!!{derivation}s where in addition to \Ax{K} we can use instances of
some !!{formula}s $!A_1$, \dots,~$!A_n$, we want every !!{derivable}
formula to be true in any model in which $!A_1$, \dots, $!A_n$ are
true.  

\begin{thm}[Soundness Theorem]\ollabel{thm:soundness}
  If every instance of $!A_1$, \dots, $!A_n$ is valid in the
  classes of models $\mClass{C}_1$, \dots, $\mClass{C}_n$,
  respectively, then $\Log{K}!A_1\dots !A_n
  \Proves !B$ implies that $!B$ is valid in the class of
  models $\mClass{C}_1 \cap \dots \cap \mClass{C}_n$.
\end{thm}

\begin{proof}
  By induction on length of proofs. For brevity, put $\mClass{C} =
  \mClass{C}_n \cap \dots \cap \mClass{C}_n$.
  \begin{enumerate}
  \item Induction Basis: If $!B$ has a proof of length~$1$, then it is
    either a tautological instance, an instance
    of~\Ax{K},\iftag{prvDiamond}{ or of~\Dual{},}{} or an instance of
    one of $!A_1$, \dots,~$!A_n$. In the first case, $!B$ is valid in
    $\mClass{C}$, since tautological instance are valid in \emph{any}
    class of models, by \olref[syn][tau]{prop:valid-taut}. Similarly
    in the second case, by
    \olref[syn][sch]{prop:Kvalid}\iftag{prvDiamond}{ and
      \olref[syn][sch]{prop:Dual-valid}}{}. Finally in the third case,
    since $!B$ is valid in $\mClass{C}_i$ and $\mClass{C} \subseteq
    \mClass{C}_i$, we have that $!B$ is valid in $\mClass{C}$ as well.
  \item Inductive step: Suppose $!B$ has a proof of length $k>1$. If
    $!B$ is a tautological instance or an instance of one of $!A_1$,
    \dots, $!A_n$, we proceed as in the previous step. So suppose $!B$ is
    obtained by \MP{} from previous !!{formula}s $!C \lif !B$ and
    $!C$. Then $!C \lif !B$ and $!C$ have proofs of length $<k$, and
    by inductive hypothesis they are valid in~$\mClass{C}$. By
    \olref[syn][sch]{prop:soundMP}, $!B$ is valid in $\mClass{C}$ as
    well. Finally suppose $!B$ is obtained by \Nec{} from $!C$ (so
    that $!B = \Box!C$). By inductive hypothesis, $!C$ is valid in
    $\mClass{C}$, and by \olref[syn][val]{prop:Nec-rule} so is~$!B$. \qedhere
  \end{enumerate}
\end{proof}

\end{document}
