% Part: normal-modal-logic
% Chapter: axioms-systems
% Section: consistency

\documentclass[../../../include/open-logic-section]{subfiles}

\begin{document}

\olfileid{nml}{prf}{con}

\olsection{Consistency}

Consistency is an important property of sets of !!{formula}s. A set of
!!{formula}s is inconsistent if a contradiction, such as~$\lfalse$, is
!!{derivable} from it; and otherwise consistent. If a set is
inconsistent, its !!{formula}s cannot all be true in a model at a
world. For the completeness theorem we prove the converse: every
consistent set is true at a world in a model, namely in the ``canonical
model.''

\begin{defn}
  A set $\Gamma$ is \emph{consistent} relatively to a system~$\Sigma$
  or, as we will say, $\Sigma$-consistent, if and only if $\Gamma
  \Proves/[\Sigma] \lfalse$.
\end{defn}

So for instance, the set $\{ \Box(p \lif q), \Box p, \lnot\Box q \}$ is
consistent relatively to propositional logic, but not
\Log{K}-consistent. Similarly, the set $\{ \Diamond p, \Box\Diamond
p \lif q, \lnot q \}$ is not \Log{K5}-consistent.

\begin{prop}\ollabel{prop:consistencyfacts}
  Let $\Gamma$ be a set of !!{formula}s. Then:
  \begin{enumerate}
  \item $\Gamma$ is $\Sigma$-consistent if and only if there is
    some !!{formula}~$!A$ such that $\Gamma \Proves/[\Sigma]
    !A$.
  \item \ollabel{prop:consistencyfacts-b}%
    $\Gamma \Proves[\Sigma] !A$ if and only if $\Gamma \cup \{
    \lnot!A \}$ is not $\Sigma$-consistent.
  \item \ollabel{prop:consistencyfacts-c}%
    If $\Gamma$ is $\Sigma$-consistent, then for any !!{formula}
    $!A$, either $\Gamma \cup \{ !A \}$ is
    $\Sigma$-consistent or $\Gamma \cup \{ \lnot!A \}$ is
    $\Sigma$-consistent.
  \end{enumerate}
\end{prop}

\begin{proof}
  These facts follow easily using classical propositional logic. We
  give the argument for \olref{prop:consistencyfacts-c}. Proceed
  contrapositively and suppose neither $\Gamma \cup \{ !A \}$ nor
  $\Gamma \cup \{ \lnot!A \}$ is $\Sigma$-consistent. Then by
  \olref{prop:consistencyfacts-b}, both $\Gamma, !A \Proves[\Sigma]
  \lfalse$ and $\Gamma, \lnot !A \Proves[\Sigma] \lfalse$. By the
  deduction theorem $\Gamma \Proves[\Sigma] !A \to \lfalse$ and
  $\Gamma \Proves[\Sigma] \lnot!A \lif \lfalse$. But $(!A \lif
  \lfalse) \lif ((\lnot!A \lif \lfalse) \lif \lfalse)$ is a
  tautological instance, hence by
  \olref[prp]{prop:derivabilityfacts}\olref[prp]{prop:derivabilityfacts-ruleT},
  $\Gamma \Proves[\Sigma] \lfalse$.
\end{proof}

\end{document}
