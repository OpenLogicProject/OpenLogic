% Part: normal-modal-logic
% Chapter: axioms-systems
% Section: logics-proofs

\documentclass[../../../include/open-logic-section]{subfiles}

\begin{document}

\olfileid{mod}{prf}{prf}

\olsection{Logics Defined by Proofs}

\begin{defn}
  Given a modal system $\Log{K} !A_1 \dots !A_n$
  and !!a{formula} $!B$ we say that $!B$ is \emph{!!{derivable}} in
  $\Log{K} !A_1 \dots !A_n$, written $\Log{K}
  !A_1 \dots !A_n \Proves !B$, if and only if
  there are !!{formula}s $!C_1$, \dots, $!C_k$ such that $!C_k
  = !B$ and each $!C_i$ is either a tautological instance,
  or an instance of the schemas $\Ax{K}$, $!A_1$, \dots,
  $!A_n$, or it follows from previous !!{formula}s by means of the
  rules \MP{} or~\Nec.
\end{defn}

The following proposition allows us to show that $!B \in \Sigma$
by exhibiting a $\Sigma$-proof of~$!B$.

\begin{prop}
  $\Log{K} !A_1 \dots !A_n = \Setabs{!B}{\Log{K} !A_1 \dots !A_n \Proves !B}$.
\end{prop}

\begin{proof}
  We use induction on the length of proofs to show that
  $\Setabs{!B}{\Log{K} !A_1 \dots !A_n \Proves !B} \subseteq \Log{K}
  !A_1 \dots !A_n$. The converse inclusion follows by showing that
  $\Setabs{!B}{\Log{K} !A_1 \dots !A_n \Proves !A }$ is a normal
  modal logic containing all the instances of the schemas $!A_1$,
  \dots, $!A_n$, and the observation that $\Log{K} !A_1 \dots !A_n$
  is, by definition, the smallest such logic.
\end{proof}

\end{document}
