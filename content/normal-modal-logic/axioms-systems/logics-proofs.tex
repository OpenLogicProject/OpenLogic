% Part: normal-modal-logic
% Chapter: axioms-systems
% Section: logics-proofs

\documentclass[../../../include/open-logic-section]{subfiles}

\begin{document}

\olfileid{nml}{prf}{prf}

\olsection{\usetoken{P}{derivation} and Modal Systems}

We first define what !!a{derivation} is for normal modal
logics. Roughly, !!a{derivation} is a sequence of !!{formula}s in
which every !!{element} is either (a substitution instance of) one of
a number of \emph{axioms}, or follows from previous !!{element}s by
one of a few inference rules. For normal modal logics, all instances
of tautologies\iftag{prvDiamond}{, \Ax{K}, and~\Dual{}}{ and~\Ax{K}}
count as axioms. This results in the modal system~$\Log{K}$, the
smallest normal modal logic. We may wish to add additional axioms to
obtain other systems, however. The rules are always modus ponens~\MP{}
and necessitation~\Nec.

\begin{defn}
  Given a modal system $\Log{K} !A_1 \dots !A_n$ and !!a{formula} $!B$
  we say that $!B$ is \emph{!!{derivable}} in $\Log{K} !A_1 \dots
  !A_n$, written $\Log{K} !A_1 \dots !A_n \Proves !B$, if and only if
  there are !!{formula}s $!C_1$, \dots, $!C_k$ such that $!C_k = !B$
  and each $!C_i$ is either a tautological instance, or an instance of
  one of $\Ax{K}$,\iftag{prvDiamond}{ $\Dual$,}{} $!A_1$,
  \dots, $!A_n$, or it follows from previous !!{formula}s by means of
  the rules \MP{} or~\Nec.
\end{defn}

The following proposition allows us to show that $!B \in \Sigma$
by exhibiting a $\Sigma$-!!{derivation} of~$!B$.

\begin{prop}
  $\Log{K} !A_1 \dots !A_n = \Setabs{!B}{\Log{K} !A_1 \dots !A_n \Proves !B}$.
\end{prop}

\begin{proof}
  We use induction on the length of !!{derivation}s to show that
  $\Setabs{!B}{\Log{K} !A_1 \dots !A_n \Proves !B} \subseteq
  \Log{K} !A_1 \dots !A_n$.

  If the !!{derivation} of~$!B$ has length~$1$, it contains a single
  !!{formula}. That !!{formula} cannot follow from previous formulas
  by \MP{} or \Nec, so must be a tautological instance, an instance of
  \Ax{K},\iftag{prvDiamond}{ $\Dual$,}{} or an instance
  of one of $!A_1$, \dots,~$!A_n$. But $\Log{K}!A_1\dots!A_n$ contains
  these as well, so $!B \in \Log{K}!A_1 \dots !A_n$.

  If the !!{derivation} of $!B$ has length~$> 1$, then $!B$ may in
  addition be obtained by \MP{} or \Nec{} from !!{formula}s not
  occurring as the last line in the !!{derivation}. If $!B$ follows
  from $!C$ and $!C \lif !B$ (by \MP), then $!C$ and $!C \lif !B \in
  \Log{K}!A_1 \dots !A_n$ by induction hypothesis. But every modal
  logic is closed under modus ponens, so $!B \in \Log{K}!A_1 \dots
  !A_n$. If $!B \equiv \Box !C$ follows from $!C$ by \Nec, then $!C
  \in \Log{K}!A_1 \dots !A_n$ by induction hypothesis. But every
  normal modal logic is closed under $\Nec$, so $!B \in
  \Log{K}!A_1\dots!A_n$.

  The converse inclusion follows by showing that
  $\Sigma = \Setabs{!B}{\Log{K} !A_1 \dots !A_n \Proves !B }$ is a normal
  modal logic containing all the instances of $!A_1$,
  \dots, $!A_n$, and the observation that $\Log{K} !A_1 \dots !A_n$
  is, by definition, the smallest such logic.
  \begin{enumerate}
    \item Every tautology~$!B$ is a tautological instance, so
      $\Log{K}!A_1\dots!A_n \Proves !B$, so $\Sigma$ contains all
      tautologies.
    \item If $\Log{K}!A_1\dots!A_n \Proves !C$ and
      $\Log{K}!A_1\dots!A_n \Proves !C \lif !B$, then
      $\Log{K}!A_1\dots!A_n \Proves !B$: Combine the !!{derivation} of
      $!C$ with that of $!C \lif !B$, and add the line~$!B$. The last
      line is justified by \MP{}. So $\Sigma$ is closed under modus
      ponens.
    \item If $!B$ has !!a{derivation}, then every substitution
      instance of~$!B$ also has a derivation: apply the substitution
      to every !!{formula} in the !!{derivation}. (Exercise: prove by
      induction on the length of !!{derivation}s that the result is
      also a correct !!{derivation}). So $\Sigma$ is closed under
      uniform substitution. (We have now established that $\Sigma$
      satisfies all conditions of a modal logic.)
    \item We have $\Log{K}!A_1\dots!A_n \Proves \Ax{K}$, so $K \in \Sigma$.
    \tagitem{prvDiamond}{We have
      $\Log{K}!A_1\dots!A_n \Proves \Dual$, so $\Dual \in \Sigma$.}{}
    \item If $\Log{K}!A_1\dots!A_n \Proves !C$, the additional
      line~$\Box !C$ is justified by~\Nec. Consequently, $\Sigma$ is
      closed under~\Nec. Thus, $\Sigma$ is normal.
  \end{enumerate}
\end{proof}

\end{document}
