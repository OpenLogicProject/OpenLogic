% Part: normal-modal-logic
% Chapter: axioms-systems
% Section: duals

\documentclass[../../../include/open-logic-section]{subfiles}

\begin{document}

\olfileid{nml}{prf}{dua}

\olsection{Dual \usetoken{P}{formula}}

\begin{defn}\ollabel{def:duals}
  Each of the !!{formula}s \Ax{T}, \Ax{B}, \Ax{4}, and
  \Ax{5} has a \emph{dual}, denoted by a subscripted diamond, as
  follows:
  \begin{align}
    \tag{\Ax{T_\Diamond}} p & \lif \Diamond p\\
    \tag{\Ax{B_\Diamond}} \Diamond\Box p & \lif p\\
    \tag{\Ax{4_\Diamond}} \Diamond\Diamond p & \lif \Diamond p\\
    \tag{\Ax{5_\Diamond}} \Diamond\Box p & \lif \Box p
    \end{align}
\end{defn}

Each of the above dual !!{formula}s is obtained from the corresponding
!!{formula} by substituting $\lnot p$ for $p$, contraposing, replacing
$\lnot\Box\lnot$ by $\Diamond$, and replacing $\lnot\Diamond\lnot$
by~$\Box$. \Ax{D}, i.e., $\Box!A \lif \Diamond!A$ is its own dual in
that sense.

\begin{prop}\ollabel{prop:dualsys}
  For each !!{formula}~$!A$ in \olref[nml][prf][dua]{def:duals}:
  $\Log{K}!A = \Log{K}!A_{\Diamond}$.
\end{prop}
\begin{proof}
  Exercise.
\end{proof}
\begin{prob}
  Prove \olref[nml][prf][dua]{prop:dualsys}.
\end{prob}

\end{document}
