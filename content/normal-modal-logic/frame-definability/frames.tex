% Part: normal-modal-logic
% Chapter: frame-correspondence
% Section: frames

\documentclass[../../../include/open-logic-section]{subfiles}

\begin{document}

\olfileid{mod}{frd}{fra}

\olsection{Frames}

\begin{defn}
  A \emph{frame} is a pair $\mModel{F} = \tuple{W,R}$ where $W$ is a
  non-empty set of worlds and $R$ a binary relation on~$W$. A model
  $\mModel{M}$ is \emph{based on} a frame $\mModel{F} = \tuple{W,R}$
  if and only if $\mModel{M} = \tuple{W, R, V}$.
\end{defn}

\begin{defn}
  If $\mClass{F}$ is a class of frames, we write $\mClass{F} \Entails
  !A$, ``$!A$ is valid in $\mClass{F}$,'' to mean that $!A$ is true in
  every model $\mModel{M}$ based on a frame~$\mModel{F} \in
  \mClass{F}$.
\end{defn}

The reason frames are interesting is that correspondence between
schemas and properties of the accessibility relation~$R$ is at the
level of frames, \emph{not of models}.

\begin{rem}
  Obviously, if !!a{formula} or a schema is valid, i.e., valid with
  respect to the class of \emph{all} models, it is also valid with
  respect to any  class~$\mClass{F}$ of frames.
\end{rem}

\end{document}
