% Part: normal-modal-logic
% Chapter: filtrations
% Section: preliminaries

\documentclass[../../../include/open-logic-section]{subfiles}

\begin{document}

\olfileid{mod}{fil}{pre}

\olsection{Preliminaries}

Filtrations allow us to establish the decidability of our systems of
modal logic by showing that they have the \emph{finite model
  property}, i.e., that any !!{formula} that is true (false) in a
model is also true (false) in a \emph{finite} model.

\begin{defn}\ollabel{def:modallyclosed}
  A set $\Gamma$ of !!{formula}s is \emph{closed under subformulas} if it
  contains every subformula of !!a{formula} in $\Gamma$. Further,
  $\Gamma$ is \emph{modally closed} if it is closed under subformulas
  and moreover $!A \in \Gamma$ implies $\Box!A,
  \Diamond!A \in \Gamma$. 
\end{defn}

\begin{defn}
Let $\mModel{M} =\tuple{W, R, V}$ and suppose $\Gamma$ is closed under
subformulas. Define a relation $\equiv$ on $W$ to hold of any two
worlds that make true the same !!{formula}s from $\Gamma$, i.e.:
\[
u \equiv v \quad \text{if and only if }\quad \forall !A \in \Gamma : \mSat{M}{!A}[u] \Leftrightarrow \mSat{N}{!A}[v].
\]
Clearly, $\equiv$ is an equivalence relation over $W$. Standardly, for
any $w \in W$, the equivalence class of $w$ is denoted by $[w]$.
\end{defn}

\end{document}
