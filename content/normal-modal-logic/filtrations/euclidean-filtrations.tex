% Part: normal-modal-logic
% Chapter: filtrations
% Section: euclidean-filtrations

\documentclass[../../../include/open-logic-section]{subfiles}

\begin{document}

\olfileid{mod}{fil}{euc}

\olsection{Filtrations of Euclidean Models}

The approach of \olref[acc]{sec} does not work in the case of models
that are euclidean or serial and euclidean. Consider the model at the
top of \olref{fig:ser-eucl}, which is both euclidean and
serial. Let $\Gamma = \{p, \Box p \}$. When taking a filtration
through $\Gamma$, then $[w_1] = [w_3]$ since $w_1$ and $w_3$ are the
only worlds that agree on $\Gamma$. Any filtration will also have the
arrow inherited from $\mModel{M}$, as depicted in
\olref{fig:ser-eucl2}. But we cannot add arrows to that model
in order to make it euclidean, for then there would be a double arrow
between $w_2$ and $w_4$, and hence also between $w_2$ and $w_5$. But
$\Box p$ is true at $w_2$ while $p$ is false at~$w_5$.

\begin{figure}[htpb]
  \centering
  \begin{tikzpicture}[node distance=2cm, auto, thick]
    \node (w1) at (0, 0) [label=90:$w_1$, label=below:$\lnot p$]{$\bullet$}; 
    \node (w2) at (2, 0) [label=90:$w_2$, label=below:$p$]{$\bullet$}; 
    \draw[->] (w1) to node {} (w2);
    \path node at (0,-1) {$\Box p$};
    \path node at (2,-1) {$\Box p$};
    \node (w3) at (0, -2.5) [label=90:$w_3$, label=below:$\lnot p$]{$\bullet$}; 
    \node (w4) at (2, -2.5) [label=135:$w_4$,
      label=below:$p$]{$\bullet$}edge [in=60,out=120,loop] (); 
    \draw[->] (w3) to node {} (w4);
    \node (w5) at (4, -2.5) [label=90:$w_5$, label=below:$\lnot p$]{$\bullet$}; 
    \draw[<->] (w4) to node {} (w5) edge [in=30,out=-30,loop] () ;
    \path node at (0,-3.5) {$\Box p$};
    \path node at (2,-3.5) {$\lnot\Box p$};
    \path node at (4,-3.5) {$\lnot\Box p$};
    \draw [rounded corners] (-1,-4) -- ++(6.25,0)  -- ++(0,5) -- ++(-6.25,0) --  cycle;
    % \draw[->, bend left] (w1) to node {$R$} (w2); 
    % \draw[->, bend left] (w2) to node {} (w1); 
    % \draw [rounded corners] (-1,-1) -- ++(0,2)  -- ++(4.25,0) -- ++(0,-2) --  cycle;
    % \path node at (2.75,0.75) {$\mModel{M}$};
  \end{tikzpicture}
  \caption{A serial and euclidean model.}\ollabel{fig:ser-eucl}
\end{figure}

\begin{figure}[ht]
  \centering
  \begin{tikzpicture}[node distance=2cm, auto, thick]
    \node (w1) at (-0.5, -1) [label=110:{$[w_1]=[w_3]$}, label=below:$\lnot
      p$]{$\bullet$}; 
    \path node at (-0.5,-2) {$\Box p$};
    \node (w2) at (2, 0) [label=90:$w_2$,
      label=0:{$p, \Box p$}]{$\bullet$}; 
    \node (w4) at (2, -2.5) [label=210:$w_4$,
      label=below:$p$]{$\bullet$}edge [in=60,out=120,loop] (); 
    \draw[->] (w1) to node {} (w2);
    \draw[->] (w1) to node {} (w4);
    \node (w5) at (4, -2.5) [label=90:$w_5$, label=below:$\lnot p$]{$\bullet$}; 
    \draw[<->] (w4) to node {} (w5) edge [in=30,out=-30,loop] () ;
    \path node at (2, -3.5) {$\lnot\Box p$};
    \path node at (4, -3.5) {$\lnot\Box p$};
    \draw [rounded corners] (-3,-4) -- ++(8.25,0)  -- ++(0,5) -- ++(-8.25,0) --  cycle;
  \end{tikzpicture}
  \caption{The filtration of the model in \olref{fig:ser-eucl}.}
  \ollabel{fig:ser-eucl2}
\end{figure}

In particular, it is not enough to consider filtrations through
arbitrary $\Gamma$'s closed under subsentences. Instead we need to
consider sets $\Gamma$ that are \emph{modally closed} (see
\olref[pre]{def:modallyclosed}). Such sets of sentences are
infinite, and therefore do not lead immediately to the decidability of
the corresponding system.

\begin{thm}
  Let $\Gamma$ be modally closed and $\mModel{M}=\tuple{W,R,V}$. If
  $\mModel{M^*} = \tuple{W^*,R^*,V^*}$ is a coarsest filtration of
  $\mModel{M}$, then $\mModel{M^*}$ is symmetric, transitive or
  euclidean if $\mModel{M}$ is symmetric, transitive, or euclidean,
  respectively.
\end{thm}

\begin{proof}
  The proof of transitivity uses the validity of both \Ax{4} and
  $\Ax{4_\Diamond}$ in all transitive models, and likewise
  euclideanness uses the fact that both \Ax{5} and
  $\Ax{5_\Diamond}$ are valid in all euclidean models, and the
  proof of symmetry likewise uses both \Ax{B} and
  $\Ax{B_\Diamond}$.

  If $\mModel{M^*}$ is a coarsest filtration, then by definition
  $R^*[u][v]$ holds if and only if $C_1(u,v)$. For transitivity,
  suppose $C_1(u,v)$ and $C_1(v,w)$: to show $C_1(u,w)$ suppose
  $\mSat{M}{\Box !A}[u]$; then $\mSat{M}{\Box\Box!A}[u]$; since
  $\Box\Box!A \in \Gamma$ by closure, also by $C_1(u,v)$,
  $\mSat{M}{\Box!A}[v]$ and by $C_1(v,w)$, also $\mSat{M}{!A}[w]$. The
  case for $\Diamond!A$ is similar.
\end{proof}

\end{document}
