% Part: first-order-logic
% Chapter: sequent-calculus
% Section: proof-theoretic-notions

\documentclass[../../../include/open-logic-section]{subfiles}

\begin{document}

\olfileid{fol}{ntd}{ptn}
\olsection{Proof-Theoretic Notions}

\begin{editorial}
This section collects the properties of the provability relation
required for the completeness theorem.  If you find the location
unmotivated, include it instead in the chapter on completeness.
\end{editorial}

\begin{explain}
Just as we've defined a number of important semantic notions
(validity, entailment, satisfiabilty), we now define corresponding
\emph{proof-theoretic notions}.  These are not defined by appeal to
satisfaction of sentences in !!{structure}s, but by appeal to the
!!{derivability} or !!{nonderivability} of certain !!{formula}s.  It was
an important discovery, due to G\"odel, that these notions coincide.
That they do is the content of the \emph{completeness theorem}.
\end{explain}

\begin{defn}[!!^{derivability}]
A !!{formula} $!A$ is \emph{!!{derivable} from} a set of
!!{formula}s~$\Gamma$, $\Gamma \Proves !A$, if there is a
!!{derivation} with end-!!{formula}~$!A$ and in which every assumption
is either !!{discharged} or is in~$\Gamma$. If $!A$ is not
!!{derivable} from $\Gamma$ we write $\Gamma \Proves/ !A$.
\end{defn}

\begin{defn}[Theorems]
A !!{formula}~$!A$ is a \emph{theorem} if there is a !!{derivation} of
$!A$ from the empty set, i.e., a !!{derivation} with
end-!!{formula}~$!A$ in which all assumptions are !!{discharged}.  We
write $\Proves !A$ if $!A$ is a theorem and $\Proves/ !A$ if it is
not.
\end{defn}

\begin{defn}[Consistency]
A set of sentences~$\Gamma$ is \emph{consistent} iff $\Gamma
\Proves/ \lfalse$.  If $\Gamma$ is not consistent, i.e., if
$\Gamma \Proves \lfalse$, we say it is \emph{inconsistent}.
\end{defn}

\begin{prop}
\ollabel{prop:prov-incons}
$\Gamma \Proves !A$ iff $\Gamma \cup \{\lnot !A\}$ is inconsistent.
\end{prop}

\begin{proof}
Exercise.
\end{proof}

\begin{prob}
Prove \olref[fol][ntd][ptn]{prop:prov-incons}
\end{prob}

\begin{prop}
\ollabel{prop:incons}
$\Gamma$ is inconsistent iff $\Gamma \Proves {!A}$ for every
  sentence~$!A$.
\end{prop}

\begin{proof}
Exercise.
\end{proof}

\begin{prob}
Prove \olref[fol][ntd][ptn]{prop:incons}
\end{prob}

\begin{prop}
\ollabel{prop:proves-compact}
$\Gamma \Proves !A$ iff for some finite $\Gamma_0 \subseteq
\Gamma$, $\Gamma_0 \Proves !A$.
\end{prop}

\begin{proof}
Any derivation of $!A$ from $\Gamma$ can only contain finitely many
!!{undischarged} assumtions.  If all these !!{undischarged} assumptions are
in~$\Gamma$, then the set of them is a finite subset of~$\Gamma$.  The
other direction is trivial, since a derivation from a subset of
$\Gamma$ is also a derivation from~$\Gamma$.
\end{proof}

\end{document}
