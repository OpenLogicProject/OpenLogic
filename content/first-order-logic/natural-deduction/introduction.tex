% Part: first-order-logic
% Chapter: natural-deduction
% Section: introduction

\documentclass[../../../include/open-logic-section]{subfiles}

\begin{document}

\olfileid{fol}{ntd}{int}

\olsection{Introduction}

Logical systems commonly have not just a semantics, but also proof
systems. The purpose of proof systems is to provide a purely syntactic
method of establishing entailment and validity.  They are purely
syntactic in the sense that a !!{derivation} in such a system is a
finite syntactic object, usually a sequence (or other finite
arrangement) of !!{formula}s.  Moreover, good proof systems have the
property that any given sequence or arrangement of !!{formula}s can be
verified mechanically to be a ``correct'' proof.  The simplest (and
historically first) proof systems for first-order logic were
axiomatic.  A sequence of !!{formula}s counts as a !!{derivation} in
such a system if each individual !!{formula} in it is either among a
fixed set of ``axioms'' or follows from !!{formula}s coming before it
in the sequence by one of a fixed number of ``inference rules''---and
it can be mechanically verified if !!a{formula} is an axiom and
whether it follows correctly from other !!{formula}s by one of the
inference rules.  Axiomatic proof systems are easy to describe---and
also easy to handle meta-theoretically---but !!{derivation}s in them are
hard to read and understand, and are also hard to produce.

Other proof systems have been developed with the aim of making it
easier to construct !!{derivation}s or easier to understand
!!{derivation}s once they are complete.  Examples are truth trees,
also known as tableaux proofs, and the sequent calculus.  Some proof
systems are designed especially with mechanization in mind, e.g., the
resolution method is easy to implement in software (but its
!!{derivation}s are essentially impossible to understand). Most of
these other proof systems represent !!{derivation}s as trees of
!!{formula}s rather than sequences. This makes it easier to see which
parts of a !!{derivation} depend on which other parts.

The proof system we will study is Gentzen's natural deduction. Natural
deduction is intended to mirror actual reasoning (especially the kind
of regimented reasoning employed by mathematicians).  Actual reasoning
proceeds by a number of ``natural'' patterns. For instance proof by
cases allows us to establish a conclusion on the basis of a
disjunctive premise, by establishing that the conclusion follows from
either of the disjuncts. Indirect proof allows us to establish a
conclusion by showing that its negation leads to a
contradiction. Conditional proof establishes a conditional claim ``if
\dots then \dots'' by showing that the consequent follows from the
antecedent.  Natural deduction is a formalization of some of these
natural inferences.  Each of the logical connectives and quantifiers
comes with two rules, an introduction and an elimination rule, and
they each correspond to one such natural inference pattern. For
instance, $\Intro{\lif}$ corresponds to conditional proof, and
$\Elim{\lor}$ to proof by cases.

One feature that distinguishes natural deduction from other proof
systems is its use of assumptions. In almost every proof system a
single !!{formula} is at the root of the tree of
!!{formula}s---usually the conclusion---and the ``leaves'' of the tree
are !!{formula}s from which the conclusion is derived.  In natural
deduction, some leaf !!{formula}s play a role inside the
!!{derivation} but are ``used up'' by the time the !!{derivation}
reaches the conclusion. This corresponds to the practice, in actual
reasoning, of introducing hypotheses which only remain in effect for a
short while.  For instance, in a proof by cases, we assume the truth
of each of the disjuncts; in conditional proof, we assume the truth of
the antecedent; in indirect proof, we assume the truth of the negation
of the conclusion.  This way of introducing hypotheticals and then
doing away with them in the service of establishing an intermediate
step is a hallmark of natural deduction. The formulas at the leaves of
a natural deduction !!{derivation} are called assumptions, and some of
the rules of inference may ``discharge'' them.  An assumption that
remains undischarged at the end of the !!{derivation} is (usually)
essential to the truth of the conclusion, and so a !!{derivation}
establishes that its undischarged assumptions entail its conclusion.

For any proof system it is crucial to verify that it in fact does what
it's supposed to: provide a way to verify that !!a{sentence} is
entailed by some others. This is called soundness; and we will prove
it for the natural deduction system we use. It is also crucial to
verify the converse: that the proof system is strong enough to verify
that $\Gamma \Entails !A$ whenever this holds, that there is a
!!{derivation} of~$!A$ from $\Gamma$ whenever $\Gamma \Entails
!A$. This is called completeness---but it is much harder to prove.

\end{document}
