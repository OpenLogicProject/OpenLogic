% Part: first-order-logic
% Chapter: completeness
% Section: lindenbaums-lemma

\documentclass[../../../include/open-logic-section]{subfiles}

\begin{document}

\iftag{FOL}
      {\olfileid{fol}{com}{lin}}
      {\olfileid{pl}{com}{lin}}

\olsection{Lindenbaum's Lemma}

\begin{explain}
We now prove a lemma that shows that any consistent set of
!!{sentence}s is contained in some set of sentences which is not just
consistent, but also !!{complete}. The proof works by adding one
!!{sentence} at a time, guaranteeing at each step that the set remains
consistent. We do this so that for every $!A$, either $!A$ or $\lnot
!A$ gets added at some stage. The union of all stages in that
construction then contains either $!A$ or its negation~$\lnot !A$ and
is thus complete. It is also consistent, since we make sure at each
stage not to introduce an inconsistency.
\end{explain}

\begin{lem}[Lindenbaum's Lemma]
\ollabel{lem:lindenbaum} Every consistent
set~$\Gamma$ in a language~$\Lang{L}$ can be
extended to !!a{complete} and consistent set~$\Gamma^*$.
\end{lem}

\begin{proof}
Let $\Gamma$ be consistent.  Let $!A_0$, $!A_1$,
\dots{} be an enumeration of all the !!{sentence}s of~$\Lang L$.
Define $\Gamma_0 = \Gamma$, and
\[
\Gamma_{n+1} =
\begin{cases}
\Gamma_n \cup \{ !A_n \} & \textrm{if $\Gamma_n \cup \{!A_n\}$ is
  consistent;} \\
\Gamma_n \cup \{ \lnot !A_n \} & \textrm{otherwise.}
\end{cases}
\]
Let $\Gamma^* = \bigcup_{n \geq 0} \Gamma_n$.

Each $\Gamma_n$ is consistent: $\Gamma_0$ is consistent by definition.
If $\Gamma_{n+1} = \Gamma_n \cup \{!A_n\}$, this is because the latter
is consistent.  If it isn't, $\Gamma_{n+1} = \Gamma_n \cup \{\lnot
!A_n\}$. We have to verify that $\Gamma_n \cup \{\lnot !A_n\}$ is
consistent. Suppose it's not. Then \emph{both} $\Gamma_n \cup
\{!A_n\}$ and $\Gamma_n \cup \{\lnot !A_n\}$ are inconsistent.  This
means that $\Gamma_n$ would be inconsistent by
\iftag{FOL}{%
  \tagrefs{prfAX/{fol:axd:prv:prop:provability-exhaustive},
    prfSC/{fol:seq:prv:prop:provability-exhaustive},
    prfND/{fol:ntd:prv:prop:provability-exhaustive},
    prfTab/{fol:tab:prv:prop:provability-exhaustive}}}{%
  \tagrefs{prfAX/{pl:axd:prv:prop:provability-exhaustive},
    prfSC/{pl:seq:prv:prop:provability-exhaustive},
    prfND/{pl:ntd:prv:prop:provability-exhaustive},
    prfTab/{pl:tab:prv:prop:provability-exhaustive}}},
contrary to the induction hypothesis.

For every~$n$ and every $i < n$, $\Gamma_i \subseteq \Gamma_n$. This
follows by a simple induction on~$n$. For $n=0$, there are no $i < 0$,
so the claim holds automatically.  For the inductive step, suppose it
is true for~$n$. We show that if $i < n+1$ then $\Gamma_i \subseteq
\Gamma_{n+1}$. We have $\Gamma_{n+1} = \Gamma_n \cup \{!A_n\}$ or $=
\Gamma_n \cup \{\lnot !A_n\}$ by construction. So $\Gamma_n \subseteq
\Gamma_{n+1}$. If $i < n+1$, then $\Gamma_i \subseteq \Gamma_n$ by
inductive hypothesis (if $i < n$) or the trivial fact that $\Gamma_n
\subseteq \Gamma_n$ (if $i = n$). We get that $\Gamma_i \subseteq
\Gamma_{n+1}$ by transitivity of~$\subseteq$.

From this it follows that $\Gamma^*$ is consistent. Here's why: Let
$\Gamma' \subseteq \Gamma^*$ be finite. Each $!B \in \Gamma'$ is also
in~$\Gamma_i$ for some~$i$. Let $n$ be the largest of these. Since
$\Gamma_i \subseteq \Gamma_n$ if $i \le n$, every $!B \in \Gamma'$ is
also $\in \Gamma_n$, i.e., $\Gamma' \subseteq \Gamma_n$, and
$\Gamma_n$~is consistent. So, every finite subset $\Gamma' \subseteq
\Gamma^*$ is consistent. By \iftag{FOL}{%
  \tagrefs{prfAX/{fol:axd:ptn:prop:proves-compact},
    prfSC/{fol:seq:ptn:prop:proves-compact},
    prfND/{fol:ntd:ptn:prop:proves-compact},
    prfTab/{fol:tab:ptn:prop:proves-compact}}}{%
  \tagrefs{prfAX/{pl:axd:ptn:prop:proves-compact},
    prfSC/{pl:seq:ptn:prop:proves-compact},
    prfND/{pl:ntd:ptn:prop:proves-compact},
    prfTab/{pl:tab:ptn:prop:proves-compact}}}, $\Gamma^*$ is
  consistent.

Every !!{sentence} of $\Frm[L]$ appears on the list used to
define~$\Gamma^*$. If $!A_n \notin \Gamma^*$, then that is because
$\Gamma_n \cup \{!A_n\}$ was inconsistent.  But then $\lnot !A_n
\in \Gamma^*$, so $\Gamma^*$ is !!{complete}.
\end{proof}

\end{document}
