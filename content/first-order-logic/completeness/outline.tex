% Part: first-order-logic
% Chapter: completeness
% Section: outline

\documentclass[../../../include/open-logic-section]{subfiles}

\begin{document}

\iftag{FOL}
      {\olfileid{fol}{com}{out}}
      {\olfileid{pl}{com}{out}}

\olsection{Outline of the Proof}

The proof of the completeness theorem is a bit complex, and upon first
reading it, it is easy to get lost.  So let us outline the proof.  The
first step is a shift of perspective, that allows us to see a route to
a proof.  When completeness is thought of as ``whenever $\Gamma
\Entails !A$ then $\Gamma \Proves !A$,'' it may be hard to even come
up with an idea: for to show that $\Gamma \Proves !A$ we have to find
!!a{derivation}, and it does not look like the hypothesis that
$\Gamma \Entails !A$ helps us for this in any way.  For some proof
systems it is possible to directly construct !!a{derivation}, but we
will take a slightly different approach.  The shift in perspective
required is this: completeness can also be formulated as: ``if
$\Gamma$ is consistent, it is satisfiable.''  Perhaps we can use the
information in~$\Gamma$ together with the hypothesis that it is
consistent to construct \iftag{FOL}{!!a{structure}}{!!a{valuation}}
that satisfies every \iftag{FOL}{!!{sentence}}{!!{formula}}
in~$\Gamma$.  After all, we know
what kind of \iftag{FOL}{!!{structure}}{!!{valuation}} we are looking
for: one that is as $\Gamma$ describes it!{}

If $\Gamma$ contains only \iftag{FOL}{atomic
  !!{sentence}s}{!!{propositional variable}s}, it is easy to construct
a model for it.\iftag{FOL}{ Suppose the atomic !!{sentence}s are all
  of the form $\Atom{P}{a_1,\dots,a_n}$ where the $a_i$ are
  !!{constant}s.}{} All we have to do is come up with
\iftag{FOL}{%
  !!a{domain}~$\Domain{M}$ and an assignment for~$P$ so that
  $\Sat{M}{\Atom{P}{a_1,\ldots,a_n}}$. But that's not very hard:
  put $\Domain{M} = \Nat$, $\Assign{\Obj c_i}{M} = i$, and for every
  $\Atom{P}{a_1,\ldots,a_n} \in \Gamma$, put the tuple $\tuple{k_1,
    \dots, k_n}$ into $\Assign{P}{M}$, where $k_i$ is the index of the
  constant symbol~$a_i$ (i.e., $a_i \ident \Obj c_{k_i}$).}{%
  !!a{valuation}~$\pAssign{v}$ such that $\pSat{v}{p}$ for all $p \in
  \Gamma$. Well, let $\pAssign{v}(p) = \True$ iff $p \in \Gamma$.}

Now suppose $\Gamma$ contains some !!{formula}~$\lnot !B$, with $!B$
atomic.  We might worry that the construction of
$\iftag{FOL}{\Struct{M}}{\pAssign{v}}$ interferes with the possibility
of making $\lnot !B$ true.  But here's where the consistency
of~$\Gamma$ comes in: if $\lnot !B \in \Gamma$, then $!B \notin
\Gamma$, or else $\Gamma$ would be inconsistent.  And if $!B \notin
\Gamma$, then according to our construction
of~$\iftag{FOL}{\Struct{M}}{\pAssign{v}}$,
$\iftag{FOL}{\Sat/{M}}{\pSat/{v}}{!B}$, so
$\iftag{FOL}{\Sat{M}}{\pSat{v}}{\lnot !B}$.  So far so good.

What if $\Gamma$ contains complex, non-atomic formulas? Say it
contains $!A \land !B$. To make that true, we should proceed as if
both $!A$ and $!B$ were in~$\Gamma$.  And if $!A \lor !B \in \Gamma$,
then we will have to make at least one of them true, i.e., proceed as
if one of them was in~$\Gamma$.

This suggests the following idea: we add additional !!{formula}s
to~$\Gamma$ so as to (a)~keep the resulting set consistent and
(b)~make sure that for every possible atomic !!{sentence}~$!A$, either
$!A$ is in the resulting set, or $\lnot !A$ is, and (c)~such that,
whenever $!A \land !B$ is in the set, so are both $!A$ and $!B$, if
$!A \lor !B$ is in the set, at least one of $!A$ or $!B$ is also, etc.
We keep doing this (potentially forever).  Call the set of all
!!{formula}s so added~$\Gamma^*$.  Then our construction above would
provide us with \iftag{FOL}{!!a{structure}~$\Struct{M}$}{!!a{valuation}~$\pAssign{v}$}
for which we could prove, by induction, that it satisfies all sentences
in~$\Gamma^*$, and hence also all sentence in~$\Gamma$
since $\Gamma \subseteq \Gamma^*$.  It turns out that guaranteeing (a)
and~(b) is enough. A set of sentences for which (b) holds is called
\emph{complete}. So our task will be to extend the consistent
set~$\Gamma$ to a consistent and complete set~$\Gamma^*$.

\iftag{FOL}{%
There is one wrinkle in this plan: if $\lexists[x][!A(x)] \in \Gamma$
we would hope to be able to pick some !!{constant}~$c$ and add $!A(c)$
in this process.  But how do we know we can always do that?  Perhaps we
only have a few !!{constant}s in our language, and for each one of
them we have $\lnot !A(c) \in \Gamma$.  We can't also add $!A(c)$,
since this would make the set inconsistent, and we wouldn't know
whether $\Struct{M}$ has to make $!A(c)$ or $\lnot !A(c)$ true.
Moreover, it might happen that $\Gamma$ contains only sentences in a
language that has no constant symbols at all (e.g., the language of
set theory).

The solution to this problem is to simply add infinitely many
constants at the beginning, plus sentences that connect them with the
quantifiers in the right way.  (Of course, we have to verify that this
cannot introduce an inconsistency.)

Our original construction works well if we only have !!{constant}s in
the atomic sentences.  But the language might also contain
!!{function}s.  In that case, it might be tricky to find the right
functions on~$\Nat$ to assign to these !!{function}s to make
everything work.  So here's another trick: instead of using $i$ to
interpret $\Obj c_i$, just take the set of !!{constant}s itself as the
domain.  Then $\Struct M$ can assign every !!{constant} to itself:
$\Assign{\Obj c_i}{M} = \Obj c_i$.  But why not go all the way: let
$\Domain{M}$ be all \emph{terms} of the language!{} If we do this,
there is an obvious assignment of functions (that take terms as
arguments and have terms as values) to !!{function}s: we assign to the
!!{function}~$\Obj f^n_i$ the function which, given $n$ terms $t_1$,
\dots,~$t_n$ as input, produces the term $\Obj f^n_i(t_1, \dots, t_n)$
as value.

The last piece of the puzzle is what to do with~$\eq$.  The
!!{predicate}~$\eq$ has a fixed interpretation: $\Sat{M}{\eq[t][t']}$
iff $\Value{t}{M} = \Value{t'}{M}$.  Now if we set things up so that the
!!{value} of a term~$t$ is $t$ itself, then this !!{structure}
will make \emph{no} sentence of the form $\eq[t][t']$ true
unless $t$ and $t'$ are one and the same term.  And of
course this is a problem, since basically every interesting theory in
a language with !!{function}s will have as theorems sentences
$\eq[t][t']$ where $t$ and $t'$ are not the same term (e.g., in
theories of arithmetic: $\eq[(\Obj 0+ \Obj 0)][\Obj 0]$).  To solve
this problem, we change the domain of~$\Struct M$: instead of using terms as
the objects in~$\Domain{M}$, we use sets of terms, and each set is so
that it contains all those terms which the sentences in~$\Gamma$
require to be equal.  So, e.g., if $\Gamma$ is a theory of arithmetic,
one of these sets will contain: $\Obj 0$, $(\Obj 0 + \Obj 0)$, $(\Obj
0 \times \Obj 0)$, etc.  This will be the set we assign to $\Obj 0$,
and it will turn out that this set is also the value of all the terms
in it, e.g., also of $(\Obj 0 + \Obj 0)$.  Therefore, the sentence
$\eq[(\Obj 0+ \Obj 0)][\Obj 0]$ will be true in this revised
!!{structure}.}{}

So here's what we'll do. First we investigate the properties of
!!{complete} consistent sets, in particular we prove that
!!a{complete} consistent set contains $!A \land !B$ iff it contains
both $!A$ and~$!B$, $!A \lor !B$ iff it contains at least one of them,
etc. (\olref[ccs]{prop:ccs}).\iftag{FOL}{ Then we define and
  investigate ``saturated'' sets of sentences. A saturated set is one
  which contains conditionals that link each quantified !!{sentence}
  to instances of it (\olref[hen]{defn:henkin-exp}). We show that any
  consistent set~$\Gamma$ can always be extended to a saturated
  set~$\Gamma'$ (\olref[hen]{lem:henkin}). If a set is consistent,
  saturated, and !!{complete} it also has the property that it
  contains \iftag{prvEx}{$\lexists[x][!A(x)]$ iff it contains $!A(t)$
    for some closed term~$t$}{}\iftag{defEx,defAll}{}{ and
  }\iftag{prvAll}{$\lforall[x][!A(x)]$ iff it contains $!A(t)$ for all
    closed terms~$t$}{} (\olref[hen]{prop:saturated-instances}).}{}
We'll then take the\iftag{FOL}{ saturated}{} consistent
set~$\iftag{FOL}{\Gamma'}{\Gamma}$ and show that it can be extended
to a \iftag{FOL}{saturated, consistent, and}{consistent and}
!!{complete} set~$\Gamma^*$ (\olref[lin]{lem:lindenbaum}). This set
$\Gamma^*$ is what we'll use to define our \iftag{FOL}{term
  model~$\Struct M(\Gamma^*)$}{!!{valuation}~$\pAssign v(\Gamma^*)$}.
\iftag{FOL}{The term model has the set of closed terms as its domain,
  and the interpretation of its !!{predicate}s is given by the atomic
  !!{sentence}s}{The valuation is determined by the !!{propositional
    variable}s} in~$\Gamma^*$ (\olref[mod]{defn:termmodel}).  We'll
use the properties of\iftag{FOL}{ saturated,}{} complete consistent
sets to show that indeed
$\iftag{FOL}{\Sat{M(\Gamma^*)}}{\pSat{v(\Gamma^*)}}{!A}$ iff $!A \in
\Gamma^*$ (\olref[mod]{lem:truth}), and thus in particular,
$\iftag{FOL}{\Sat{M(\Gamma^*)}}{\pSat{v(\Gamma^*)}}{\Gamma}$.
\iftag{FOL}{Finally, we'll consider how to define a term model if
  $\Gamma$ contains~$\eq$ as well
  (\olref[ide]{defn:term-model-factor}) and show that it
  satisfies~$\Gamma^*$ (\olref[ide]{lem:truth}).}{}

\end{document}
