% Part: first-order-logic
% Chapter: completeness
% Section: henkin-expansion

\documentclass[../../../include/open-logic-section]{subfiles}

\begin{document}

\olfileid{fol}{com}{hen}

\olsection{Henkin Expansion}

\begin{explain}
Part of the challenge in proving the completeness theorem is that the
model we construct from a complete consistent set~$\Gamma$ must make
all the quantified !!{formula}s in~$\Gamma$ true.  In order to
guarantee this, we use a trick due to Leon Henkin.  In essence, the
trick consists in expanding the language by infinitely many !!{constant}s
and adding, for each !!{formula} with one free !!{variable} $!A(x)$ a
formula of the form
\iftag{prvEx}
      {$\lexists[x][!A] \lif !A(c)$}
      {$\lnot\lforall[x][!A] \lif \lnot !A(c)$},
where $c$ is one of the new !!{constant}s.  When we construct the
!!{structure} satisfying~$\Gamma$, this will guarantee that each
\iftag{prvEx}
{true existential sentence has a witness}
{false universal sentence has a counterexample}
among the new constants.
\end{explain}

\begin{prop}
\ollabel{prop:lang-exp}
If $\Gamma$ is consistent in $\Lang L$ and $\Lang L'$ is obtained from
$\Lang L$ by adding !!a{denumerable} set of new !!{constant}s $\Obj d_0$,
$\Obj d_1$, \dots, then $\Gamma$ is consistent in~$\Lang L'$.
\end{prop}

\begin{defn}[Saturated set]
A set $\Gamma$ of !!{formula}s of a language $\Lang {L}$ is
\emph{saturated} iff for each !!{formula} $!A(x) \in \Frm[L]$ with one
free !!{variable}~$x$ there is !!a{constant}~$c \in \Lang{L}$ such
that
\iftag{prvEx}
      {$\lexists[x][!A(x)] \lif !A(c) \in \Gamma$}
      {$\lnot\lforall[x][!A(x)] \lif \lnot !A(c) \in \Gamma$}.
\end{defn}

The following definition will be used in the proof of the next theorem.

\begin{defn}
\ollabel{defn:henkin-exp}
Let $\Lang L'$ be as in \olref{prop:lang-exp}.  Fix an enumeration
$!A_0(x_0)$, $!A_1(x_1)$, \dots of all !!{formula}s~$!A_i(x_i)$
of~$\Lang L'$ in which one variable ($x_i$) occurs free.  We define
the !!{sentence}s~$!D_n$ by induction on~$n$.

Let $c_0$ be the first !!{constant} among the $\Obj d_i$ we added
to~$\Lang{L}$ which does not occur in~$!A_0(x_0)$.  Assuming that
$!D_0$, \dots,~$!D_{n-1}$ have already been defined, let $c_n$ be the
first among the new !!{constant}s~$\Obj d_i$ that occurs neither in
$!D_0$, \dots,~$!D_{n-1}$ nor in~$!A_n(x_n)$.

Now let $!D_{n}$ be the !!{formula} \iftag{prvEx}
{$\lexists[x_{n}][!A_{n}(x_{n})] \lif
  !A_{n}(c_{n})$}{$\lnot\lforall[x_{n}][!A_{n}(x_{n})] \lif \lnot
  !A_{n}(c_{n})$}.
\end{defn}

\begin{lem}
\ollabel{lem:henkin}
Every consistent set~$\Gamma$ can be extended to a saturated
consistent set~$\Gamma'$.
\end{lem}

\begin{proof}
Given a consistent set of sentences~$\Gamma$ in a language~$\Lang{L}$,
expand the language by adding !!a{denumerable} set of new
!!{constant}s to form~$\Lang{L'}$.  By \olref{prop:lang-exp}, $\Gamma$
is still consistent in the richer language.  Further, let $!D_i$ be as
in \olref{defn:henkin-exp}.  Let
\begin{align*}
\Gamma_0 & = \Gamma \\
\Gamma_{n+1} & = \Gamma_n \cup \{!D_n \}
\end{align*}
i.e., $\Gamma_{n+1} = \Gamma \cup \{ !D_0, \dots, !D_n \}$, and let
$\Gamma' = \bigcup_{n} \Gamma_n$.  $\Gamma'$ is clearly saturated.

If $\Gamma'$ were inconsistent, then for some $n$, $\Gamma_n$ would be
inconsistent (Exercise: explain why).  So to show that $\Gamma'$ is
consistent it suffices to show, by induction on~$n$, that each
set~$\Gamma_n$ is consistent.

The induction basis is simply the claim that $\Gamma_0 = \Gamma$ is
consistent, which is the hypothesis of the theorem.  For the induction
step, suppose that $\Gamma_{n}$ is consistent but $\Gamma_{n+1} =
\Gamma_n \cup \{!D_n\}$ is inconsistent.  Recall that $!D_n$~is
\iftag{prvEx}
{$\lexists[x_{n}][!A_{n}(x_n)] \lif !A_{n}(c_{n})$}
{$\lnot\lforall[x_{n}][!A_{n}(x_n)] \lif \lnot !A_{n}(c_{n})$},
where $!A_n(x_n)$ is a !!{formula} of $\Lang{L'}$ with only the
variable~$x_n$ free. By the way we've chosen the~$c_n$ (see
\olref{defn:henkin-exp}), $c_n$ does not occur in~$!A_n(x_n)$ nor
in~$\Gamma_n$.

If $\Gamma_n \cup \{!D_n\}$ is inconsistent, then $\Gamma_n
\Proves \lnot !D_n$, and hence both of the following hold:
\iftag{prvEx}{
\[
\Gamma_n \Proves \lexists[x_n][!A_n(x_n)]
\qquad
\Gamma_n \Proves \lnot !A_n(c_n)
\]}{
\[
\Gamma_n \Proves \lnot\lforall[x_n][!A_n(x_n)]
\qquad
\Gamma_n \Proves !A_n(c_n)
\]}
Since $c_n$ does not occur in
$\Gamma_n$ or in~$!A_n(x_n)$, 
\tagrefs{prfAX/{fol:axd:qpr:thm:strong-generalization},prfSC/{fol:seq:qpr:thm:strong-generalization},prfND/{fol:ntd:qpr:thm:strong-generalization},prfTab/{fol:tab:qpr:thm:strong-generalization}} applies.
From \iftag{prvEx}
{$\Gamma_n \Proves \lnot !A_n(c_n)$}
{$\Gamma_n \Proves !A_n(c_n)$},
we obtain
\iftag{prvEx}
{$\Gamma_n \Proves \lforall[x_n][\lnot !A_n(x_n)]$}
{$\Gamma_n \Proves \lforall[x_n][!A_n(x_n)]$}.
Thus we have that both
\iftag{prvEx}
{$\Gamma_n \Proves \lexists[x_n][!A_n]$ and
$\Gamma_n \Proves \lforall[x_n][\lnot !A_n(x_n)]$}
{$\Gamma_n \Proves \lnot\lforall[x_n][!A_n(x_n)]$ and
$\Gamma_n \Proves \lforall[x_n][!A_n(x_n)]$},
so $\Gamma_n$ itself is inconsistent.
\iftag{prvEx}
{(Note that
\iftag{prvAll}
{$\lforall[x_n][\lnot !A_n(x_n)] \Proves
  \lnot\lexists[x_n][!A_n(x_n)]$}
{$\lforall[x_n][\lnot !A_n]$ is defined as
  $\lnot\lexists[x_n][\lnot\lnot !A_n(x_n)]$ and
  $\lnot\lexists[x_n][\lnot\lnot !A_n(x_n)] \Proves
  \lnot\lexists[x_n][!A_n(x_n)]$}.)}{}
Contradiction: $\Gamma_n$ was supposed to be consistent.  Hence
$\Gamma_n \cup \{ !D_n\}$ is consistent.
\end{proof}

\begin{explain}
We'll now show that \emph{complete}, consistent sets which are
saturated have the property that \iftag{prvAll}{it contains a
  universally quantified !!{sentence} iff it contains all its
  instances}{}\iftag{defAll,defEx}{}{ and }\iftag{prvAll}{it contains an
  existentially quantified !!{sentence} iff it contains at least one
  instance}{}. We'll use this to show that the !!{structure} we'll
generate from a complete, consistent, saturated set makes all its
quantified sentences true.
\end{explain}

\begin{prop}\ollabel{prop:saturated-instances}
Suppose $\Gamma$ is complete, consistent, and saturated.
\begin{tagenumerate}{prvEx,prvAll}
\tagitem{prvEx}{$\lexists[x][!A(x)] \in \Gamma$ iff $!A(t) \in \Gamma$
  for at least one closed term~$t$.}{}
\tagitem{prvAll}{$\lforall[x][!A(x)] \in \Gamma$ iff $!A(t) \in \Gamma$
  for all closed terms~$t$.}{}
\end{tagenumerate}
\end{prop}

\begin{proof}
  \begin{tagenumerate}{prvEx,prvAll}
  \tagitem{prvEx}{%
    \iftag{probEx}{Exercise.}{First suppose that $\lexists[x][!A(x)]
      \in \Gamma$.  Because $\Gamma$ is saturated,
      $(\lexists[x][!A(x)] \lif !A(c)) \in \Gamma$ for some
      !!{constant}~$c$. By
      \tagrefs{prfAX/{fol:axd:ppr:prop:provability-lif},prfSC/{fol:seq:ppr:prop:provability-lif},prfND/{fol:ntd:ppr:prop:provability-lif},prfTab/{fol:tab:ppr:prop:provability-lif}}, item~(1),
      and \olref[ccs]{prop:ccs}\olref[ccs]{prop:ccs-prov-in}, $!A(c)
      \in \Gamma$.

    For the other direction, saturation is not necessary: Suppose
    $!A(t) \in \Gamma$. Then $\Gamma \Proves \lexists[x][!A(x)]$ by
      \tagrefs{prfAX/{fol:axd:qpr:prop:provability-quantifiers},prfSC/{fol:seq:qpr:prop:provability-quantifiers},prfND/{fol:ntd:qpr:prop:provability-quantifiers},prfTab/{fol:tab:qpr:prop:provability-quantifiers}}, item~(1). By
    \olref[ccs]{prop:ccs}\olref[ccs]{prop:ccs-prov-in},
    $\lexists[x][!A(x)] \in \Gamma$.}}{}
  \tagitem{prvAll}{%
    \iftag{probAll}{Exercise.}{Suppose that $!A(t) \in \Gamma$ for
      all closed terms~$t$. By way of contradiction, assume
      $\lforall[x][!A(x)] \notin \Gamma$. Since $\Gamma$ is complete,
      $\lnot\lforall[x][!A(x)] \in \Gamma$. By saturation,
      \iftag{prvEx}{$(\lexists[x][\lnot !A(x)] \lif \lnot !A(c)) \in
        \Gamma$}{$(\lnot\lforall[x][!A(x)] \lif \lnot !A(c)) \in
        \Gamma$} for some !!{constant}~$c$. By assumption, since $c$
      is a closed term, $!A(c) \in \Gamma$. But this would make
      $\Gamma$ inconsistent. (Exercise: give the derivation that shows
      \[
      \lnot \lforall[x][!A(x)],
      \iftag{prvEx}{\lexists[x][\lnot !A(x)] \lif \lnot
        !A(c)}{\lnot\lforall[x][!A(x)] \lif \lnot
        !A(c)}, !A(c)
      \]
      is inconsistent.)

      For the reverse direction, we do not need saturation: Suppose
      $\lforall[x][!A(x)] \in \Gamma$.  Then $\Gamma \Proves !A(t)$
      by
      \tagrefs{prfAX/{fol:axd:qpr:prop:provability-quantifiers},prfSC/{fol:seq:qpr:prop:provability-quantifiers},prfND/{fol:ntd:qpr:prop:provability-quantifiers},prfTab/{fol:tab:qpr:prop:provability-quantifiers}},
      item~(2). We get $!A(t) \in \Gamma$ by
      \olref[ccs]{prop:ccs}.}}{}
  \end{tagenumerate}
\end{proof}

\end{document}
