% Part: first-order-logic 
% Chapter: axiomatic-deduction 
% Section: deduction-theorem

% verification of properties of provability needed for maximally 
% consistent sets in the completeness chapter.

\documentclass[../../../include/open-logic-section]{subfiles}

\begin{document}

\iftag{FOL}
      {\olfileid{fol}{axd}{ded}}
      {\olfileid{pl}{axd}{ded}}

\olsection{The Deduction Theorem}

As we've seen, giving !!{derivation}s in an axiomatic system is
cumbersome, and !!{derivation}s may be hard to find. Rather than
actually write out long lists of !!{formula}s, it is generally easier
to argue that such !!{derivation}s exist, by making use of a few
simple results. We've already established three such results:
\olref[ptn]{prop:reflexivity} says we can always assert that $\Gamma
\Proves !A$ when we know that $!A \in
\Gamma$. \olref[ptn]{prop:monotony} says that if $\Gamma \Proves !A$
then also $\Gamma \cup \{!B\} \Proves !A$. And
\olref[ptn]{prop:transitivity} implies that if $\Gamma \Proves !A$ and
$!A \Proves !B$, then $\Gamma \Proves !B$. Here's another simple
result, a ``meta''-version of modus ponens:

\begin{prop}
  \ollabel{prop:mp} If $\Gamma \Proves !A$ and $\Gamma \Proves !A \lif
  !B$, then $\Gamma \Proves !B$.
\end{prop}

\begin{proof}
  We have that $\{!A, !A \lif !B\} \Proves !B$:
  \begin{derivation}
    1. & $!A$ & Hyp.\\
    2. & $!A \lif !B$ & Hyp.\\
    3. & $!B$ & 1, 2, MP
  \end{derivation}
  By \olref[ptn]{prop:transitivity}, $\Gamma \Proves !B$.
\end{proof}

The most important result we'll use in this context is the deduction
theorem:

\begin{thm}[Deduction Theorem]
\ollabel{thm:deduction-thm} $\Gamma \cup \{!A\} \Proves !B$ if and
only if $\Gamma \Proves !A \lif !B$.
\end{thm}

\begin{proof}
The ``if'' direction is immediate.  If $\Gamma \Proves !A \lif !B$
then also $\Gamma \cup \{!A\} \Proves !A \lif !B$ by
\olref[ptn]{prop:monotony}.  Also, $\Gamma \cup \{!A\} \Proves !A$ by
\olref[ptn]{prop:reflexivity}. So, by \olref{prop:mp}, $\Gamma \cup
\{!A\} \Proves !B$.

For the ``only if'' direction, we proceed by induction on the length
of the !!{derivation} of $!B$ from $\Gamma \cup \{!A\}$.

For the induction basis, we prove the claim for every !!{derivation}
of length~$1$. !!^a{derivation} of~$!B$ from $\Gamma \cup \{!A\}$ of
length~$1$ consists of $!B$ by itself; and if it is correct $!B$ is
either $\in \Gamma \cup \{!A\}$ or is an axiom.  If $!B \in \Gamma$ or
is an axiom, then $\Gamma \Proves !B$. We also have that $\Gamma
\Proves !B \lif (!A \lif !B)$ by \olref[prp]{ax:lif1}, and
\olref{prop:mp} gives $\Gamma \Proves !A \lif !B$. If $!B \in \{ !A\}$
then $\Gamma \Proves !A \lif !B$ because then last !!{sentence}~$!A
\lif !B$ is the same as $!A \lif !A$, and we have !!{derive}d that in
\olref[pro]{ex:identity}.

For the inductive step, suppose !!a{derivation} of~$!B$ from
$\Gamma \cup \{!A\}$ ends with a step~$!B$ which is justified by modus
ponens. (If it is not justified by modus ponens, $!B \in \Gamma$, $!B
\ident !A$, or $!B$ is an axiom, and the same reasoning as in the
induction basis applies.) Then some previous steps in the
!!{derivation} are $!C \lif !B$ and $!C$, for some !!{formula}~$!C$,
i.e., $\Gamma \cup \{!A\} \Proves !C \lif !B$ and $\Gamma \cup \{!A\}
\Proves !C$, and the respective derivations are shorter, so the
inductive hypothesis applies to them. We thus have both:
\begin{align*}
  & \Gamma \Proves !A \lif (!C \lif !B); \\
  & \Gamma \Proves !A \lif !C.
\end{align*}
But also
\[
\Gamma \Proves (!A \lif (!C \lif !B)) \lif
((!A\lif !C)  \lif (!A \lif !B)),
\]
by \olref[prp]{ax:lif2}, and two applications of \olref{prop:mp} give
$\Gamma \Proves !A \lif !B$, as required.
\end{proof}

Notice how \olref[prp]{ax:lif1} and \olref[prp]{ax:lif2} were chosen
precisely so that the Deduction Theorem would hold.

The following are some useful facts about !!{derivability}, which we
leave as exercises.

\begin{prop}
  \ollabel{prop:derivfacts}
\begin{enumerate}
\item $\Proves (!A \lif !B) \lif ((!B \lif !C)
  \lif (!A \lif !C)$; \ollabel{derivfacts:a}
\item If $\Gamma \cup \{ \lnot !A\}
  \Proves \lnot !B$ then $\Gamma \cup \{ !B\} \Proves
  !A$ (Contraposition); \ollabel{derivfacts:b}
\item  $\{ !A, \lnot!A\} \Proves
    !B$ (Ex Falso Quodlibet, Explosion); \ollabel{derivfacts:c}
\item  $\{ \lnot\lnot!A\} \Proves
  !A$ (Double Negation Elimination);\ollabel{derivfacts:d}
\item If $\Gamma \Proves \lnot\lnot!A$ then $\Gamma \Proves
  !A$;\ollabel{derivfacts:e}
\end{enumerate}
\end{prop}

\tagprob{FOL}
\begin{prob}
  Prove \olref[fol][axd][ded]{prop:derivfacts}
\end{prob}
\tagendprob

\tagprob{notFOL}
\begin{prob}
  Prove \olref[pl][axd][ded]{prop:derivfacts}
\end{prob}
\tagendprob

\end{document}
