% Part: first-order-logic
% Chapter: tableaux
% Section: identity

\documentclass[../../../include/open-logic-section]{subfiles}

\begin{document}

\olfileid{fol}{tab}{ide}

\olsection{\usetoken{P}{tableau} with \usetoken{S}{identity}}

!!^{tableau}s with !!{identity} require additional inference rules.
The rules for $\eq$ are ($t$, $t_1$, and $t_2$ are closed terms):

\begin{defish}
\AxiomC{}
\RightLabel{$\eq$}
\UnaryInfC{\sFmla{\True}{\eq[t][t]}}
\DisplayProof
\hfill
\AxiomC{\sFmla{\True}{\eq[t_1][t_2]}}
\noLine
\UnaryInfC{\sFmla{\True}{!A(t_1)}}
\RightLabel{$\TRule{\True}{\eq}$}
\UnaryInfC{\sFmla{\True}{!A(t_2)}}
\DisplayProof
\hfill
\AxiomC{\sFmla{\True}{\eq[t_1][t_2]}}
\noLine
\UnaryInfC{\sFmla{\False}{!A(t_1)}}
\RightLabel{$\TRule{\False}{\eq}$}
\UnaryInfC{\sFmla{\False}{!A(t_2)}}
\DisplayProof
\end{defish}
Note that in contrast to all the other rules, $\TRule{\True}{\eq}$ and
$\TRule{\False}{\eq}$ require that \emph{two} signed !!{formula}s
already appear on the branch, namely both $\sFmla{\True}{\eq[t_1][t_2]}$
and $\sFmla{S}{!A(t_1)}$.

\begin{ex}
If $s$ and $t$ are closed terms, then $\eq[s][t], !A(s)
\Proves !A(t)$:
\begin{oltableau}
  [\sFmla{\False}{\formula{A}(t)}, just = \TAss
    [\sFmla{\True}{\eq[s][t]}, just = \TAss
      [\sFmla{\True}{\formula{A}(s)}, just = \TAss
        [\sFmla{\True}{\formula{A}(t)}, just={\TRule{\True}{\eq}[2, 3]}, close]
      ]
    ]
  ]
\end{oltableau}
This may be familiar as the principle of substitutability of
identicals, or Leibniz' Law.

!!^{tableau}s prove that $\eq$ is symmetric, i.e., that $\eq[s_1][s_2]
\Proves \eq[s_2][s_1]$:
\begin{oltableau}
  [\sFmla{\False}{\eq[s_2][s_1]}, just = \TAss
    [\sFmla{\True}{\eq[s_1][s_2]}, just = \TAss
      [\sFmla{\True}{\eq[s_1][s_1]}, just = {$\eq$}
        [\sFmla{\True}{\eq[s_2][s_1]}, just = {\TRule{\True}{\eq}[2, 3]}, close]
      ]
    ]
  ]
\end{oltableau}
Here, line $2$ is the first prerequisite
!!{formula}~$\sFmla{\True}{\eq[s_1][s_2]}$ of $\TRule{\True}{\eq}$.
Line~$3$ is the second one, of the form $\sFmla{\True}{!A(s_2)}$---think of
$!A(x)$ as
$\eq[x][s_1]$, then $!A(s_1)$ is $\eq[s_1][s_1]$ and $!A(s_2)$ is $\eq[s_2][s_1]$.

They also prove that $\eq$ is transitive, i.e., that $\eq[s_1][s_2],
\eq[s_2][s_3] \Proves \eq[s_1][s_3]$:
\begin{oltableau}
  [\sFmla{\False}{\eq[s_1][s_3]}, just = \TAss
    [\sFmla{\True}{\eq[s_1][s_2]}, just = \TAss
      [\sFmla{\True}{\eq[s_2][s_3]}, just = \TAss
        [\sFmla{\True}{\eq[s_1][s_3]}, just = {\TRule{\True}{\eq}[3, 2]}, close]
      ]
    ]
  ]
\end{oltableau}
In this !!{tableau}, the first prerequisite !!{formula} of
$\TRule{\True}{\eq}$ is line~$3$, $\sFmla{\True}{\eq[s_2][s_3]}$
($s_2$ plays the role of~$t_1$, and $s_3$ the role of~$t_2$). The
second prerequisite, of the form~$\sFmla{\True}{!A(s_2)}$ is line~$2$.
Here, think of $!A(x)$ as $\eq[s_1][x]$; that makes $!A(s_2)$ into
$\eq[t_1][t_2]$ (i.e., line~$2$) and $!A(s_3)$ into the
!!{formula}~$\eq[s_1][s_3]$ in the conclusion.
\end{ex}

\begin{prob}
Give closed !!{tableau}s for the following:
\begin{enumerate}
\item $\sFmla{\False}{\lforall[x][\lforall[y][((x = y \land !A(x))
      \lif !A(y))]]}$
\item $\sFmla{\False}{\lexists[x][(!A(x) \land
    \lforall[y][(!A(y) \lif y = x)])]}$,\\
  $\sFmla{\True}{\lexists[x][!A(x)] \land
    \lforall[y][\lforall[z][((!A(y) \land !A(z)) \lif y = z)]]}$
\end{enumerate}
\end{prob}

\end{document}
