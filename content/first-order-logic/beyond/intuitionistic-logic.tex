% Part: first-order-logic
% Chapter: beyond
% Section: intuitionistic-logic

\documentclass[../../../include/open-logic-section]{subfiles}

\begin{document}

\olfileid{fol}{byd}{il}

\olsection{Intuitionistic Logic}

In constrast to second-order and higher-order logic, intuitionistic
first-order logic represents a restriction of the classical version,
intended to model a more ``constructive'' kind of reasoning. The
following examples may serve to illustrate some of the underlying
motivations.

Suppose someone came up to you one day and announced that they had
determined a natural number~$x$, with the property that if $x$ is
prime, the Riemann hypothesis is true, and if $x$ is composite, the
Riemann hypothesis is false. Great news!{} Whether the Riemann
hypothesis is true or not is one of the big open questions of
mathematics, and here they seem to have reduced the problem to one of
calculation, that is, to the determination of whether a specific
number is prime or not.

What is the magic value of $x$? They describe it as follows: $x$ is
the natural number that is equal to $7$ if the Riemann hypothesis is
true, and $9$ otherwise.

Angrily, you demand your money back. From a classical point of view,
the description above does in fact determine a unique value of $x$;
but what you really want is a value of $x$ that is given
\emph{explicitly}.

To take another, perhaps less contrived example, consider the
following question. We know that it is possible to raise an irrational
number to a rational power, and get a rational result. For example,
$\sqrt{2}^2 = 2$. What is less clear is whether or not it is possible
to raise an irrational number to an \emph{irrational} power, and get a
rational result. The following theorem answers this in the
affirmative:

\begin{thm}
There are irrational numbers $a$ and $b$ such that $a^b$ is rational.
\end{thm}

\begin{proof}
Consider $\sqrt{2}^{\sqrt{2}}$. If this is rational, we are done:
we can let $a = b = \sqrt{2}$. Otherwise, it is irrational. Then we
have
\[
(\sqrt{2}^{\sqrt{2}})^{\sqrt{2}} = \sqrt{2}^{\sqrt{2} \cdot
  \sqrt{2}} = \sqrt{2}^2 = 2,
\]
which is certainly rational. So, in this case, let $a$ be
$\sqrt{2}^{\sqrt{2}}$, and let $b$ be~$\sqrt 2$.
\end{proof}

Does this constitute a valid proof? Most mathematicians feel that it
does. But again, there is something a little bit unsatisfying here: we
have proved the existence of a pair of real numbers with a certain
property, without being able to say \emph{which} pair of numbers it
is.  It is possible to prove the same result, but in such a way that
the pair $a$, $b$ \emph{is} given in the proof: take $a = \sqrt{3}$
and $b = \log_3 4$. Then
\[
a^b = \sqrt{3}^{\log_3 4} = 3^{1/2 \cdot \log_3 4} = (3^{\log_3
  4})^{1/2} = 4^{1/2}= 2,
\]
since $3^{\log_3 x} = x$.

Intuitionistic logic is designed to model a kind of reasoning where
moves like the one in the first proof are disallowed. Proving the
existence of an $x$ satisfying~$!A(x)$ means that you have to give a
specific~$x$, and a proof that it satisfies $!A$, like in the second
proof. Proving that $!A$ or $!B$ holds requires that you can prove one
or the other.

Formally speaking, intuitionistic first-order logic is what you get if
you omit restrict !!a{derivation} system for first-order logic in a certain
way. Similarly, there are intuitionistic versions of second-order or
higher-order logic. From the mathematical point of view, these are
just formal deductive systems, but, as already noted, they are
intended to model a kind of mathematical reasoning. One can take this
to be the kind of reasoning that is justified on a certain
philosophical view of mathematics (such as Brouwer's intuitionism);
one can take it to be a kind of mathematical reasoning which is more
``concrete'' and satisfying (along the lines of Bishop's
constructivism); and one can argue about whether or not the formal
description captures the informal motivation. But whatever
philosophical positions we may hold, we can study intuitionistic logic
as a formally presented logic; and for whatever reasons, many
mathematical logicians find it interesting to do so.

There is an informal constructive interpretation of the intuitionist
connectives, usually known as the BHK interpretation (named after
Brouwer, Heyting, and Kolmogorov). It runs as follows: a proof of $!A
\land !B$ consists of a proof of $!A$ paired with a proof of $!B$; a
proof of $!A \lor !B$ consists of either a proof of $!A$, or a proof
of $!B$, where we have explicit information as to which is the case; a
proof of $!A \lif !B$ consists of a procedure, which transforms a
proof of $!A$ to a proof of~$!B$; a proof of $\lforall[x][!A(x)]$
consists of a procedure which returns a proof of $!A(x)$ for any value
of~$x$; and a proof of $\lexists[x][!A(x)]$ consists of a value
of~$x$, together with a proof that this value satisfies~$!A$. One can
describe the interpretation in computational terms known as the
``Curry-Howard isomorphism'' or the ``!!{formula}s-as-types
paradigm'': think of !!a{formula} as specifying a certain kind of
data type, and proofs as computational objects of these data types
that enable us to see that the corresponding !!{formula} is true.

Intuitionistic logic is often thought of as being classical logic
``minus'' the law of the excluded middle. This following theorem makes
this more precise.
\begin{thm}
Intuitionistically, the following axiom schemata are equivalent:
\begin{enumerate}
\item $(!A \lif \lfalse) \lif \lnot !A$.
\item $!A \lor \lnot !A$
\item $\lnot \lnot !A \lif !A$
\end{enumerate}
\end{thm}
Obtaining instances of one schema from either of the others is a good
exercise in intuitionistic logic.

The first deductive systems for intuitionistic propositional
logic, put forth as formalizations of Brouwer's intuitionism, are due,
independently, to Kolmogorov, Glivenko, and Heyting. The first
formalization of intuitionistic first-order logic (and parts of
intuitionist mathematics) is due to Heyting. Though a number of
classically valid schemata are not intuitionistically valid, many are.

The \emph{double-negation translation} describes an important
relationship between classical and intuitionist logic. It is defined
inductively follows (think of $!A^N$ as the ``intuitionist''
translation of the classical !!{formula}~$!A$):
\begin{align*}
!A^N & \ident \lnot\lnot !A \quad \text{for atomic !!{formula}s $!A$} \\
(!A \land !B)^N & \ident (!A^N \land !B^N) \\
(!A \lor !B)^N & \ident  \lnot\lnot (!A^N \lor !B^N) \\
(!A \lif !B)^N & \ident (!A^N \lif !B^N) \\
(\lforall[x][!A])^N & \ident \lforall[x][!A^N] \\
(\lexists[x][!A])^N & \ident \lnot\lnot\lexists[x][!A^N]
\end{align*}
Kolmogorov and Glivenko had versions of this translation for
propositional logic; for predicate logic, it is due to G\"odel and
Gentzen, independently. We have

\begin{thm}
\begin{enumerate}
\item $!A \liff !A^N$ is provable classically
\item If $!A$ is provable classically, then $!A^N$ is provable
  intuitionistically.
\end{enumerate}
\end{thm}

We can now envision the following dialogue. Classical mathematician:
``I've proved $!A$!'' Intuitionist mathematician: ``Your proof isn't
valid. What you've really proved is $!A^N$.'' Classical mathematician:
``Fine by me!'' As far as the classical mathematician is concerned, the
intuitionist is just splitting hairs, since the two are
equivalent. But the intuitionist insists there is a difference.

Note that the above translation concerns pure logic only; it does not
address the question as to what the appropriate \emph{nonlogical}
axioms are for classical and intuitionistic mathematics, or what the
relationship is between them. But the following slight extension of
the theorem above provides some useful information:

\begin{thm}
If $\Gamma$ proves $!A$ classically, $\Gamma^N$ proves $!A^N$
intuitionistically.
\end{thm}

In other words, if $!A$ is provable from some hypotheses classically,
then $!A^N$ is provable from their double-negation translations.

To show that a sentence or propositional !!{formula} is intuitionistically
valid, all you have to do is provide a proof. But how can you show
that it is not valid? For that purpose, we need a semantics
that is sound, and preferrably complete. A semantics
due to Kripke nicely fits the bill.

We can play the same game we did for classical logic: define the
semantics, and prove soundness and completeness. It is worthwhile,
however, to note the following distinction. In the case of classical
logic, the semantics was the ``obvious'' one, in a sense implicit in
the meaning of the connectives. Though one can provide some intuitive
motivation for Kripke semantics, the latter does not offer the same
feeling of inevitability. In addition, the notion of a classical
!!{structure} is a natural mathematical one, so we can either take the
notion of !!a{structure} to be a tool for studying classical first-order
logic, or take classical first-order logic to be a tool for studying
mathematical !!{structure}s. In contrast, Kripke !!{structure}s can only be
viewed as a logical construct; they don't seem to have independent
mathematical interest.

A Kripke !!{structure}~$\mModel M = \tuple{W, R, V}$ for a propositional
language consists of a set~$W$, partial order~$R$ on~$W$ with a least
!!{element}, and an ``monotone'' assignment of propositional variables to
the !!{element}s of~$W$. The intuition is that the !!{element}s of $W$
represent ``worlds,'' or ``states of knowledge''; an element $v \geq
u$ represents a ``possible future state'' of~$u$; and the
propositional variables assigned to~$u$ are the propositions that are
known to be true in state~$u$. The forcing relation $\mSat{M}{!A}[w]$
then extends this relationship to arbitrary !!{formula}s in the
language; read $\mSat{M}{!A}[w]$ as ``$!A$ is true in state~$w$.'' The
relationship is defined inductively, as follows:
\begin{enumerate}
\item $\mSat{M}{\Obj p_i}[w]$ iff $\Obj p_i$ is one of the
  propositional variables assigned to~$w$.
\item $\mSat/{M}{\lfalse}[w]$.
\item $\mSat{M}{(!A \land !B)}[w]$ iff $\mSat{M}{!A}[w]$ and $\mSat{M}{!B}[w]$.
\item $\mSat{M}{(!A \lor !B)}[w]$ iff $\mSat{M}{!A}[w]$ or $\mSat{M}{!B}[w]$.
\item $\mSat{M}{(!A \lif !B)}[w]$ iff, whenever $w' \geq w$ and
  $\mSat{M}{!A}[w']$, then $\mSat{M}{!B}[w']$.
\end{enumerate}
It is a good exercise to try to show that $\lnot (p \land q) \lif
(\lnot p \lor \lnot q)$ is not intuitionistically valid, by cooking up a
Kripke !!{structure} that provides a counterexample.

\end{document}
