% Part: first-order-logic
% Chapter: proof-systems
% Section: introduction

\documentclass[../../../include/open-logic-section]{subfiles}

\begin{document}

\iftag{FOL}
      {\olfileid{fol}{prf}{int}}
      {\olfileid{pl}{prf}{int}}

\olsection{Introduction}

Logics commonly have both a semantics and !!a{derivation}
system. The semantics concerns concepts such as truth, satisfiability,
validity, and entailment.  The purpose of !!{derivation} systems is to
provide a purely syntactic method of establishing entailment and
validity.  They are purely syntactic in the sense that a
!!{derivation} in such a system is a finite syntactic object, usually
a sequence (or other finite arrangement) of !!{sentence}s or
!!{formula}s.  Good !!{derivation} systems have the property that any
given sequence or arrangement of !!{sentence}s or !!{formula}s can be
verified mechanically to be ``correct.''  

The simplest (and historically first) !!{derivation} systems for
first-order logic were \emph{axiomatic}.  A sequence of !!{formula}s
counts as a !!{derivation} in such a system if each individual
!!{formula} in it is either among a fixed set of ``axioms'' or follows
from !!{formula}s coming before it in the sequence by one of a fixed
number of ``inference rules''---and it can be mechanically verified if
!!a{formula} is an axiom and whether it follows correctly from other
!!{formula}s by one of the inference rules.  Axiomatic proof systems
are easy to describe---and also easy to handle
meta-theoretically---but !!{derivation}s in them are hard to read and
understand, and are also hard to produce.

Other !!{derivation} systems have been developed with the aim of making it
easier to construct !!{derivation}s or easier to understand
!!{derivation}s once they are complete.  Examples are natural
deduction, truth trees, also known as tableaux proofs, and the sequent
calculus.  Some !!{derivation} systems are designed especially with
mechanization in mind, e.g., the resolution method is easy to
implement in software (but its !!{derivation}s are essentially
impossible to understand). Most of these other proof systems represent
!!{derivation}s as trees of !!{formula}s rather than sequences. This
makes it easier to see which parts of !!a{derivation} depend on which
other parts.

So for a given logic, such as first-order logic, the different
!!{derivation} systems will give different explications of what it is
for !!a{sentence} to be a \emph{theorem} and what it means for
!!a{sentence} to be !!{derivable} from some others. However that is
done (via axiomatic !!{derivation}s, natural deductions, sequent
!!{derivation}s, truth trees, resolution refutations), we want these
relations to match the semantic notions of validity and
entailment. Let's write $\Proves !A$ for ``$!A$~is a theorem'' and
``$\Gamma \Proves !A$'' for ``$!A$~is !!{derivable} from~$\Gamma$.''
However $\Proves$~is defined, we want it to match up with $\Entails$,
that is:
\begin{enumerate}
\item $\Proves !A$ if and only if $\Entails !A$
\item $\Gamma \Proves !A$ if and only if $\Gamma \Entails !A$
\end{enumerate}
The ``only if'' direction of the above is called
\emph{soundness}. !!^a{derivation} system is sound if !!{derivability}
guarantees entailment (or validity). Every decent !!{derivation}
system has to be sound; unsound !!{derivation} systems are not useful
at all. After all, the entire purpose of !!a{derivation} is to provide
a syntactic guarantee of validity or entailment. We'll prove soundness
for the !!{derivation} systems we present.

The converse ``if'' direction is also important: it is called
\emph{completeness}. A complete !!{derivation} system is strong enough
to show that $!A$~is a theorem whenever $!A$~is valid, and that
there $\Gamma \Proves !A$ whenever $\Gamma \Entails !A$.
Completeness is harder to establish, and some logics have no complete
!!{derivation} systems. First-order logic does. Kurt G\"odel was the
first one to prove completeness for a !!{derivation} system of
first-order logic in his 1929 dissertation.

Another concept that is connected to !!{derivation} systems is that of
\emph{consistency}. A set of !!{sentence}s is called inconsistent if
anything whatsoever can be !!{derive}d from it, and consistent
otherwise.  Inconsistency is the syntactic counterpart to
unsatisfiablity: like unsatisfiable sets, inconsistent sets of
!!{sentence}s do not make good theories, they are defective in a
fundamental way. Consistent sets of !!{sentence}s may not be true or
useful, but at least they pass that minimal threshold of logical
usefulness.  For different !!{derivation} systems the specific
definition of consistency of sets of !!{sentence}s might differ, but
like $\Proves$, we want consistency to coincide with its semantic
counterpart, satisfiability. We want it to always be the case that
$\Gamma$ is consistent if and only if it is satisfiable. Here, the
``if'' direction amounts to completeness (consistency guarantees
satisfiability), and the ``only if'' direction amounts to soundness
(satisfiability guarantees consistency).  In fact, for classical
first-order logic, the two versions of soundness and completeness are
equivalent.

\end{document}
