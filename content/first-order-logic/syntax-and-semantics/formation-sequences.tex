% Part: first-order-logic
% Chapter: syntax-and-semantics
% Section: formation-sequences

\documentclass[../../../include/open-logic-section]{subfiles}

\begin{document}

\olfileid{fol}{syn}{fseq}

\olsection{Formation Sequences}

Defining !!{formula}s via an inductive definition, and the
complementary technique of proving properties of !!{formula}s via
induction, is an elegant and efficient approach. However, it can
also be useful to consider a more bottom-up, step-by-step approach
to the construction of !!{formula}s, which we do here using the
notion of a \emph{formation sequence}.
%
To show how terms and !!{formula}s can be introduced in this way
without needing to refer to their inductive definitions, we first
introduce the notion of an arbitrary string of symbols drawn from
some language~$\Lang L$.

\begin{defn}[Strings]
\ollabel{defn:string}
Suppose $\Lang L$ is a first-order language. An \emph{$\Lang
L$-string} is a finite sequence of symbols of~$\Lang L$. Where the
language~$\Lang L$ is clearly fixed by the context, we will often
refer to a $\Lang L$-string simply as a \emph{string}.
\end{defn}

\begin{ex}
For any first-order language $\Lang L$, all
$\Lang L$-!!{formula}s are $\Lang L$-strings, but not
conversely. For example, \[)(\Obj v_0\lif\lexists\] is an
$\Lang L$-string but not an $\Lang L$-!!{formula}.
\end{ex}

\begin{defn}[Formation sequences for terms]
\ollabel{defn:fseq-trm}
A finite sequence of $\Lang L$-strings $\tuple{t_0,\dotsc,t_n}$ is a
\emph{formation sequence} for a term $t$ if $t \ident t_n$ and for all
$i \leq n$, either $t_i$ is !!a{variable} or !!a{constant}, or $\Lang
L$ contains a $k$-ary !!{function}~$f$ and there exist
$m_0,\dotsc,m_k < i$ such that $t_i \ident f(t_{m_0},\dotsc,t_{m_k})$.
When it is necessary to distinguish, we will refer to formation
sequences for terms as \emph{term formation sequences}.
\end{defn}

\begin{ex}
The sequence
\[
    \tuple{\Obj c_0, \Obj v_0, \Atom{\Obj f^2_0}{\Obj c_0, \Obj v_0}, \Atom{\Obj f^1_0}{\Atom{\Obj f^2_0}{\Obj c_0, \Obj v_0}}}
\]
is a formation sequence for the term $\Atom{\Obj f^1_0}{\Atom{\Obj
f^2_0}{\Obj c_0, \Obj v_0}}$, as is
\[
    \tuple{\Obj v_0, \Obj c_0, \Atom{\Obj f^2_0}{\Obj c_0, \Obj v_0}, \Atom{\Obj f^1_0}{\Atom{\Obj f^2_0}{\Obj c_0, \Obj v_0}}}.
\]
\end{ex}

\begin{defn}[Formation sequences for formulas]
\ollabel{defn:fseq-frm}
A finite sequence of $\Lang L$-strings $\tuple{!A_0,\dotsc,!A_n}$
is a \emph{formation sequence} for~$!A$ if $!A \ident !A_n$ and
for all $i \leq n$, either $!A_i$ is an atomic !!{formula} or there
exist $j,k < i$ and !!a{variable}~$x$ such that one of the following
holds:
\begin{enumerate}
    \tagitem{prvNot}{$!A_i \ident \lnot !A_j$.}{}%
    \tagitem{prvAnd}{$!A_i \ident (!A_j \land !A_k)$.}{}%
    \tagitem{prvOr}{$!A_i \ident (!A_j \lor !A_k)$.}{}%
    \tagitem{prvIf}{$!A_i \ident (!A_j \lif !A_k)$.}{}%
    \tagitem{prvIff}{$!A_i \ident (!A_j \liff !A_k)$.}{}%
    \tagitem{prvAll}{$!A_i \ident \lforall[x][!A_j]$.}{}%
    \tagitem{prvEx}{$!A_i \ident \lexists[x][!A_j]$.}{}%
\end{enumerate}
When it is necessary to distinguish, we will refer to formation
sequences for formulas as \emph{formula formation sequences}.
\end{defn}

\begin{ex}
\[
\tuple{
    \Atom{\Obj A^1_0}{\Obj v_0},
    \Atom{\Obj A^1_1}{\Obj c_1},
    (\Atom{\Obj A^1_1}{\Obj c_1} \land \Atom{\Obj A^1_0}{\Obj v_0}),
    \lexists[\Obj v_0][(\Atom{\Obj A^1_1}{\Obj c_1} \land \Atom{\Obj A^1_0}{\Obj v_0})]
}
\]
is a formation sequence of $\lexists[\Obj v_0][(\Atom{\Obj A^1_1}{\Obj
c_1} \land \Atom{\Obj A^1_0}{\Obj v_0})]$, as is
\begin{multline*}
\tuple{
    \Atom{\Obj A^1_0}{\Obj v_0},
    \Atom{\Obj A^1_1}{\Obj c_1},
    (\Atom{\Obj A^1_1}{\Obj c_1} \land \Atom{\Obj A^1_0}{\Obj v_0}),
    \Atom{\Obj A^1_1}{\Obj c_1},\\
    \lforall[\Obj v_1][\Atom{\Obj A^1_0}{\Obj v_0}],
    \lexists[\Obj v_0][(\Atom{\Obj A^1_1}{\Obj c_1} \land \Atom{\Obj A^1_0}{\Obj v_0})]
}.
\end{multline*}
%
As can be seen from the second example, formation sequences
may contain ``junk'': !!{formula}s which are redundant or do not
contribute to the construction.
\end{ex}

\begin{prop}\ollabel{prop:formed}
Every !!{formula}~$!A$ in~$\Frm[L]$ has a formation sequence.
\end{prop}

\begin{proof}
Suppose $!A$ is atomic. Then the sequence $\tuple{!A}$ is a
formation sequence for~$!A$.
%
Now suppose that $!B$ and~$!C$ have formation sequences
$\tuple{!B_0,\dotsc,!B_n}$ and $\tuple{!C_0,\dotsc,!C_m}$
respectively.
%
\begin{enumerate}
  \tagitem{prvNot}{If $!A \ident \lnot !B$,
      then $\tuple{!B_0,\dotsc,!B_n,\lnot !B_n}$
      is a formation sequence for~$!A$.}{}
  \tagitem{prvAnd}{If $!A \ident (!B \land !C)$,
      then $\tuple{!B_0,\dotsc,!B_n,!C_0,\dotsc,!C_m,(!B_n \land !C_m)}$
      is a formation sequence for~$!A$.}{}
  \tagitem{prvOr}{If $!A \ident (!B \lor !C)$,
      then $\tuple{!B_0,\dotsc,!B_n,!C_0,\dotsc,!C_m,(!B_n \lor !C_m)}$
      is a formation sequence for~$!A$.}{}
  \tagitem{prvIf}{If $!A \ident (!B \lif !C)$,
      then $\tuple{!B_0,\dotsc,!B_n,!C_0,\dotsc,!C_m,(!B_n \lif !C_m)}$
      is a formation sequence for~$!A$.}{}
  \tagitem{prvIff}{If $!A \ident (!B \liff !C)$,
      then $\tuple{!B_0,\dotsc,!B_n,!C_0,\dotsc,!C_m,(!B_n \liff !C_m)}$
      is a formation sequence for~$!A$.}{}
  \tagitem{prvAll}{If $!A \ident \lforall[x][!B]$,
      then $\tuple{!B_0,\dotsc,!B_n,\lforall[x][!B_n]}$
      is a formation sequence for~$!A$.}{}
  \tagitem{prvEx}{If $!A \ident \lexists[x][!B]$,
      then $\tuple{!B_0,\dotsc,!B_n,\lexists[x][!B_n]}$
      is a formation sequence for~$!A$.}{}
\end{enumerate}
By the principle of induction on !!{formula}s,
every !!{formula} has a formation sequence.
\end{proof}

We can also prove the converse. This is important because it shows
that our two ways of defining formulas are equivalent: they give
the same results. It also means that we can prove theorems about
formulas by using ordinary induction on the length of formation
sequences.

\begin{lem}
\ollabel{lem:fseq-init}
Suppose that $\tuple{!A_0,\dotsc,!A_n}$ is a formation sequence
for~$!A_n$, and that $k \leq n$. Then $\tuple{!A_0,\dotsc,!A_k}$
is a formation sequence for~$!A_k$.
\end{lem}

\begin{proof}
Exercise.
\end{proof}

\begin{prob}
Prove \olref[fol][syn][fseq]{lem:fseq-init}.
\end{prob}

\begin{thm}
\ollabel{thm:fseq-frm-equiv}
$\Frm[L]$ is the set of all $\Lang L$-strings $\varphi$ such that
there exists a formula formation sequence for $\varphi$.
\end{thm}

\begin{proof}
Let $F$ be the set of all strings of symbols in the language~$\Lang L$
that have a formation sequence. We have seen in
\olref[fol][syn][fseq]{prop:formed} that $\Frm[L] \subseteq F$, so now
we prove the converse.

Suppose $!A$ has a formation sequence $\tuple{!A_0,\dotsc,!A_n}$.
We prove that $!A \in \Frm[L]$ by strong induction on~$n$.
Our induction hypothesis is that every string of symbols with a
formation sequence of length $m < n$ is in $\Frm[L]$.
By the definition of a formation sequence, either $!A \ident !A_n$ is
atomic or there must exist $j,k < n$ such that one of the
following is the case:
\begin{enumerate}
    \tagitem{prvNot}{$!A \ident \lnot !A_j$.}{}%
    \tagitem{prvAnd}{$!A \ident (!A_j \land !A_k)$.}{}%
    \tagitem{prvOr}{$!A \ident (!A_j \lor !A_k)$.}{}%
    \tagitem{prvIf}{$!A \ident (!A_j \lif !A_k)$.}{}%
    \tagitem{prvIff}{$!A \ident (!A_j \liff !A_k)$.}{}%
    \tagitem{prvAll}{$!A \ident \lforall[x][!A_j]$.}{}%
    \tagitem{prvEx}{$!A \ident \lexists[x][!A_j]$.}{}%
\end{enumerate}
Now we reason by cases. If $!A$ is atomic then
$!A_n \in \Frm[L_0]$. Suppose instead that
$!A \equiv (!A_j \land !A_k)$. By 
\olref[fol][syn][fseq]{lem:fseq-init},
$\tuple{!A_0,\dotsc,!A_j}$ and $\tuple{!A_0,\dotsc,!A_k}$ are
formation sequences for $!A_j$ and~$!A_k$, respectively. Since
these are proper initial subsequences of the formation sequence
for~$!A$, they both have length less than~$n$. Therefore by
the induction hypothesis, $!A_j$ and~$!A_k$ are in~$\Frm[L_0]$,
and by the definition of !!a{formula}, so is
$(!A_j \land !A_k)$. The other cases follow by parallel
reasoning.
\end{proof}

Formation sequences for terms have similar properties to those
for !!{formula}s.

\begin{prop}
\ollabel{prop:fseq-trm-equiv}
$\Trm[L]$ is the set of all $\Lang L$-strings $t$
such that there exists a term formation sequence for~$t$.
\end{prop}

\begin{proof}
Exercise.
\end{proof}

\begin{prob}
Prove \olref[fol][syn][fseq]{prop:fseq-trm-equiv}.
Hint: use a similar strategy to that used in the proof of
\olref[fol][syn][fseq]{thm:fseq-frm-equiv}.
\end{prob}

There are two types of ``junk'' that can appear in formation
sequences: repeated elements, and elements that are irrelevant
to the construction of the formation or term. We can eliminate
both by looking at minimal formation sequences.

\begin{defn}[Minimal formation sequences]
\ollabel{defn:minimal-fseq}
A formation sequence $\tuple{!A_0, \ldots, !A_n}$ for a
formula~$!A$ is a \emph{minimal formation sequence} for~$!A$
if for every other formation sequence~$s$ for~$!A$,
the length of~$s$ is greater than or equal to~$n+1$.

Similarly, a formation sequence $\tuple{t_0, \ldots, t_n}$
for a term~$t$ is a \emph{minimal formation sequence}
for~$t$ if for every other formation sequence~$s$ for~$t$,
the length of~$s$ is greater than or equal to~$n+1$.
\end{defn}

Note that a formula or term can have more than one minimal
formation sequence, but they will contain exactly the same
strings.

\begin{prop}
\ollabel{prop:subformula-equivs}
The following are equivalent:
\begin{enumerate}
    \item $!B$ is a sub-!!{formula} of~$!A$.
    \item $!B$ occurs in every formation sequence of~$!A$.
    \item $!B$ occurs in a minimal formation sequence of~$!A$.
\end{enumerate}
\end{prop}

\begin{proof}
Exercise.
\end{proof}

\begin{prob}
Prove \olref[fol][syn][fseq]{prop:subformula-equivs}.
\end{prob}

\begin{history}
Formation sequences were introduced by Raymond Smullyan in his
textbook \emph{First-Order Logic} \citep{Smullyan1968}.
Additional properties of formation sequences were established by
\citet{Zuckerman1973}.
\end{history}

\end{document}
