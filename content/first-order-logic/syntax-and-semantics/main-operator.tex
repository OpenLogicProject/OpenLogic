% Part: first-order-logic
% Chapter: syntax-and-semantics
% Section: main-operator

\documentclass[../../../include/open-logic-section]{subfiles}

\begin{document}

\olfileid{fol}{syn}{mai}

\olsection{\printtoken{S}{main operator} of a Formula}

\begin{explain}
It is often useful to talk about the last operator used in
constructing !!a{formula}~$!A$.  This operator is called the \emph{main
  operator} of~$!A$. Intuitively, it is the ``outermost'' operator
of $!A$. For example, the main operator of $\lnot !A$ is $\lnot$,
the main operator of $(!A \lor !B)$ is $\lor$, etc.
\end{explain}


\begin{defn}[!!^{main operator}]
\ollabel{def:main-op}
The \emph{!!{main operator}} of !!a{formula}~$!A$ is
defined as follows:
\begin{enumerate}
\item \indcase*{!A}{!A}{$\indfrm$ has no !!{main operator}.}

\tagitem{prvNot}{\indcase{!A}{\lnot !B}{the !!{main operator} of $\indfrm$
  is~$\lnot$.}}{}

\tagitem{prvAnd}{\indcase{!A}{(!B \land !C)}{the !!{main operator} of
  $\indfrm$ is~$\land$.}}{}

\tagitem{prvOr}{\indcase{!A}{(!B \lor !C)}{the !!{main operator} of
  $\indfrm$ is~$\lor$.}}{}

\tagitem{prvIf}{\indcase{!A}{(!B \lif !C)}{the !!{main operator} of
  $\indfrm$ is~$\lif$.}}{}

\tagitem{prvIff}{\indcase{!A}{(!B \liff !C)}{the !!{main operator} of
  $\indfrm$ is~$\liff$.}}{}

\tagitem{prvAll}{\indcase{!A}{\lforall[x][!B]}{the !!{main operator}
  of $\indfrm$ is~$\lforall$.}}{}

\tagitem{prvEx}{\indcase{!A}{\lexists[x][!B]}{the !!{main operator} of
  $\indfrm$ is~$\lexists$.}}{}
\end{enumerate}
\end{defn}

In each case, we intend the specific indicated \emph{occurrence} of
the !!{main operator} in the formula. For instance, since the formula
$((!D \lif !E) \lif (!E \lif !D))$ is of the form $(!B \lif !C)$ where
$!B$ is $(!D \lif !E)$ and $!C$ is $(!E \lif !D)$, the second
occurrence of $\lif$ is the !!{main operator}.

\begin{explain}
This is a \emph{recursive} definition of a function which maps all
non-atomic !!{formula}s to their !!{main operator} occurrence. Because
of the way !!{formula}s are defined inductively, every
!!{formula}~$!A$ satisfies one of the cases in \olref{def:main-op}.
This guarantees that for each non-atomic !!{formula}~$!A$ !!a{main
  operator} exists. Because each !!{formula} satisfies only one of these
conditions, and because the smaller !!{formula}s from which $!A$ is
constructed are uniquely determined in each case, the !!{main
  operator} occurrence of~$!A$ is unique, and so we have defined a
function.
\end{explain}

We call !!{formula}s by the following names depending on which symbol their
!!{main operator} is:

\begin{figure}[!h]
\centering
\begin{tabular}{c | c | c}
!!^{main operator} & Type of !!{formula} & Example\\
\hline
none & atomic (!!{formula}) &
\iftag{prvFalse}{$\lfalse$,}{}
\iftag{prvTrue}{$\ltrue$,}{}
$\Atom{R}{t_1, \dots, t_n}$,
$\eq[t_1][t_2]$\\
$\lnot$ & negation & $\lnot !A$ \\
$\land$ & conjunction & $(!A \land !B$) \\
$\lor$ & disjunction & $(!A \lor !B$) \\
$\lif$ & !!{conditional} & $(!A \lif !B$) \\
$\liff$ & !!{biconditional} & $(!A \liff !B$) \\
\iftag{prvIff,defIff}{}{$\liff$ & !!{biconditional} & $!A \liff !B$ \\}
$\lforall[][]$ & universal (!!{formula})& $\lforall[x][!A]$ \\
$\lexists[][]$ & existential (!!{formula})& $\lexists[x][!A]$
\end{tabular}
\end{figure}

\end{document}
