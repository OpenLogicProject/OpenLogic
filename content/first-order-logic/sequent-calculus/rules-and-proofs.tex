% Part: first-order-logic
% Chapter: sequent-calculus
% Section: rules-and-proofs

\documentclass[../../../include/open-logic-section]{subfiles}

\begin{document}

\olfileid{fol}{seq}{rul}

\olsection{Rules and \usetoken{P}{derivation}}

For the following, let $\Gamma, \Delta, \Pi, \Lambda$ represent finite
sequences of !!{sentence}s.

\begin{defn}[Sequent]
A \emph{sequent} is an expression of the form
\[
\Gamma \Sequent \Delta
\]
where $\Gamma$ and $\Delta$ are finite (possibly empty) sequences of
!!{sentence}s of the language $\Lang L$. $\Gamma$ is called the
\emph{antecedent}, while $\Delta$ is the \emph{succedent}.
\end{defn}

\begin{explain}
The intuitive idea behind a sequent is: if all of the !!{sentence}s in
the antecedent hold, then at least one of the !!{sentence}s in the
succedent holds. That is, if $\Gamma = \tuple{!A_1, \dots, !A_m}$ and
$\Delta = \tuple{!B_1, \dots, !B_n}$, then $\Gamma \Sequent \Delta$
holds iff
\[
(!A_1 \land \cdots \land !A_m) \lif (!B_1 \lor \cdots \lor
!B_n)
\]
holds. There are two special cases: where $\Gamma$ is empty and when
$\Delta$~is empty. When $\Gamma$ is empty, i.e., $m = 0$, $\quad
\Sequent \Delta$ holds iff $!B_1 \lor \dots \lor !B_n$ holds. When
$\Delta$ is empty, i.e., $n = 0$, $\Gamma \Sequent \quad$ holds iff
$\lnot(!A_1 \land \dots \land !A_m)$ does.  We say a sequent is valid
iff the corresponding !!{sentence} is valid.
\end{explain}

If $\Gamma$ is a sequence of !!{sentence}s, we write $\Gamma, !A$ for
the result of appending $!A$ to the right end of~$\Gamma$ (and $!A,
\Gamma$ for the result of appending $!A$ to the left end
of~$\Gamma$). If $\Delta$ is a sequence of !!{sentence}s also, then $\Gamma,
\Delta$ is the concatenation of the two sequences.

\begin{defn}[Initial Sequent]
An \emph{initial sequent} is a sequent
\iftag{prvFalse,prvTrue}{of one of the following forms:
  \begin{enumerate}
    \item $!A \Sequent !A$
    \tagitem{prvTrue}{$\quad \Sequent \ltrue$}{}
    \tagitem{prvFalse}{$\lfalse \Sequent \quad$}{}
  \end{enumerate}}
{of the form $!A \Sequent !A$} for any !!{sentence} $!A$ in the language.
\end{defn}

!!^{derivation}s in the sequent calculus are certain trees of
sequents, where the topmost sequents are initial sequents, and if a
sequent stands below one or two other sequents, it must follow
correctly by a rule of inference.  The rules for $\Log{LK}$ are
divided into two main types: \emph{logical} rules and
\emph{structural} rules.  The logical rules are named for the !!{main
  operator} of the !!{sentence} containing $!A$ and/or $!B$ in the
lower sequent. Each one comes in two versions, one for inferring a
sequent with the !!{sentence} containg the !!{operator} on the left,
and one with the !!{sentence} on the right.

\end{document}
