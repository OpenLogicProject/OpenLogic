% Part: first-order-logic
% Chapter: sequent-calculus
% Section: proof-theoretic-notions

\documentclass[../../../include/open-logic-section]{subfiles}

\begin{document}

\begin{editorial}
  This section collects the properties of the provability relation
  required for the completeness theorem.  If you find the location
  unmotivated, include it instead in the chapter on completeness.
\end{editorial}

\olfileid{fol}{seq}{ptn}
\olsection{Proof-Theoretic Notions}

\begin{explain}
Just as we've defined a number of important semantic notions
(validity, entailment, satisfiabilty), we now define corresponding
\emph{proof-theoretic notions}.  These are not defined by appeal to
satisfaction of sentences in !!{structure}s, but by appeal to the
!!{derivability} or !!{nonderivability} of certain sequents.  It was
an important discovery, due to G\"odel, that these notions coincide.
That they do is the content of the \emph{completeness theorem}.
\end{explain}

\begin{defn}[Theorems]
A sentence~$!A$ is a \emph{theorem} if there is a !!{derivation}
in~$\Log{LK}$ of the sequent $\quad \Sequent !A$.  We write
$\Proves[\Log{LK}] !A$ if $!A$ is a theorem and $\Proves/[\Log{LK}]
!A$ if it is not.
\end{defn}

\begin{defn}[!!^{derivability}]
A sentence $!A$ is \emph{!!{derivable} from} a set of
sentences~$\Gamma$, $\Gamma \Proves[\Log{LK}] !A$, iff there is a
finite subset~$\Gamma_0 \subseteq \Gamma$ and a sequence $\Gamma_0'$
of the !!{formula}s in~$\Gamma_0$ such that $\Log{LK}$ !!{derive}s
$\Gamma_0' \Sequent !A$.  If $!A$ is not !!{derivable} from $\Gamma$
we write $\Gamma \Proves/[\Log{LK}] !A$.
\end{defn}

Because of the contraction, weakening, and exchange rules, the order
and number of !!{formula}s in~$\Gamma_0'$ does not matter: if a
sequent $\Gamma_0' \Sequent !A$ is !!{derivable}, then so is
$\Gamma_0'' \Sequent !A$ for any $\Gamma_0''$ that contans the same
!!{formula}s as~$\Gamma_0'$.  For instance, if $\Gamma_0 = \{!B, !C\}$
then both $\Gamma_0' = \tuple{!B, !B, !C}$ and $\Gamma_0'' =
\tuple{!C, !C, !B}$ are sequences containing just the !!{formula}s
in~$\Gamma_0$. If a sequent containing one is !!{derivable}, so is the
other, e.g.:
\begin{prooftree}
  \AxiomC{}
  \Deduce$!B, !B, !C \fCenter !A$
  \RightLabel{\LeftR{\Contraction}}
  \UnaryInf$!B, !C \fCenter !A$
  \RightLabel{\LeftR{\Exchange}}
  \UnaryInf$!C, !B \fCenter !A$
  \RightLabel{\LeftR{\Weakening}}
  \UnaryInf$!C, !C, !B \fCenter !A$
\end{prooftree}
From now on we'll say that if $\Gamma_0$ is a finite set of
!!{formula}s then $\Gamma_0 \Sequent !A$ is any sequent where the
antecedent is a sequence of !!{formula}s in~$\Gamma_0$ and tacitly include
contractions, exchanges, and weakenings if necessary.

\begin{defn}[Consistency]
A set of sentences~$\Gamma$ is \emph{inconsistent} iff there is a
finite subset~$\Gamma_0 \subseteq \Gamma$ such that $\Log{LK}$
!!{derive}s $\Gamma_0 \Sequent \quad$. If $\Gamma$ is not
inconsistent, i.e., if for every finite $\Gamma_0 \subseteq \Gamma$,
$\Log{LK}$ does not !!{derive} $\Gamma_0 \Sequent \quad$, we say it is
\emph{consistent}.
\end{defn}

\begin{prop}
\ollabel{prop:prov-incons}
$\Gamma \Proves[\Log{LK}] !A$ iff $\Gamma \cup \{\lnot !A\}$ is inconsistent.
\end{prop}

\begin{proof}
Exercise.
\end{proof}

\begin{prob}
Prove \olref[fol][seq][ptn]{prop:prov-incons}
\end{prob}

\begin{prop}
\ollabel{prop:incons}
$\Gamma$ is inconsistent iff $\Gamma \Proves[\Log{LK}] {!A}$ for every
  sentence~$!A$.
\end{prop}

\begin{proof}
Exercise.
\end{proof}

\begin{prob}
Prove \olref[fol][seq][ptn]{prop:incons}
\end{prob}

\begin<prvFalse>{prop}
\ollabel{prop:incons-lfalse}
$\Gamma$ is inconsistent iff $\Gamma_0 \Proves[\Log{LK}] \lfalse$.
\end{prop}

\begin{proof}
Suppose $\Gamma$ is inconsistent, i.e., $\Log{LK}$ !!{derive}s
$\Gamma_0 \Sequent \quad$ for some finite $\Gamma_0 \subseteq
\Gamma$. Then, using \RightR{\Weakening}, it also !!{derive}s
$\Gamma_0 \Sequent \lfalse$.  If, on the other hand, $\Log{LK}$
!!{derive}s $\Gamma_0 \Sequent \lfalse$, we obtain !!a{derivation} of
$\Gamma_0 \Sequent \quad$ using cut and an initial sequent:
\begin{prooftree}
  \AxiomC{}
  \Deduce$\Gamma_0 \fCenter \lfalse$
  \Axiom$\lfalse \fCenter$
  \RightLabel{\Cut}
  \BinaryInf$\Gamma_0 \fCenter$
\end{prooftree}
\end{proof}

\end{document}
