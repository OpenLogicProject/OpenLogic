% Part: first-order-logic
% Chapter: sequent-calculus
% Section: proving-things

\documentclass[../../../include/open-logic-section]{subfiles}

\begin{document}

\iftag{FOL}
      {\olfileid{fol}{seq}{pro}}
      {\olfileid{pl}{seq}{pro}}

\olsection{Examples of \usetoken{P}{derivation}}

\begin{ex}
Give an $\Log{LK}$-derivation for the sequent $!A \land !B \Sequent !A$.

We begin by writing the desired end-sequent at the bottom of the derivation.
\begin{prooftree}
\AxiomC{}
\UnaryInf$!A\land !B \fCenter !A$
\end{prooftree}
Next, we need to figure out what kind of inference could have a lower
sequent of this form. This could be a structural rule, but it is a
good idea to start by looking for a logical rule. The only logical
connective occurring in the lower sequent is $\land$,
so we're looking for an $\land$ rule, and since the $\land$ symbol
occurs in the antecedent, we're looking at the \LeftR{\land}
rule.
\begin{prooftree}
\AxiomC{}
\RightLabel{\LeftR{\land}}
\UnaryInf$!A\land !B \fCenter !A$
\end{prooftree}
There are two options for what could have been the upper sequent of
the \LeftR{\land} inference: we could have an upper sequent of $!A
\Sequent !A$, or of $!B \Sequent !A$. Clearly, $!A \Sequent !A$ is an
initial sequent (which is a good thing), while $!B \Sequent !A$ is not
derivable in general. We fill in the upper sequent:
\begin{prooftree}
\Axiom$!A \fCenter !A$
\RightLabel{\LeftR{\land}}
\UnaryInf$!A\land !B \fCenter !A$
\end{prooftree}
We now have a correct $\Log{LK}$-derivation of the sequent $!A \land
!B \Sequent !A$.
\end{ex}

\begin{ex}
Give an $\Log{LK}$-derivation for the sequent $\lnot !A \lor !B
\Sequent !A \lif !B$.

Begin by writing the desired end-sequent at the bottom of the derivation.
\begin{prooftree}
\AxiomC{}
\UnaryInf$\lnot !A \lor !B \fCenter !A \lif !B$
\end{prooftree}
To find a logical rule that could give us this end-sequent, we look at
the logical connectives in the end-sequent: $\lnot$, $\lor$, and
$\lif$. We only care at the moment about $\lor$ and $\lif$ because
they are !!{main operator}s of !!{sentence}s in the end-sequent,
while $\lnot$ is inside the scope of another connective, so we will
take care of it later. Our options for logical rules for the final
inference are therefore the \LeftR{\lor} rule and the \RightR{\lif}
rule. We could pick either rule, really, but let's pick the \RightR{\lif}
rule (if for no reason other than it allows us to put off
splitting into two branches). According to the form of \RightR{\lif}
inferences which can yield the lower sequent, this must look like:
\begin{prooftree}
\AxiomC{}
\UnaryInf$ !A, \lnot !A \lor !B \fCenter !B $
\RightLabel{\RightR{\lif}} \UnaryInf$ \lnot !A \lor !B \fCenter !A \lif !B $
\end{prooftree}

If we move $\lnot !A \lor !B$ to the outside of the antecedent, we can
apply the \LeftR{\lor} rule. According to the schema, this must split
into two upper sequents as follows:
\begin{prooftree}
\AxiomC{}
\UnaryInf$\lnot !A, !A \fCenter !B$
\AxiomC{}
\UnaryInf$!B, !A \fCenter !B$
\RightLabel{\LeftR{\lor}}
\BinaryInf$ \lnot !A \lor !B, !A \fCenter !B $
\RightLabel{\RightR{\Exchange}}
\UnaryInf$ !A, \lnot !A \lor !B \fCenter !B $
\RightLabel{\RightR{\lif}}
\UnaryInf$ \lnot !A \lor !B \fCenter !A \lif !B $
\end{prooftree}
Remember that we are trying to wind our way up to initial sequents; we
seem to be pretty close!{} The right branch is just one weakening and
one exchange away from an initial sequent and then it is done:
\begin{prooftree}
\AxiomC{}
\UnaryInf$\lnot !A, !A \fCenter !B$
\Axiom$!B \fCenter !B$
\RightLabel{\LeftR{\Weakening}}
\UnaryInf$!A, !B \fCenter !B$
\RightLabel{\LeftR{\Exchange}}
\UnaryInf$!B, !A \fCenter !B$
\RightLabel{\LeftR{\lor}}
\BinaryInf$\lnot !A \lor !B, !A \fCenter !B $
\RightLabel{\RightR{\Exchange}}
\UnaryInf$ !A, \lnot !A \lor !B \fCenter !B $
\RightLabel{\RightR{\lif}}
\UnaryInf$ \lnot !A \lor !B \fCenter !A \lif !B $
\end{prooftree}

Now looking at the left branch, the only logical connective in any
!!{sentence} is the $\lnot$ symbol in the antecedent !!{sentence}s, so
we're looking at an instance of the \LeftR{\lnot} rule.
\begin{prooftree}
\AxiomC{}
\UnaryInf$ !A \fCenter !B, !A$
\RightLabel{\LeftR{\lnot}}
\UnaryInf$\lnot !A, !A \fCenter !B$
\Axiom$!B \fCenter !B$
\RightLabel{\LeftR{\Weakening}}
\UnaryInf$!A, !B \fCenter !B$
\RightLabel{\LeftR{\Exchange}}
\UnaryInf$!B, !A \fCenter !B$
\RightLabel{\LeftR{\lor}}
\BinaryInf$\lnot !A \lor !B, !A \fCenter !B $
\RightLabel{\RightR{\Exchange}}
\UnaryInf$ !A, \lnot !A \lor !B \fCenter !B $
\RightLabel{\RightR{\lif}}
\UnaryInf$ \lnot !A \lor !B \fCenter !A \lif !B $
\end{prooftree}
Similarly to how we finished off the right branch, we are just one
weakening and one exchange away from finishing off this left branch as well.
\begin{prooftree}
\Axiom$!A \fCenter !A$
\RightLabel{\RightR{\Weakening}}
\UnaryInf$ !A \fCenter !A, !B$
\RightLabel{\RightR{\Exchange}}
\UnaryInf$ !A \fCenter !B, !A$
\RightLabel{\LeftR{\lnot}}
\UnaryInf$\lnot !A, !A \fCenter !B$
\Axiom$!B \fCenter !B$
\RightLabel{\LeftR{\Weakening}}
\UnaryInf$!A, !B \fCenter !B$
\RightLabel{\LeftR{\Exchange}}
\UnaryInf$!B, !A \fCenter !B$
\RightLabel{\LeftR{\lor}}
\BinaryInf$\lnot !A \lor !B, !A \fCenter !B $
\RightLabel{\RightR{\Exchange}}
\UnaryInf$ !A, \lnot !A \lor !B \fCenter !B $
\RightLabel{\RightR{\lif}}
\UnaryInf$ \lnot !A \lor !B \fCenter !A \lif !B $
\end{prooftree}
\end{ex}

\begin{ex}
Give an $\Log{LK}$-derivation of the sequent $\lnot !A \lor \lnot !B
\Sequent \lnot (!A \land !B)$

Using the techniques from above, we start by writing the desired
end-sequent at the bottom.
\begin{prooftree}
\AxiomC{}
\UnaryInf$ \lnot !A \lor \lnot !B \fCenter \lnot (!A \land !B) $
\end{prooftree}
The available main connectives of !!{sentence}s in the end-sequent are
the $\lor$ symbol and the $\lnot$ symbol. It would work to apply
either the \LeftR{\lor} or the \RightR{\lnot} rule here, but we start
with the \RightR{\lnot} rule because it avoids splitting up into two
branches for a moment:
\begin{prooftree}
\AxiomC{}
\UnaryInf$!A \land !B, \lnot !A \lor \lnot !B \fCenter $
\RightLabel{\RightR{\lnot}}
\UnaryInf$\lnot !A \lor \lnot !B \fCenter \lnot (!A \land !B)$
\end{prooftree}
Now we have a choice of whether to look at the \LeftR{\land} or the
\LeftR{\lor} rule. Let's see what happens when we apply the \LeftR{\land}
rule: we have a choice to start with either the sequent $!A,
\lnot !A \lor !B \Sequent \quad$ or the sequent $!B, \lnot !A
\lor !B \Sequent \quad$. Since the !!{derivation} is symmetric with
regards to $!A$ and $!B$, let's go with the former:
\begin{prooftree}
\AxiomC{}
\UnaryInf$!A, \lnot !A \lor \lnot !B \fCenter $
\RightLabel{\LeftR{\land}}
\UnaryInf$!A \land !B, \lnot !A \lor \lnot !B \fCenter $
\RightLabel{\RightR{\lnot}}
\UnaryInf$\lnot !A \lor \lnot !B \fCenter \lnot (!A \land !B)$
\end{prooftree}
Continuing to fill in the derivation, we see that we run into a problem:
\begin{prooftree}
\Axiom$!A \fCenter !A$
\RightLabel{\LeftR{\lnot}}
\UnaryInf$ \lnot !A, !A \fCenter$
\AxiomC{}
\RightLabel{?}
\UnaryInf$!A \fCenter !B$
\RightLabel{\LeftR{\lnot}}
\UnaryInf$ \lnot !B, !A \fCenter$
\RightLabel{\LeftR{\lor}}
\BinaryInf$\lnot !A \lor \lnot !B, !A \fCenter $
\RightLabel{\LeftR{\Exchange}}
\UnaryInf$!A, \lnot !A \lor \lnot !B \fCenter $
\RightLabel{\LeftR{\land}}
\UnaryInf$!A \land !B, \lnot !A \lor \lnot !B \fCenter $
\RightLabel{\RightR{\lnot}}
\UnaryInf$\lnot !A \lor \lnot !B \fCenter \lnot (!A \land !B)$
\end{prooftree}
The top of the right branch cannot be reduced any further, and it
cannot be brought by way of structural inferences to an initial
sequent, so this is not the right path to take. So clearly, it was a
mistake to apply the \LeftR{\land} rule above. Going back to what we
had before and carrying out the \LeftR{\lor} rule instead, we get
\begin{prooftree}
\AxiomC{}
\UnaryInf$\lnot !A, !A \land !B \fCenter $

\AxiomC{}
\UnaryInf$\lnot !B, !A \land !B \fCenter $

\RightLabel{\LeftR{\lor}}
\BinaryInf$\lnot !A \lor \lnot !B, !A \land !B \fCenter $
\RightLabel{\LeftR{\Exchange}}
\UnaryInf$!A \land !B, \lnot !A \lor \lnot !B \fCenter $
\RightLabel{\RightR{\lnot}}
\UnaryInf$\lnot !A \lor \lnot !B \fCenter \lnot (!A \land !B)$
\end{prooftree}
Completing each branch as we've done before, we get
\begin{prooftree}
\Axiom$ !A \fCenter!A$
\RightLabel{\LeftR{\land}}
\UnaryInf$!A \land !B \fCenter !A$
\RightLabel{\LeftR{\lnot}}
\UnaryInf$\lnot !A, !A \land !B \fCenter $

\Axiom$ !B \fCenter !B$
\RightLabel{\LeftR{\land}}
\UnaryInf$!A \land !B \fCenter !B$
\RightLabel{\LeftR{\lnot}}
\UnaryInf$\lnot !B, !A \land !B \fCenter $

\RightLabel{\LeftR{\lor}}
\BinaryInf$\lnot !A \lor \lnot !B, !A \land !B \fCenter $
\RightLabel{\LeftR{\Exchange}}
\UnaryInf$!A \land !B, \lnot !A \lor \lnot !B \fCenter $
\RightLabel{\RightR{\lnot}}
\UnaryInf$\lnot !A \lor \lnot !B \fCenter \lnot (!A \land !B)$
\end{prooftree}
(We could have carried out the $\land$ rules lower than the $\lnot$
rules in these steps and still obtained a correct derivation).
\end{ex}

\begin{ex}
So far we haven't used the contraction rule, but it is sometimes
required. Here's an example where that happens.  Suppose we want to
prove $\quad \Sequent !A \lor \lnot !A$. Applying $\RightR{\lor}$
backwards would give us one of these two !!{derivation}s:
\begin{prooftree}
\AxiomC{}
\UnaryInf$ \fCenter !A$
\RightLabel{\RightR{\lor}}
\UnaryInf$ \fCenter !A \lor \lnot !A$
\DisplayProof\qquad\bottomAlignProof
\AxiomC{}
\UnaryInf$!A \fCenter $
\RightLabel{\RightR{\lnot}}
\UnaryInf$ \fCenter \lnot !A$
\RightLabel{\RightR{\lor}}
\UnaryInf$ \fCenter !A \lor \lnot !A$
\end{prooftree}
Neither of these of course ends in an initial sequent.  The trick is
to realize that the contraction rule allows us to combine two copies
of !!a{sentence} into one---and when we're searching for a proof,
i.e., going from bottom to top, we can keep a copy of $!A \lor \lnot
!A$ in the premise, e.g.,
\begin{prooftree}
\AxiomC{}
\UnaryInf$ \fCenter !A \lor \lnot !A, !A$
\RightLabel{\RightR{\lor}}
\UnaryInf$ \fCenter !A \lor \lnot !A, !A \lor \lnot !A$
\RightLabel{\RightR{\Contraction}}
\UnaryInf$ \fCenter !A \lor \lnot !A$
\end{prooftree}
Now we can apply $\RightR{\lor}$ a second time, and also get~$\lnot
!A$, which leads to a complete !!{derivation}.
\begin{prooftree}
\Axiom$!A \fCenter !A$
\RightLabel{\RightR{\lnot}}
\UnaryInf$\fCenter !A, \lnot !A$
\RightLabel{\RightR{\lor}}
\UnaryInf$\fCenter !A, !A \lor \lnot !A$
\RightLabel{\RightR{\Exchange}}
\UnaryInf$ \fCenter !A \lor \lnot !A, !A$
\RightLabel{\RightR{\lor}}
\UnaryInf$ \fCenter !A \lor \lnot !A, !A \lor \lnot !A$
\RightLabel{\RightR{\Contraction}}
\UnaryInf$ \fCenter !A \lor \lnot !A$
\end{prooftree}
\end{ex}

\begin{prob}
Give !!{derivation}s of the following sequents:
\begin{enumerate}
\item $!A \land (!B \land !C) \Sequent (!A \land !B) \land !C$.
\item $!A \lor (!B \lor !C) \Sequent (!A \lor !B) \lor !C$.
\item $!A \lif (!B \lif !C) \Sequent !B \lif (!A \lif !C)$.
\item $!A \Sequent \lnot\lnot !A$.
\end{enumerate}
\end{prob}

\begin{prob}
Give !!{derivation}s of the following sequents:
\begin{enumerate}
\item $(!A \lor !B) \lif !C \Sequent !A \lif !C$.
\item $(!A \lif !C) \land (!B \lif !C) \Sequent (!A \lor !B) \lif !C$.
\item $\Sequent \lnot(!A \land \lnot !A)$.
\item $!B \lif !A \Sequent \lnot !A \lif \lnot !B$.
\item $\Sequent (!A \lif \lnot !A) \lif \lnot !A$.
\item $\Sequent \lnot(!A \lif !B) \lif \lnot !B$.
\item $!A \lif !C \Sequent \lnot (!A \land \lnot !C)$.
\item $!A \land \lnot !C \Sequent \lnot (!A \lif !C)$.
\item $!A \lor !B, \lnot !B \Sequent !A$.
\item $\lnot !A \lor \lnot !B \Sequent \lnot(!A \land !B)$.
\item $\Sequent (\lnot !A \land \lnot !B) \lif\lnot(!A \lor !B)$.
\item $\Sequent \lnot(!A \lor !B) \lif (\lnot !A \land \lnot !B)$.
\end{enumerate}
\end{prob}

\begin{prob}
Give !!{derivation}s of the following sequents:
\begin{enumerate}
\item $\lnot(!A \lif !B) \Sequent !A$.
\item $\lnot(!A \land !B) \Sequent \lnot !A \lor \lnot !B$.
\item $!A \lif !B \Sequent \lnot !A \lor !B$.
\item $\Sequent \lnot \lnot !A \lif !A$.
\item $!A \lif !B, \lnot !A \lif !B \Sequent !B$.
\item $(!A \land !B) \lif !C \Sequent (!A \lif !C) \lor (!B \lif !C)$.
\item $(!A \lif !B) \lif !A \Sequent !A$.
\item $\Sequent (!A \lif !B) \lor (!B \lif !C)$.
\end{enumerate}
(These all require the $\RightR{\Contraction}$~rule.)
\end{prob}
\end{document}
