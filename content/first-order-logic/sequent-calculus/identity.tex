% Part: first-order-logic
% Chapter: sequent-calculus
% Section: equality

\documentclass[../../../include/open-logic-section]{subfiles}

\begin{document}

\olfileid{fol}{seq}{ide}

\olsection{\usetoken{P}{derivation} with \usetoken{S}{identity}}

!!^{derivation}s with the !!{identity} require additional inference rules.

\paragraph{Initial sequents for $\eq$:}

If $t$ is a closed term, then ${} \Sequent \eq[t][t]$ is an initial sequent.

\paragraph{Rules for $\eq$:}

\[
\Axiom$ \eq[t_1][t_2], \Gamma \fCenter \Delta, !A(t_1) $
\RightLabel{$\eq$}
\UnaryInf$\eq[t_1][t_2], \Gamma \fCenter \Delta, !A(t_2)$
\DisplayProof
\quad
\textrm{  and  }
\quad
\Axiom$\eq[t_1][t_2], \Gamma \fCenter \Delta, !A(t_2) $
\RightLabel{$\eq$}
\UnaryInf$\eq[t_1][t_2], \Gamma  \fCenter \Delta, !A(t_1)$
\DisplayProof
\]
where $t_1$ and $t_2$ are closed terms.

\begin{ex}
If $s$ and $t$ are closed terms, then $\eq[s][t], !A(s)
\Proves !A(t)$:
\begin{prooftree}
\Axiom$ !A(s) \fCenter !A(s)$
\RightLabel{\LeftR{\Weakening}}
\UnaryInf$\eq[s][t], !A(s)  \fCenter !A(s)$
\RightLabel{$\eq$}
\UnaryInf$\eq[s][t], !A(s)  \fCenter !A(t)$
\end{prooftree}
This may be familiar as the principle of substitutability of
identicals, or Leibniz' Law.

$\Log{LK}$ proves that $\eq$ is symmetric and transitive:
\[
\Axiom$ \fCenter \eq[t_1][t_1] $
\RightLabel{\LeftR{\Weakening}}
\UnaryInf$ \eq[t_1][t_2] \fCenter \eq[t_1][t_1] $
\RightLabel{$\eq$}
\UnaryInf$ \eq[t_1][t_2] \fCenter \eq[t_2][t_1]$
\DisplayProof
\qquad
\Axiom$ \eq[t_1][t_2] \fCenter \eq[t_1][t_2] $
\RightLabel{\LeftR{\Weakening}}
\UnaryInf$\eq[t_2][t_3], \eq[t_1][t_2]  \fCenter \eq[t_1][t_2] $
\RightLabel{$\eq$}
\UnaryInf$\eq[t_2][t_3], \eq[t_1][t_2]  \fCenter \eq[t_1][t_3]$
\RightLabel{\LeftR{\Exchange}}
\UnaryInf$\eq[t_1][t_2], \eq[t_2][t_3]  \fCenter \eq[t_1][t_3]$
\DisplayProof
\]
In the proof on the left, the !!{formula}~$\eq[x][t_1]$ is our
$!A(x)$. On the right, we take $!A(x)$ to be~$\eq[t_1][x]$.
\end{ex}

\begin{prob}
Give !!{derivation}s of the following sequents:
\begin{enumerate}
\item $\Sequent \lforall[x][\lforall[y][((x = y \land !A(x)) \lif !A(y))]]$
\item $\lexists[x][!A(x)] \land \lforall[y][\lforall[z][((!A(y) \land
    !A(z)) \lif y = z)]] \Sequent 
\lexists[x][(!A(x) \land \lforall[y][(!A(y) \lif y = x)])]$
\end{enumerate}
\end{prob}

\begin{prop}
$\Log{LK}$ with initial sequents and rules for identity is sound.
\end{prop}

\begin{proof}
Initial sequents of the form ${} \Sequent \eq[t][t]$ are valid, since
for every !!{structure}~$\Struct M$, $\Sat{M}{\eq[t][t]}$. (Note that
we assume the term $t$ to be closed, i.e., it contains no variables,
so variable assignments are irrelevant).

Suppose the last inference in !!a{derivation} is $=$. Then the premise
is $\eq[t_1][t_2], \Gamma \Sequent \Delta, !A(t_1)$ and the conclusion
is $\eq[t_1][t_2], \Gamma \Sequent \Delta, !A(t_2)$. Consider
!!a{structure}~$\Struct M$. We need to show that the conclusion is
valid, i.e., if $\Sat{M}{\eq[t_1][t_2]}$ and $\Sat{M}{\Gamma}$, then
either $\Sat{M}{!C}$ for some $!C \in \Delta$ or $\Sat{M}{!A(t_2)}$.

By induction hypothesis, the premise is valid. This means that if
$\Sat{M}{\eq[t_1][t_2]}$ and $\Sat{M}{\Gamma}$ either (a) for some $!C
\in \Delta$, $\Sat{M}{!C}$ or (b) $\Sat{M}{!A(t_1)}$. In case (a) we
are done. Consider case (b).  Let $s$ be a variable assignment with
$s(x) = \Value{t_1}{M}$.  By \olref[syn][ass]{prop:sentence-sat-true},
$\Sat{M}{!A(t_1)}[s]$. Since $s \sim_x s$, by
\olref[syn][ext]{prop:ext-formulas}, $\Sat{M}{!A(x)}[s]$. since
$\Sat{M}{\eq[t_1][t_2]}$, we have $\Value{t_1}{M} = \Value{t_2}{M}$,
and hence $s(x) = \Value{t_2}{M}$.  By applying
\olref[syn][ext]{prop:ext-formulas} again, we also have
$\Sat{M}{!A(t_2)}[s]$.  By \olref[syn][ass]{prop:sentence-sat-true},
$\Sat{M}{!A(t_2)}$.
\end{proof}

\end{document}
