\documentclass[../../../include/open-logic-section]{subfiles}

\begin{document}
\olfileid{sth}{card-arithmetic}{expotough}

\olsection[Some Simplifications]{Some Simplification with Cardinal Exponentiation}

Whilst defining $\canonord$ was a little involved, the upshot is a
useful result concerning cardinal addition and multiplication,
\olref[simp]{cardplustimesmax}. Transfinite exponentiation, however,
cannot be simplified so straightforwardly. To explain why, I start
with a result which extends a familiar pattern from the finitary case
(though its proof  at quite a high level of abstraction):

\begin{prop}\ollabel{simplecardexpo}
$\cardexpo{\cardfont{a}}{\cardfont{b} \cardplus \cardfont{c}} =
\cardexpo{\cardfont{a}}{\cardfont{b}} \cardtimes
\cardexpo{\cardfont{a}}{\cardfont{c}}$ and
$\cardexpo{(\cardexpo{\cardfont{a}}{\cardfont{b}})}{\cardfont{c}} =
\cardexpo{\cardfont{a}}{\cardfont{b} \cardtimes \cardfont{c}}$, for
any cardinals $\cardfont{a}, \cardfont{b}, \cardfont{c}$.
\end{prop}

\begin{proof}
For the first claim, consider a function $f \colon
(\cardfont{b}\disjointsum\cardfont{c}) \to \cardfont{a}$. Now ``split
this'', by defining $f_\cardfont{b}(\beta) = f(\beta, 0)$ for each
$\beta \in \cardfont{b}$, and $f_\cardfont{c}(\gamma) = f(\gamma, 1)$
for each $\gamma \in \cardfont{c}$. The map $f \mapsto
(f_{\cardfont{b}} \times f_\cardfont{c})$ is !!a{bijection}
$\funfromto{\cardfont{b} \disjointsum \cardfont{c}}{\cardfont{a}} \to
(\funfromto{\cardfont{b}}{\cardfont{a}} \times
\funfromto{\cardfont{c}}{\cardfont{a}})$. 

For the second claim, consider a function $f \colon \cardfont{c} \to
(\funfromto{\cardfont{b}}{\cardfont{a}})$; so for each $\gamma \in
\cardfont{c}$ we have some function $f(\gamma) \colon \cardfont{b} \to
\cardfont{a}$. Now define $f^*(\beta, \gamma) = (f(\gamma))(\beta)$
for each $\tuple{\beta, \gamma} \in \cardfont{b} \times \cardfont{c}$.
The map $f \mapsto f^*$ is !!a{bijection}
$\funfromto{\cardfont{c}}{(\funfromto{\cardfont{b}}{\cardfont{a}})}
\to \funfromto{\cardfont{b} \cardtimes \cardfont{c}}{\cardfont{a}}$. 
\end{proof}

Now, what we would \emph{like} is an easy way to compute
$\cardexpo{\cardfont{a}}{\cardfont{b}}$ when we are dealing with
infinite cardinals. Here is a nice step in this direction:

\begin{prop}\ollabel{cardexpo2reduct}
If $2 \leq \cardfont{a} \leq \cardfont{b}$ and $\cardfont{b}$ is
infinite, then $\cardexpo{\cardfont{a}}{\cardfont{b}} =
\cardexpo{2}{\cardfont{b}}$
\end{prop}

\begin{proof}
\begin{align*}
	\cardexpo{2}{\cardfont{b}} &\leq 
	\cardexpo{\cardfont{a}}{\cardfont{b}}\text{, as $2 \leq \cardfont{a}$}\\
	&\leq \cardexpo{(2^\cardfont{a})}{\cardfont{b}}
	\text{, by \olref[opps]{lem:SizePowerset2Exp}}\\
	&= \cardexpo{2}{\cardfont{a} \cardtimes \cardfont{b}}
	\text{, by \olref{simplecardexpo}} \\
	&= \cardexpo{2}{\cardfont{b}}
	\text{, by \olref[simp]{cardplustimesmax}}
\end{align*}
\end{proof}

We should not really expect to be able to simplify this any further,
since $\cardfont{b} < \cardexpo{2}{\cardfont{b}}$ by
\olref[card-arithmetic][opps]{lem:SizePowerset2Exp}.
However, this does not tell us what to say about
$\cardexpo{\cardfont{a}}{\cardfont{b}}$ when $\cardfont{b} <
\cardfont{a}$. Of course, if $\cardfont{b}$ is \emph{finite}, we know
what to do.

\begin{prop}
If $\cardfont{a}$ is infinite and $n \in \omega$ then
$\cardexpo{\cardfont{a}}{n} = \cardfont{a}$
\end{prop}

\begin{proof}
$\cardexpo{\cardfont{a}}{n} = \cardfont{a} \cardtimes \cardfont{a}
\cardtimes \ldots \cardtimes \cardfont{a} = \cardfont{a}$, by $n-1$
applications of \olref[simp]{cardplustimesmax}.
\end{proof}

Additionally, in certain other cases, we can control the size of
$\cardexpo{\cardfont{a}}{\cardfont{b}}$:

\begin{prop}
If $2 \leq \cardfont{b} < \cardfont{a} \leq
\cardexpo{2}{\cardfont{b}}$ and $\cardfont{b}$ is infinite, then
$\cardexpo{\cardfont{a}}{\cardfont{b}} = \cardexpo{2}{\cardfont{b}}$
\end{prop}

\begin{proof}
$\cardexpo{2}{\cardfont{b}}\leq \cardexpo{\cardfont{a}}{\cardfont{b}}
\leq \cardexpo{(\cardexpo{2}{\cardfont{b}})}{\cardfont{b}} =
\cardexpo{2}{\cardfont{b}\cardtimes\cardfont{b}} =
\cardexpo{2}{\cardfont{b}}$, reasoning as in \olref{cardexpo2reduct}.
\end{proof}

But, beyond this point, things become rather more subtle.

\end{document}