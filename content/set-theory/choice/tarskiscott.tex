\documentclass[../../../include/open-logic-section]{subfiles}

\begin{document}

\olfileid{sth}{choice}{tarskiscott}
\olsection{The Tarski--Scott Trick}

In \olref[cardinals][cardsasords]{defcardinalasordinal}, we
defined cardinals as ordinals. To do this, we assumed the Axiom of
Well-Ordering. We did this, for no other reason than that it is the
``industry standard''.

Before we discuss any of the philosophical issues surrounding
Well-Ordering, then, it is important to be clear that we \emph{can}
depart from the industry standard, and develop a theory of cardinals
\emph{without} assuming Well-Ordering. We can still employ the
definitions of $\cardeq{A}{B}$, $\cardle{A}{B}$ and $\cardless{A}{B}$,
as they appeared in \olref[sfr][siz][]{chap}. We will just need a new
notion of \emph{cardinal}.

A na\"ive thought would be to attempt to define $A$'s cardinality thus:
\[
	\Setabs{x}{\cardeq{A}{x}}.
\]
You might want to compare this with Frege's definition of $\# x Fx$,
sketched at the very end of \olref[cardinals][hp]{sec}. And, for
reasons we gestured at there, this definition fails. Any singleton set
is equinumerous with $\{\emptyset\}$. But new singleton sets are
formed at every successor stage of the hierarchy (just consider the
singleton of the previous stage). So $\Setabs{x}{\cardeq{A}{x}}$ does
not exist, since it cannot have a rank.

To get around this problem, we use a trick due to Tarski and Scott:\footnote{A reminder: all formulas may have parameters (unless explicitly stated otherwise).}

\begin{defn}[Tarski--Scott]
For any formula $\phi(x)$, let
$[ x : \phi(x)] $ be the set of all $x$, of least possible rank, such
that $\phi(x)$ (or $\emptyset$, if there are no $\phi$s).
\end{defn}

We should check that this definition is legitimate. Working in $\ZF$,
\olref[spine][foundation]{zfentailsregularity} guarantees that
$\setrank{x}$ exists for every $x$. Now, if there are any entities
satisfying $\phi$, then we can let $\alpha$ be the least rank such
that  $(\exists x\subseteq V_\alpha)\phi(x)$, i.e., $(\forall \beta
\in \alpha)(\forall x \subseteq V_\beta)\lnot \phi(x)$. We can then
define $[x : \phi(x)]$ by Separation as $\Setabs{x \in
V_{\alpha+1}}{\phi(x)}$. 

Having justified the Tarski--Scott trick, we can now use it to define
a notion of cardinality:

\begin{defn}
The \textsc{ts}-cardinality of $A$ is $\text{tsc}(A) = [x :
\cardeq{A}{x}]$.
\end{defn}

The definition of a \textsc{ts}-cardinal does not use Well-Ordering.
But, even without that Axiom, we can show that
\emph{\textsc{ts}-cardinals} behave rather like \emph{cardinals} as
defined in \olref[cardinals][cardsasords]{defcardinalasordinal}.
For example, if we restate
\olref[cardinals][cardsasords]{lem:CardinalsBehaveRight} and
\olref[card-arithmetic][opps]{lem:SizePowerset2Exp} in terms of
\textsc{ts}-cardinals, the proofs go through just fine in $\ZF$,
without assuming Well-Ordering. 

Whilst we are on the topic, it is worth noting that we can also
develop a theory of ordinals using the Tarski--Scott trick. Where
$\tuple{A, <}$ is a well-ordering, let $\text{tso}(A, <) = [\tuple{X,
R} : \ordeq{\tuple{A, <}}{\tuple{X, R}}]$. For more on this treatment
of cardinals and ordinals, see \citet[chs.~9--12]{Potter2004}.

\end{document}