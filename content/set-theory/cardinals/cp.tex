\documentclass[../../../include/open-logic-section]{subfiles}

\begin{document}

\olfileid{sth}{cardinals}{cp}
\olsection{Cantor's Principle}

Cast your mind back to \olref[ordinals][vn]{sec}. We were discussing
well-ordered sets, and suggested that it would be nice to have objects
which go proxy for well-orders. With this is mind, we introduced
ordinals, and then showed in
\olref[ordinals][ordtype]{ordtypesworklikeyouwant} that these
behave as we would want them to, i.e.:
\[
	\ordtype{A, <} = \ordtype{B, \lessdot} 
	\text{ iff } \tuple{A, <} \isomorphic \tuple{B, \lessdot}.
\]
Cast your mind back even further, to \olref[sfr][siz][equ]{sec}.
There, working na\"ively, we introduced the notion of the ``size'' of
a set. Specifically, we said that two sets are equinumerous,
$\cardeq{A}{B}$, just in case there is !!a{bijection} $f \colon A \to
B$. This is an intrinsically {simpler} notion than that of a
well-ordering: we are only interested in !!{bijection}s, and not (as
with order-isomorphisms) whether the !!{bijection}s ``preserve any
structure''.

This all gives rise to an obvious thought. Just as we introduced
certain objects, \emph{ordinals}, to calibrate well-orders, we can
introduce certain objects, \emph{cardinals}, to calibrate size. That
is the aim of this chapter. 

Before we say what these cardinals will be, we should lay down a
principle which they ought to satisfy. Writing $\card{X}$ for the
cardinality of the set $X$, we would want them to obey:
\[
	\card{A} = \card{B} \text{ iff } \cardeq{A}{B}.
\]
We'll call this \emph{Cantor's} Principle, since Cantor was probably
the first to have it very clearly in mind. (We'll say more about its
relationship to \emph{Hume's} Principle in \olref[hp]{sec}.) So
our aim is to define $\card{X}$, for each $X$, in such a way that it
delivers Cantor's Principle.

\end{document}