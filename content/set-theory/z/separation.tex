\documentclass[../../../include/open-logic-section]{subfiles}

\begin{document}

\olfileid{sth}{z}{sep}
\olsection{Separation}

We start with a principle to replace Na\"{i}ve Comprehension:

\begin{axiom}[Scheme of Separation] For every formula $\phi(x)$, this is an
axiom: for any $A$, the set $\Setabs{x \in A}{\phi(x)}$ exists.
\end{axiom}

Note that this is not a single axiom. It is a \emph{scheme} of axioms.
There are \emph{infinitely many} Separation axioms; one for every
formula $\phi(x)$. The scheme can equally well be (and normally is)
written down as follows:

\begin{defish}
For any formula $\phi(x)$ which does not contain ``$S$'', this is an
axiom:
\[
	\forall A \exists S \forall x(x \in S \liff (\phi(x) \land x \in A)).
\]
\end{defish}

In keeping with the convention noted at the start of
\olref[sth][][]{part}, the formulas~$\phi$ in the Separation axioms
may have parameters.\footnote{For an explanation of what this means,
see the discussion immediately after
\olref[sfr][infinite][induction]{natinductionschema}.}

Separation is immediately justified by our cumulative-iterative
conception of sets we have been telling. To see why, let $A$ be a set.
So $A$ is formed by some stage~$S$ (by \stageshier). Since $A$ was
formed at stage~$S$, all of $A$'s members were formed before stage $S$
(by \stagesacc). Now in particular, consider all the sets which are
members of $A$ and which also satisfy $\phi$; clearly all of these
sets, too, were formed before stage~$S$. So they are formed into a set
$\Setabs{x \in A}{\phi(x)}$ at stage~$S$ too (by \stagesacc).

Unlike Na\"ive Comprehension, this avoid Russell's Paradox. For we
cannot simply assert the existence of the set $\Setabs{x}{x \notin
x}$. Rather, \emph{given} some set~$A$, we can assert the existence of
the set $R_A = \Setabs{x \in A}{x \notin x}$. But all this proves is
that $R_A \notin R_A$ and $R_A \notin A$, none of which is very
worrying.

However, Separation has an immediate and striking consequence:

\begin{thm}\ollabel{thm:NoUniversalSet}
There is no \emph{universal} set, i.e., $\Setabs{x}{x = x}$ does not exist.
\end{thm}

\begin{proof}
For reductio, suppose $V$ is a universal set. Then by Separation, $R =
\Setabs{x \in V}{x \notin x} = \Setabs{x}{x \notin x}$ exists,
contradicting Russell's Paradox.
\end{proof}

The absence of a universal set---indeed, the open-endedness of the
hierarchy of sets---is one of the most fundamental ideas behind the
cumulative-iterative conception. So it is worth seeing that,
intuitively, we could reach it via a different route. A universal set
must be !!a{element} of itself. But, on our cumulative-iterative
conception, every set appears (for the first time) in the hierarchy at
the first stage immediately after all of its !!{element}s. But this
entails that \emph{no} set is self-membered. For any self-membered set
would have to first occur immediately after the stage at which it
first occurred, which is absurd. (We will see in
\olref[sth][spine][rank]{defnsetrank} how to make this explanation
more rigorous, by using the notion of the ``rank'' of a set. However,
we will need to have a few more axioms in place to do this.)

Here are a few more consequences of Separation and Extensionality.

\begin{prop}\ollabel{prop:emptyexists}
If any set exists, then $\emptyset$ exists.
\end{prop}

\begin{proof}
If $A$ is a set, $\emptyset = \Setabs{x \in A}{x \neq x}$ exists by Separation.
\end{proof}

\begin{prop}
$A \setminus B$ exists for any sets $A$ and $B$
\end{prop}

\begin{proof}
$A \setminus B = \Setabs{x \in A}{x \notin B}$ exists by Separation.
\end{proof}

It also turns out that (almost) arbitrary intersections exist:

\begin{prop}\ollabel{prop:intersectionsexist}
If $A \neq \emptyset$, then $\bigcap A = \Setabs{x}{(\forall y \in A)x \in y}$ exists.
\end{prop}

\begin{proof}
Let $A \neq \emptyset$, so there is some $c \in A$. Then $\bigcap A =
\Setabs{x}{(\forall y \in A)x \in y} = \Setabs{x \in c}{(\forall y \in
A)x \in y}$, which exists by Separation.
\end{proof}

Note the condition that $A \neq \emptyset$, though; for $\bigcap
\emptyset$ would be the universal set, vacuously, contradicting
\olref{thm:NoUniversalSet}.

\end{document}
