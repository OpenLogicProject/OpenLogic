% Part: incompleteness
% Chapter: theories-computability
% Section: intepretability

\documentclass[../../../include/open-logic-section]{subfiles}

\begin{document}

\olfileid{inc}{tcp}{itp}

\olsection{Theories in which $\Th{Q}$ is Interpretable are Undecidable}

We can strengthen these results even more. Informally, an
interpretation of a language $\Lang{L_1}$ in another language
$\Lang{L_2}$ involves defining the universe, relation symbols, and
function symbols of $\Lang{L_1}$ with !!{formula}s in
$\Lang{L_2}$. Though we won't take the time to do this, one can make
this definition precise.

\begin{thm}
  Suppose $\Th{T}$ is a theory in a language in which one can
  interpret the language of arithmetic, in such a way that $\Th{T}$ is
  consistent with the interpretation of $\Th{Q}$. Then $\Th{T}$ is
  undecidable. If $\Th{T}$ proves the interpretation of the axioms of
  $\Th{Q}$, then no consistent extension of $\Th{T}$ is decidable.
\end{thm}

The proof is just a small modification of the proof of the last
theorem; one could use a counterexample to get a separation of $\Th{Q}$ and
$\Th{\bar Q}$. One can take $\Th{ZFC}$, Zermelo-Fraenkel set theory with the
axiom of choice, to be an axiomatic foundation that is powerful enough
to carry out a good deal of ordinary mathematics. In $\Th{ZFC}$ one
can define the natural numbers, and via this interpretation, the
axioms of $\Th{Q}$ are true. So we have

\begin{cor}
There is no decidable extension of $\Th{ZFC}$.
\end{cor}

\begin{cor}
There is no complete, consistent, computably !!{axiomatizable} extension of
$\Th{ZFC}$. 
\end{cor}

The language of $\Th{ZFC}$ has only a single binary relation,
$\in$. (In fact, you don't even need equality.) So we have

\begin{cor}
First-order logic for any language with a binary relation symbol is
undecidable.
\end{cor}

This result extends to any language with two unary function symbols,
since one can use these to simulate a binary relation symbol. The
results just cited are tight: it turns out that first-order logic for
a language with only \emph{unary} relation symbols and at most one
\emph{unary} function symbol is decidable.

One more bit of trivia. We know that the set of sentences in the
language $\Obj 0$, $'$, $+$, $\times$, $<$ true in the standard model
is undecidable. In fact, one can define $<$ in terms of the other
symbols, and then one can define $+$ in terms of $\times$ and $'$. So
the set of true sentences in the language $\Obj 0$, $'$, $\times$ is
undecidable. On the other hand, Presburger has shown that the set of
sentences in the language $\Obj 0$, $'$, $+$ true in the language of
arithmetic is decidable. The procedure is computationally infeasible,
however.

\end{document}
