% Part: incompleteness
% Chapter: theories-computability
% Section: consis-with-q

\documentclass[../../../include/open-logic-section]{subfiles}

\begin{document}

\olfileid{inc}{tcp}{con}

\olsection{Theories Consistent with $\Th{Q}$ are Undecidable}
The following theorem says that not only is $\Th{Q}$ undecidable, but, in
fact, any theory that does not disagree with $\Th{Q}$ is undecidable.

\begin{thm}
Let $\Th{T}$ be any theory in the language of arithmetic that is
consistent with $\Th{Q}$ (i.e., $\Th{T} \cup \Th{Q}$ is
consistent). Then $\Th{T}$ is undecidable.
\end{thm}

\begin{proof}
Remember that $\Th{Q}$ has a finite set of axioms, $!A_1$, \dots,
$!A_8$. We can even replace these by a single axiom, $!E = !A_1 \land
\dots \land !A_8$.

Suppose $\Th{T}$ is a decidable theory consistent with $\Th{Q}$. Let
\[
C = \Setabs{!A}{\Th{T} \Proves !E \lif !A}.
\]
We show that $C$ would be a computable separation of $\Th{Q}$ and
$\Th{\bar Q}$, a contradiction. First, if $!A$ is in $\Th{Q}$, then
$!A$ is provable from the axioms of $\Th{Q}$; by the deduction
theorem, there is a proof of $!E \lif !A$ in first-order logic. So
$!A$ is in $C$.

On the other hand, if $!A$ is in $\Th{\bar Q}$, then there is a proof
of $!E \lif \lnot !A$ in first-order logic. If $\Th{T}$ also proves
$!E \lif !A$, then $\Th{T}$ proves $\lnot !E$, in which case $\Th{T}
\cup \Th{Q}$ is inconsistent. But we are assuming $\Th{T} \cup \Th{Q}$
is consistent, so $\Th{T}$ does not prove $!E \lif !A$, and so $!A$
is not in $C$.

We've shown that if $!A$ is in $\Th{Q}$, then it is in $C$, and if $!A$
is in $Q'$, then it is in $\num C$. So $C$ is a computable separation,
which is the contradiction we were looking for.
\end{proof}

This theorem is very powerful. For example, it implies:

\begin{cor}
  First-order logic for the language of arithmetic (that is, the set
  $\Setabs{!A}{\text{$!A$ is provable in first-order logic}}$) is
  undecidable.
\end{cor}

\begin{proof}
First-order logic is the set of consequences of $\emptyset$,
which is consistent with $\Th{Q}$.
\end{proof}

\end{document}
