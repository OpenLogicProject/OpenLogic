% Part: incompleteness
% Chapter: representability-in-q
% Section: beta-function

\documentclass[../../../include/open-logic-section]{subfiles}

\begin{document}

\olfileid{inc}{req}{bet}
\olsection{The Beta Function Lemma}


In order to show that we can carry out primitive recursion if
addition, multiplication, and $\Char{=}$ are available, we need to
develop functions that handle sequences. (If we had exponentiation as
well, our task would be easier.) When we had primitive recursion, we
could define things like the ``$n$-th prime,'' and pick a fairly
straightforward coding. But here we do not have primitive
recursion---in fact we want to show that we can do primitive recursion
using minimization---so we need to be more clever.  

\begin{lem}
\ollabel{lem:beta}
There is a function $\beta(d,i)$ such that for every sequence $a_0$,
\dots,~$a_n$ there is a number~$d$, such that for every $i \le n$,
$\beta(d,i) = a_i$. Moreover, $\beta$ can be defined from the basic
functions using just composition and regular minimization.
\end{lem}

Think of $d$ as coding the sequence $\tuple{a_0, \dots, a_n}$, and
$\beta(d,i)$ returning the $i$-th element. (Note that this ``coding''
does \emph{not} use the power-of-primes coding we're already familiar
with!). The lemma is fairly minimal; it doesn't say we can concatenate
sequences or append elements, or even that we can \emph{compute}~$d$
from $a_0$, \dots,~$a_n$ using functions definable by composition and
regular minimization. All it says is that there is a ``decoding''
function such that every sequence is ``coded.''

The use of the notation $\beta$ is G\"odel's. To repeat, the hard part
of proving the lemma is defining a suitable~$\beta$ using the
seemingly restricted resources, i.e., using just composition and
minimization---however, we're allowed to use addition, multiplication,
and~$\Char{=}$. There are various ways to prove this lemma, but one of
the cleanest is still G\"odel's original method, which used a
number-theoretic fact called the Chinese Remainder theorem.

\begin{defn}
Two natural numbers $a$ and $b$ are \emph{relatively prime} iff their
greatest common divisor is~$1$; in other words, they have no other
divisors in common.
\end{defn}

\begin{defn}
Natural numbers $a$ and $b$ are \emph{congruent modulo~$c$}, $a \equiv b \mod c$, iff $c \mid (a-b)$, i.e., $a$ and $b$ have the
same remainder when divided by~$c$.
\end{defn}

Here is the \emph{Chinese Remainder theorem}:
\begin{thm}
Suppose $x_0$, \dots,~$x_n$ are (pairwise) relatively prime. Let
$y_0$, \dots,~$y_n$ be any numbers. Then there is a number $z$ such that
\begin{align*}
z & \equiv y_0 \mod x_0 \\
z & \equiv y_1 \mod x_1 \\
& \vdots  \\
z & \equiv y_n \mod x_n.
\end{align*}
\end{thm}

Here is how we will use the Chinese Remainder theorem: if $x_0$,
\dots,~$x_n$ are bigger than $y_0$, \dots,~$y_n$ respectively, then we
can take $z$ to code the sequence $\tuple{y_0, \dots,y_n}$. To
recover~$y_i$, we need only divide $z$ by~$x_i$ and take the
remainder. To use this coding, we will need to find suitable values
for $x_0$, \dots,~$x_n$.

A couple of observations will help us in this regard. Given
$y_0$, \dots,~$y_n$, let
\[
j = \max(n, y_0, \dots, y_n) + 1,
\]
and let
\begin{align*}
x_0 & = 1 + \fact{j} \\
x_1 & = 1 + 2 \cdot \fact{j} \\
x_2 & = 1 + 3 \cdot \fact{j} \\
& \vdots  \\
x_n & = 1 + (n+1) \cdot \fact{j}
\end{align*}
Then two things are true:
\begin{enumerate}
\item\ollabel{rel-prime} $x_0$, \dots,~$x_n$ are relatively prime.
\item\ollabel{less} For each $i$, $y_i < x_i$.
\end{enumerate}
To see that \olref{rel-prime} is true, note that if $p$ is a prime number and $p
\mid x_i$ and $p \mid x_k$, then $p \mid 1 + (i+1) \fact{j}$ and $p \mid
1 + (k+1) \fact{j}$. But then $p$ divides their difference,
\[
(1 + (i+1)\fact{j}) - (1+ (k+1)\fact{j}) = (i-k) \fact{j}.
\]
Since $p$ divides $1 + (i+1)\fact{j}$, it can't divide $\fact{j}$ as well
(otherwise, the first division would leave a remainder of~$1$). So $p$
divides $i-k$, since $p$ divides $(i-k)\fact{j}$. But $\left| i-k
\right|$ is at most~$n$, and we have chosen $j > n$, so this implies
that $p \mid \fact{j}$, again a contradiction. So there is no prime
number dividing both $x_i$ and $x_k$. Clause~\olref{less} is easy: we have $y_i <
j < \fact{j} < x_i$.

Now let us prove the $\beta$ function lemma. Remember that we can use
$0$, successor, plus, times, $\Char{=}$, projections, and any function
defined from them using composition and minimization applied to
regular functions. We can also use a relation if its characteristic
function is so definable. As before we can show that these relations
are closed under Boolean combinations and bounded quantification; for
example:
\begin{align*}
\fn{not}(x) & \defis \Char{=}(x,0)\\
\bmin{x \leq z}{R(x,y)} & \defis \umin{x}{(R(x,y) \lor x = z)}\\
\bexists{x \leq z}{R(x,y)} & \defiff R(\bmin{x \leq z}{R(x,y)}, y)
\end{align*}
We can then show that all of the following are also definable without
primitive recursion:
\begin{enumerate}
\item The pairing function, $J(x,y) = \frac{1}{2}[(x+y)(x+y+1)] + x$;
% maybe explain more what is going on here, a bit confusing.
\item the projection functions
\begin{align*}
K(z) & = \bmin{x \leq z}{\bexists{y \leq z}{z = J(x,y)}},\\
L(z) & = \bmin{y \leq z}{\bexists{x \leq z}{z = J(x,y)}};
\end{align*}
\item the less-than relation $x < y$;
\item the divisibility relation $x \mid y$;
% \item $x \tsub y$
% \item $\fn{Prime}(x)$
% \item Assuming $p$ is prime, the relation ``$x$ is a power of $p$'':
% \[
% \bforall{y \leq x}{(y \mid x \lif y = 1 \lor y = x)}.
% \]
\item the function $\fn{rem}(x,y)$ which returns the remainder when
  $y$ is divided by~$x$.
\end{enumerate}
Now define
\begin{align*}
\beta^*(d_0,d_1,i) & = \fn{rem}(1+(i+1) d_1,d_0) \text{ and}\\
\beta(d,i) & = \beta^*(K(d),L(d),i).
\end{align*}
This is the function we want. Given $a_0$, \dots,~$a_n$ as above, let
\[
j = \max(n,a_0,\dots,a_n)+1,
\]
and let $d_1 = \fact{j}$. By \olref{rel-prime} above, we know that
$1+d_1$, $1+2 d_1$, \dots, $1+(n+1) d_1$ are relatively prime, and
by~\olref{less} that all are greater than $a_0$, \dots,~$a_n$. By the
Chinese Remainder theorem there is a value~$d_0$ such that for
each~$i$,
\[
d_0 \equiv a_i \mod (1+(i+1)d_1)
\]
and so (because $d_1$ is greater than~$a_i$),
\[
a_i = \fn{rem}(1+(i+1)d_1,d_0).
\]
Let $d = J(d_0,d_1)$. Then for each $i \le n$, we have
\begin{align*}
\beta(d,i) & =  \beta^*(d_0,d_1,i) \\
& =  \fn{rem}(1+(i+1) d_1,d_0) \\
& =  a_i
\end{align*}
which is what we need. This completes the proof of the
$\beta$-function lemma.

\begin{prob}
  Show that the relations $x < y$, $x \mid y$, and the
  function~$\fn{rem}(x,y)$ can be defined without primitive recursion.
  You may use $0$, successor, plus, times, $\Char{=}$, projections,
  and bounded minimization and quantification.
\end{prob}

\end{document}
