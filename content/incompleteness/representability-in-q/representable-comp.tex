% Part: incompleteness
% Chapter: representability-in-q
% Section: representable-comp

\documentclass[../../../include/open-logic-section]{subfiles}

\begin{document}

\olfileid{inc}{req}{rpc}
\olsection{Functions Representable in $\Th{Q}$ are Computable}

\begin{lem}
Every function that is representable in $\Th{Q}$ is computable.
\end{lem}

\begin{proof}
All we need to know is that we can code terms, !!{formula}s, and
!!{derivation}s in such a way that the relation ``$d$ is
!!a{derivation} of~$!A$ from the axioms of~$\Th{Q}$'' is computable,
as well as the function which returns the result of substituting the
numeral corresponding to~$n$ for the variable $x$ in the
!!{formula}~$!A$. In terms of G\"odel numbers,
$\fn{SubNumeral}(\Gn{!A},n, \Gn{x})$,
returns~$\Gn{\Subst{!A}{\num{n}}{x}}$.

Assuming this, suppose the function~$f$ is represented
by $!A_f(x_0, \dots, x_{k}, y)$. Then the algorithm for computing~$f$
is as follows: on input $n_0$, \dots, $n_{k}$, search for a number $m$
and !!a{derivation} of the !!{formula}~$!A_f(\num{n_0}, \dots, \num{n_k},
\num{m})$; when you find one, output~$m$. Since $f$ is represented
by~$!A_f(x_0, \dots, x_k, y)$, such an $m$ exists, namely, $m = f(n_0,
\dots, n_k)$. Using sequences and minimization, we can write $f$ as
\[
f(n_0,\dots,n_{k}) = (\umin{s}{\text{``$(s)_0$ is !!a{derivation}
    of~$!A_f(\num{n_0}, \dots, \num{n_k}, \num{(s)_1})$ in
    $\Th{Q}$''}})_1.
\]
This completes the proof, modulo the (involved but routine) details of
coding and defining the function and relation above.
\end{proof}

\end{document}
