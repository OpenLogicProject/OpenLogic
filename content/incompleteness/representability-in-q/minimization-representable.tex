% Part: incompleteness
% Chapter: representability-in-q
% Section: minimization-representable

\documentclass[../../../include/open-logic-section]{subfiles}

\begin{document}

\olfileid{inc}{req}{min}
\olsection{Regular Minimization is Representable in $\Th{Q}$}

Let's consider unbounded search. Suppose $g(x, z)$ is regular and
representable in $\Th{Q}$, say by the !!{formula}~$!A_g(x, z, y)$. Let
$f$ be defined by $f(z) = \umin{x}{[g(x, z) = 0]}$. We would like to find
!!a{formula}~$!A_f(z, y)$ representing~$f$.  The value of~$f(z)$ is
that number~$x$ which (a) satisfies $g(x, z) = 0$ and (b) is the least
such, i.e., for any $w < x$, $g(w, z) \neq 0$.  So the following is a
natural choice:
\[
!A_f(z,y) \ident !A_g(y, z, \Obj 0) \land \lforall[w][(w < y \lif \lnot
  !A_g(w, z, \Obj 0))].
\]
In the general case, of course, we would have to replace $z$ with
$z_0$, \dots, $z_k$.

The proof, again, will involve some lemmas about things $\Th{Q}$ is
strong enough to prove.

\begin{lem}
\ollabel{lem:succ} For every !!{constant}~$a$ and every natural
number~$n$,
\[
\Th{Q} \Proves \eq[(a' + \num n)][(a + \num n)'].
\]
\end{lem}

\begin{proof}
The proof is, as usual, by induction on $n$. In the base case, $n =
0$, we need to show that $\Th{Q}$ proves $\eq[(a' + \Obj 0)][(a + \Obj
0)']$. But we have:
\begin{align}
  \Th{Q} & \Proves \eq[(a' + \Obj 0)][a'] \quad \text{by axiom $Q_4$}
  \ollabel{step1}\\
  \Th{Q} & \Proves  \eq[(a + \Obj 0)][a] \quad \text{by axiom $Q_4$}
  \ollabel{step2} \\
  \Th{Q} & \Proves \eq[(a + \Obj 0)'][a'] \quad \text{by \olref{step2}}
  \ollabel{step3} \\
  \Th{Q} & \Proves \eq[(a' + \Obj 0)][(a + \Obj 0)'] \quad
  \text{by \olref{step1} and \olref{step3}}\notag
\end{align}
In the induction step, we can assume that we have shown that $\Th{Q}
\Proves \eq[(a' + \num n)][(a + \num n)']$. Since
$\num{n+1}$ is $\num{n}'$, we need to show that $\Th{Q}$ proves
$\eq[(a' + \num{n}')][(a + \num{n}')']$. We have:
\begin{align}
  \Th{Q} & \Proves \eq[(a' + \num n')][(a' + \num n)'] \quad
  \text{by axiom $!Q_5$} \ollabel{step5}\\
  \Th{Q} & \Proves \eq[(a' + \num n')][(a + \num n')'] \quad
  \text{inductive hypothesis} \ollabel{step6}\\
  \Th{Q} & \Proves \eq[(a' + \num n)'][(a + \num n')'] \quad
  \text{by \olref{step5} and \olref{step6}.} \notag
\end{align}
\end{proof}

It is again worth mentioning that this is weaker than saying that
$\Th{Q}$ proves $\lforall[x][\lforall[y][(x' + y) = (x + y)']]$.
Although this !!{sentence} is true in~$\Struct{N}$, $\Th{Q}$ does not
prove it.

\begin{lem}
\ollabel{lem:less-zero}
$\Th{Q} \Proves \lforall[x][\lnot x < \Obj 0]$.
\end{lem}

\begin{proof}
We give the proof informally (i.e., only giving hints as to
how to construct the formal derivation).

We have to prove $\lnot a < \Obj 0$ for an arbitrary~$a$. By the
definition of $<$, we need to prove $\lnot
\lexists[y][\eq[(y'+a)][\Obj 0]]$ in $\Th{Q}$. We'll assume
$\lexists[y][\eq[(y'+a)][\Obj 0]]$ and prove a contradiction. Suppose
$\eq[(b'+a)][\Obj 0]$. Using $!Q_3$, we
have that $\eq[a][\Obj 0] \lor \lexists[y][\eq[a][y']]$. We
distinguish cases. 

Case 1: $\eq[a][\Obj 0]$ holds. From $\eq[(b'+a)][\Obj 0]$, we have
$\eq[(b' + \Obj 0)][\Obj 0]$. By axiom~$!Q_4$ of $\Th{Q}$, we have
$\eq[(b' + \Obj 0)][b']$, and hence $\eq[b'][\Obj 0]$. But by
axiom~$!Q_2$ we also have $\eq/[b'][\Obj 0]$, a contradiction. 

Case 2: For some $c$, $\eq[a][c']$. But then we have $\eq[(b' +
c')][\Obj 0]$. By axiom~$!Q_5$, we have $\eq[(b' + c)'][\Obj 0]$, again
contradicting axiom~$Q_2$.
\end{proof}

\begin{lem}
\ollabel{lem:less-nsucc}
For every natural number~$n$, 
  \[
  \Th{Q} \Proves
  \lforall[x][(x < \num {n+1} \lif (\eq[x][\Obj 0] \lor \dots \lor
    \eq[x][\num n]))].
  \]
\end{lem}

\begin{proof}
We use induction on~$n$. Let us consider the base case, when $n = 0$.
In that case, we need to show $a < \num 1 \lif \eq[a][\Obj 0]$, for
arbitrary~$a$. Suppose $a < \num 1$. Then by the defining axiom for
$<$, we have $\lexists[y][\eq[(y'+a)][\Obj 0']]$ (since $\num 1 \ident
\Obj 0'$).

Suppose $b$ has that property, i.e., we have $\eq[(b'+a)][\Obj 0']$.
We need to show $\eq[a][\Obj 0]$. By axiom~$!Q_3$, we have either
$\eq[a][\Obj 0]$ or that there is a $c$ such that $\eq[a][c']$. In the
former case, there is nothing to show. So suppose $\eq[a][c']$. Then
we have $\eq[(b' + c')][\Obj 0']$. By axiom~$!Q_5$ of $\Th{Q}$, we have
$\eq[(b'+c)'][\Obj 0']$. By axiom~$!Q_1$, we have $\eq[(b' + c)][\Obj
0]$. But this means, by axiom~$!Q_8$, that $c < \Obj 0$, contradicting
\olref{lem:less-zero}.

Now for the inductive step. We prove the case for $n+1$, assuming the
case for~$n$. So suppose $a < \num {n+2}$. Again using $!Q_3$ we can
distinguish two cases: $\eq[a][\Obj 0]$ and for some $b$,
$\eq[a][c']$. In the first case, $\eq[a][\Obj 0] \lor \dots \lor
\eq[a][\num{n+1}]$ follows trivially. In the second case, we have $c'
< \num {n+2}$, i.e., $c' < \num{n+1}'$. By axiom~$!Q_8$, for some $d$,
$\eq[(d'+c')][\num{n+1}']$. By axiom $!Q_5$,
$\eq[(d'+c)'][\num{n+1}']$. By axiom~$!Q_1$, $\eq[(d'+c)][\num{n+1}]$,
and so $c < \num{n+1}$ by axiom~$!Q_8$. By inductive hypothesis,
$\eq[c][\Obj 0] \lor \dots \lor \eq[c][\num{n}]$. From this, we get
$\eq[c'][\Obj 0'] \lor \dots \lor \eq[c'][\num{n}']$ by logic, and so
$\eq[a][\num{1}] \lor \dots \lor \eq[a][\num{n+1}]$ since
$\eq[a][c']$.
\end{proof}

\begin{lem}
  \ollabel{lem:trichotomy} For every natural number~$m$,
  \[
  \Th{Q} \Proves
  \lforall[y][((y < \num{m} \lor \num{m} < y) \lor \eq[y][\num{m}])].
  \]
\end{lem}

\begin{proof}
By induction on~$m$. First, consider the case $m=0$. $\Th{Q} \Proves
\lforall[y][(\eq[y][\Obj 0] \lor \lexists[z][\eq[y][z']])]$ by~$!Q_3$.
Let $a$ be arbitrary. Then either $\eq[a][\Obj 0]$ or for some~$b$,
$\eq[a][b']$. In the former case, we also have $(a < \Obj 0 \lor \Obj
0 < a) \lor \eq[a][\Obj 0]$. But if $\eq[a][b']$, then $\eq[(b' +
\Obj 0)][(a + \Obj 0)]$ by the logic of~$\eq$. By $!Q_4$, $\eq[(a +
\Obj 0)][a]$, so we have $\eq[(b' + \Obj 0)][a]$, and hence
$\lexists[z][\eq[(z' + \Obj 0)][a]]$. By the definition of $<$ in
$!Q_8$, $\Obj 0 < a$.  If $\Obj 0 < a$, then also $(\Obj 0 < a \lor a
< \Obj 0) \lor \eq[a][\Obj 0]$. 

Now suppose we have
\begin{align*}
  \Th{Q} & \Proves \lforall[y][((y < \num{m} \lor \num{m} < y) \lor
    \eq[y][\num{m}])]
  \intertext{and we want to show}
  \Th{Q} & \Proves \lforall[y][((y < \num{m+1} \lor \num{m+1} < y) \lor
    \eq[y][\num{m+1}])]
\end{align*}
Let $a$ be arbitrary. By $!Q_3$, either $\eq[a][\Obj 0]$ or for
some~$b$, $\eq[a][b']$. In the first case, we have $\eq[\num{m}' +
a][\num{m+1}]$ by $!Q_4$, and so $a < \num{m+1}$ by $!Q_8$.

Now consider the second case, $\eq[a][b']$. By the induction
hypothesis, $(b < \num{m} \lor \num{m} < b) \lor
    \eq[b][\num{m}]$.

The first disjunct $b < \num{m}$ is equivalent (by $!Q_8$) to
$\lexists[z][\eq[(z' + b)][\num{m}]]$. Suppose $c$ has this property.
If $\eq[(c' + b)][\num{m}]$, then also $\eq[(c' + b)'][\num{m}']$. By
$!Q_5$, $\eq[(c' + b)'][(c' + b')]$. Hence, $\eq[(c' +
b')][\num{m}']$. We get $\lexists[u][\eq[(u' + b')][\num{m+1}]]$ by
existentially generalizing on~$c'$ and keeping in mind that $\num{m}'
\ident \num{m+1}$. Hence, if $b < \num{m}$ then $b' < \num{m+1}$ and
so $a < \num{m+1}$.

Now suppose $\num{m} < b$, i.e., $\lexists[z][\eq[(z' +
\num{m})][b]]$.  Suppose $c$ is such a~$z$, i.e., $\eq[(c' +
\num{m})][b]$. By logic, $\eq[(c' + \num{m})'][b']$. By $!Q_5$,
$\eq[(c' + \num{m}')][b']$. Since $\eq[a][b']$ and $\num{m}' \ident
\num{m+1}$, $\eq[(c' + \num{m+1})][a]$. By $!Q_8$, $\num{m+1} < a$.

Finally, assume $\eq[b][\num{m}]$. Then, by logic,
$\eq[b'][\num{m}']$, and so $\eq[a][\num{m+1}]$.

Hence, from each disjunct of the case for~$m$ and~$b$, we can obtain
the corresponding disjunct for for~$m+1$ and~$a$.
\end{proof}
  
\begin{prop}
\ollabel{prop:rep-minimization}
If $!A_g(x, z, y)$ represents $g(x, z)$ in~$\Th{Q}$, then
\[
!A_f(z,y) \ident !A_g(y, z, \Obj 0) \land \lforall[w][(w < y \lif \lnot
  !A_g(w, z, \Obj 0))]
\]
represents $f(z) = \umin{x}{[g(x, z) = 0]}$.
\end{prop}

\begin{proof}
First we show that if $f(n) = m$, then $\Th{Q} \Proves !A_f(\num n, \num m)$,
i.e.,  
\begin{align}
  \Th{Q} & \Proves !A_g(\num{m}, \num{n}, \Obj 0) \land \lforall[w][(w <
    \num{m} \lif \lnot !A_g(w, \num{n}, \Obj 0))]. \notag
  \intertext{Since $!A_g(x, z, y)$ represents $g(x, z)$ and $g(m, n) =
    0$ if $f(n) = m$, we have}
\Th{Q} & \Proves !A_g(\num{m}, \num{n}, \Obj 0). \notag
\intertext{If $f(n) = m$, then for every $k < m$, $g(k, n) \neq 0$. So}
\Th{Q} & \Proves \lnot !A_g(\num{k}, \num{n}, \Obj 0). \notag
\intertext{We get that}
\Th{Q} & \Proves \lforall[w][(w < \num{m} \lif \lnot
  !A_g(w, \num{n}, \Obj 0))]. \ollabel{rep-less}
\end{align}
by \olref{lem:less-zero} in case $m = 0$ and by \olref{lem:less-nsucc} otherwise.

Now let's show that if $f(n) = m$, then $\Th{Q} \Proves
\lforall[y][(!A_f(\num{n}, y) \lif \eq[y][\num{m}])]$.  We again
sketch the argument informally, leaving the formalization to the
reader.

Suppose $!A_f(\num{n}, b)$. From this we get (a) $!A_g(b, \num{n},
\Obj 0)$ and (b) $\lforall[w][(w < b \lif \lnot !A_g(w, \num{n}, \Obj
  0))]$.  By \olref{lem:trichotomy}, $(b < \num{m} \lor \num{m} < b)
\lor \eq[b][\num{m}]$. We'll show that both $b < \num{m}$ and $\num{m}
< b$ leads to a contradiction.

If $\num{m} < b$, then $\lnot !A_g(\num{m}, \num{n}, \Obj 0)$
from~(b). But $m = f(n)$, so $g(m, n) = 0$, and so $\Th{Q} \Proves
!A_g(\num{m}, \num{n}, \Obj 0)$ since $!A_g$ represents~$g$. So we
have a contradiction.

Now suppose $b < \num{m}$. Then since $\Th{Q} \Proves \lforall[w][(w <
  \num{m} \lif \lnot !A_g(w, \num{n}, \Obj 0))]$ by \olref{rep-less}, we
get $\lnot !A_g(b, \num{n}, \Obj 0)$. This again contradicts~(a).
\end{proof}

\end{document}
