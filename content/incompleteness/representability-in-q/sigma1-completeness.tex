% Part: incompleteness
% Chapter: representability-in-q
% Section: sigma1-completeness

\documentclass[../../../include/open-logic-section]{subfiles}

\begin{document}

\olfileid{inc}{inp}{s1c}
\olsection{\texorpdfstring{$\Sigma_1$}{Sigma-1} completeness}

Despite the incompleteness of $\Th{Q}$ and its consistent, axiomatizable
extensions, we have seen that $\Th{Q}$ does prove many basic facts about
numerals. In fact, this can be extended quite considerably. To understand
the scope of what can be proved in $\Th{Q}$, we introduce the notions of
$\Delta_0$, $\Sigma_1$, and $\Pi_1$ !!{formula}s. Roughly speaking, a
$\Sigma_1$ !!{formula} is one of the form $\lexists{x}!A(x)$, where $!A$
is constructed using only propositional connectives and bounded
quantifiers. We shall show that if $!A$ is a $\Sigma_1$ !!{sentence}
which is true in $\Struct{N}$, then $\Th{Q} \Proves !A$
(\olref{thm:sigma1-completeness}).

\begin{defn}
\ollabel{defn:bd-quant}
A \emph{bounded existential !!{formula}} is one of the form
$\lexists[x][(x < t \land !A(x))]$ where $t$ is any term, which we
conventionally write as $\bexists{x < t}{!A(x)}$.
%
A \emph{bounded universal !!{formula}} is one of the form
$\lforall[x][(x < t \lif !A(x))]$ where $t$ is any term, which we
conventionally write as $\bforall{x < t}{!A(x)}$.
\end{defn}

\begin{defn}
\ollabel{defn:delta0-sigma1-pi1-frm}
A !!{formula} $!B$ is $\Delta_0$ if it is built up from atomic
!!{formula}s using only propositional connectives and bounded
quantification.
%
A !!{formula} $!A$ is $\Sigma_1$ if $!A \ident \lexists[x][!B(x)]$
where $!B$ is $\Delta_0$.
%
A !!{formula} $!A$ is $\Pi_1$ if $!A \ident \lforall[x][!B(x)]$
where $!B$ is $\Delta_0$.
\end{defn}

\begin{lem}
\ollabel{lem:q-proves-clterm-id} Suppose $t$ is a closed term such that
$\Assign{t}{N} = n$. Then $\Th{Q} \Proves \eq[t][\num n]$.
\end{lem}

\begin{proof}
We prove this by induction on the complexity of $t$. For the base case,
${\Obj 0}^\Struct{N} = 0$, and $\Th{Q} \Proves \eq[\Obj 0][\num 0]$
since $\num 0 \ident \Obj 0$.
%
For the inductive case, let $t_1$ and $t_2$ be terms such that
$t_1^\Struct{N} = n_1$, $t_2^\Struct{N} = n_2$,
$\Th{Q} \Proves \eq[t_1][\num n_1]$, and
$\Th{Q} \Proves \eq[t_2][\num n_2]$.

Then $(t_1')^\Struct{N} = n_1 + 1$, and we have that $\Th{Q} \Proves
\eq[t_1'][{\num n_1}']$ by the first-order rules for identity applied
to the induction hypothesis and the !!{formula}
$\eq[\num{n_1}'][\num{n_1}']$,
so we have $\Th{Q} \Proves \eq[t_1'][\num{n_1 + 1}]$
by the definition of numerals.

For sums we have
$$
      (t_1 + t_2)^\mathfrak{N}
    = t_1^\mathfrak{N} + t_2^\mathfrak{N}
    = n_1 + n_2.
$$
By the induction hypothesis and the rules for identity,
$\Th{Q} \Proves \eq[t_1 + t_2][\num n_1 + t_2]$, and then
$\Th{Q} \Proves \eq[t_1 + t_2][\num n_1 + \num n_2]$
by a second application of the rules for identity.
By \olref[inc][req][bre]{lem:q-proves-add},
$\Th{Q} \Proves \eq[\num n_1 + \num n_2][\num{n_1 + n_2}]$,
so $\Th{Q} \Proves \eq[t_1 + t_2][\num{n_1 + n_2}]$.

Similar reasoning also works for $\times$, using
\olref[inc][req][bre]{lem:q-proves-mult}.
%
Since this exhausts the closed terms of arithmetic, we have that
$\Th{Q} \Proves \eq[t][\num n]$ for all closed terms $t$ such that
$t^\Struct{N} = n$.
\end{proof}

\begin{prob}
Prove in detail the part of \olref{lem:q-proves-clterm-id}
involving $\times$.
\end{prob}

\begin{lem}
\ollabel{lem:atomic-completeness}
Suppose $t_1$ and $t_2$ are closed terms. Then
\begin{enumerate}
\item If $t_1^\Struct{N} = t_2^\Struct{N}$,
    then $\Th{Q} \Proves \eq[t_1][t_2]$.
\item If $t_1^\Struct{N} \neq t_2^\Struct{N}$,
    then $\Th{Q} \Proves \eq/[t_1][t_2]$.
\item If $t_1^\Struct{N} < t_2^\Struct{N}$,
    then $\Th{Q} \Proves t_1 < t_2$.
\item If $t_2^\Struct{N} \leq t_1^\Struct{N}$,
    then $\Th{Q} \Proves \lnot(t_1 < t_2)$.
\end{enumerate}
\end{lem}

\begin{proof}
Given terms $t_1$ and $t_2$, we fix $n = t_1^\mathfrak{N}$ and
$m = t_2^\mathfrak{N}$.

Suppose $!A \ident t_1 = t_2$. By \olref{lem:q-proves-clterm-id},
$\Th{Q} \Proves \eq[t_1][\num n]$ and $\Th{Q} \Proves \eq[t_2][\num n]$.
If $n = m$, then $\Th{Q} \Proves \eq[\num n][\num m]$ and hence
$\Th{Q} \Proves \eq[t_1][t_2]$ by the transitivity of identity.
If $n \neq m$ then $\Th{Q} \Proves \eq/[\num n][\num m]$,
and by the transitivity of identity again,
$\Th{Q} \Proves \eq/[t_1][t_2]$.

Now let $!A \ident t_1 < t_2$. For both cases, we rely on axiom $!Q_8$,
which states that $x < y \leftrightarrow \lexists[z][\eq[z' + x][y]]$
for all $x,y$.

Suppose $\Sat{N}{t_1 < t_2}$. Then there exists some $k \in \Nat$
such that $n + k + 1 = m$. By \olref{lem:q-proves-clterm-id},
$\Th{Q} \Proves \eq[t_1][\num n]$ and $\Th{Q} \Proves \eq[t_2][\num m]$,
and by the first part of this lemma,
$\Th{Q} \Proves \eq[\num n + {\num k}'][\num m]$.
By the transitivity of identity it follows that
$\Th{Q} \Proves \eq[{\num k}' + t_1][t_2]$,
so $\Th{Q} \Proves \lexists[z][\eq[z' + t_1][t_2]]$.
By the right-to-left direction of $!Q_8$, $\Th{Q} \Proves t_1 < t_2$.

Suppose instead that $\Sat/{N}{t_1 < t_2}$, i.e.\ $m \leq n$.
%
We work in $\Th{Q}$ and assume that $t_1 < t_2$. By the left-to-right
direction of $!Q_8$, there is some $z$ such that $\eq[z' + t_1][t_2]$.
Since $\Th{Q} \Proves \eq[t_1][\num n]$ and
$\Th{Q} \Proves \eq[t_2][\num m]$, $\eq[z' + \num n][\num m]$.
%
By an external induction on $m$ using $!Q_5$,
$\eq[z' + \num{n - m}][\Obj 0]$.
If $m = n$ then $\eq/[z'][\Obj 0]$, giving a contradiction via $!Q_3$.
If $m < n$ then $\eq[(z' + \num{n - m - 1})'][\Obj 0]$ by $!Q_5$ again,
giving a contradiction via $!Q_3$.
So $\Th{Q} \Proves \lnot(t_1 < t_2)$.
\end{proof}

\begin{lem}
\ollabel{lem:bounded-quant-equiv}
Suppose $!A$ is a !!{formula}. Then
\begin{enumerate}
\item $\Th{Q} \Proves \bforall{x<t}{!A(x)}$ iff $\Th{Q} \Proves
    !A(\num 0) \land \dotsc \land !A(\num{t^\Struct{N}-1})$.
\item $\Th{Q} \Proves \bexists{x<t}{!A(x)}$ iff $\Th{Q} \Proves
    !A(\num 0) \lor \dotsc \lor !A(\num{t^\Struct{N}-1})$.
\end{enumerate}
\end{lem}

\begin{proof}
    We prove the case for the bounded universal quantifier.
    If $t^\Struct{N} = 0$ then the left-hand side of the
    equivalence is provable in $\Th{Q}$, because there is no
    $x<\num 0$ by \olref[inc][req][min]{lem:less-zero}.
    Similarly, we can take an empty disjunction to be simply
    $\ltrue$, which is also provable in $\Th{Q}$.
    %
    We therefore suppose that $t^\Struct{N} = k+1$ for some
    natural number $k$. By \olref{lem:q-proves-clterm-id} we
    can assume that we are working with a !!{formula} of the
    form $\bforall{x<\num{k+1}}{!A(x)}$.
    
    Suppose that $\Th{Q} \Proves \bforall{x<\num{k+1}}{!A(x)}$,
    and let $n \leq k$. Since $\Th{Q} \Proves \num n < \num{k+1}$
    by \olref{lem:atomic-completeness}, it follows by logic that
    $\Th{Q} \Proves !A(\num n)$. Applying this fact $k+1$ times
    for each $n \leq k$, we get that $\Th{Q} \Proves !A(\num 0)
    \land \dotsc \land !A(\num k)$ as desired.
    
    For the other direction, suppose that $\Th{Q} \Proves
    !A(\num 0) \land \dotsc \land !A(\num k)$. Working in
    $\Th{Q}$, suppose that $x < \num{k+1}$.
    By \olref[inc][req][min]{lem:less-nsucc} we have that
    $x = \num 0 \lor \dotsc \lor x = \num k$, so by logic it
    follows that $!A(x)$, and hence the universal claim
    $\bforall{x<\num{k+1}}!A(x)$ follows.
    
    The proof of the equivalence for bounded existentially
    quantified !!{formula}s is similar.
\end{proof}

\begin{prob}
Give a detailed proof of the existential case in
\olref{lem:bounded-quant-equiv}.
\end{prob}

\begin{lem}
\ollabel{lem:delta0-completeness}
If $!A$ is a $\Delta_0$ !!{sentence} which is true in
$\Struct{N}$, then $\Th{Q} \Proves !A$.
\end{lem}

\begin{proof}
We prove this by induction on !!{formula} complexity.
%
The base case is given by \olref{lem:atomic-completeness},
so we move to the induction step. For simplicity we split
the case of negation into subcases depending on the
structure of the !!{formula} to which the negation is
applied.

\begin{enumerate}
\item Suppose $(!A \land !B)$ is true in $\Struct{N}$,
so $!A$ and $!B$ are true in $\Struct{N}$.
By the induction hypothesis, $\Th{Q} \Proves !A$ and
$\Th{Q} \Proves !B$,
so $\Th{Q} \Proves (!A \land !B)$ by logic.
%
\item Suppose $\lnot (!A \land !B)$ is true in $\Struct{N}$,
so either $\lnot !A$ or $\lnot !B$ is true in $\Struct{N}$.
Without loss of generality, suppose the former. By the
induction hypothesis $\Th{Q} \Proves \lnot !A$, and hence
$\Th{Q} \Proves \lnot (!A \land !B)$ by logic.
%
\item Suppose $(!A \lor !B)$ is true in $\Struct{N}$, so
either $!A$ is true in $\Struct{N}$ or $!B$ is true in
$\Struct{N}$. Without loss of generality, suppose the former
holds. By the induction hypothesis $\Th{Q} \Proves !A$, and
hence $\Th{Q} \Proves (!A \lor !B)$ by logic.
%
\item Suppose $\lnot(!A \lor !B)$ is true in $\Struct{N}$,
so $\lnot !A$ and $\lnot !B$ are true in $\Struct{N}$.
Then $\Th{Q} \Proves \lnot !A$ and $\Th{Q} \Proves \lnot !B$
by the induction hypothesis. Consequently,
$\Th{Q} \Proves \lnot(!A \lor !B)$ by logic.
%
\item Suppose that $\bforall{x<t}!A(x)$ is true in
$\Struct{N}$, where $t$ is a closed term. By the induction
hypothesis and logic, if $!A(\num n)$ is true in $\Struct{N}$
for all $n < t^\Struct{N}$ then $\Th{Q} \Proves
!A(\num 0) \land \dots \land !A(\num{t^\Struct{N}-1})$.
By \olref{lem:bounded-quant-equiv} it follows that
$\Th{Q} \Proves \bforall{x<t}!A(x)$.
%
\item The case for the bounded existential quantifier, where
we have a !!{sentence} of the form $\bexists{x < t}!A(x)$,
is similar to that for the bounded universal quantifier.
%
\item Suppose that $\lnot \bforall{x<t}!A(x)$ is true in
$\Struct{N}$, where $t$ is a closed term. This !!{sentence}
is equivalent to the !!{sentence} $\bexists{x<t}\lnot!A(x)$,
with the equivalence derivable in $\Th{Q}$, so we may apply
the reasoning for bounded existential quantifiers.
%
\item Similarly, suppose that $\lnot \bexists{x<t}!A(x)$ is
true in $\Struct{N}$, where $t$ is a closed term. This
!!{sentence} is equivalent in $\Th{Q}$ to
$\bforall{x<t}\lnot!A(x)$, and so we may apply the reasoning
for bounded universal quantifiers.
%
\item Finally, suppose $\lnot !A$ is true in $\Struct{N}$.
The only cases remaining are when $!A$ is atomic and when
$\lnot !A \ident \lnot\lnot !B$ for some $\Delta_0$
!!{sentence} $!B$. If $!A$ is atomic then by
\olref{lem:atomic-completeness}, $\Th{Q} \Proves \lnot !A$.
If $\lnot !A \ident \lnot\lnot !B$, then by logic it is
provably equivalent in $\Th{Q}$ to $!B$, which is true in
$\Struct{N}$ since $\lnot !A$ is true in $\Struct{N}$.
By the induction hypothesis we therefore have that
$\Th{Q} \Proves \lnot !A$.
\end{enumerate}
\end{proof}

\begin{prob}
Give a detailed proof of the existential case in
\olref{lem:delta0-completeness}.
\end{prob}

\begin{thm}
\ollabel{thm:sigma1-completeness}
If $!A$ is a $\Sigma_1$ !!{sentence} which is true in
$\Struct{N}$, then $\Th{Q} \Proves !A$.
\end{thm}

\begin{proof}
If $\lexists{x}!A(x)$ is a $\Sigma_1$ !!{sentence} which
is true in $\Struct{N}$, then there exists a natural number
$n$ and a variable assignment $s$ such that $s(x) = n$ and
$\Struct{N},s \Entails !A(x)$. By standard facts about
the satisfaction relation it follows that
$\Struct{N} \Entails !A(\num n)$. But $!A(\num n)$ is a
$\Delta_0$ !!{formula}, so by \olref{lem:delta0-completeness}
we have that $\Th{Q} \Proves !A(\num n)$, and hence by
logic we also have that $\Th{Q} \Proves \exists{x}!A(x)$.
\end{proof}

\end{document}
