% Part: incompleteness
% Chapter: representability-in-q
% Section: composition-representable

\documentclass[../../../include/open-logic-section]{subfiles}

\begin{document}

\olfileid{inc}{req}{cmp}
\olsection{Composition is Representable in $\Th{Q}$}

Suppose $h$ is defined by
\[
h(x_0,\dots,x_{l-1}) = f(g_0(x_0,\dots,x_{l-1}), \dots,
g_{k-1}(x_0,\dots,x_{l-1})).
\]
where we have already found !!{formula}s $!A_f, !A_{g_0}, \dots,
!A_{g_{k-1}}$ representing the functions $f$, and $g_0$,
\dots,~$g_{k-1}$, respectively. We have to find !!a{formula}~$!A_h$
representing $h$.

Let's start with a simple case, where all functions are $1$-place,
i.e., consider $h(x) = f(g(x))$. If $!A_f(y, z)$ represents~$f$, and
$!A_g(x, y)$ represents~$g$, we need !!a{formula}~$!A_h(x, z)$ that
represents~$h$.  Note that $h(x) = z$ iff there is a~$y$ such that
both $z = f(y)$ and $y = g(x)$. (If $h(x) = z$, then $g(x)$ is such
a~$y$; if such a $y$ exists, then since $y = g(x)$ and $z = f(y)$, $z
= f(g(x))$.) This suggests that $\lexists[y][(!A_g(x, y) \land !A_f(y,
  z))]$ is a good candidate for~$!A_h(x, z)$. We just have to verify
that $\Th{Q}$ proves the relevant !!{formula}s.

\begin{prop}
\ollabel{prop:rep1}
If $h(n) = m$, then $\Th{Q} \Proves !A_h(\num{n}, \num{m})$.
\end{prop}

\begin{proof}
Suppose $h(n) = m$, i.e., $f(g(n)) = m$.  Let $k = g(n)$. Then
\begin{align*}
  \Th{Q} & \Proves !A_g(\num{n}, \num{k})
  \intertext{since $!A_g$ represents~$g$, and}
  \Th{Q} & \Proves !A_f(\num{k}, \num{m})
  \intertext{since $!A_f$ represents~$f$. Thus,}
  \Th{Q} & \Proves !A_g(\num{n}, \num{k}) \land !A_f(\num{k}, \num{m})
  \intertext{and consequently also}
  \Th{Q} & \Proves \lexists[y][(!A_g(\num{n}, y) \land !A_f(y, \num{m}))],
\end{align*}
i.e., $\Th{Q} \Proves !A_h(\num{n}, \num{m})$.
\end{proof}

\begin{prop}
\ollabel{prop:rep2}
If $h(n) = m$, then $\Th{Q} \Proves \lforall[z][(!A_h(\num{n}, z) \lif
  z = \num{m})]$.
\end{prop}

\begin{proof}
Suppose $h(n) = m$, i.e., $f(g(n)) = m$.  Let $k = g(n)$. Then
\begin{align*}
  \Th{Q} & \Proves \lforall[y][(!A_g(\num{n}, y) \lif \eq[y][\num{k}])]
  \intertext{since $!A_g$ represents~$g$, and}
  \Th{Q} & \Proves \lforall[z][(!A_f(\num{k}, z) \lif \eq[z][\num{m}])]
  \intertext{since $!A_f$ represents~$f$. Using just a little bit of
    logic, we can show that also}
  \Th{Q} & \Proves \lforall[z][(\lexists[y][(!A_g(\num{n}, y) \land
      !A_f(y, z))] \lif \eq[z][\num{m}])].
\end{align*}
i.e., $\Th{Q} \Proves \lforall[y][(!A_h(\num n, y) \lif \eq[y][\num m])]$.
\end{proof}

The same idea works in the more complex case where $f$ and~$g_i$ have
arity greater than~$1$.

\begin{prop}
\ollabel{prop:rep-composition}
If $!A_f(y_0, \dots, y_{k-1}, z)$ represents $f(y_0, \dots, y_{k-1})$
in~$\Th{Q}$, and $!A_{g_i}(x_0, \dots, x_{l-1}, y)$ represents
$g_i(x_0, \dots, x_{l-1})$ in~$\Th{Q}$, then
\begin{multline*}
  \lexists[y_0\dots][\lexists[y_{k-1}][(!A_{g_0}(x_0,\dots,x_{l-1},y_0) \land
      \dots \land {}]]\\
  !A_{g_{k-1}}(x_0,\dots,x_{l-1},y_{k-1}) \land !A_f(y_0,\dots,y_{k-1},z))
\end{multline*}
represents
\[
h(x_0, \dots, x_{l-1}) = f(g_0(x_0, \dots, x_{l-1}), \dots, g_{k-1}(x_0,
\dots, x_{l-1})).
\]
\end{prop}

\begin{proof}
Exercise.
\end{proof}

\begin{prob}
Using the proofs of \olref[inc][req][cmp]{prop:rep2} and
\olref[inc][req][cmp]{prop:rep2} as a guide, carry out the proof of
\olref[inc][req][cmp]{prop:rep-composition} in detail.
\end{prob}

\end{document}
