% Part: incompleteness
% Chapter: representability-in-q
% Section: undecidability

\documentclass[../../../include/open-logic-section]{subfiles}

\begin{document}

\olfileid{inc}{req}{und}
\olsection{Undecidability}

We call a theory $\Th{T}$ \emph{undecidable} if there is no
computational procedure which, after finitely many steps and
unfailingly, provides a correct answer to the question ``does $\Th{T}$
prove~$!A$?'' for any sentence~$!A$ in the language of~$\Th{T}$.  So
$\Th{Q}$ would be decidable iff there were a computational procedure
which decides, given a sentence~$!A$ in the language of arithmetic,
whether $\Th{Q} \Proves !A$ or not.  We can make this more precise by
asking: Is the relation~$\Prov[\Th{Q}](y)$, which holds of~$y$
iff $y$ is the G\"odel number of a sentence provable in~$\Th{Q}$,
recursive?  The answer is: no.

\begin{thm}
$\Th{Q}$ is undecidable, i.e., the relation
\[
\Prov[\Th{Q}](y) \defiff \fn{Sent}(y) \land
\lexists[x][\Prf[\Th{Q}](x, y)]
\]
is not recursive.
\end{thm}

\begin{proof}
Suppose it were.  Then we could solve the halting problem as follows:
Given $e$ and $n$, we know that $\cfind{e}(n) \fdefined$ iff there is
an~$s$ such that $T(e, n, s)$, where $T$ is Kleene's predicate from
\olref[cmp][rec][nft]{thm:kleene-nf}.  Since $T$ is primitive recursive
it is representable in~$\Th{Q}$ by a formula $!B_T$, that is, $\Th{Q}
\Proves !B_T(\num{e}, \num{n}, \num{s})$ iff $T(e, n, s)$.  If $\Th{Q}
\Proves !B_T(\num{e}, \num{n}, \num{s})$ then also $ \Th{Q} \Proves
\lexists[y][!B_T(\num{e}, \num{n}, y)]$.  If no such $s$ exists, then
$\Th{Q} \Proves \lnot !B_T(\num{e}, \num{n}, \num{s})$ for
every~$s$.  But $\Th{Q}$ is $\omega$-consistent, i.e., if $\Th{Q}
\Proves \lnot !A(\num{n})$ for every~$n \in \Nat$, then $\Th{Q}
\Proves/ \lexists[y][!A(y)]$.  We know this because the axioms of
$\Th{Q}$ are true in the standard model~$\Struct{N}$.  So, $\Th{Q}
\Proves/ \lexists[y][!B_T(\num{e}, \num{n}, y)]$.  In other words,
$\Th{Q} \Proves \lexists[y][!B_T(\num{e}, \num{n}, y)]$ iff there is
an $s$ such that $T(e, n, s)$, i.e., iff $\cfind{e}(n) \fdefined$.
From $e$ and~$n$ we can compute $\Gn{\lexists[y][!B_T(\num{e},
    \num{n}, y)]}$, let $g(e, n)$ be the primitive recursive function
which does that.  So
\[
h(e, n) =
\begin{cases}
1 & \text{if $\Prov[\Th{Q}](g(e, n))$}\\
0 & \text{otherwise}.
\end{cases}
\]
This would show that $h$ is recursive if $\Prov[\Th{Q}]$ is. But~$h$
is not recursive, by \olref[cmp][rec][hlt]{thm:halting-problem}, so
$\Prov[\Th{Q}]$ cannot be either.
\end{proof}

\begin{cor}
First-order logic is undecidable.
\end{cor}

\begin{proof}
If first-order logic were decidable, provability in~$\Th{Q}$ would be
as well, since $\Th{Q} \Proves !A$ iff $\Proves !T \lif !A$, where
$!T$ is the conjunction of the axioms of~$\Th{Q}$.
\end{proof}

\end{document}
