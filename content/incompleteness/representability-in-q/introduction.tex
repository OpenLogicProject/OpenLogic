% Part: incompleteness
% Chapter: representability-in-q
% Section: introduction

\documentclass[../../../include/open-logic-section]{subfiles}

\begin{document}

\olfileid{inc}{req}{int}
\olsection{Introduction}

The incompleteness theorems apply to theories in which basic facts
about computable functions can be expressed and proved.  We will
describe a very minimal such theory called ``$\Th{Q}$'' (or, sometimes,
``Robinson's $Q$,'' after Raphael Robinson). We will say what it means
for a function to be \emph{representable} in $\Th{Q}$, and then we
will prove the following:
\begin{quote}
  A function is representable in $\Th{Q}$ if and only if it is computable.
\end{quote}
For one thing, this provides us with another model of
computability. But we will also use it to show that the set
$\Setabs{!A}{\Th{Q} \Proves !A}$ is not decidable, by reducing the
halting problem to it. By the time we are done, we will have proved
much stronger things than this.

The language of $\Th{Q}$ is the language of
arithmetic; $\Th{Q}$ consists of the following axioms
(to be used in conjunction with the other axioms and rules of
first-order logic with !!{identity}):
\begin{align*}
& \lforall[x][\lforall[y][(\eq[x'][y'] \lif \eq[x][y])]] \tag{$!Q_1$}\\
& \lforall[x][\eq/[\Obj 0][x']] \tag{$!Q_2$}\\
& \lforall[x][(\eq[x][\Obj 0] \lor \lexists[y][\eq[x][y']])] \tag{$!Q_3$}\\
& \lforall[x][\eq[(x + \Obj 0)][x]] \tag{$!Q_4$}\\
& \lforall[x][\lforall[y][\eq[(x + y')][(x + y)']]] \tag{$!Q_5$}\\
& \lforall[x][\eq[(x \times \Obj 0)][\Obj 0]] \tag{$!Q_6$}\\
& \lforall[x][\lforall[y][\eq[(x \times y')][((x \times y) + x)]]] \tag{$!Q_7$}\\
& \lforall[x][\lforall[y][(x < y \liff \lexists[z][\eq[(z' + x)][y]])]] \tag{$!Q_8$}
\end{align*}
For each natural number $n$, define the numeral $\num{n}$ to be the
term $0^{\prime\prime\ldots\prime}$ where there are $n$ tick marks in
all.  So, $\num{0}$ is the !!{constant}~$\Obj{0}$ by itself, $\num{1}$
is $\Obj{0}'$, $\num{2}$ is $\Obj{0}''$, etc.

As a theory of arithmetic, $\Th{Q}$ is \emph{extremely} weak; for
example, you can't even prove very simple facts like
$\lforall[x][\eq/[x][x']]$ or $\lforall[x][\lforall[y][(x + y) = (y +
    x)]]$. But we will see that much of the reason that $\Th{Q}$ is so
interesting is \emph{because} it is so weak. In fact, it is just
barely strong enough for the incompleteness theorem to hold. Another
reason $\Th{Q}$ is interesting is because it has a \emph{finite} set
of axioms.

A stronger theory than $\Th{Q}$ (called \emph{Peano arithmetic} $\Th{PA}$)
is obtained by adding a schema of induction to~$\Th{Q}$:
\[
(!A(\Obj 0) \land \lforall[x][(!A(x) \lif !A(x'))]) \lif \lforall[x][!A(x)]
\]
where $!A(x)$ is any formula. If $!A(x)$ contains free !!{variable}s
other than $x$, we add universal quantifiers to the front to bind all
of them (so that the corresponding instance of the induction schema is
a !!{sentence}). For instance, if $!A(x, y)$ also contains the
!!{variable}~$y$ free, the corresponding instance is
\[
\lforall[y][((!A(\Obj 0) \land \lforall[x][(!A(x) \lif !A(x'))]) \lif
  \lforall[x][!A(x)])]
\]
Using instances of the induction schema, one can prove much more from
the axioms of~$\Th{PA}$ than from those of $\Th{Q}$. In fact, it takes
a good deal of work to find ``natural'' statements about the natural
numbers that can't be proved in Peano arithmetic!{}

\begin{defn}
\ollabel{defn:representable-fn}
  A function $f(x_0,\ldots,x_k)$ from the natural numbers to
  the natural numbers is said to be {\em representable in $\Th{Q}$} if
  there is a formula $!A_f(x_0,\dots,x_k,y)$ such that whenever
  $f(n_0,\dots,n_k) = m$, $\Th{Q}$ proves
\begin{enumerate}
\item $!A_f(\num{n_0}, \dots, \num{n_k}, \num{m})$
\item $\lforall[y][(!A_f(\num{n_0}, \dots, \num{n_k}, y) \lif \num{m} = y)]$.
\end{enumerate}
\end{defn}

There are other ways of stating the definition; for example, we could
equivalently require that $\Th{Q}$ proves $\lforall[y][(!A_f(\num{n_0}, \dots,
\num{n_k}, y) \liff \eq[y][\num{m}])]$.

\begin{thm}
\ollabel{thm:representable-iff-comp}
A function is representable in $\Th{Q}$ if and only if it is computable.
\end{thm}

There are two directions to proving the theorem. The left-to-right
direction is fairly straightforward once arithmetization of syntax is
in place. The other direction requires more work.  Here is the basic
idea: we pick ``general recursive'' as a way of making ``computable''
precise, and show that every general recursive function is
representable in~$\Th{Q}$. Recall that a function is general recursive
if it can be defined from $\Zero$, the successor function~$\Succ$, and the
projection functions~$\Proj{n}{i}$, using composition, primitive recursion,
and regular minimization. So one way of showing that every general
recursive function is representable in~$\Th{Q}$ is to show that the
basic functions are representable, and whenever some functions are
representable, then so are the functions defined from them using
composition, primitive recursion, and regular minimization. In other
words, we might show that the basic functions are representable, and
that the representable functions are ``closed under'' composition,
primitive recursion, and regular minimization.  This guarantees that
every general recursive function is representable.

It turns out that the step where we would show that representable
functions are closed under primitive recursion is hard. In order to
avoid this step, we show first that in fact we can do without
primitive recursion. That is, we show that every general recursive
function can be defined from basic functions using composition and
regular minimization alone.  To do this, we show that primitive
recursion can actually be done by a specific regular minimization.
However, for this to work, we have to add some additional basic
functions: addition, multiplication, and the characteristic function
of the identity relation~$\Char{=}$.  Then, we can prove the theorem
by showing that all of \emph{these} basic functions are representable
in~$\Th{Q}$, and the representable functions are closed under
composition and regular minimization.

\end{document}
