% Part: incompleteness
% Chapter: introduction
% Section: overview

\documentclass[../../../include/open-logic-section]{subfiles}

\begin{document}

\olfileid{inc}{int}{ovr}

\olsection{Overview of Incompleteness Results}

Hilbert expected that mathematics could be formalized in an
!!{axiomatizable} theory which it would be possible to prove
!!{complete} and !!{decidable}. Moreover, he aimed to prove the
consistency of this theory with very weak, ``finitary,'' means, which
would defend classical mathematics against the challenges of
intuitionism.  G\"odel's incompleteness theorems showed that these
goals cannot be achieved.

G\"odel's first incompleteness theorem showed that a version of
Russell and Whitehead's \emph{Principia Mathematica} is not
!!{complete}.  But the proof was actually very general and applies to
a wide variety of theories.  This means that it wasn't just that
\emph{Principia Mathematica} did not manage to completely capture
mathematics, but that \emph{no} acceptable theory does.  It took a
while to isolate the features of theories that suffice for the
incompleteness theorems to apply, and to generalize G\"odel's proof to
apply make it depend only on these features.  But we are now in a
position to state a very general version of the first incompleteness
theorem for theories in the language $\Lang L_A$ of arithmetic.

\begin{thm}
If $\Gamma$ is a consistent and !!{axiomatizable} theory in~$\Lang
L_A$ which !!{represents} all computable functions and !!{decidable}
relations, then $\Gamma$ is not !!{complete}.
\end{thm}

To say that $\Gamma$ is not !!{complete} is to say that for at least
one !!{sentence}~$!A$, $\Gamma \Proves/ !A$ and $\Gamma \Proves/ \lnot
!A$.  Such !!a{sentence} is called \emph{independent} (of~$\Gamma)$.
We can in fact relatively quickly prove that there must be independent
sentences. But the power of G\"odel's proof of the theorem lies in the
fact that it exhibits a \emph{specific example} of such an independent
!!{sentence}. The intriguing construction produces
!!a{sentence}~$!G_\Gamma$, called a \emph{G\"odel sentence}
for~$\Gamma$, which is unprovable because in $\Gamma$, $!G_\Gamma$ is
equivalent to the claim that $!G_\Gamma$ is unprovable in~$\Gamma$.  It
does so \emph{constructively}, i.e., given an axiomatization
of~$\Gamma$ and a description of the !!{derivation} system, the proof gives a
method for actually writing down~$!G_\Gamma$.

The construction in G\"odel's proof requires that we find a way to
express in $\Lang L_A$ the properties of and operations on terms and
!!{formula}s of $\Lang L_A$ itself. These include properties such as
``$!A$ is !!a{sentence},'' ``$\delta$ is !!a{derivation} of~$!A$,''
and operations such as $\Subst{!A}{t}{x}$.  This way must (a) express
these properties and relations via a ``coding'' of symbols and
sequences thereof (which is what terms, !!{formula}s, !!{derivation}s,
etc. are) as natural numbers (which is what $\Lang L_A$ can talk
about). It must (b) do this in such a way that $\Gamma$ will prove the
relevant facts, so we must show that these properties are coded by
!!{decidable} properties of natural numbers and the operations
correspond to computable functions on natural numbers. This is called
``arithmetization of syntax.''

Before we investigate how syntax can be arithmetized, however, we will
consider the condition that~$\Gamma$ is ``strong enough,'' i.e.,
!!{represents} all computable functions and !!{decidable} relations.
This requires that we give a precise definition of ``computable.''
This can be done in a number of ways, e.g., via the model of Turing
machines, or as those functions computable by programs in some
general-purpose programming language.  Since our aim is to
!!{represents}s these functions and relations in a theory in the
language~$\Lang L_A$, however, it is best to pick a simple definition
of computability of just numerical functions.  This is the notion of
\emph{recursive function}.  So we will first discuss the recursive
functions. We will then show that $\Th{Q}$ already !!{represents} all
recursive functions and relations.  This will allow us to apply the
incompleteness theorem to specific theories such as $\Th{Q}$ and
$\Th{PA}$, since we will have established that these are examples of
theories that are ``strong enough.''

The end result of the arithmetization of syntax is
!!a{formula} $\OProv[\Gamma](x)$ which, via the coding of !!{formula}s
as numbers, expresses provability from the axioms of~$\Gamma$.
Specifically, if $!A$ is coded by the number~$n$, and $\Gamma \Proves
!A$, then $\Gamma \Proves \Prov[\Gamma](\num{n})$.  This ``provability
predicate'' for $\Gamma$ allows us also to express, in a certain
sense, the consistency of $\Gamma$ as !!a{sentence} of~$\Lang L_A$:
let the ``consistency statement'' for~$\Gamma$ be the !!{sentence}
$\lnot \Prov[\Gamma](\num{n})$, where we take $n$ to be the code of a
contradiction, e.g., of~$\lfalse$.  The second incompleteness theorem
states that consistent !!{axiomatizable} theories also do not prove
their own consistency statements.  The conditions required for this
theorem to apply are a bit more stringent than just that the theory
represents all computable functions and !!{decidable} relations, but
we will show that $\Th{PA}$ satisfies them.

\end{document}
