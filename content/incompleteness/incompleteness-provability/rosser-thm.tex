% Part: incompleteness
% Chapter: incompleteness-provability
% Section: rosser-thm

\documentclass[../../../include/open-logic-section]{subfiles}

\begin{document}

\olfileid{inc}{inp}{ros}

\olsection{Rosser's Theorem}

Can we modify G\"odel's proof to get a stronger result, replacing
``$\omega$-consistent'' with simply ``consistent''? The answer is
``yes,'' using a trick discovered by Rosser.  Rosser's trick is to use
a ``modified'' !!{derivability} predicate $\ORProv_T(y)$ instead of
$\OProv[\Th{T}](y)$.

\begin{thm}
\ollabel{thm:rosser}
Let $\Th{T}$ be any consistent, !!{axiomatizable} theory
extending $\Th{Q}$. Then $\Th{T}$ is not complete.
\end{thm}

\begin{proof}
Recall that $\OProv[\Th{T}](y)$ is defined as $\lexists[x][\OPrf[\Th{T}](x,
  y)]$, where $\OPrf[\Th{T}](x, y)$ represents the decidable relation which
holds iff $x$ is the G\"odel number of !!a{derivation} of the
!!{sentence} with G\"odel number~$y$. The relation that holds between
$x$ and~$y$ if $x$~is the G\"odel number of a \emph{refutation} of the
sentence with G\"odel number~$y$ is also decidable. Let $\fn{not}(x)$
be the primitive recursive function which does the following: if $x$
is the code of a formula $!A$, $\fn{not}(x)$ is a code of $\lnot
!A$. Then $\Refut[\Th{T}](x, y)$ holds iff $\Prf[\Th{T}](x, \fn{not}(y))$.  Let
$\ORefut[\Th{T}](x, y)$ represent it.  Then, if $\Th{T} \Proves \lnot !A$
and $\delta$ is a corresponding !!{derivation}, $\Th{Q} \Proves
\ORefut[\Th{T}](\gn{\delta}, \gn{!A})$.  We define $\ORProv[\Th{T}](y)$ as
\[
\lexists[x][(\OPrf[\Th{T}](x,y) \land \lforall[z][(z < x \lif \lnot
  \ORefut[\Th{T}](z,y))])].
\]
Roughly, $\ORProv[\Th{T}](y)$ says ``there is a proof of $y$ in $\Th{T}$,
and there is no shorter refutation of~$y$.''  Assuming $\Th{T}$ is
consistent, $\ORProv[\Th{T}](y)$ is true of the same numbers as
$\OProv[\Th{T}](y)$; but from the point of view of \emph{provability}
in~$\Th{T}$ (and we now know that there is a difference between truth
and provability!) the two have different properties. If $\Th{T}$ is
\emph{in}consistent, then the two do \emph{not} hold of the same
numbers!{} ($\ORProv[\Th{T}](y)$ is often read as ``$y$ is Rosser
provable.'' Since, as just discussed, Rosser provability is not some
special kind of provability---in inconsistent theories, there are
!!{sentence}s that are provable but not Rosser provable---this may be
confusing. To avoid the confusion, you could instead read it as
``$y$ is shmovable.'')

By the fixed-point lemma, there is a formula $!R_\Th{T}$ such that
\begin{equation}
  \Th{Q} \Proves !R_\Th{T} \liff \lnot \ORProv[\Th{T}](\gn{!R_\Th{T}}).
  \ollabel{RT}
\end{equation}
In contrast to the proof of \olref[1in]{thm:first-incompleteness},
here we claim that if $\Th{T}$ is consistent, $\Th{T}$ doesn't !!{derive}
$!R_\Th{T}$, and $\Th{T}$ also doesn't !!{derive} $\lnot !R_\Th{T}$. (In
other words, we don't need the assumption of $\omega$-consistency.)

First, let's show that $\Th{T} \Proves/ !R_{\Th{T}}$.  Suppose it did, so
there is !!a{derivation} of~$!R_{\Th{T}}$ from~$T$; let $n$ be its G\"odel
number. Then $\Th{Q} \Proves \OPrf[\Th{T}](\num{n}, \gn{!R_{\Th{T}}})$, since
$\OPrf[\Th{T}]$ represents $\Prf[\Th{T}]$ in~$\Th{Q}$. Also, for each $k < n$,
$k$ is not the G\"odel number of !!a{derivation} of $\lnot !R_{\Th{T}}$, since $\Th{T}$ is
consistent. So for each $k < n$, $\Th{Q} \Proves \lnot
\ORefut[\Th{T}](\num{k}, \gn{!R_{\Th{T}}})$. By \olref[req][min]{lem:less-nsucc},
$\Th{Q} \Proves \lforall[z][(z < \num{n} \lif \lnot \ORefut[\Th{T}](z,
  \gn{!R_{\Th{T}}}))]$. Thus,
\[
\Th{Q} \Proves \lexists[x][(\OPrf[\Th{T}](x,\gn{!R_{\Th{T}}}) \land \lforall[z][(z
    < x \lif \lnot \ORefut[\Th{T}](z,\gn{!R_{\Th{T}}}))])],
\]
but that's just $\ORProv[\Th{T}](\gn{!R_{\Th{T}}})$. By \olref{RT}, $\Th{Q}
\Proves \lnot !R_{\Th{T}}$. Since $\Th{T}$ extends $\Th{Q}$, also $\Th{T}
\Proves \lnot !R_{\Th{T}}$. We've assumed that $\Th{T} \Proves !R_{\Th{T}}$, so
$\Th{T}$ would be inconsistent, contrary to the assumption of the
theorem.

Now, let's show that $\Th{T} \Proves/ \lnot !R_{\Th{T}}$. Again, suppose it
did, and suppose $n$ is the G\"odel number of !!a{derivation}
of~$\lnot !R_{\Th{T}}$. Then $\Refut[\Th{T}](n, \Gn{!R_{\Th{T}}})$ holds, and since
$\ORefut[\Th{T}]$ represents $\Refut[\Th{T}]$ in $\Th{Q}$, $\Th{Q} \Proves
\ORefut[\Th{T}](\num{n}, \gn{!R_{\Th{T}}})$. We'll again show that $\Th{T}$ would
then be inconsistent because it would also !!{derive}~$!R_{\Th{T}}$.  Since
\begin{align*}
\Th{Q} & \Proves !R_{\Th{T}} \liff \lnot \ORProv[\Th{T}](\gn{!R_{\Th{T}}}), 
\intertext{and since $\Th{T}$ extends~$\Th{Q}$, it suffices to show that}
\Th{Q} & \Proves \lnot \ORProv[\Th{T}](\gn{!R_{\Th{T}}}).
\end{align*}
The !!{sentence} $\lnot
\ORProv[\Th{T}](\gn{!R_{\Th{T}}})$, i.e.,
\begin{align*}
  \lnot & \lexists[x][(\OPrf[\Th{T}](x,\gn{!R_{\Th{T}}}) \land
    \lforall[z][(z < x \lif \lnot \ORefut[\Th{T}](z,\gn{!R_{\Th{T}}}))])],
  \intertext{is logically equivalent to}
  & \lforall[x][(\OPrf[\Th{T}](x,\gn{!R_{\Th{T}}}) \lif
    \lexists[z][(z < x \land \ORefut[\Th{T}](z,\gn{!R_{\Th{T}}}))])].
\end{align*}
We argue informally using logic, making use of facts about what
$\Th{Q}$ !!{derive}s. Suppose $x$ is arbitrary and $\OPrf[\Th{T}](x,
\gn{!R_{\Th{T}}})$. We already know that $\Th{T} \Proves/ !R_{\Th{T}}$, and so for
every $k$, $\Th{Q} \Proves \lnot \OPrf[\Th{T}](\num{k}, \gn{!R_{\Th{T}}})$. Thus,
for every $k$ it follows that $\eq/[x][\num{k}]$. In particular, we
have (a) that $\eq/[x][\num{n}]$.  We also have $\lnot(\eq[x][\num{0}]
\lor \eq[x][\num{1}] \lor \dots \lor \eq[x][\num{n-1}])$ and so by
\olref[req][min]{lem:less-nsucc}, (b) $\lnot(x < \num{n})$. By
\olref[req][min]{lem:trichotomy}, $\num{n} < x$. Since $\Th{Q} \Proves
\ORefut[\Th{T}](\num{n}, \gn{!R_{\Th{T}}})$, we have $\num{n} < x \land
\ORefut[\Th{T}](\num{n}, \gn{!R_{\Th{T}}})$, and from that $\lexists[z][(z < x
\land \ORefut[\Th{T}](z,\gn{!R_{\Th{T}}}))]$. Since $x$ was arbitrary we get, as
required, that
\[
\lforall[x][(\OPrf[\Th{T}](x,\gn{!R_{\Th{T}}}) \lif
  \lexists[z][(z < x \land \ORefut[\Th{T}](z,\gn{!R_{\Th{T}}}))])].
\]
\end{proof}

\begin{prob}
Two sets $A$ and $B$ of natural numbers are said to be
\emph{computably inseparable} if there is no !!{decidable} set~$X$
such that $A \subseteq X$ and $B \subseteq \Complement{X}$
($\Complement{X}$ is the complement, $\Nat \setminus X$, of~$X$). Let
$\Th{T}$ be a consistent !!{axiomatizable} extension of $\Th{Q}$.
Suppose $A$ is the set of G\"odel numbers of !!{sentence}s provable
in~$\Th{T}$ and $B$ the set of G\"odel numbers of sentences
refutable in~$\Th{T}$. Prove that $A$ and~$B$ are computably
inseparable.
\end{prob}

\end{document}
