% Part: second-order-logic
% Chapter: metatheory
% Section: introduction

\documentclass[../../../include/open-logic-section]{subfiles}

\begin{document}

\olfileid{sol}{met}{int}

\olsection{Introduction}

First-order logic has a number of nice properties. We know it is not
decidable, but at least it is axiomatizable. That is, there are proof
systems for first-order logic which are sound and complete, i.e., they
give rise to !!a{derivability} relation~$\Proves$ with the property
that for any set of !!{sentence}s~$\Gamma$ and !!{sentence}~$!Q$,
$\Gamma \Entails !A$ iff $\Gamma \Proves !A$.  This means in
particular that the validities of first-order logic are !!{computably
  enumerable}. There is a computable function~$f\colon \Nat \to
\Sent[L]$ such that the values of~$f$ are all and only the valid
!!{sentence}s of~$\Lang{L}$. This is so because !!{derivation}s can be
enumerated, and those that !!{derive} a single~!!{sentence} are then
mapped to that !!{sentence}.  Second-order logic is more expressive
than first-order logic, and so it is in general more complicated to
capture its validities.  In fact, we'll show that second-order logic
is not only undecidable, but its validities are not even !!{computably
  enumerable}. This means there can be no sound and complete proof
system for second-order logic (although sound, but incomplete proof
systems are available and in fact are important objects of research).

First-order logic also has two more properties: it is compact (if
every finite subset of a set~$\Gamma$ of !!{sentence}s is satisfiable,
$\Gamma$ itself is satisfiable) and the L\"owenheim-Skolem Theorem
holds for it (if $\Gamma$ has an infinite model it has
!!a{denumerable} model). Both of these results fail for second-order
logic. Again, the reason is that second-order logic can express facts
about the size of !!{domain}s that first-order logic cannot.

\end{document}
