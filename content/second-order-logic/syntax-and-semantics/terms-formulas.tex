% Part: second-order-logic
% Chapter: syntax-and-semantics
% Section: terms-formulas

\documentclass[../../../include/open-logic-section]{subfiles}

\begin{document}

\olfileid{sol}{syn}{frm}

\olsection{Terms and \printtoken{P}{formula}}

Like in first-order logic, expressions of second-order logic are built
up from a basic vocabulary containing \emph{!!{variable}s},
\emph{!!{constant}s}, \emph{!!{predicate}s} and sometimes
\emph{!!{function}s}.  From them, together with logical connectives,
quantifiers, and punctuation symbols such as parentheses and commas,
\emph{terms} and \emph{!!{formula}s} are formed.  The difference is
that in addition to variables for objects, second-order logic also
contains variables for relations and functions, and allows
quantification over them. So the logical symbols of second-order logic
are those of first-order logic, plus:

\begin{enumerate}
\item A !!{denumerable}s set of second-order relation !!{variable}s of
  every arity~$n$: $\Obj V_0^n$, $\Obj V_1^n$, $\Obj V_2^n$, \dots
\item A !!{denumerable}s set of second-order function !!{variable}s:
  $\Obj u_0^n$, $\Obj u_1^n$, $\Obj u_2^n$, \dots
\end{enumerate}

Just as we use $x$, $y$, $z$ as meta-variables for first-order
variables $\Obj v_i$, we'll use $X$, $Y$, $Z$, etc., as metavariables
for $\Obj V_i^n$ and $u$, $v$, etc., as meta-variables for~$\Obj u_i^n$.

\begin{explain}
The non-logical symbols of a second-order language are specified the
same way a first-order language is: by listing its !!{constant}s,
!!{function}s, and !!{predicate}s.

In first-order logic, the !!{identity}~$\eq$ is usually included. In
first-order logic, the non-logical symbols of a language~$\Lang{L}$
are crucial to allow us to express anything interesting. There are of
course !!{sentence}s that use no non-logical symbols, but with
only~$\eq$ it is hard to say anything interesting.  In second-order
logic, since we have an unlimited supply of relation and function
variables, we can say anything we can say in a first-order language
even without a special supply of non-logical symbols.
\end{explain}

\begin{defn}[Second-order Terms]
The set of \emph{second-order terms} of~$\Lang L$, $\TrmSOL[L]$, is
defined by adding to \olref[fol][syn][frm]{defn:terms} the clause
\begin{enumerate}
\item If $u$ is an $n$-place function variable and $t_1$, \dots, $t_n$
  are terms, then $\Atom{u}{t_1, \ldots, t_n}$ is a term.
\end{enumerate}
\end{defn}

\begin{explain}
So, a second-order term looks just like a first-order term, except
that where a first-order term contains !!a{function}~$\Obj{f^n_i}$, a
second-order term may contain a function variable~$\Obj{u^n_i}$ in its
place.
\end{explain}

\begin{defn}[Second-order \usetoken{s}{formula}]
The set of \emph{second-order !!{formula}s}~$\FrmSOL[L]$ of the
language~$\Lang L$ is defined by adding to
\olref[fol][syn][frm]{defn:terms} the clauses
\begin{enumerate}
\item If $X$ is an $n$-place predicate variable and $t_1$, \dots,
  $t_n$ are second-order terms of~$\Lang L$, then
  $\Atom{X}{t_1,\ldots, t_n}$ is an atomic !!{formula}.

\tagitem{prvAll}{If $!A$ is !!a{formula} and $u$ is a function variable,
  then $\lforall[u][!A]$ is !!a{formula}.}{}

\tagitem{prvAll}{If $!A$ is !!a{formula} and $X$ is a predicate variable,
  then $\lforall[X][!A]$ is !!a{formula}.}{}

\tagitem{prvEx}{If $!A$ is !!a{formula} and $u$ is a function variable,
  then $\lexists[u][!A]$ is !!a{formula}.}{}

\tagitem{prvEx}{If $!A$ is !!a{formula} and $X$ is a predicate variable,
  then $\lexists[X][!A]$ is !!a{formula}.}{}
\end{enumerate}
\end{defn}

\end{document}
