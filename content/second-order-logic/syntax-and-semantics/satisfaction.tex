% Part: second-order-logic
% Chapter: syntax-and-semantics
% Section: satisfaction

\documentclass[../../../include/open-logic-section]{subfiles}

\begin{document}

\olfileid{sol}{syn}{sat}

\olsection{Satisfaction}

\begin{explain}
To define the satisfaction relation $\Sat{M}{!A}[s]$ for second-order
!!{formula}s, we have to extend the definitions to cover second-order
variables.  
\end{explain}

\begin{defn}[Variable Assignment]
A \emph{variable assignment}~$s$ for a !!{structure}~$\Struct{M}$ is a
function which maps each
\begin{enumerate}
\item object !!{variable}~$\Obj{v_i}$ to an element of~$\Domain M$,
  i.e., $s(\Obj{v_i}) \in \Domain{M}$
\item $n$-place relation variable~$\Obj{V_i^n}$ to an $n$-place
  relation on~$\Domain{M}$, i.e., $s(\Obj{V_i^n}) \subseteq \Domain{M}^n$;
\item $n$-place function variable~$\Obj{u_i^n}$ to an $n$-place
  function from $\Domain{M}$ to $\Domain{M}$, i.e.,
  $s(\Obj{u_i^n})\colon \Domain{M}^n \to \Domain{M}$;
\end{enumerate}
\end{defn}

\begin{explain}
A !!{structure} assigns a !!{value} to each !!{constant} and
!!{function}, and a second-order variable assigns objects and
functions to each object and function variable. Together, they let us
assign a value to very term.
\end{explain}

\begin{defn}[\usetoken{S}{value} of a Term]
If $t$ is a term of the language~$\Lang L$, $\Struct M$ is a
!!{structure} for~$\Lang L$, and $s$ is a !!{variable} assignment
for~$\Struct M$, the \emph{!!{value}}~$\Value{t}{M}[s]$ is defined as
for first-order terms, plus the following clause:
\begin{quote}
\indcase{t}{\Atom{u}{t_1, \ldots, t_n}}{
\[
\Value{\indfrm}{M}[s] = s(u)(\Value{t_1}{M}[s], \ldots,
\Value{t_n}{M}[s]).
\]}
\end{quote}
\end{defn}

\begin{defn}[Satisfaction]
For second-order !!{formula}s~$!A$, the definition of satisfaction is
like \olref[fol][syn][sat]{defn:satisfaction} with the addition of:
\begin{enumerate}
\item \indcase{!A}{X^n{t_1, \dots, t_n}}{$\Sat{M}{\indfrm}[s]$
  iff $\langle \Value{t_1}{M}[s], \dots, \Value{t_n}{M}[s] \rangle \in
  s(X^n)$.}
\tagitem{prvAll}{%
  \indcase{!A}{\lforall[X][!B]}{$\Sat{M}{\indfrm}[s]$ iff for every
    $X$-variant $s'$ of $s$, $\Sat{M}{!B}[s']$.}}{}

\tagitem{prvEx}{%
  \indcase{!A}{\lexists[X][!B]}{$\Sat{M}{\indfrm}[s]$ iff there is an
    $X$-variant $s'$ of $s$ so that $\Sat{M}{!B}[s']$.}}{}

\tagitem{prvAll}{%
  \indcase{!A}{\lforall[u][!B]}{$\Sat{M}{\indfrm}[s]$ iff for every
    $u$-variant $s'$ of $s$, $\Sat{M}{!B}[s']$.}}{}

\tagitem{prvEx}{%
  \indcase{!A}{\lexists[u][!B]}{$\Sat{M}{\indfrm}[s]$ iff there is an
    $u$-variant $s'$ of $s$ so that $\Sat{M}{!B}[s']$.}}{}
\end{enumerate}
\end{defn}

\begin{ex}
$\Sat{M}{\lforall[z][(Xz \liff \lnot Yz)]}[s]$ whenever $s(Y) =
  \Domain{M} \setminus s(X)$. So for instance, let $\Domain{M} = \{1,
  2, 3\}$, $s(X) = \{1, 2\}$ and $s(Y) = \{3\}$.

$\Sat{M}{\lexists[Y][(\lexists[y][Yy] \land \lforall[z][(Xz \liff
        \lnot Yz)])]}[s]$ if there is an $s' \sim_Y s$ such that
  $\Sat{M}{(\lexists[y][Yy] \land \lforall[z][(Xz \liff \lnot
      Yz)])}[s]$. And that is the case iff $s'(Y) \neq \emptyset$ (so
  that $\Sat{M}{\lexists[y][Yy]}[s']$) and, as before, $s'(Y) =
  \Domain{M} \setminus s'(X)$. In other words,
  $\Sat{M}{\lexists[Y][(\lexists[y][Yy] \land \lforall[z][(Xz \liff
        \lnot Yz)])]}[s]$ iff $\Domain{M} \setminus s(X)$ is
  non-empty, or, $s(X) \neq \Domain{M}$. So, the !!{formula} is
  satisfied, e.g., if $s(X) = \{1, 2\}$ but not if $s(X) = \{1, 2,
  3\}$.  
\end{ex}

\end{document}
