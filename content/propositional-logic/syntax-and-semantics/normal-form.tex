% Part: propositional-logic
% Chapter: syntax-and-semantics
% Section: normal-form

\documentclass[../../../include/open-logic-section]{subfiles}

\begin{document}

\olfileid{pl}{syn}{nor}

\olsection{Normal forms}

In this section, we prove two \emph{normal form} theorems for propositional logic. These guarantee that, for any !!{formula}, there is a semantically equivalent !!{formula} in some canonical normal form. Moreover, we shall give methods for finding these normal-form equivalents.

Say that a !!{formula} is in \emph{disjunctive normal form} \emph{iff} it meets all of the following conditions:
	\begin{itemize}
		\item No connectives occur in the !!{formula} other than negations, conjunctions and disjunctions;
		\item Every occurrence of negation has minimal scope (i.e.\ any `$\lnot$' is immediately followed by an atomic !!{formula});
		\item No disjunction occurs within the scope of any conjunction.
	\end{itemize}
(See \olref[fol][syn][sbf]{sec} and \olref[fol][syn][fvs]{sec} for a definition of the scope of connective.) So, here are are some !!{formula}s in disjunctive normal form:
	\begin{center}
		$!A$\\
		$(!A \land !B) \lor (!A \land \lnot !B)$\\
		$(!A \land !B) \lor (!A \land  !B \land !C \land \lnot !D \land \lnot !E)$\\
		$!A \lor (!C \land \lnot \Obj p_{234} \land \Obj p_{233} \land !D) \lor \lnot !B$
	\end{center}
Note that we have allowed ourselfs to employ the relaxed bracketing-conventions that allow conjunctions and disjunctions to be of arbitrary length. These conventions make it easier to see when a !!{formula} is in disjunctive normal form. 

To further illustrate the idea of disjunctive normal form, we shall introduce some more notation. We write `$\pm !A$' to indicate that $!A$ is an atomic !!{formula} which may or may not be prefaced with an occurrence of negation. Then a !!{formula} in disjunctive normal form has the following shape:
	$$(\pm {!A}_1 \land \ldots \land \pm {!A}_i) \lor (\pm {!A}_{i+1} \land \ldots \land \pm{!A}_j) \lor \ldots \lor (\pm{!A}_{m+1} \land \ldots \land \pm {!A}_n)$$
We now know what it is for a !!{formula} to be in disjunctive normal form. The result that we are aiming at is:
	\begin{prop}\ollabel{prop:dnf}
	For any !!{formula}, there is a semantically equivalent !!{formula} in disjunctive normal form.
	\end{prop}
Henceforth, we shall abbreviate `Disjunctive Normal Form' by `DNF'. 


The proof of the DNF Theorem employs truth tables. We shall first illustrate the technique for finding an equivalent !!{formula} in DNF, and then turn this illustration into a rigorous proof. 

Let's suppose we have some !!{formula}, ${!D}$, which contains three atomic !!{formula}s, `$!A$', `$!B$' and `$!C$'. The very first thing to do is fill out a complete truth table for ${!D}$. Maybe we end up with this:
\begin{center}
\begin{tabular}{c c c | c}
$!A$ & $!B$ & $!C$ & ${!D}$\\
\hline
 T & T & T & T \\
 T & T & F & F \\
 T & F & T & T \\
 T & F & F & F \\
 F & T & T & F \\
 F & T & F & F \\
 F & F & T & T \\
 F & F & F & T
\end{tabular}
\end{center}

As it happens, ${!D}$ is true on four lines of its truth table, namely lines 1, 3, 7 and 8. Corresponding to each of those lines, I shall write down four !!{formula}s, whose only connectives are negations and conjunctions, where every negation has minimal scope:
	\begin{itemize}
		\item[\textbullet]  `$!A \land !B \land !C$'\hfill which is true on line 1 (and only then)
		\item[\textbullet] `$!A \land \lnot !B \land !C$' \hfill which is true on line 3 (and only then)
		\item[\textbullet] `$\lnot !A \land \lnot !B \land !C$' \hfill which is true on line 7 (and only then)
		\item[\textbullet] `$\lnot !A \land \lnot !B \land \lnot !C$' \hfill which is true on line 8 (and only then)
	\end{itemize}

But if we now disjoin all of these conjunctions, like so:
$$(!A \land !B \land !C) \lor (!A \land \lnot !B \land !C) \lor (\lnot !A \land \lnot !B \land !C) \lor (\lnot !A \land \lnot !B \land \lnot !C)$$
we have a !!{formula} in DNF which is true on exactly those lines where one of the disjuncts is true, i.e.\ it is true on (and only on) lines 1, 3, 7, and 8. So this !!{formula} has exactly the same truth table as $!D$. So I have a !!{formula} in DNF that is semantically equivalent to $!D$. Which is exactly what we wanted!

Now, this strategy did not depend on the specifics of ${!D}$; it is perfectly general. Consequently, we can use it to obtain a simple proof of the DNF Theorem.

\begin{proof}[Proof of DNF Theorem]
Pick any arbitrary !!{formula}, ${!D}$, and let ${!A}_1, \ldots, {!A}_n$ be the atomic !!{formula}s that occur in ${!D}$. To obtain a !!{formula} in DNF that is semantically equivalent ${!D}$, we consider ${!D}$'s truth table. There are two cases to consider:
	\begin{enumerate}
		\item ${!D}$ is false on every line of its truth table. Then, ${!D}$ is a contradiction. In that case, the contradiction $({!A}_1 \land \lnot {!A}_1) \cong {!D}$, and $({!A}_1 \land \lnot {!A}_1)$ is in DNF. 
	
		\item ${!D}$ is true on at least one line of its truth table.
		For each line $i$ of the truth table, let ${!B}_i$ be a conjunction of the form 
		$$(\pm{!A}_1 \land \ldots \land \pm{!A}_n)$$
		where the following rules determine whether or not to include a negation in front of each atomic !!{formula}:
			\begin{align*}
				{!A}_m\text{ is a conjunct of }{!B}_i&\emph{ iff }{!A}_m\text{ is true on line }i\\
				\lnot{!A}_m\text{ is a conjunct of }{!B}_i&\emph{ iff }{!A}_m\text{ is false on line }i
			\end{align*}
		Given these rules, a trivial proof by induction shows that ${!B}_i$ is true on (and only on) line $i$ of the truth table which considers all possible valuations of ${!A}_1, \ldots, {!A}_n$ (i.e.\ ${!D}$'s truth table). 
		
		Next, let $i_1, i_2, \ldots, i_m$ be the numbers of the lines of the truth table where ${!D}$ is \emph{true}. Now let ${!C}$ be the !!{formula}:
		$${!B}_{i_1} \lor {!B}_{i_2} \lor \ldots \lor {!B}_{i_m}$$
		Since ${!D}$ is true on at least one line of its truth table, ${!C}$ is indeed well-defined; and in the limiting case where ${!D}$ is true on exactly one line of its truth table, ${!C}$ is just ${!B}_{i_1}$, for some $i_1$.
		
		By construction, ${!C}$ is in DNF. Moreover, by construction, for each line $i$ of the truth table: ${!D}$ is true on line $i$ of the truth table \emph{iff} one of ${!C}$'s disjuncts (namely, ${!B}_i$) is true on, and only on, line $i$. (Again, this is shown by a trivial proof by induction.) Hence ${!D}$ and ${!C}$ have the same truth table, and so are tautologically equivalent.
	\end{enumerate}
	These two cases are exhaustive and, either way, we have a !!{formula} in DNF that is semantically  equivalent to ${S}$.
\end{proof}\noindent
So we have proved the DNF Theorem. 

So far we have discussed \emph{disjunctive} normal form. Given the duality of disjunction and conjunction, it may not come as a surprise to hear that there is also such a thing as \emph{conjunctive normal form} (CNF).

The definition of CNF is exactly analogous to the definition of DNF. So, a !!{formula} is in CNF \emph{iff} it meets all of the following conditions:
	\begin{itemize}
		\item No connectives occur in the !!{formula} other than negations, conjunctions and disjunctions;
		\item Every occurrence of negation has minimal scope;
		\item No conjunction occurs within the scope of any disjunction. 
	\end{itemize}
Generally, then, a !!{formula} in CNF looks like this
	$$(\pm {!A}_1 \lor \ldots \lor \pm {!A}_i) \land (\pm {!A}_{i+1} \lor \ldots \lor \pm{!A}_j) \land \ldots \land (\pm{!A}_{m+1} \lor\ldots \lor \pm {!A}_n)$$
where each ${!A}_k$ is an atomic !!{formula}.

It should be immediate clear that if a !!{formula} is in DNF, then its dual is in CNF; and \emph{vice versa}. Armed with this insight, we can immediately prove another normal form theorem:
	\begin{prop} For any !!{formula}, there is a semantically equivalent !!{formula} in conjunctive normal form.
	\end{prop}
\begin{proof}
	Let $!A$ be any !!{formula}. Let $!B$ be a DNF !!{formula} semantically equivalent to $!A^d$ by using \olref{prop:dnf}. Now, $!B^d$ is on CNF by the observation above. So, by \olref[sem]{thm:dual}, we have $(!A^d)^d \cong !B^d$, i.e., the CNF !!{formula} $!B^d$ is semantiaclly equivalent to $!A$.
\end{proof}

This slick proof is a further illustration of the power of duality. However, it might suggest that the DNF Theorem enjoys some kind of `precedence' over the CNF Theorem. That would be misleading.  We can easily prove the CNF Theorem directly, using the same proof techniques that we used to prove the DNF Theorem (whereupon the DNF Theorem could be proved as a consequence of the CNF Theorem and duality). 

\begin{prob}
Consider the following !!{formula}s:
	\begin{itemize}
		\item $(A \lif \lnot B)$
		\item $\lnot (A \liff B)$
		\item $(\lnot A \lor \lnot (A \land B))$
		\item $(\lnot (A \lif B ) \land (A \lif C))$
		\item $(\lnot (A \lor B) \liff ((\lnot C \land \lnot A) \lif \lnot B))$
		\item $((\lnot (A \land \lnot B) \lif C) \land \lnot (A \land D))$
	\end{itemize}
For each !!{formula}:
	\begin{itemize}
		\item write down !!{formula}s in DNF that are semantically equivalent to these !!{formula}s.
		\item write down !!{formula}s in CNF that are semantiaclly equivalent to these !!{formula}s. 
	\end{itemize}
\end{prob}


\end{document}