%Part: propositional-logic
% Chapter: syntax-and-semantics
% Section: normal-form

\documentclass[../../../include/open-logic-section]{subfiles}

\begin{document}

\olfileid{pl}{syn}{nor}

\olsection{Normal forms}

\newcommand*{\intoDNF}{\texttt{IntoDNF}}
\newcommand\dual[1]{\ensuremath{\underline{#1}}} 


In this chapter, we prove two \emph{normal form} theorems for propositional logic. These guarantee that, for any !!{formula}, there is a semantically equivalent !!{formula} in some canonical normal form. Moreover, we shall give methods for finding these normal-form equivalents.

\subsection{Disjunctive Normal Form}%\label{s:DNFDefined}

Say that a !!{formula} is in \emph{disjunctive normal form} \emph{iff} it meets all of the following conditions:
	\begin{itemize}
		\item[(\textsc{dnf1})] No connectives occur in the !!{formula} other than negations, conjunctions and disjunctions;
		\item[(\textsc{dnf2})] Every occurrence of negation has minimal scope (i.e.\ any `$\lnot$' is immediately followed by an atomic !!{formula});
		\item[(\textsc{dnf3})] No disjunction occurs within the scope of any conjunction.
	\end{itemize}
(For a reminder of the definition of the \emph{scope} of a connective, see XX) So, here are are some !!{formula}s in disjunctive normal form:
	\begin{center}
		$A$\\
		$(A \land B) \lor (A \land \lnot B)$\\
		$(A \land B) \lor (A \land  B \land C \land \lnot D \land \lnot E)$\\
		$A \lor (C \land \lnot P_{234} \land P_{233} \land Q) \lor \lnot B$
	\end{center}
Note that I have here broken one of the maxims of this book (see \S\ref{s:InductionOnLength}) and \emph{temporarily} allowed myself to employ the relaxed bracketing-conventions that allow conjunctions and disjunctions to be of arbitrary length (see XXX). These conventions make it easier to see when a !!{formula} is in disjunctive normal form. I shall continue to help myself to these relaxed conventions, without further comment, until \S\ref{s:DNFsubstitution}. 

To further illustrate the idea of disjunctive normal form, I shall introduce some more notation. I write `$\pm\meta{A}$' to indicate that $\meta{A}$ is an atomic !!{formula} which may or may not be prefaced with an occurrence of negation. Then a !!{formula} in disjunctive normal form has the following shape:
	$$(\pm \meta{A}_1 \land \ldots \land \pm \meta{A}_i) \lor (\pm \meta{A}_{i+1} \land \ldots \land \pm\meta{A}_j) \lor \ldots \lor (\pm\meta{A}_{m+1} \land \ldots \land \pm \meta{A}_n)$$
We now know what it is for a !!{formula} to be in disjunctive normal form. The result that we are aiming at is:
	\begin{thm}\label{thm:dnf}\textbf{Disjunctive Normal Form Theorem.} For any !!{formula}, there is a tautologically equivalent !!{formula} in disjunctive normal form.
	\end{thm}\noindent 
Henceforth, I shall abbreviate `Disjunctive Normal Form' by `DNF'. 


\subsection{Proof of DNF Theorem via truth tables}\label{s:DNFTruthTable}
My first proof of the DNF Theorem employs truth tables. I shall first illustrate the technique for finding an equivalent !!{formula} in DNF, and then turn this illustration into a rigorous proof. 

Let's suppose we have some !!{formula}, $\meta{S}$, which contains three atomic !!{formula}s, `$A$', `$B$' and `$C$'. The very first thing to do is fill out a complete truth table for $\meta{S}$. Maybe we end up with this:
\begin{center}
\begin{tabular}{c c c | c}
$A$ & $B$ & $C$ & $\meta{S}$\\
\hline
 T & T & T & T \\
 T & T & F & F \\
 T & F & T & T \\
 T & F & F & F \\
 F & T & T & F \\
 F & T & F & F \\
 F & F & T & T \\
 F & F & F & T
\end{tabular}
\end{center}
%Now, consider a !!{formula}, whose only connectives are negations and conjunctions, where no connective occurs within the scope of any negation, e.g.:
%	$$A \land \lnot B \land C$$
%This !!{formula} is true when, and only when, `$A$' is true, `$B$' is false and `$C$' is true. Similarly, the !!{formula}:
%	$$\lnot A \land \lnot B \land C$$
%this is true when, and only when, `$A$' is false, `$B$' is false and `$C$' is true. 
%
%A disjunction is true when, and only when, at least one of the disjuncts is true. So if I write down a disjunction of !!{formula}s of the above form, perhaps
%	$$(A \land \lnot B \land C) \lor (\lnot A \land \lnot B \land C)$$
%then it will be true on exactly \emph{two} lines of the truth table which describes all possible valuations of `$A$', `$B$' and `$C$'. 
%
As it happens, $\meta{S}$ is true on four lines of its truth table, namely lines 1, 3, 7 and 8. Corresponding to each of those lines, I shall write down four !!{formula}s, whose only connectives are negations and conjunctions, where every negation has minimal scope:
	\begin{itemize}
		\item[\textbullet]  `$A \land B \land C$'\hfill which is true on line 1 (and only then)
		\item[\textbullet] `$A \land \lnot B \land C$' \hfill which is true on line 3 (and only then)
		\item[\textbullet] `$\lnot A \land \lnot B \land C$' \hfill which is true on line 7 (and only then)
		\item[\textbullet] `$\lnot A \land \lnot B \land \lnot C$' \hfill which is true on line 8 (and only then)
	\end{itemize}
But if I now disjoin all of these conjunctions, like so:
$$(A \land B \land C) \lor (A \land \lnot B \land C) \lor (\lnot A \land \lnot B \land C) \lor (\lnot A \land \lnot B \land \lnot C)$$
I have a !!{formula} in DNF which is true on exactly those lines where one of the disjuncts is true, i.e.\ it is true on (and only on) lines 1, 3, 7, and 8. So this !!{formula} has exactly the same truth table as $\meta{S}$. So I have a !!{formula} in DNF that is tautologically equivalent to $\meta{S}$. Which is exactly what I required!

Now, the strategy that I just adopted did not depend on the specifics of $\meta{S}$; it is perfectly general. Consequently, we can use it to obtain a simple proof of the DNF Theorem.

\begin{proof}[Proof of DNF Theorem via truth tables]
Pick any arbitrary !!{formula}, $\meta{S}$, and let $\meta{A}_1, \ldots, \meta{A}_n$ be the atomic !!{formula}s that occur in $\meta{S}$. To obtain a !!{formula} in DNF that is tautologically equivalent $\meta{S}$, we consider $\meta{S}$'s truth table. There are two cases to consider:
	\begin{itemize}
		\item[\emph{Case 1: $\meta{S}$ is false on every line of its truth table.}] Then, $\meta{S}$ is a contradiction. In that case, the contradiction $(\meta{A}_1 \land \lnot \meta{A}_1) \cong \meta{S}$, and $(\meta{A}_1 \land \lnot \meta{A}_1)$ is in DNF. 
	
		\item[\emph{Case 2:  $\meta{S}$ is true on at least one line of its truth table.}]
		For each line $i$ of the truth table, let $\meta{B}_i$ be a conjunction of the form 
		$$(\pm\meta{A}_1 \land \ldots \land \pm\meta{A}_n)$$
		where the following rules determine whether or not to include a negation in front of each atomic !!{formula}:
			\begin{align*}
				\meta{A}_m\text{ is a conjunct of }\meta{B}_i&\emph{ iff }\meta{A}_m\text{ is true on line }i\\
				\lnot\meta{A}_m\text{ is a conjunct of }\meta{B}_i&\emph{ iff }\meta{A}_m\text{ is false on line }i
			\end{align*}
		Given these rules, a trivial proof by induction shows that $\meta{B}_i$ is true on (and only on) line $i$ of the truth table which considers all possible valuations of $\meta{A}_1, \ldots, \meta{A}_n$ (i.e.\ $\meta{S}$'s truth table). 
		
		Next, let $i_1, i_2, \ldots, i_m$ be the numbers of the lines of the truth table where $\meta{S}$ is \emph{true}. Now let $\meta{D}$ be the !!{formula}:
		$$\meta{B}_{i_1} \lor \meta{B}_{i_2} \lor \ldots \lor \meta{B}_{i_m}$$
		Since $\meta{S}$ is true on at least one line of its truth table, $\meta{D}$ is indeed well-defined; and in the limiting case where $\meta{S}$ is true on exactly one line of its truth table, $\meta{D}$ is just $\meta{B}_{i_1}$, for some $i_1$.
		
		By construction, $\meta{D}$ is in DNF. Moreover, by construction, for each line $i$ of the truth table: $\meta{S}$ is true on line $i$ of the truth table \emph{iff} one of $\meta{D}$'s disjuncts (namely, $\meta{B}_i$) is true on, and only on, line $i$. (Again, this is shown by a trivial proof by induction.) Hence $\meta{S}$ and $\meta{D}$ have the same truth table, and so are tautologically equivalent.
	\end{itemize}
	These two cases are exhaustive and, either way, we have a !!{formula} in DNF that is tautologically equivalent to $\meta{S}$.
\end{proof}\noindent
So we have proved the DNF Theorem. Before I say any more, though, I should immediately flag that I am hereby returning to the austere definition of a (TFL) !!{formula}, according to which we can assume that any conjunction has exactly two conjuncts, and any disjunction has exactly two disjuncts.\footnote{NB: if you did the practice exercises for chapter \ref{ch:Induction}, you will know how to justify these conventions rigorously; and you will also, essentially, have performed the two `trivial proofs by induction' mentioned in the preceding proof.}  


\subsection{Proof of DNF Theorem via substitution}\label{s:DNFsubstitution}
I now want to offer a second proof of the DNF Theorem. 

At this point, you might reasonably ask: Why bother with \emph{two} proofs? Certainly one proof is sufficient to establish a result. But there are several possible reasons for offering new proofs of known results. For example:
	\begin{itemize}
		\item a new proof might offer more insight (or different insights) than the original proof;
		\item a new proof-technique might be useful in other ways: it might help you to prove other results (that you have not yet proved);
		\item sheer curiosity;
	\end{itemize}
and more besides. All these reasons apply here. 

Our first proof made use of truth tables. Our second proof will employ Lemma \ref{lem:SubstitutionTautEquivalent}. The aim is to define an \emph{algorithm} which will slowly replace sub!!{formula}s with tautologically equivalent sub!!{formula}s, pushing us in the direction of a DNF !!{formula}, whilst preserving tautological equivalence, until eventually we end up with a tautologically equivalent !!{formula} in DNF. (For a reminder about what an algorithm is, see \S\ref{s:Interpolation}.) 

That is the big idea. But the details are conceptually trickier than those in the first proof. So we shall proceed slowly. I shall start with a relatively informal overview of the algorithm, which we shall call \intoDNF{}, and shall show how it applies to a particular input !!{formula}. With the ideas fixed by the example, I shall offer a rigorous proof that the algorithm does what we require.

Our algorithm, \intoDNF{}, takes an input !!{formula}, $\meta{S}$, and then brings us closer and closer to DNF. The initial step of the algorithm, then, is obvious:
	\begin{itemize}
		\item[\emph{Step 0.}] Write down your initial !!{formula}, $\meta{S}$.
	\end{itemize}
To make matters tractable, we shall work a particular application of the algorithm as we describe it. So, let us suppose that this is our input !!{formula}:
	\begin{align*}
	 ((\lnot A \lif \lnot (B \lor \lnot C)) &\lif \lnot (D \liff \lnot E))
	\end{align*}
The next step of the algorithm is designed to yield a tautologically equivalent !!{formula} which satisfies (\textsc{dnf1}). Otherwise put, our aim is to eliminate all occurrences of conditionals and biconditionals from a !!{formula}, whilst preserving tautological equivalence. So, here is what we should do:
	\begin{itemize}
		\item[\emph{Step 1a.}] Replace all sub!!{formula}s of the form $(\meta{A} \lif \meta{B})$ with $(\lnot\meta{A} \lor \meta{B})$. 
		\item[\emph{Step 1b.}] Replace all sub!!{formula}s of the form  $(\meta{A} \liff \meta{B})$ with $((\meta{A} \land \meta{B}) \lor (\lnot \meta{A} \land \lnot\meta{B}))$. 
	\end{itemize}
Here is the result of running through Step 1 on our example. (The markers on the left-side indicate the Step we have just finished.)
{\small	\begin{align*}
	 0: & & ((\lnot A \lif \lnot (B \lor \lnot  C)) &\lif \lnot (D \liff \lnot E))\\
	 \text{1a:} & & (\lnot (\lnot \lnot A \lor \lnot (B \lor \lnot  C)) &\lor \lnot (D \liff \lnot E)) \\
	 \text{1b:} & & (\lnot (\lnot \lnot A \lor \lnot (B \lor \lnot  C)) &\lor \lnot ((D \land \lnot E) \lor (\lnot D \land \lnot \lnot E)))
	\end{align*}}
We are a little closer to DNF, since this clearly satisfies (\textsc{dnf1}). 

The next step of \intoDNF{} is designed to bring us to a tautologically equivalent !!{formula} which satisfies (\textsc{dnf2}). Roughly, we need to `push' the negations as deeply into the !!{formula} as we possibly can. This is what the second step of \intoDNF{} does:
	\begin{itemize}
		\item[\emph{Step 2a:}] Replace all sub!!{formula}s of the form $\lnot\lnot\meta{A}$ with $\meta{A}$.
		\item[\emph{Step 2b:}] Replace the first sub!!{formula} of the form  $\lnot (\meta{A} \land \meta{B})$ with $(\lnot \meta{A} \lor \lnot \meta{B})$.\footnote{Here `first' means `first, working from left to right'. In fact, inspecting the proof of Lemma \ref{lem:IntoDNFHalts}, it is clear that the \emph{order} in which we perform these manipulations does not matter at all; we just need to do them one at a time.}
		\item[\emph{Step 2c:}] Replace the first sub!!{formula} of the form  $\lnot (\meta{A} \lor \meta{B})$ with $(\lnot \meta{A} \land \lnot \meta{B})$. 
		\item[\emph{Step 2d:}] Repeat Steps 2a--2c, until no further change can occur.\footnote{And we can check \emph{that} no further change can occur, just by repeatedly running through Steps 2a--2c, until changes stop occurring.}
	\end{itemize}
Applying Step 2 to to our example, we get the following sequence:
{\small	\begin{align*}
		\text{1:} & & (\lnot (\lnot \lnot A \lor \lnot (B \lor \lnot C)) &\lor \lnot ((D \land \lnot E) \lor (\lnot D \land \lnot \lnot E)))\\
		\text{2a:} & & (\lnot (A \lor \lnot (B \lor \lnot  C)) &\lor \lnot ((D \land \lnot E) \lor (\lnot D \land  E)))\\
		\text{2b:} & & (\lnot (A \lor \lnot (B \lor \lnot  C)) &\lor \lnot ((D \land \lnot E) \lor (\lnot D \land  E)))\\
		\text{2c:} & & ((\lnot A \land \lnot \lnot (B \lor \lnot  C)) &\lor \lnot ((D \land \lnot E) \lor (\lnot D \land  E)))\\
		\text{2a:} & & ((\lnot A \land (B \lor \lnot  C)) &\lor \lnot ((D \land \lnot E) \lor (\lnot D \land  E)))\\
		\text{2b:} & & ((\lnot A \land (B \lor \lnot  C)) &\lor \lnot ((D \land \lnot E) \lor (\lnot D \land  E)))\\
		\text{2c:} & & ((\lnot A \land (B \lor \lnot  C)) &\lor  (\lnot(D \land \lnot E) \land \lnot (\lnot D \land  E)))\\
		\text{2a:} & & ((\lnot A \land (B \lor \lnot  C)) &\lor  (\lnot(D \land \lnot E) \land \lnot (\lnot D \land  E)))\\
		\text{2b:} & & ((\lnot A \land (B \lor \lnot  C)) &\lor  ((\lnot D \land \lnot \lnot E) \land \lnot (\lnot D \land  E)))\\
		\text{2c:} & & ((\lnot A \land (B \lor \lnot  C)) &\lor  ((\lnot D \lor \lnot \lnot E) \land \lnot (\lnot D \land  E)))\\
		\text{2a:} & & ((\lnot A \land (B \lor \lnot  C)) &\lor  ((\lnot D \lor E) \land \lnot (\lnot D \land  E)))\\
		\text{2b:} & & ((\lnot A \land (B \lor \lnot  C)) &\lor  ((\lnot D \lor E) \land (\lnot \lnot D \lor  \lnot E)))\\
		\text{2c:} & & ((\lnot A \land (B \lor \lnot  C)) &\lor  ((\lnot D \lor E) \land (\lnot \lnot D \lor  \lnot E)))\\
		\text{2a:} & & ((\lnot A \land (B \lor \lnot  C)) &\lor  ((\lnot D \lor E) \land (D \lor  \lnot E))) 
	\end{align*}}\noindent 
Note that the !!{formula} does not necessarily change at every sub-step. But after the last written step, no further changes \emph{can} occur; so we are done with Step 2. And we are even closer to DNF, since this satisfies (\textsc{dnf2}). 

All that remains is to meet condition (\textsc{dnf3}), i.e.\ to ensure that no disjunction occurs within the scope of any conjunction. Roughly, we need to `push' the conjunctions into the scope of disjunctions. And that is the idea behind the final step of \intoDNF:
	\begin{itemize}
		\item[\emph{Step 3a:}]  Replace the first sub!!{formula} of the form $(\meta{A} \land (\meta{B} \lor \meta{C}))$ with $((\meta{A} \land \meta{B}) \lor (\meta{A} \land \meta{C}))$. 		
		\item[\emph{Step 3b:}]  Replace the first sub!!{formula} of the form $((\meta{A} \lor \meta{B}) \land \meta{C})$ with $((\meta{A} \land \meta{C}) \lor (\meta{B} \land \meta{C}))$. 
	\item[\emph{Step 3c:}] Repeat Steps 3a--3b, until no further changes can occur.
	\end{itemize}
Applying Step 3 to our example, we get:
{\small	\begin{align*}
		\text{2}: & & ((\lnot A \land (B \lor \lnot  C)) &\lor  ((\lnot D \lor E) \land (D \lor  \lnot E)))\\
		\text{3a}: & & (((\lnot A \land B) \lor (\lnot A \land \lnot  C)) &\lor  ((\lnot D \lor E) \land (D \lor  \lnot E)))\\	
		\text{3b}: & & (((\lnot A \land B) \lor (\lnot A \land \lnot  C)) &\lor  ((\lnot D \land (D \lor \lnot E)) \lor (E \land (D \lor  \lnot E))))\\
		\text{3a}: & & (((\lnot A \land B) \lor (\lnot A \land \lnot  C)) &\lor  (((\lnot D \land D) \lor (\lnot D \land \lnot E)) \lor (E \land (D \lor  \lnot E))))\\
		\text{3b}: & & (((\lnot A \land B) \lor (\lnot A \land \lnot  C)) &\lor  (((\lnot D \land D) \lor (\lnot D \land \lnot E)) \lor (E \land (D \lor  \lnot E))))\\
		\text{3a}: & & (((\lnot A \land B) \lor (\lnot A \land \lnot  C)) &\lor  (((\lnot D \land D) \lor (\lnot D \land \lnot E)) \lor ((E \land D) \lor  (E \land \lnot E))))
	\end{align*}}\noindent 
And this is in DNF, as may become clearer if we \emph{temporarily} allow ourselves to remove brackets, following the notational conventions of ForAllX{} \S10.3:
		\begin{align*}
		(\lnot A \land B) \lor (\lnot A \land \lnot  C) \lor  (\lnot D \land D) \lor (\lnot D \land \lnot E) \lor (E \land D) \lor  (E \land \lnot E)
		\end{align*}
So we are done! Finally, it is worth noting that this algorithm -- although lengthy -- was \emph{much} more efficient at generating a DNF !!{formula} than the truth-table method would have been. In fact, the truth table for our example would have $32$ lines, $22$ of which are true.

\
\\So far, I have defined the algorithm, \intoDNF, via Steps 0 through to 3c. I have also provided a worked-application of \intoDNF. But it remains to \emph{prove} that \intoDNF{} really is fit for purpose. 

First, we need to prove that any outputs from \intoDNF{} are tautologically equivalent to the inputs. This is actually quite simple:
\begin{lem}\label{lem:IntoDNFEquivalent}
	For any input !!{formula}: any outputs of \intoDNF{} are tautologically equivalent to the input !!{formula}.
	\begin{proof}
		At Step 0, we simply write our input. Each instance of Steps 1a--3b then involves replacing a sub!!{formula} with a tautologically equivalent sub!!{formula}. In particular, we make use of the following equivalences:
	\begin{align*}
		\text{in Step 1a:} & & (\meta{A} \lif \meta{B}) &\cong (\lnot \meta{A} \lor \meta{B})\\
		\text{in Step 1b:} & & (\meta{A} \liff \meta{B}) &\cong ((\meta{A} \land \meta{B}) \lor (\lnot \meta{A} \land \lnot\meta{B}))\\
		\text{in Step 2a:} & & \lnot \lnot \meta{A} &\cong \meta{A}\\
		\text{in Step 2b:} & & \lnot (\meta{A} \land \meta{B})&\cong(\lnot \meta{A} \lor \lnot \meta{B})\\
		\text{in Step 2c:} & & \lnot (\meta{A} \lor \meta{B}) &\cong (\lnot \meta{A} \land \lnot \meta{B})\\
		\text{in Step 3a:} & & 		(\meta{A} \land (\meta{B} \lor \meta{C})) &\cong ((\meta{A} \land \meta{B}) \lor (\meta{A} \land \meta{C}))\\
		\text{in Step 3b:} & & ((\meta{A} \lor \meta{B}) \land \meta{C}) &\cong ((\meta{A} \land \meta{C}) \lor (\meta{B} \land \meta{C}))
		\end{align*}
	Now, by Lemma \ref{lem:SubstitutionTautEquivalent}, if we replace some sub!!{formula} of a !!{formula} with a tautologically equivalent sub!!{formula}s, we obtain a !!{formula} that is tautologically equivalent to what we started with. So any instance of any Step of \intoDNF{} preserves tautological equivalence. So, provided the algorithm yields an output within finitely many steps, that output is indeed (by a trivial induction) tautologically equivalent to the input.
	\end{proof}
\end{lem}\noindent
The next task is to show that \intoDNF{} \emph{always} yields an output. We need to ensure, for example, that the algorithm never gets trapped on an infinite loop. At the same time as we prove this, we shall also show that the output is in DNF, as we would hope. 

Unfortunately, the proof is surprisingly difficult. So I shall start with an intuitive overview of how the proof. We will aim, intuitively, to show each of the following:
\begin{itemize}	
	\item \emph{The output of Step 1 is a !!{formula} satisfying} (\textsc{dnf1}).

	This will be fairly trivial.
	
	\item \emph{The output of Step 2 is a !!{formula} also satisfying} (\textsc{dnf2}). 
	
	Intuitively, this is because the negation-signs get pushed progressively deeper into the formula, until they precede only atomic !!{formula}s.
	
	\item \emph{The output of Step 3 is a !!{formula} also satisfying} (\textsc{dnf3}).
	
	Intuitively, this is because the conjunction-signs get pushed progressively deeper into the formula, until they have no disjunctions in their scope.
\end{itemize}
That is the main idea. As a student, your aim should be to grasp this main idea fully. Morally speaking, it just \emph{must} be right. And, for just this reason, many books at this level leave matters here, and claim that they have proved the DNF Theorem. But, \emph{just because we can},\footnote{Can, and \emph{therefore should}? There is a deep philosophical-cum-technical question: how much rigour does a proof require? But this is a book in formal logic, not philosophical logic; so I leave this as a cliffhanger.} here is a full proof, which brings these intuitive points into sharp relief:
\begin{lem}\label{lem:IntoDNFHalts}
	On any input, \intoDNF{} is guaranteed to terminate, yielding an output in DNF.
	\begin{proof}
		\emph{Concerning Step 0:} trivially, we will complete Step 0 (eventually), since this Step just involves writing down a single (finitely long) input !!{formula}.
		
		\emph{Concerning Step 1:} The input !!{formula} only contained a finite number of instances of `$\lif$' and `$\liff$'. These will all be eliminated (eventually) during Step 1. As such, the output at the end of Step 1 satisfies (\textsc{dnf1}). 
		
		\emph{Concerning Step 2:} This is a bit harder. For each !!{formula} $\meta{A}$, we define its not-height, $nh(\meta{A})$, recursively:\footnote{Why do we define not-height this way? The glib answer is: because \emph{it works} in the proof. In fact, this glib answer really says it all. I have deliberately manufactured an arithmetical function, $nh$, with no other end in mind, than that successive applications of Steps 2a--2c of the algorithm are guaranteed to yield a decreasing sequence of $nh$ values.}
			\begin{align*}
				nh(\meta{A}) &= 1, \text{if }\meta{A}\text{ is atomic}\\
				nh(\lnot \meta{A}) &= 3\times{nh(\meta{A})} - 1\\
				nh(\meta{A} \land \meta{B}) = nh(\meta{A} \lor \meta{B}) &= nh(\meta{A}) + nh(\meta{B})
			\end{align*}
		I claim that, \emph{if an instance of Step 2a, 2b or 2c has any effect on the !!{formula}, then it reduces the !!{formula}'s not-height.} In the case of Step 2a, observe that:
		\begin{align*}
		nh(\lnot\lnot\meta{A}) &= 3 \times nh(\lnot\meta{A}) -1 \\
		&= 3\times(3 \times nh(\meta{A})-1) -1 \\
		& = 9\times nh(\meta{A}) - 4
		\end{align*}
		and this is greater than $nh(\meta{A})$ for any $\meta{A}$, since a !!{formula}'s not-height must always be at least $1$. In the case of Step 2b, observe that:
	\begin{align*}
				nh(\lnot(\meta{A} \land \meta{B})) &= 3\times (nh(\meta{A} \land \meta{B})) -1 \\
				&= 3 \times(nh(\meta{A}) + nh(\meta{B})) - 1\\\
			\end{align*}
		and this is always greater than:
		\begin{align*}
		nh(\lnot\meta{A} \lor \lnot \meta{B}) &= nh(\lnot \meta{A}) + nh(\lnot\meta{B})\\
		&= 3\times nh(\meta{A}) -1 + 3 \times nh(\meta{B}) - 1\\
		&= 3\times(nh(\meta{A}) + nh(\meta{B})) - 2
		\end{align*}
		The case of Step 2c is exactly similar, and this establishes the claim. 
		
		Consequently, repeatedly running through Steps 2a--2c yields a sequence of !!{formula}s with decreasing not-heights. Since the input !!{formula} to Step 2 (i.e.\ the output from Step 1) has some finite not-height, after a finite number of applications of Steps 2a--2c, it will be impossible to reduce the not-height any further. This brings Step 2 to an end. Trivially, the resulting !!{formula} still satisfies (\textsc{dnf1}). And it also satisfies (\textsc{dnf2}), since the not-height of any !!{formula} satisfying (\textsc{dnf1}) but not satisfying (\textsc{dnf2}) clearly \emph{can} be decreased by some application of Step 2a, 2b or 2c.
		
		\emph{Concerning Step 3:} This is like Step 2. For each !!{formula} $\meta{A}$, we define its conjunction-height, $ch(\meta{A})$, recursively:
			\begin{align*}
				ch(\meta{A}) &= 2, \text{if }\meta{A}\text{ is atomic}\\
				ch(\lnot \meta{A}) &= ch(\meta{A})\\
				ch(\meta{A} \land \meta{B}) &= ch(\meta{A}) \times ch(\meta{B})\\
				ch(\meta{A} \lor \meta{B}) &=  ch(\meta{A}) + ch(\meta{B}) + 1
			\end{align*}
		I claim that, \emph{if an instance of Step 3a or 3b has any effect on the !!{formula}, then it reduces the !!{formula}'s conjunction-height}. In the case of Step 3a, observe that 
			\begin{align*}
				ch(\meta{A} \land (\meta{B} \lor \meta{C})) &= ch(\meta{A}) \times  (ch(\meta{B}) + ch(\meta{C}) + 1)\\
					&= ch(\meta{A})\times ch(\meta{B}) + ch(\meta{A}) \times ch(\meta{C}) + ch(\meta{A})
			\end{align*}
		and, since $ch(\meta{A}) > 1$ for any $\meta{A}$, this must be larger than:
			\begin{align*}
				ch((\meta{A} \land \meta{B}) \lor (\meta{A} \land \meta{C})) &= ch(\meta{A}) \times ch(\meta{B}) + ch(\meta{A}) \times ch(\meta{C}) + 1 
			\end{align*}
		The case of Step 3b is exactly similar. This establishes the claim. 
		
		Now, as above: running through Steps 3a--3b repeatedly yields a sequence of !!{formula}s with decreasing conjunction-heights. So, after a finite number of applications of Steps 3a--3b, it will be impossible to reduce the conjunction-height any further. Trivially, the resulting !!{formula} still satisfies (\textsc{dnf1}) and (\textsc{dnf2}). And it also satisfies (\textsc{dnf3}), since the conjunction-height of any !!{formula} satisfying both (\textsc{dnf1}) and (\textsc{dnf2}) but not (\textsc{dnf3}) \emph{can} be decreased by some application of Step 3a or 3b.
	\end{proof}
\end{lem}\noindent
Now if we just combine combine Lemma \ref{lem:IntoDNFEquivalent} with Lemma \ref{lem:IntoDNFHalts}, we have a new proof of the DNF Theorem, via substitution.





\subsection{Cleaning up DNF !!{formula}s}\label{s:DNFCleanUp}
I have offered two different proofs of the DNF Theorem. They also described two different algorithms for putting !!{formula}s into DNF. But these algorithms do not always leave you with the most elegant output. Indeed, just as we can `clean up' the output of the algorithm for calculating interpolants (see \S\ref{s:Interpolation}), we can `clean up' the output of our algorithms for calculating DNF-equivalents. Here are the clean-up rules; I leave it as an exercise to \emph{justify} them. First, we rewrite our DNF !!{formula} using the relaxed notational conventions of ForAllX{} \S10.3. Now we apply the following:

\
\\\emph{Rule 1: remove repetitious conjuncts.}  If any disjunct contains the same conjunct more than once, remove all but one instance of that conjunct (respecting bracketing conventions as appropriate).
\\\emph{Example:} if the output is:
$$(A \land B) \lor (A \land \lnot C) \lor D \lor (A \land B)$$
then applying Rule 1 we get:
$$(A \land B) \lor (A \land \lnot C) \lor D$$
\noindent \emph{Rule 2: remove overly-specific disjuncts.} Delete any disjunct which entails any other disjunct. (If two or more disjuncts entail each other, delete all but the first of those disjuncts.)
\\\emph{Example:} the output of our example in \S\ref{s:DNFsubstitution} was:
		\begin{align*}
		(A \land B) \lor (A \land \lnot C) &\lor  (\lnot D \land D) \lor (\lnot E \land D) \lor (\lnot D \land E) \lor (\lnot E \land  E)		
		\end{align*}
		Noting that a contradiction entails \emph{everything}, we should remove both contradictory disjuncts, obtaining (still with relaxed conventions):
		\begin{align*}
		(A \land B) \lor (A \land \lnot C) &\lor  (\lnot E \land D) \lor (\lnot D \land E)
		\end{align*}		
\emph{Rule 3: invoke excluded middle.} Suppose there is a disjunct whose conjuncts are just $\meta{A}_1, \ldots,  \meta{A}_m$ and $\meta{B}$ (in any order), and another disjunct whose conjuncts are just $\meta{A}_1, \ldots, \meta{A}_m$ and $\lnot \meta{B}$ (in any order), so that the two disjuncts disagree only on whether whether $\meta{B}$ should be prefixed with a negation. In that case, delete both disjuncts and replace them with the simpler disjunct $(\meta{A}_1 \land \ldots \land \meta{A}_m)$. (However, if we have \emph{just} $\meta{B}$ and $\lnot\meta{B}$ as disjuncts, then delete the entire !!{formula} and leave the tautology $(\meta{B} \lor \lnot \meta{B})$.)
\\\emph{Example:} the output of the example in \S\ref{s:DNFTruthTable} was:
		$$(A \land B \land C) \lor (A \land \lnot B \land C) \lor (\lnot A \land \lnot B \land C) \lor (\lnot A \land \lnot B \land \lnot C)$$
			which can be simplified to:
		$$(A \land C) \lor (\lnot A \land \lnot B)$$


\subsection{Conjunctive Normal Form}\label{s:CNF}
So far in this chapter, I have discussed \emph{disjunctive} normal form. Given the duality of disjunction and conjunction (see \S\ref{s:Duality}), it may not come as a surprise to hear that there is also such a thing as \emph{conjunctive normal form} (CNF).

The definition of CNF is exactly analogous to the definition of DNF. So, a !!{formula} is in CNF \emph{iff} it meets all of the following conditions:
	\begin{itemize}
		\item[(\textsc{cnf1})] No connectives occur in the !!{formula} other than negations, conjunctions and disjunctions;
		\item[(\textsc{cnf2})] Every occurrence of negation has minimal scope;
		\item[(\textsc{cnf3})] No conjunction occurs within the scope of any disjunction. 
	\end{itemize}
Generally, then, a !!{formula} in CNF looks like this
	$$(\pm \meta{A}_1 \lor \ldots \lor \pm \meta{A}_i) \land (\pm \meta{A}_{i+1} \lor \ldots \lor \pm\meta{A}_j) \land \ldots \land (\pm\meta{A}_{m+1} \lor\ldots \lor \pm \meta{A}_n)$$
where each $\meta{A}_k$ is an atomic !!{formula}.

Since `$\lnot$' is its own dual, and `$\lor$' and `$\land$' are the duals of each other, it is immediate clear that if a !!{formula} is in DNF, then its dual is in CNF; and \emph{vice versa}. Armed with this insight, we can immediately prove another normal form theorem:
	\begin{thm}\label{thm:cnf}\textbf{Conjunctive Normal Form Theorem.} For any !!{formula}, there is a tautologically equivalent !!{formula} in conjunctive normal form.
\begin{proof}
	Let $\meta{S}$ be any !!{formula}. Applying the DNF Theorem to $\dual{\meta{S}}$, there is a DNF !!{formula}, $\meta{D}$, such that $\dual{\meta{S}} \cong \meta{D}$. So, by Theorem \ref{thm:DualEquivalence}, $\dual{\dual{\meta{S}}} \cong \dual{\meta{D}}$. Whence, by Lemma \ref{lem:DualDual}, $\meta{S} \cong \dual{\meta{D}}$. But, since $\meta{D}$ is in DNF, $\dual{\meta{D}}$ is in CNF. 	
\end{proof}
\end{thm}\noindent
This slick proof is a further illustration of the power of duality. However, it might suggest that the DNF Theorem enjoys some kind of `precedence' over the CNF Theorem. That would be misleading.  We can easily prove the CNF Theorem directly, using either of the two proof techniques that we used to prove the DNF Theorem (whereupon the DNF Theorem could be proved as a consequence of the CNF Theorem and duality). I shall sketch the main ideas of the direct proofs of the CNF Theorem; I leave it as an exercise to make these ideas more precise.
\begin{proof}[Proof sketch of CNF Theorem via truth tables]
	Given a TFL !!{formula}, $\meta{S}$, we begin by writing down the complete truth table for $\meta{S}$.
	
	If $\meta{S}$ is \emph{true} on every line of the truth table, then $\meta{S} \cong (\meta{A}_1 \lor \lnot \meta{A}_1)$.
	
	If $\meta{S}$ is \emph{false} on at least one line of the truth table then, for every line on the truth table where $\meta{S}$ is false, write down a disjunction $(\pm\meta{A}_1 \lor \ldots \lor \pm\meta{A}_n)$ which is \emph{false} on (and only on) that line. Let $\meta{C}$ be the conjunction of all of these disjuncts; by construction, $\meta{C}$ is in CNF and $\meta{S} \cong \meta{C}$.
\end{proof}
\begin{proof}[Proof sketch of CNF Theorem via substitution.] Using Steps 0--2 of \intoDNF, we  obtain a !!{formula} satisfying (\textsc{cnf1}) and (\textsc{cnf2}). To turn this into a !!{formula} in CNF, we simply modify Step 3, using substitutions based on the duals (what else!)\ of the Distribution Laws used in Step 3 of \intoDNF{}; i.e.:
	\begin{align*}
		((\meta{A} \land \meta{B}) \lor \meta{C}) &\cong ((\meta{A} \lor \meta{C}) \land (\meta{B} \lor \meta{C}))\\
		(\meta{A} \lor (\meta{B} \land \meta{C})) &\cong ((\meta{A} \lor \meta{B}) \land (\meta{A} \lor \meta{C}))	
	\end{align*}
This allows us to pull conjunctions outside the scope of any disjunction.
\end{proof}\noindent
Of course, after obtaining a !!{formula} in CNF -- however we chose to do it -- we can clean-up the resulting !!{formula} using rules similar to those of \S\ref{s:DNFCleanUp}. I leave it as an exercise, to determine what the appropriate rules should be.

\begin{prob}
\label{pr.DNF}
Consider the following !!{formula}s:
	\begin{itemize}
		\item $(A \lif \lnot B)$
		\item $\lnot (A \liff B)$
		\item $(\lnot A \lor \lnot (A \land B))$
		\item $(\lnot (A \lif B ) \land (A \lif C))$
		\item $(\lnot (A \lor B) \liff ((\lnot C \land \lnot A) \lif \lnot B))$
		\item $((\lnot (A \land \lnot B) \lif C) \land \lnot (A \land D))$
	\end{itemize}
For each !!{formula}:
	\begin{itemize}
		\item use both algorithms (by truth table, and by substitution) to write down !!{formula}s in DNF that are tautologically equivalent to these !!{formula}s.
		\item use both algorithms to write down !!{formula}s in CNF that are tautologically equivalent to these !!{formula}s. 
	\end{itemize}
\end{prob}

\begin{prob}
Offer proofs of the two `trivial inductions' mentioned in the proof of Theorem \ref{thm:dnf} via truth tables.\\
\end{prob}

\begin{prob}
Explain why the proof of Theorem \ref{lem:IntoDNFEquivalent} required `a trivial induction' (in the last !!{formula}). Prove that induction.\\
\end{prob}

\begin{prob}
Justify each of the rules mentioned in \S\ref{s:DNFCleanUp}, i.e.\ prove that applying each of them preserves tautological equivalence.\\
\end{prob}

\begin{prob}
Fill out the details of the two proof sketches of the CNF Theorem. Determine the appropriate `clean up' rules for CNF !!{formula}s.
\end{prob}

\end{document}