% Part: first-order-logic
% Chapter: syntax-and-semantics
% Section: formation-sequences

\documentclass[../../../include/open-logic-section]{subfiles}

\begin{document}

\olfileid{pl}{syn}{fseq}

\olsection{Formation Sequences}

Defining !!{formula}s via an inductive definition, and the
complementary technique of proving properties of !!{formula}s via
induction, is an elegant and efficient approach. However, it can
also be useful to consider a more bottom-up, step-by-step approach
to the construction of !!{formula}s, which we do here using the
notion of a \emph{formation sequence}.

\begin{defn}[Formation sequences for formulas]
\ollabel{defn:fseq-frm}
A finite sequence $\tuple{!A_0,\dotsc,!A_n}$ of strings of
symbols from the language~$\Lang L_0$ is a \emph{formation
sequence} for $!A$ if $!A \ident !A_n$ and for all $i \leq n$,
either $!A_i$ is an atomic formula or there exist $j,k < i$
such that one of the following holds:
\begin{enumerate}
    \tagitem{prvNot}{$!A_i \ident \lnot !A_j$.}{}%
    \tagitem{prvAnd}{$!A_i \ident (!A_j \land !A_k)$.}{}%
    \tagitem{prvOr}{$!A_i \ident (!A_j \lor !A_k)$.}{}%
    \tagitem{prvIf}{$!A_i \ident (!A_j \lif !A_k)$.}{}%
    \tagitem{prvIff}{$!A_i \ident (!A_j \liff !A_k)$.}{}%
\end{enumerate}
\end{defn}

\begin{ex}
\[
\tuple{
    \Obj p_0,
    \Obj p_1,
    (\Obj p_1 \land \Obj p_0),
    \lnot (\Obj p_1 \land \Obj p_0)
}
\]
is a formation sequence of
$\lnot (\Obj p_1 \land \Obj p_0)$, as is
\[
\tuple{
    \Obj p_0,
    \Obj p_1,
    \Obj p_0,
    (\Obj p_1 \land \Obj p_0),
    (\Obj p_0 \lif \Obj p_1),
    \lnot (\Obj p_1 \land \Obj p_0)
}.
\]
%
As can be seen from the second example, formation sequences
may contain `junk': formulas which are redundant or do not
contribute to the construction.
\end{ex}

\begin{prop}
\ollabel{prop:formed}
Every !!{formula}~$!A$ in~$\Frm[L_0]$ has a formation sequence.
\end{prop}

\begin{proof}
Suppose $!A$ is atomic. Then the sequence~$\tuple{!A}$ is a
formation sequence for~$!A$.
%
Now suppose that $!B$ and~$!C$ have formation sequences
$\tuple{!B_0,\dotsc,!B_n}$ and $\tuple{!C_0,\dotsc,!C_m}$
respectively.
%
\begin{enumerate}
  \tagitem{prvNot}{If $!A \ident \lnot !B$,
      then $\tuple{!B_0,\dotsc,!B_n,\lnot !B_n}$
      is a formation sequence for~$!A$.}{}
  \tagitem{prvAnd}{If $!A \ident (!B \land !C)$,
      then $\tuple{!B_0,\dotsc,!B_n,!C_0,\dotsc,!C_m,(!B_n \land !C_m)}$
      is a formation sequence for~$!A$.}{}
  \tagitem{prvOr}{If $!A \ident (!B \lor !C)$,
      then $\tuple{!B_0,\dotsc,!B_n,!C_0,\dotsc,!C_m,(!B_n \lor !C_m)}$
      is a formation sequence for~$!A$.}{}
  \tagitem{prvIf}{If $!A \ident (!B \lif !C)$,
      then $\tuple{!B_0,\dotsc,!B_n,!C_0,\dotsc,!C_m,(!B_n \lif !C_m)}$
      is a formation sequence for~$!A$.}{}
  \tagitem{prvIff}{If $!A \ident (!B \liff !C)$,
      then $\tuple{!B_0,\dotsc,!B_n,!C_0,\dotsc,!C_m,(!B_n \liff !C_m)}$
      is a formation sequence for~$!A$.}{}
\end{enumerate}
By the principle of induction on !!{formula}s,
every !!{formula} has a formation sequence.
\end{proof}

We can also prove the converse. This is important because it shows
that our two ways of defining formulas are equivalent: they give
the same results. It also means that we can prove theorems about
formulas by using ordinary induction on the length of formation
sequences.

\begin{lem}
\ollabel{lem:fseq-init}
Suppose that $\tuple{!A_0,\dotsc,!A_n}$ is a formation sequence
for~$!A_n$, and that $k \leq n$. Then $\tuple{!A_0,\dotsc,!A_k}$
is a formation sequence for~$!A_k$.
\end{lem}

\begin{proof}
Exercise.
\end{proof}

\begin{thm}
\ollabel{thm:fseq-frm-equiv}
$\Frm[L_0]$ is the set of all strings of symbols
in the language~$\Lang L_0$ with a formation sequence.
\end{thm}

\begin{proof}
Let $F$ be the set of all strings of symbols in the
language~$\Lang L_0$ that have a formation sequence.
We have seen in \olref[pl][syn][fseq]{prop:formed} that
$\Frm[L_0] \subseteq F$, so now we prove the converse.

Suppose $!A$ has a formation sequence $\tuple{!A_0,\dotsc,!A_n}$.
We prove that $!A \in \Frm[L_0]$ by strong induction on~$n$.
Our induction hypothesis is that every string of symbols with a
formation sequence of length $m < n$ is in~$\Frm[L_0]$.
By the definition of a formation sequence, either $!A_n$ is
atomic or there must exist $j,k < n$ such that one of the
following is the case:
\begin{enumerate}
    \tagitem{prvNot}{$!A_n \ident \lnot !A_j$.}{}%
    \tagitem{prvAnd}{$!A_n \ident (!A_j \land !A_k)$.}{}%
    \tagitem{prvOr}{$!A_n \ident (!A_j \lor !A_k)$.}{}%
    \tagitem{prvIf}{$!A_n \ident (!A_j \lif !A_k)$.}{}%
    \tagitem{prvIff}{$!A_n \ident (!A_j \liff !A_k)$.}{}%
\end{enumerate}
Now we reason by cases. If $!A_n$ is atomic then
$!A_n \in \Frm[L_0]$. Suppose instead that $!A \equiv
(!A_j \land !A_k)$. By \olref[pl][syn][fseq]{lem:fseq-init},
$\tuple{!A_0,\dotsc,!A_j}$ and $\tuple{!A_0,\dotsc,!A_k}$ are
formation sequences for $!A_j$ and~$!A_k$ respectively. Since
these are proper initial subsequences of the formation sequence
for~$!A$, they both have length less than~$n$. Therefore by
the induction hypothesis, $!A_j$ and~$!A_k$ are in~$\Frm[L_0]$,
and so by the definition of !!a{formula}, so is
$(!A_j \land !A_k)$. The other cases follow by parallel
reasoning.
\end{proof}

\end{document}
