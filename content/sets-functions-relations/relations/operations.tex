% Part: sets-functions-relations
% Chapter: relations
% Section: operations

\documentclass[../../../include/open-logic-section]{subfiles}

\begin{document}

\olfileid{sfr}{rel}{ops}
\olsection{Operations on Relations}

It is often useful to modify or combine relations. We've already used
the union of relations above (which is just the union of two relations
considered as sets of pairs). Here are some other ways:

\begin{defn} Let $R$, $S \subseteq X^2$ be relations and $Y$ a set.
\begin{enumerate}
\item The \emph{inverse}~$R^{-1}$ of $R$ is $R^{-1} = \Setabs{\tuple{y,
    x}}{\tuple{x, y} \in R}$.
\item The \emph{relative product}~$R \mid S$ of $R$ and $S$ is
\[
(R \mid S) = \{\tuple{x, z} : \text{for some } y, Rxy \text{ and } Syz\}
\]
\item The \emph{restriction}~$R \restrict Y$ of $R$ to $Y$ is $R \cap Y^2$
\item The \emph{application}~$R[Y]$ of $R$ to $Y$ is
\[
R[Y] = \{y : \text{for some } x \in Y, Rxy\}
\]
\end{enumerate}
\end{defn}

\begin{ex}
Let $S \subseteq \Int^2$ be the successor relation on~$\Int$, i.e.,
the set of pairs $\tuple{x, y}$ where $x + 1 = y$, for $x, y \in
\Int$. $Sxy$ holds iff $y$ is the successor of~$x$.
\begin{enumerate}
\item The inverse $S^{-1}$ of $S$ is the predecessor relation, i.e.,
  $S^{-1}xy$ iff $x-1 = y$.
\item The relative product $S\mid S$ is the relation $x$ bears to $y$
  if $x+2 = y$.
\item The restriction of $S$ to $\Nat$ is the successor relation
  on~$\Nat$.
\item The application of $S$ to a set, e.g., $S[\{1, 2, 3\}]$ is $\{2,
  3, 4\}$.
\end{enumerate}

\end{ex}

\begin{defn}[Transitive closure]
The \emph{transitive closure}~$R^+$ of a relation $R \subseteq X^2$ is
$R^+ = \bigcup_{i=1}^\infty R^i$ where $R^1 = R$ and $R^{i+1} = R^i
\mid R$.

The \emph{reflexive transitive closure} of $R$ is $R^* = R^+ \cup
\Id{X}$.
\end{defn}

\begin{ex}
Take the successor relation $S \subseteq \Int^2$. $S^2xy$ iff $x + 2 =
y$, $S^3xy$ iff $x + 3 = y$, etc. So $R^*xy$ iff for some $i \ge 1$,
$x + i = y$. In other words, $S^+xy$ iff $x < y$ (and $R^*xy$ iff $x
\le y$).
\end{ex}

\begin{prob}
Show that the transitive closure of $R$ is in fact transitive.
\end{prob}

\end{document}

