% Part: sets-functions-relations
% Chapter: relations
% Section: trees

\documentclass[../../../include/open-logic-section]{subfiles}

\begin{document}

\olfileid{sfr}{rel}{tre}
\olsection{Trees}

A particular kind of partial order which plays an important role in
all parts of logic is a \emph{tree}. Finite trees occur in elementary
parts of logic: for example, !!{formula}s can be understood in terms
of their decomposition into a syntax tree, while !!{derivation}s in
natural deduction also take the form of a finite tree.
%
Infinite trees appear already in the proof of the completeness
theorems for propositional and first-order logic, and are used
throughout mathematical logic. For example, in descriptive set theory,
many pointclasses of real numbers (such as Borel sets or analytic sets)
have representations in terms of trees.

\begin{defn}[Tree]
A \emph{tree} is a pair $T = \tuple{X,\le}$ such that $X$ is a set
and $\le$ is a partial order on $X$ with a unique minimal element
$r \in X$ (called a \emph{root}) such that for all $t \in X$,
the set $\Setabs{s}{s \le t}$ is well-ordered by $\le$.
\end{defn}

\begin{defn}[Successors]
Suppose $T = \tuple{X,\le}$ is a tree.
%
If $t,s \in X$, $t < s$, and there is no $s' \in X$ such that
$t < s' < s$, then we say that $s$ is a \emph{successor} of $t$.
\end{defn}

\begin{defn}[Infinite and finitely branching trees]
Suppose that $T = \tuple{X,\le}$ is a tree.
%
$T$ is said to be \emph{infinite} if $X$ is an infinite set,
\emph{finite} otherwise.
%
If $T$ is such that every $t \in X$ has only finitely many
successors, then we say that $T$ is \emph{finitely branching}.
\end{defn}

\begin{defn}[Branches]
Given a tree $T = \tuple{X,\le}$, a \emph{branch} of $T$ is a
maximal chain in $T$, i.e.\ a set $B \subseteq X$ such that
for any $a,b \in B$ either $a \le b$ or $b \le a$, and for any
$c \in X \setminus B$ there exists $d \in B$ such that neither
$c \le d$ nor $d \le c$.
%
We use $[T]$ to denote the set of all branches of $T$.
\end{defn}

\begin{ex}
A classic example of a finitely branching tree is the
\emph{binary tree} of finite sequences of $0$s and $1$s,
sometimes denoted $\{0,1\}^*$, ordered by the extension
relation $\sqsubseteq$ (e.g., $101 \sqsubseteq 101101$).
Since any binary string can always be extended by adding
a $0$ or a $1$ on the end, this tree contains infinitely
many elements. Its root is the empty sequence $\emptyseq$.
\end{ex}

\begin{prop}[K\H{o}nig's lemma]
If $T = \tuple{X,\le}$ is a finitely branching infinite tree,
then $T$ has an infinite branch.
\end{prop}

A special case of K\H{o}nig's lemma widely used in
computability theory, known as \emph{weak K\H{o}nig's lemma},
is the following:
any infinite subtree of $\{0,1\}^*$ has an infinite branch.

\end{document}
