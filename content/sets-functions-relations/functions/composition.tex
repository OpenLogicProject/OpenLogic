% Part: sets-functions-relations
% Chapter: functions
% Section: composition

\documentclass[../../../include/open-logic-section]{subfiles}

\begin{document}

\olfileid{sfr}{fun}{cmp}
\olsection{Composition of Functions}

\begin{explain}
\oliflabeldef{sfr:fun:inv:sec}{We saw in \olref[inv]{sec} that the
inverse~$f^{-1}$ of !!a{bijection}~$f$ is itself a function. Another
operation on functions is composition: w}{W}e can define a new
function by composing two functions, $f$ and~$g$, i.e., by first
applying $f$ and then~$g$. Of course, this is only possible if the
ranges and domains match, i.e., the range of~$f$ must be a subset of
the domain of~$g$. \oliflabeldef{sfr:rel:ops:sec}{This operation on
functions is the analogue of the operation of relative product on
relations from \olref[rel][ops]{sec}.}{}

A diagram might help to explain the idea of composition. In
\olref{fig:composition}, we depict two functions $f \colon A \to B$
and $g \colon B \to C$ and their composition~$(\comp{f}{g})$. The
function $(\comp{f}{g}) \colon A \to C$ pairs each !!{element} of~$A$
with !!a{element} of~$C$. We specify which !!{element} of~$C$
!!a{element} of $A$ is paired with as follows: given an input $x \in
A$, first apply the function $f$ to~$x$, which will output some $f(x)
= y \in B$, then apply the function $g$ to~$y$, which will output some
$g(f(x)) = g(y) = z \in C$.
\begin{figure}
  \olasset{assets/diagrams/composition.tikz}
  \caption{The composition $g \circ f$ of two functions $f$ and~$g$.}
  \ollabel{fig:composition}
\end{figure}
\end{explain}

\begin{defn}[Composition]
Let $f\colon A \to B$ and $g\colon B \to C$ be functions. The
\emph{composition} of $f$ with~$g$ is $\comp{f}{g} \colon A \to C$,
where $(\comp{f}{g})(x) = g(f(x))$.
\end{defn}

\begin{ex}
Consider the functions $f(x) = x + 1$, and $g(x) = 2x$. Since
$(\comp{f}{g})(x) = g(f(x))$, for each input~$x$ you must first take
its successor, then multiply the result by two. So their composition
is given by $(\comp{f}{g})(x) = 2(x+1)$.
\end{ex}

\begin{prob}
Show that if $f \colon A \to B$ and $g \colon B \to C$ are both
!!{injective}, then $\comp{f}{g}\colon A \to C$ is !!{injective}.
\end{prob}

\begin{prob}
Show that if $f \colon A \to B$ and $g \colon B \to C$ are both
!!{surjective}, then $\comp{f}{g}\colon A \to C$ is !!{surjective}.
\end{prob}

\begin{prob}
Suppose $f \colon A \to B$ and $g \colon B \to C$. Show that the graph
of $\comp{f}{g}$ is $R_f \mid R_g$.
\end{prob}

\end{document}
