% Part: sets-functions-relations
% Chapter: infinite
% Section: dedekind-induction

\documentclass[../../../include/open-logic-section]{subfiles}

\begin{document}

\olfileid{sfr}{infinite}{induction}
\olsection[Arithmetical Induction]{Dedekind Algebras and Arithmetical Induction}

Crucially, now, a Dedekind algebra---indeed, \emph{any} Dedekind
algebra---will serve as a surrogate for the natural numbers. This is
thanks to the following trivial consequence:

\begin{thm}[Arithmetical induction]\ollabel{thm:dedinfiniteinduction}
Let $N, s, o$ comprise a Dedekind algebra. Then for any set $X$:  
	\begin{center}
		if $o \in X$ and $(\forall {n} \in N \cap X){s}({n}) \in X$, {then} $N \subseteq X$.
	\end{center}
\end{thm}

\begin{proof}
By the definition of a Dedekind algebra, $N = \closureofunder{s}{o}$.
Now if both ${o} \in X$ and $(\forall {n} \in N)(n \in X \lif
{s}({n}) \in X)$, then $N = \closureofunder{s}{o} \subseteq X$.
\end{proof}

Since induction is characteristic of the natural numbers, the point is
this. Given any Dedekind infinite set, we can form a Dedekind algebra,
and use that algebra as our surrogate for the natural numbers. 

Admittedly, \olref{thm:dedinfiniteinduction} formulates induction in
\emph{set-theoretic} terms. But we can easily put the principle in
terms which might be more familiar:

\begin{cor}\ollabel{natinductionschema}
Let $N, s, o$ comprise a Dedekind algebra. Then for any formula
$\phi(x)$, which may have parameters:
\begin{center}
	if $\phi(o)$ and $(\forall {n} \in N)(\phi(n)\lif
	\phi({s}({n})))$, {then} $(\forall n \in N)\phi(n)$
\end{center}
\end{cor}

\begin{proof}
Let $X = \Setabs{n \in N}{\phi(n)}$, and now use
\olref{thm:dedinfiniteinduction}
\end{proof}

In this result, we spoke of a formula ``having parameters''. What this
means, roughly, is that for any objects $c_1, \ldots, c_k$, we can
work with $\phi(x, c_1, \ldots, c_k)$. More precisely, we can state
the result without mentioning ``parameters'' as follows. For any
formula $\phi(x, v_1, \ldots, v_k)$, whose free variables are all
displayed, we have:
	\begin{align*}
			\forall v_1 \ldots \forall v_k((&\phi(o, v_1,\ldots, v_k) \land {}\\
			&	(\forall x \in N)(\phi(x,v_1, \ldots, v_k) \lif \phi(s(x), v_1,\ldots, v_k))) \lif {}\\
			&\hspace{3em} (\forall x \in N)\phi(x, v_1,\ldots, v_k))
	\end{align*}
Evidently, speaking of ``having parameters'' can make things much
easier to read. (In \olref[sth][][]{part}, we will use this device
rather frequently.)

Returning to Dedekind algebras: given any Dedekind algebra, we can
also define the usual arithmetical functions of addition,
multiplication and exponentiation. This is non-trivial, however, and
it involves the technique of \emph{recursive definition}. That is a
technique which we shall introduce and justify much later, and in a
much more general context. (Enthusiasts might want to revisit this
after \olref[sth][ord-arithmetic][]{chap}, or perhaps read an alternative
treatment, such as \citealt[pp.~95--8]{Potter2004}.) But, where $N, s, o$
comprise a Dedekind algebra, we will ultimately be able to stipulate the
following:
\begin{align*}
	{a} + {o} &= {a} & & & {a} \times {o} &= {o} & & & {a}^{o} &= s(o)\\
{a} + {s}({b}) &= {s}({a}+{b}) &&& {a} \times {s}({b}) &= ({a}\times {b}) + {a}  & & & {a}^{{s}({b})} &= {a}^{b} \times {a}
\end{align*}
and show that these behave as one would hope.

\end{document}
