% Part:sets-functions-relations
% Chapter: sets
% Section: comparing-sizes

\documentclass[../../../include/open-logic-section]{subfiles}

\begin{document}

\olfileid{sfr}{set}{car}

\olsection{Comparing Sizes of Sets}

\begin{explain}
Just like we were able to make precise when two sets have the same
size in a way that also accounts for the size of infinite sets, we can
also compare the sizes of sets in a precise way. Our definition of
``is smaller than (or equinumerous)'' will require, instead of a
!!{bijection} between the sets, a total !!{injective} function from the first
set to the second. If such a function exists, the size of the first
set is less than or equal to the size of the second. Intuitively,
!!a{injective} function from one set to another guarantees that the range of
the function has at least as many elements as the domain, since no two
!!{element}s of the domain map to the same !!{element} of the range.
\end{explain}

\begin{defn}
$X$ is \emph{no larger than}~$Y$, $\cardle{X}{Y}$, if and only if there
  is an !!{injective} function~$f \colon X \to Y$.
\end{defn}

\begin{thm}[Schr\"oder-Bernstein]
  Let $X$ and $Y$ be sets. If $\cardle{X}{Y}$ and $\cardle{Y}{X}$,
  then $\cardeq{X}{Y}$.
\end{thm}

\begin{explain}
In other words, if there is a total !!{injective} function from $X$ to
$Y$, and if there is a total !!{injective} function from $Y$ back to~$X$,
then there is a total !!{bijection} from $X$ to~$Y$. Sometimes, it can be
difficult to think of a !!{bijection} between two equinumerous sets, so
the Schr\"oder-Bernstein theorem allows us to break the comparison
down into cases so we only have to think of !!a{injection} from the
first to the second, and vice-versa. The Schr\"oder-Bernstein theorem,
apart from being convenient, justifies the act of discussing the
``sizes'' of sets, for it tells us that set cardinalities have the
familiar anti-symmetric property that numbers have.
\end{explain}

\begin{defn}
$X$ is \emph{smaller than}~$Y$, $\cardless{X}{Y}$, if and only if
  there is !!a{injective} function~$f\colon X \to Y$ but no
  !!{bijective}~$g\colon X \to Y$.
\end{defn}

\begin{thm}[Cantor]
  \ollabel{thm:cantor}
For all $X$, $\cardless{X}{\Pow{X}}$.
\end{thm}

\begin{proof}
The function~$f \colon X \to \Pow{X}$ that maps any $x \in X$ to its
singleton~$\{x\}$ is !!{injective}, since if $x \neq y$ then also $f(x) =
\{x\} \neq \{y\} = f(y)$.

There cannot be !!a{surjective} function~$g\colon X \to \Pow{X}$, let
alone a !!{bijective} one. For suppose that $g\colon X \to \Pow{X}$.
Since $g$ is total, every $x \in X$ is mapped to a subset $g(x)
\subseteq X$. We show that $g$ cannot be surjective. To do this, we
define a subset~$Y \subseteq X$ which by definition cannot be in the
range of~$g$. Let
\[
\overline{Y} = \Setabs{x \in X}{x \notin g(x)}.
\]
Since $g(x)$ is defined for all $x \in X$, $\overline{Y}$ is clearly a
well-defined subset of~$X$.  But, it cannot be in the range
of~$g$. Let $x \in X$ be arbitrary, we show that $\overline{Y} \neq
g(x)$.  If $x \in g(x)$, then it does not satisfy $x \notin g(x)$, and
so by the definition of~$\overline{Y}$, we have $x \notin
\overline{Y}$.  If $x \in \overline{Y}$, it must satisfy the defining
property of~$\overline{Y}$, i.e., $x \notin g(x)$.  Since $x$ was
arbitrary this shows that for each $x \in X$, $x \in g(x)$ iff $x
\notin \overline{Y}$, and so $g(x) \neq \overline{Y}$.  So
$\overline{Y}$ cannot be in the range of~$g$, contradicting the
assumption that~$g$ is surjective.
\end{proof}

\begin{explain}
  It's instructive to compare the proof of \olref{thm:cantor} to that
  of \olref[nen]{thm-nonenum-pownat}. There we showed that for any
  list $Z_1$, $Z_2$, \dots, of subsets of~$\Int^+$ one can construct a
  set~$\overline{Z}$ of numbers guaranteed not to be on the list. It
  was guaranteed not to be on the list because, for every $n \in
  \Int^+$, $n \in Z_n$ iff $n \notin \overline{Z}$. This way, there is
  always some number that is !!a{element} of one of $Z_n$ and
  $\overline{Z}$ but not the other. We follow the same idea here,
  except the indices~$n$ are now !!{element}s of~$X$ instead
  of~$\Int^+$. The set $\overline{Y}$ is defined so that it is
  different from~$g(x)$ for each $x \in X$, because $x \in g(x)$ iff
  $x \notin \overline{Y}$. Again, there is always !!a{element} of~$X$
  which is !!a{element} of one of $g(x)$ and $\overline{Y}$ but not
  the other. And just as $\overline{Z}$ therefore cannot be on the
  list $Z_1$, $Z_2$, \dots, $\overline{Y}$ cannot be in the range
  of~$g$.
\end{explain}

\begin{prob}
  Show that there cannot be !!a{injective} function $g\colon \wp(X) \to
  X$, for any set $X$. Hint: Suppose $g\colon \wp(X) \to X$ is
  !!{injective}. Then for each $x \in X$ there is at most one $Y \subseteq
  X$ such that $g(Y) = x$. Define a set $\overline{Y}$ such that for
  every $x \in X$, $g(\overline{Y}) \neq x$.
\end{prob}

\end{document}
