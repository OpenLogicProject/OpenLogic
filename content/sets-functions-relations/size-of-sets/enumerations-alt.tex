% Part:sets-functions-relations
% Chapter: size-of-sets
% Section: enumerations-alt

\documentclass[../../../include/open-logic-section]{subfiles}

\begin{document}

\olfileid{sfr}{siz}{enm-alt}

\olsection{Enumerations and \usetoken{S}{enumerable} Sets}

\begin{editorial}
  This section defines enumerations as bijections with (initial
  segments) of $\Nat$, the way it's done in set theory. So it
  conflicts slightly with the definitions in \olref[enm]{sec}, and
  repeats all the examples there. It is also a bit more terse than
  that section.
\end{editorial}

We can specify finite set is by simply enumerating its
!!{element}s. We do this when we define a set like so:
\[
  A = \{a_1, a_2, \ldots, a_n\}.
\]
Assuming that the !!{element}s $a_1$, \dots, $a_n$ are all distinct,
this gives us !!a{bijection} between $A$ and the first $n$ natural
numbers $0$, \dots, $n-1$. Conversely, since every finite set has only
finitely many !!{element}s, every finite set can be put into such a
correspondence. In other words, if $A$ is finite, there is
!!a{bijection} between $A$ and $\{0, \dots, n-1\}$, where $n$ is the
number of !!{element}s of~$A$.

If we allow for certain kinds of infinite sets, then we will also
allow some infinite sets to be enumerated. We can make this precise by
saying that an infinite set is enumerated by !!a{bijection} between it
and all of~$\Nat$.

\begin{defn}[Enumeration, set-theoretic] 
An \emph{enumeration} of a set $A$ is !!a{bijection} whose range is
$A$ and whose domain is either an initial set of natural numbers $\{0,
1, \ldots, n\}$ {or} the entire set of natural numbers~$\Nat$. 
\end{defn}

\begin{explain}
There is an intuitive underpinning to this use of the word
\emph{enumeration}. For to say that we have enumerated a set $A$ is to
say that there is !!a{bijection} $f$ which allows us to count out the
elements of the set $A$. The $0$th element is $f(0)$, the 1st is
$f(1)$, \ldots the $n$th is $f(n)$\ldots.\footnote{Yes, we count
from $0$. Of course we could also start with~$1$. This would
make no big difference. We would just have to replace~$\Nat$
by~$\PosInt$.} The rationale for this may be made even clearer by
adding the following:
\end{explain}

\begin{defn}
  \ollabel{defn:enumerable}
  A set~$A$ is !!{enumerable} iff either $A = \emptyset$ or there is
  an enumeration of~$A$. We say that $A$ is !!{nonenumerable} iff $A$
  is not !!{enumerable}.
\end{defn}

\begin{explain}
So a set is !!{enumerable} iff it is empty or you can use an
enumeration to count out its !!{element}s.
\end{explain}

\begin{ex}
A function enumerating the natural numbers is simply the identity
function $\Id{\Nat} \colon \Nat \to \Nat$ given by $\Id{\Nat}(n) = n$. A
function enumerating the \emph{positive} natural numbers, $\Nat^+ =
\Nat \setminus \{0\}$, is the function $g(n) = n + 1$, i.e.\ the
successor function.
\end{ex}

\begin{prob}
Show that a set $A$ is !!{enumerable} iff either $A = \emptyset$ or
there is !!a{surjection} $f\colon \Nat \to A$. Show that $A$ is
!!{enumerable} iff there is !!a{injection} $g\colon A \to \Nat$. 
\end{prob}

\begin{ex}
The functions $f\colon \Nat \to \Nat$ and $g \colon \Nat \to \Nat$
given by
\begin{align*}
  f(n) & = 2n \text{ and}\\
  g(n) & = 2n+1
\end{align*}
respectively enumerate the even natural numbers and the odd natural
numbers. But neither is !!{surjective}, so neither is an enumeration
of $\Nat$.
\end{ex}

\begin{prob}
Define an enumeration of the square numbers $1$, $4$, $9$, $16$, \dots
\end{prob}

\begin{ex}
Let $\lceil x \rceil$ be the \emph{ceiling} function, which rounds $x$
up to the nearest integer. Then the function $f \colon \Nat \to \Int$
given by:
\[
  f(n) = (-1)^{n} \left\lceil\tfrac{n}{2}\right\rceil
\]
enumerates the set of
integers~$\Int$ as follows:
\[
\begin{array}{c c c c c c c c}
f(0) & f(1) & f(2) & f(3) & f(4) & f(5) & f(6) & \dots \\ \\
\big\lceil \tfrac{0}{2} \big\rceil & -\big\lceil \tfrac{1}{2}\big\rceil &  \big\lceil \tfrac{2}{2} \big\rceil & -\big\lceil \tfrac{3}{2} \big\rceil & \big\lceil \tfrac{4}{2} \big\rceil  & -\big\lceil \tfrac{5}{2}\big\rceil & \big\lceil \tfrac{6}{2} \big\rceil & \dots \\ \\
0 & -1 & 1 & -2 & 2 & -3 & 3& \dots
\end{array}
\]
Notice how $f$ generates the values of $\Int$ by ``hopping'' back and
forth between positive and negative integers. You can also think of
$f$ as defined by cases as follows:
\[
f(n) = \begin{cases}
  \frac{n}{2} & \text{if $n$ is even}\\
  -\frac{n+1}{2} & \text{if $n$ is odd}
  \end{cases}
\]
\end{ex}

\begin{prob}
Show that if $A$ and $B$ are !!{enumerable}, so is $A \cup B$.
\end{prob}

\begin{prob}
Show by induction on $n$ that if $A_1$, $A_2$, \dots, $A_n$ are all
!!{enumerable}, so is $A_1 \cup \dots \cup A_n$.
\end{prob}

\end{document}