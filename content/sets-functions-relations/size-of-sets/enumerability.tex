% Part:sets-functions-relations
% Chapter: sets
% Section: enumerability

\documentclass[../../../include/open-logic-section]{subfiles}

\begin{document}

\olfileid{sfr}{siz}{enm}

\olsection{\printtoken{S}{enumerable} Sets}

One way of specifying a finite set is by listing its !!{element}s. But
conversely, since there are only finitely many !!{element}s in a set,
every finite set can be enumerated.  By this we mean: its elements can
be put into a list (a list with a beginning, where each !!{element} of
the list other than the first has a unique predecessor).  Some
infinite sets can also be enumerated, such as the set of positive
integers.

\begin{defn}[Enumeration]
Informally, an \emph{enumeration} of a set~$X$ is a list (possibly
infinite) of !!{element}s of~$X$ such that every !!{element} of $X$
appears on the list at some finite position. If $X$ has an
enumeration, then $X$ is said to be \emph{!!{enumerable}}.  If $X$ is
!!{enumerable} and infinite, we say $X$ is !!{denumerable}.
\end{defn}

\begin{explain}
A couple of points about enumerations:
\begin{enumerate}
\item We count as enumerations only lists which have a beginning and
  in which every !!{element} other than the first has a single
  !!{element} immediately preceding it.  In other words, there are
  only finitely many elements between the first !!{element} of the
  list and any other !!{element}. In particular, this means that every
  !!{element} of an enumeration has a finite position: the first
  !!{element} has position~$1$, the second position~$2$, etc.
\item We can have different enumerations of the same set~$X$ which
  differ by the order in which the !!{element}s appear: $4$, $1$,
  $25$, $16$,~$9$ enumerates the (set of the) first five square
  numbers just as well as $1$, $4$, $9$, $16$,~$25$ does.
\item Redundant enumerations are still enumerations: $1$, $1$, $2$,
  $2$, $3$, $3$,~\dots{} enumerates the same set as $1$, $2$,
  $3$,~\dots{} does.
\item Order and redundancy \emph{do} matter when we specify an
  enumeration: we can enumerate the positive integers beginning with
  $1$, $2$, $3$, $1$, \dots{}, but the pattern is easier to see when
  enumerated in the standard way as $1$, $2$, $3$, $4$,~\dots
\item Enumerations must have a beginning: \dots, $3$, $2$, $1$ is not
  an enumeration of the positive integers because it has no first
  !!{element}. To see how this follows from the informal definition,
  ask yourself, ``at what position in the list does the number 76
  appear?''
\item The following is not an enumeration of the positive integers:
  $1$, $3$, $5$, \dots, $2$, $4$, $6$, \dots\@ The problem is that the
  even numbers occur at places $\infty + 1$, $\infty + 2$, $\infty +
  3$, rather than at finite positions.
\item Lists may be gappy: $2$, $-$, $4$, $-$, $6$, $-$, \dots{}
  enumerates the even positive integers.
\item The empty set is enumerable: it is enumerated by the empty list!{}
\end{enumerate}
\end{explain}

\begin{prop}
  If $X$ has an enumeration, it has an enumeration without gaps or
  repetitions.
\end{prop}

\begin{proof}
  Suppose $X$ has an enumeration $x_1$, $x_2$, \dots{} in which each
  $x_i$ is an !!{element} of~$X$ or a gap.  We can remove repetitions
  from an enumeration by replacing repeated !!{element}s by gaps. For
  instance, we can turn the enumeration into a new one in which $x_i'$
  is $x_i$ if $x_i$ is !!a{element} of~$X$ that is not among $x_1$,
  \dots, $x_{i-1}$ or is $-$ if it is. We can remove gaps by closing
  up the elements in the list. To make precise what ``closing up''
  amounts to is a bit difficult to describe. Roughly, it means that we
  can generate a new enumeration $x_1''$, $x_2''$, \dots, where each
  $x_i''$ is the first !!{element} in the enumeration $x_1'$, $x_2'$,
  \dots{} after $x_{i-1}''$ (if there is one).
\end{proof}

The last argument shows that in order to get a good handle on
enumerations and !!{enumerable} sets and to prove things about them,
we need a more precise definition.  The following provides it.

\begin{defn}[Enumeration]
An \emph{enumeration} of a set $X$ is any !!{surjective} function
$f \colon \Int^+ \to X$.
\end{defn}

\begin{explain}
Let's convince ourselves that the formal definition and the informal
definition using a possibly gappy, possibly infinite list are
equivalent. !!^a{surjective} function (partial or total) from $\Int^+$
to a set $X$ enumerates~$X$. Such a function determines an enumeration
as defined informally above: the list $f(1)$, $f(2)$, $f(3)$,
\dots. Since $f$ is !!{surjective}, every !!{element} of $X$ is
guaranteed to be the value of $f(n)$ for some~$n \in \Int^+$. Hence,
every !!{element} of $X$ appears at some finite position in the
list. Since the function may not be !!{injective}, the list may be
redundant, but that is acceptable (as noted above).

On the other hand, given a list that enumerates all !!{element}s
of~$X$, we can define !!a{surjective} function $f\colon \Int^+ \to X$ by
letting $f(n)$ be the $n$th !!{element} of the list that is not a gap,
or the final !!{element} of the list if there is no $n$th !!{element}.
There is one case in which this does not produce !!a{surjective}
function: if $X$ is empty, and hence the list is empty. So, every
non-empty list determines !!a{surjective} function $f\colon \Int^+ \to
X$.
\end{explain}

\begin{defn}
  \ollabel{defn:enumerable}
  A set~$X$ is !!{enumerable} iff it is empty or has an enumeration.
\end{defn}

\begin{ex}
A function enumerating the positive integers ($\Int^+$) is simply the
identity function given by $f(n) = n$. A function enumerating the
natural numbers $\Nat$ is the function $g(n) = n - 1$.
\end{ex}

\begin{prob}
  According to \olref[sfr][siz][enm]{defn:enumerable}, a set $X$ is
  enumerable iff $X = \emptyset$ or there is !!a{surjective} $f\colon
  \Int^+ \to X$.  It is also possible to define ``!!{enumerable} set''
  precisely by: a set is enumerable iff there is !!a{injective}
  function $g\colon X \to \Int^+$.  Show that the definitions are
  equivalent, i.e., show that there is !!a{injective} function
  $g\colon X \to \Int^+$ iff either $X = \emptyset$ or there is
  !!a{surjective} $f\colon \Int^+ \to X$.
\end{prob}

\begin{ex}
The functions $f\colon \Int^+ \to \Int^+$ and $g \colon \Int^+ \to
\Int^+$ given by
\begin{align*}
f(n) & = 2n \text{ and}\\
g(n) & = 2n+1
\end{align*}
enumerate the even positive integers and the odd positive integers,
respectively. However, neither function is an enumeration of
$\Int^+$, since neither is !!{surjective}.
\end{ex}

\begin{prob}
Define an enumeration of the positive squares $4$, $9$, $16$, \dots
\end{prob}

\begin{ex}
The function $f(n) = (-1)^{n} \lceil \frac{(n-1)}{2}\rceil$ (where
$\lceil x \rceil$ denotes the \emph{ceiling} function, which rounds
$x$ up to the nearest integer) enumerates the set of
integers~$\Int$. Notice how $f$ generates the values of $\Int$ by
``hopping'' back and forth between positive and negative integers:
\[
\begin{array}{c c c c c c c c}
f(1) & f(2) & f(3) & f(4) & f(5) & f(6) & f(7) & \dots \\ \\
- \lceil \tfrac{0}{2} \rceil & \lceil \tfrac{1}{2}\rceil & - \lceil \tfrac{2}{2} \rceil & \lceil \tfrac{3}{2} \rceil & - \lceil \tfrac{4}{2} \rceil  & \lceil \tfrac{5}{2}
\rceil & - \lceil \tfrac{6}{2} \rceil & \dots \\ \\
0 & 1 & -1 & 2 & -2 & 3 & \dots
\end{array}
\]
You can also think of $f$ as defined by cases as follows:
\[
f(n) = \begin{cases}
  0 & \text{if $n = 1$}\\
  n/2 & \text{if $n$ is even}\\
  -(n-1)/2 & \text{if $n$ is odd and $>1$}
  \end{cases}
\]
\end{ex}

\begin{prob}
Show that if $X$ and $Y$ are !!{enumerable}, so is $X \cup Y$.
\end{prob}

\begin{prob}
  Show by induction on $n$ that if $X_1$, $X_2$, \dots, $X_n$ are all
  !!{enumerable}, so is $X_1 \cup \dots \cup X_n$.
\end{prob}

\begin{explain}
That is fine for ``easy'' sets. What about the set of, say, pairs of
positive integers?{}
\[
\Int^+ \times \Int^+ = \Setabs{\tuple{n,m}}{n,m \in \Int^+}
\]
We can organize the pairs of positive integers 
in an \emph{array}, such as the following:
\[
\begin{array}{ c | c | c | c | c | c}
& \textbf 1 & \textbf 2 & \textbf 3 & \textbf 4 & \dots \\
\hline
\textbf 1 & \tuple{1,1} & \tuple{1,2} & \tuple{1,3} & \tuple{1,4} & \dots \\
\hline
\textbf 2 & \tuple{2,1} & \tuple{2,2} & \tuple{2,3} & \tuple{2,4} & \dots \\
\hline
\textbf 3 & \tuple{3,1} & \tuple{3,2} & \tuple{3,3} & \tuple{3,4} & \dots \\
\hline
\textbf 4 & \tuple{4,1} & \tuple{4,2} & \tuple{4,3} & \tuple{4,4} & \dots \\
\hline
\vdots & \vdots & \vdots & \vdots & \vdots & \ddots\\
\end{array}
\]

Clearly, every ordered pair in $\Int^+ \times \Int^+$ will appear
exactly once in the array. In particular, $\tuple{n,m}$ will appear in
the $n$th column and $m$th row. But how do we organize the elements of
such an array into a one-way list? The pattern in the array below
demonstrates one way to do this:
\[
\begin{array}{ c | c | c | c | c | c }
& & & & & \\
\hline
& 1 & 2 & 4 & 7 & \dots \\
\hline
& 3 & 5 & 8 & \dots & \dots \\
\hline
& 6 & 9 & \dots & \dots & \dots \\
\hline
& 10 & \dots & \dots & \dots & \dots \\
\hline
& \vdots & \vdots & \vdots & \vdots & \ddots\\
\end{array}
\]
This pattern is called \emph{Cantor's zig-zag method}. Other patterns
are perfectly permissible, as long as they ``zig-zag'' through every
cell of the array. By Cantor's zig-zag method, the enumeration for
$\Int^+ \times \Int^+$ according to this scheme would be:
\[
\tuple{1,1}, \tuple{1,2}, \tuple{2,1}, \tuple{1,3}, \tuple{2,2},
\tuple{3,1}, \tuple{1,4}, \tuple{2,3}, \tuple{3,2}, \tuple{4,1}, \dots
\]

What ought we do about enumerating, say, the set of ordered triples
of positive integers?
\[
\Int^+ \times \Int^+ \times \Int^+ = \Setabs{\tuple{n,m,k}}{n,m,k \in \Int^+}
\]
We can think of $\Int^+ \times \Int^+ \times \Int^+$ as the Cartesian
product of $\Int^+ \times \Int^+$ and $\Int^+$, that is,
\[
(\Int^+)^3 = (\Int^+ \times \Int^+) \times \Int^+ =
\Setabs{\tuple{\tuple{n,m},k}}{\tuple{n,m} \in \Int^+ \times \Int^+, k
  \in \Int^+ }
\]
and thus we can enumerate $(\Int^+)^3$ with an array by labelling one
axis with the enumeration of $\Int^+$, and the other axis with the
enumeration of $(\Int^+)^2$:
\[
\begin{array}{ c | c | c | c | c | c}
& \textbf 1 & \textbf 2 & \textbf 3 & \textbf 4 & \dots \\
\hline
\mathbf{\tuple{1,1}} & \tuple{1,1,1} & \tuple{1,1,2} & \tuple{1,1,3} & \tuple{1,1,4} & \dots \\
\hline
\mathbf{\tuple{1,2}} & \tuple{1,2,1} & \tuple{1,2,2} & \tuple{1,2,3} & \tuple{1,2,4} & \dots \\
\hline
\mathbf{\tuple{2,1}} & \tuple{2,1,1} & \tuple{2,1,2} & \tuple{2,1,3} & \tuple{2,1,4} & \dots \\
\hline
\mathbf{\tuple{1,3}} & \tuple{1,3,1} & \tuple{1,3,2} & \tuple{1,3,3} & \tuple{1,3,4} & \dots\\
\hline
\vdots & \vdots & \vdots & \vdots & \vdots & \ddots \\
\end{array}
\]
Thus, by using a method like Cantor's zig-zag method, we may
similarly obtain an enumeration of~$(\Int^+)^3$.

Cantor's zig-zag method makes the enumerability of $(\Int^+)^2$ (and analogously, $(\Int^+)^3$, etc.) visually evident.  Following the zig-zag line in the array and counting the places, we can tell that $\tuple{2,3}$ is at place 8, but specifying the inverse $g\colon (\Int^+)^2 \to \Int^+$ of the zig-zag enumeration such that
\[
g(\tuple{1,1}) = 1, \ \ g(\tuple{1,2}) = 2, \ \ g(\tuple{2,1}) = 3, \ \ \dots \ \ g(\tuple{2,3}) = 8, \ \ \dots
\]
would be helpful. To calculate the position of each pair in the enumeration, we can use the function below. (The exact derivation of the function is somewhat messy, so we are skipping it here.)
\[
g(n,m) = \frac{(n + m - 2)(n + m - 1)}{2}+n
\]
Accordingly, the pair $\tuple{2,3}$ is in position $((2 + 3 - 2)(2 + 3 - 1) / 2) + 2 = (3 \cdot 4 / 2) + 2 = (12 / 2) + 2 = 8$; pair $\tuple{3,7}$ is in position $((3 + 7 - 2)(3 + 7 - 1) / 2) + 3 = 39$.
\end{explain}

Functions like $g$ above, i.e., inverses of enumerations of sets of pairs, are called \emph{pairing functions}.

\begin{defn}[Pairing function]
A function $f\colon X \times Y \to \Int^+$ is an arithmetical \emph{pairing function} if $f$ is total and injective. We also say that $f$ \emph{encodes} $X \times Y$, and that for $f(\tuple{x,y})=n$, $n$ is the \emph{code} for $\tuple{x,y}$.
\end{defn}

\begin{explain}
The idea is that we can use such functions to encode, e.g., pairs of positive integers in $\Int^+$, or, in other words, represent pairs of positive integers as positive integers. Using the inverse of the pairing function, we can \emph{decode} the integer, i.e., find out which pair of positive integers is represented.

There are other enumerations of $(\Int^+)^2$ that make it easier to figure out what their inverses are. Here is one. Instead of visualizing the enumeration in an array, start with the list of positive integers associated with (initially) empty spaces. Imagine filling these spaces successively with pairs $\tuple{n,m}$ as follow. Starting with the pairs that have 1 in  the first place (i.e., pairs $\tuple{1,m}$), put the first (i.e., $\tuple{1,1}$) in the first empty place, then skip an empty space, put the second (i.e., $\tuple{1,2}$) in the next empty place, skip one again, and so forth. The (incomplete) beginning of our enumeration now looks like this
\[
\begin{array}{c c c c c c c c c c c}
f(1) & f(2) & f(3) & f(4) & f(5) & f(6) & f(7) & f(8) & f(9) & f(10) & \dots \\ \\
\tuple{1,1} &  & \tuple{1,2} &  & \tuple{1,3} & & \tuple{1,4} &  & \tuple{1,5} &  & \dots \\
\end{array}
\]
Repeat this with pairs $\tuple{2,m}$ for the place that still remain empty, again skipping every other empty place:
\[
\begin{array}{c c c c c c c c c c c}
f(1) & f(2) & f(3) & f(4) & f(5) & f(6) & f(7) & f(8) & f(9) & f(10) & \dots \\ \\
\tuple{1,1} & \tuple{2,1} & \tuple{1,2} &  & \tuple{1,3} & \tuple{2,2} & \tuple{1,4} &  & \tuple{1,5} &  \tuple{2,3} & \dots \\
\end{array}
\]
Enter pairs $\tuple{3,m}$, $\tuple{4,m}$, etc., in the same way. Our completed enumeration thus starts like this:
\[
\begin{array}{c c c c c c c c c c c}
f(1) & f(2) & f(3) & f(4) & f(5) & f(6) & f(7) & f(8) & f(9) & f(10) & \dots \\ \\
\tuple{1,1} & \tuple{2,1} & \tuple{1,2} & \tuple{3,1}  & \tuple{1,3} & \tuple{2,2} & \tuple{1,4} & \tuple{4,1}  & \tuple{1,5} &  \tuple{2,3} & \dots \\
\end{array}
\]
If we number the cells in the array above according to this enumeration, we will not find a neat zig-zag line, but this arrangement:
\[
\begin{array}{ c | c | c | c | c | c | c | c }
& \textbf 1 & \textbf 2 & \textbf 3 & \textbf 4 & \textbf 5 & \textbf 6 & \dots \\
\hline
\textbf 1 & 1 & 3 & 5 & 7 & 9 & 11 & \dots \\
\hline
\textbf 2 & 2 & 6 & 10 & 14 & 18 & \dots & \dots \\
\hline
\textbf 3 & 4 & 12 & 20 & 28 & \dots & \dots & \dots \\
\hline
\textbf 4 & 8 & 24 & 40 & \dots & \dots & \dots & \dots \\
\hline
\textbf 5 & 16 & 48 & \dots & \dots & \dots & \dots & \dots \\
\hline
\textbf 6 & 32 & \dots & \dots & \dots & \dots & \dots & \dots \\
\hline
\vdots & \vdots & \vdots & \vdots & \vdots & \vdots & \vdots & \ddots\\
\end{array}
\]

We can see that the pairs in the first row are in the odd numbered places of our enumeration, i.e., pair $\tuple{1,m}$ is in place $2m-1$; pairs in the second row, $\tuple{1,m}$, are in places whose number is the double of an odd number, specifically,  $2 \cdot (2m-1)$; pairs in the third row, $\tuple{1,m}$, are in places whose number is four times an odd number, $4 \cdot (2m-1)$; and so on. The factors of $(2m-1)$ for each row, $1, 2, 4, 8, \dots$, are powers of 2: $2^0, 2^1, 2^2, 2^3, \dots$ In fact, the relevant exponent is one less than the first member of the pair in question. Thus, for pair $\tuple{n,m}$ the factor is $n-1$.  This gives us the general formula: $2^{n-1} \cdot (2m-1)$, and hence:
\end{explain}
\begin{ex}
The function $f\colon (\Int^+)^2 \to \Int^+$ given by
\[
h(n,m) = 2^{n-1} (2m-1)
\]
is a pairing function for the set of pairs of positive integers $(\Int^+)^2$.
\end{ex}
\begin{explain}
Accordingly, in our second enumeration of $(\Int^+)^2$, the pair $\tuple{2,3}$ is in position $2^{2-1} \cdot (2 \cdot 3 - 1) = 2 \cdot 5 = 10$; pair $\tuple{3,7}$ is in position $2^{3-1} \cdot (2 \cdot 7 - 1)  = 52$.
\end{explain}

Another common pairing function that encodes $(\Int^+)^2$ is the following:
\begin{ex}
The function $f\colon (\Int^+)^2 \to \Int^+$ given by
\[
j(n,m) = 2^n3^m
\]
is a pairing function for the set of pairs of positive integers $(\Int^+)^2$.
\end{ex}

\begin{explain}
$j$ is injective, but nor surjective.  That means the inverse of $j$ is a partial, surjective function, and hence an enumeration of $(\Int^+)^2$. (Exercise.)
\end{explain}

\begin{prob}
Give an enumeration of the set of all positive rational numbers. (A
positive rational number is one that can be written as a fraction
$n/m$ with $n, m \in \Int^+$).
\end{prob}

\begin{prob}
Show that $\Rat$ is !!{enumerable}. (A rational number is one that can
be written as a fraction $z/m$ with $z \in \Int$, $m \in \Int^+$).
\end{prob}

\begin{prob}
Define an enumeration of $\Bin^*$.
\end{prob}

\begin{prob}
Recall from your introductory logic course that each possible truth
table expresses a truth function. In other words, the truth functions
are all functions from $\Bin^k \to \Bin$ for some~$k$. Prove that the
set of all truth functions is enumerable.
\end{prob}

\begin{prob}
Show that the set of all finite subsets of an arbitrary infinite
enumerable set is enumerable.
\end{prob}

\begin{prob}
A set of positive integers is said to be \emph{cofinite} iff
it is the complement of a finite set of positive integers. Let
\emph{I} be the set that contains all the finite and cofinite sets of
positive integers. Show that \emph{I} is enumerable.
\end{prob}

\begin{prob}
Show that the !!{enumerable} union of !!{enumerable} sets is
!!{enumerable}. That is, whenever $X_1$, $X_2$, \dots{} are sets, and
each $X_i$ is !!{enumerable}, then the union $\bigcup_{i=1}^\infty
X_i$ of all of them is also !!{enumerable}.
\end{prob}

\begin{prob}
Let $f: X \times Y \to \Int^+$ be an arbitrary pairing function. Show that the inverse of $f$ is an enumeration of $X \times Y$.
\end{prob}

\begin{prob}
Specify a function that encodes $\Nat^3$.
\end{prob}

\end{document}
