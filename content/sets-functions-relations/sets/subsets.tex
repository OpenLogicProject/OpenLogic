% Part: sets-functions-relations
% Chapter: sets
% Section: subsets

\documentclass[../../../include/open-logic-section]{subfiles}

\begin{document}

\olfileid{sfr}{set}{sub}
\olsection{Subsets and Power Sets}

\begin{explain}
We will often want to compare sets. And one obvious kind of comparison
one might make is as follows: \emph{everything in one set is in the
other too}. This situation is sufficiently important for us to
introduce some new notation.
\end{explain}

\begin{defn}[Subset]
If every !!{element} of a set $A$ is also !!a{element} of~$B$, then we
say that $A$ is a \emph{subset} of~$B$, and write $A \subseteq B$. If
$A$ is not a subset of~$B$ we write $A \not\subseteq B$.
If $A \subseteq B$ but $A \neq B$, we write $A \subsetneq B$ and say
that $A$ is a \emph{proper subset} of $B$.
\end{defn}

\begin{ex}
Every set is a subset of itself, and $\emptyset$ is a subset of every
set. The set of even numbers is a subset of the set of natural
numbers. Also, $\{ a, b \} \subseteq \{ a, b, c \}$. But $\{ a, b, e
\}$ is not a subset of $\{ a, b, c \}$.
\end{ex}

\begin{ex}
The number $2$ is an !!{element} of the set of integers, whereas the
set of even numbers is a subset of the set of integers. However, a set
may happen to \emph{both} be !!a{element} and a subset of some other
set, e.g., $\{0\} \in \{0, \{0\}\}$ and also $\{0\} \subseteq \{0,
\{0\}\}$.
\end{ex}

Extensionality gives a criterion of identity for sets: $A = B$ iff
every !!{element} of~$A$ is also !!a{element} of~$B$ and vice versa.
The definition of ``subset'' defines $A \subseteq B$ precisely as the
first half of this criterion: every !!{element} of~$A$ is also
!!a{element} of~$B$. Of course the definition also applies if we
switch $A$ and $B$: that is, $B \subseteq A$ iff every !!{element}
of~$B$ is also !!a{element} of~$A$. And that, in turn, is exactly the
``vice versa'' part of extensionality. In other words, extensionality
entails that sets are equal iff they are subsets of one another.

\begin{prop}
$A = B$ iff both $A \subseteq B$ and $B \subseteq A$.
\end{prop}

Now is also a good opportunity to introduce some further bits of
helpful notation. In defining when $A$ is a subset of~$B$ we said that
``every !!{element} of~$A$ is \dots,'' and filled the ``$\dots$'' with
``!!a{element} of $B$''. But this is such a common \emph{shape} of
expression that it will be helpful to introduce some formal notation
for it.

\begin{defn}\ollabel{forallxina}
$(\forall x \in A)\phi$ abbreviates $\forall x(x \in A \lif
\phi)$. Similarly, $(\exists x \in A)\phi$ abbreviates $\exists x(x
\in A \land \phi)$. 
\end{defn}

Using this notation, we can say that $A \subseteq B$ iff $(\forall
x \in A)x \in B$. 

Now we move on to considering a certain kind of set: the set of all
subsets of a given set. 

\begin{defn}[Power Set]
The set consisting of all subsets of a set~$A$ is called the
\emph{power set of}~$A$, written $\Pow{A}$.
  \[
    \Pow{A} = \Setabs{B}{B \subseteq A} 
  \]
\end{defn}

\begin{ex}
What are all the possible subsets of $\{ a, b, c \}$? They are:
$\emptyset$, $\{a \}$, $\{b\}$, $\{c\}$, $\{a, b\}$, $\{a, c\}$, $\{b,
c\}$, $\{a, b, c\}$. The set of all these subsets is
$\Pow{\{a,b,c\}}$:
\[
\Pow{\{ a, b, c \}} = \{\emptyset, \{a \}, \{b\}, \{c\}, \{a, b\},
\{b, c\}, \{a, c\}, \{a, b, c\}\}
\]
\end{ex}

\begin{prob}
List all subsets of $\{a, b, c, d\}$.
\end{prob}

\begin{prob}
Show that if $A$ has $n$ !!{element}s, then $\Pow{A}$ has $2^n$
!!{element}s.
\end{prob}

\end{document}
