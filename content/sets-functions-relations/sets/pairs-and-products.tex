% Part: sets-functions-relations
% Chapter: sets
% Section: pairs-and-products

\documentclass[../../../include/open-logic-section]{subfiles}

\begin{document}

\olfileid{sfr}{set}{pai}
\olsection{Pairs, Tuples, Cartesian Products}

\begin{explain}
Sets have no order to their elements. We just think of them as an
unordered collection. So if we want to represent order, we use
\emph{ordered pairs} $\tuple{x, y}$. In an unordered pair $\{x, y\}$,
the order does not matter: $\{x, y\} = \{y, x\}$. In an ordered pair,
it does: if $x \neq y$, then $\tuple{x, y} \neq \tuple{y, x}$.

Sometimes we also want ordered sequences of more than two objects,
e.g., \emph{triples} $\tuple{x, y, z}$, \emph{quadruples} $\tuple{x,
  y, z, u}$, and so on.  In fact, we can think of triples as special
ordered pairs, where the first element is itself an ordered pair:
$\tuple{x, y, z}$ is short for $\tuple{\tuple{x, y},z}$. The same is
true for quadruples: $\tuple{x,y,z,u}$ is short for
$\tuple{\tuple{\tuple{x,y},z},u}$, and so on. In general, we talk of
\emph{ordered $n$-tuples} $\tuple{x_1, \dots, x_n}$.
\end{explain}

\begin{defn}[Cartesian product]
Given sets $X$ and $Y$, their \emph{Cartesian product} $X \times Y$ is
$\Setabs{\tuple{x, y}}{x \in X \text{ and } y \in Y}$.
\end{defn}

\begin{ex}
If $X = \{0, 1\}$, and $Y = \{1, a, b\}$, then their product is
\[
X \times Y = \{ \tuple{0, 1}, \tuple{0, a}, \tuple{0, b},
    \tuple{1, 1}, \tuple{1, a}, \tuple{1, b} \}.
\]
\end{ex}

\begin{ex}
If $X$ is a set, the product of $X$ with itself, $X \times X$, is also
written~$X^2$. It is the set of \emph{all} pairs $\tuple{x, y}$ with
$x, y \in X$. The set of all triples $\tuple{x, y, z}$ is $X^3$, and
so on. We can give an inductive definition:
\begin{align*}
  X^1 & = X\\
  X^{k+1} & = X^k \times X
\end{align*}
\end{ex}

\begin{prob}
List all !!{element}s of $\{1, 2, 3\}^3$.
\end{prob}

\begin{prop}
If $X$ has $n$ !!{element}s and $Y$ has $m$ !!{element}s, then $X
\times Y$ has $n\cdot m$ elements.
\end{prop}

\begin{proof}
For every !!{element}~$x$ in~$X$, there are $m$ !!{element}s of the
form $\tuple{x, y} \in X \times Y$. Let $Y_x = \Setabs{\tuple{x, y}}{y
  \in Y}$. Since whenever $x_1 \neq x_2$, $\tuple{x_1, y} \neq
\tuple{x_2, y}$, $Y_{x_1} \cap Y_{x_2} = \emptyset$. But if $X = \{x_1,
\dots, x_n\}$, then $Y = Y_{x_1} \cup \dots \cup Y_{x_n}$, so has
$n\cdot m$ !!{element}s.

To visualize this, arrange the !!{element}s of~$X \times Y$ in a grid:
\[
\begin{array}{rcccc}
  Y_{x_1} = & \{\tuple{x_1, y_1} & \tuple{x_1, y_2} & \dots & \tuple{x_1, y_m}\}\\
  Y_{x_2} = & \{\tuple{x_2, y_1} & \tuple{x_2, y_2} & \dots & \tuple{x_2, y_m}\}\\
  \vdots & & \vdots\\
  Y_{x_n} = & \{\tuple{x_n, y_1} & \tuple{x_n, y_2} & \dots & \tuple{x_n, y_m}\}
\end{array}
\]
Since the $x_i$ are all different, and the $y_j$ are all different, no
two of the pairs in this grid are the same, and there are $n\cdot m$
of them.
\end{proof}

\begin{prob}
Show, by induction on~$k$, that for all $k \ge 1$, if $X$ has $n$
!!{element}s, then $X^k$ has $n^k$ !!{element}s.
\end{prob}

\begin{ex}
If $X$ is a set, a \emph{word} over~$X$ is any sequence of
!!{element}s of~$X$. A sequence can be thought of as an $n$-tuple of
!!{element}s of~$X$. For instance, if $X = \{a, b, c\}$, then the
sequence ``$bac$'' can be thought of as the triple~$\tuple{b, a, c}$.
Words, i.e., sequences of symbols, are of crucial importance in
computer science, of course. By convention, we count !!{element}s of
$X$ as sequences of length~1, and $\emptyset$ as the sequence of
length~0. The set of \emph{all} words over~$X$ then is
\[
X^* = \{\emptyset\} \cup X \cup X^2 \cup X^3 \cup \dots
\]
\end{ex}

\end{document}
