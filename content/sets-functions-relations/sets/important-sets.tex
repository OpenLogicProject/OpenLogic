% Part: sets-functions-relations
% Chapter: sets
% Section: important-sets

\documentclass[../../../include/open-logic-section]{subfiles}

\begin{document}

\olfileid{sfr}{set}{imp}
\olsection{Some Important Sets}

\begin{ex}
We will mostly be dealing with sets whose !!{element}s are
mathematical objects. Four such sets are important enough to have
specific names:
\begin{multline*}
    \Nat = \{0, 1, 2, 3, \ldots\} \\
    \shoveright{\text{the set of natural numbers}}\\
    \shoveleft{\Int = \{\ldots, -2, -1, 0, 1, 2, \ldots\}} \\
    \shoveright{\text{the set of integers}}\\
    \shoveleft{\Rat = \Setabs{\nicefrac{m}{n}}{m, n \in \Int\text{ and }n \neq 0}}\\
    \shoveright{\text{the set of rationals}}\\
    \shoveleft{\Real = (-\infty, \infty)}\\
    \text{the set of real numbers (the continuum)}
\end{multline*}
These are all \emph{infinite} sets, that is, they each have
infinitely many !!{element}s.

As we move through these sets, we are adding \emph{more} numbers to
our stock. Indeed, it should be clear that $\Nat \subseteq \Int
\subseteq \Rat \subseteq \Real$: after all, every natural number is an
integer; every integer is a rational; and every rational is a real.
Equally, it should be clear that $\Nat \subsetneq \Int \subsetneq
\Rat$, since $-1$ is an integer but not a natural number, and
$\nicefrac{1}{2}$ is rational but not integer. It it is less obvious
that $\Rat \subsetneq \Real$, i.e., that there are some real numbers
which are not rational\oliflabeldef{sfr:arith:real:realline}{, but we'll
return to this in \olref[arith][real]{realline}}{}. 

We'll sometimes also use the set of positive integers $\PosInt = \{1,
2, 3, \dots\}$ and the set containing just the first two natural
numbers $\Bin = \{0, 1\}$.
\end{ex}

\begin<compsci>{ex}[Strings] 
Another interesting example  is the set $A^{*}$ of \emph{finite
strings} over an alphabet $A$: any finite sequence of elements of~$A$
is a string over $A$. We include the \emph{empty string $\Lambda$}
among the strings over~$A$, for every alphabet~$A$. For instance,
\begin{multline*}
\Bin^*
=\{\Lambda,0,1,00,01,10,11,\\
000,001,010,011,100,101,110,111,0000,\ldots\}.
\end{multline*}
If $x=x_{1}\ldots x_{n}\in A^{*}$is a string consisting of $n$
``letters'' from $A$, then we say \emph{length} of the string is~$n$
and write $\len{x}=n$.
\end{ex}

\begin{ex}[Infinite sequences]
For any set $A$ we may also consider the set~$A^\omega$ of infinite
sequences of !!{element}s of~$A$. An infinite sequence
$a_1a_2a_3a_4\dots$ consists of a one-way infinite list of objects,
each one of which is !!a{element} of~$A$.
\end{ex}

\end{document}
