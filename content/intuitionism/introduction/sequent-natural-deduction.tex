\documentclass[../../../include/open-logic-section]{subfiles}

\begin{document}

\olsection{Sequent Natural Deduction}

Let us write $\Gamma \fCenter !A$ if there is a deduction tree with
$\Gamma$ (a set of formulas) as assumptions and $!A$ as conclusion;
, or $\fCenter!A$ for empty $\Gamma$.

We write $\Gamma, !A_1, \dots, !A_n$ for $\Gamma \cup
\{!A_1,\dots,!A_n\}$; $\Gamma, \Delta$ for $\Gamma \cup \Delta$.

Observe that when we have $\Gamma \fCenter !A_1 \land !A_2$, meaning we
have a derivation tree with $\Gamma$ as premises and $!A_1 \land !A_2$ as
conclusion, then by applying $\Elim{\land}$ at the bottom, we can always get a
derivation tree with the same premises and $!A_1$ as conclusion; thus
$\Gamma \fCenter !A_1$.
\begin{gather*}
  \def\fCenter{\ \Proves \ }
  \Axiom$\Gamma \fCenter !A_1 \land !A_2$
  \RightLabel{$\Elim{\land}$}
  \UnaryInf$\Gamma \fCenter !A_i$
  \DisplayProof
  \;
  i \in \{1,2\}
\end{gather*}
The right label $\Elim{\land}$ hints its relation with the rule of
the same name in natural deduction.

Likewise, if we have $\Gamma, !A \fCenter !B$, meaning we have a
derivation tree with premises $\Gamma$ and $!A$,  and conclusion $!B$;
now if we apply the $\Intro{\lif}$ rule, we have a tree with $\Gamma$
as premises and $!A \lif !B$ as the conclusion; Thus $\Gamma \fCenter
!A \lif !B$. Note how this has made discharging more explicit.
\begin{gather*}
  \def\fCenter{\ \Proves \ }
  \Axiom $\Gamma, !A \fCenter !B$
  \RightLabel{$\Intro{\lif}$}
  \UnaryInf $\Gamma \fCenter !A \lif !B$
  \DisplayProof
\end{gather*}

We can draw conclusions from other rules in the same fashion, which is
spelled out as follows:
\begin{gather*}
  \def\fCenter{\ \Proves \ }
  \Axiom $\Gamma \fCenter !A_1$
  \Axiom $\Delta \fCenter !A_2$
  \RightLabel{$\Intro{\land}$}
  \BinaryInf $\Gamma,\Delta \fCenter !A_1 \land !A_2$
  \DisplayProof
  \quad
  \Axiom $\Gamma \fCenter !A_1 \land !A_2$
  \RightLabel{$\Elim{\land}_i$}
  \UnaryInf $\Gamma \fCenter !A_i$
  \DisplayProof
  \;
  i \in \{1,2\}
  \\
  \def\fCenter{\ \Proves \ }
  \Axiom $\Gamma \fCenter !A_i$
  \RightLabel{$\Intro{\lor}_i$}
  \UnaryInf $\Gamma \fCenter !A_1 \lor !A_2$
  \DisplayProof
  \;
  i \in \{1,2\}
  \quad
  \Axiom $\Gamma \fCenter !A_1 \lor !A_2$
  \Axiom $\Delta, !A_1 \fCenter !C$
  \Axiom $\Delta', !A_2 \fCenter !C$
  \RightLabel{$\Elim{\lor}$}
  \TrinaryInf $\Gamma, \Delta, \Delta' \fCenter !C$
  \DisplayProof
  \\
  \def\fCenter{\ \Proves \ }
  \Axiom $\Gamma, !A \fCenter !B$
  \RightLabel{$\Intro{\lif}$}
  \UnaryInf $\Gamma \fCenter !A \lif !B$
  \DisplayProof
  \quad
  \Axiom $\Gamma \fCenter !A$
  \Axiom $\Delta \fCenter !A \lif !B$
  \RightLabel{$\Elim{\lif}$}
  \BinaryInf $\Gamma, \Delta \fCenter !B$
  \DisplayProof
  \\
  \def\fCenter{\ \Proves \ }
  \Axiom $\Gamma \fCenter \lfalse$
  \RightLabel{$\Elim{\lfalse}$}
  \UnaryInf $\Gamma \fCenter !A$
  \DisplayProof
\end{gather*}

Note that all the above observations require existing derivation(s),
but where does the first derivation comes from? Consider when the
derivation tree is trivial, i.e., without any deduction (horizontal
line); in such cases the premise is exactly the conclusion, which
leads to the following observation:

\begin{prooftree}
  \AxiomC{$ $}
  \UnaryInfC{$!A \Proves !A$}
\end{prooftree}

Let's now redo some proofs in this fashion:
\begin{gather*}
  \def\fCenter{\ \Proves \ }
  \Axiom$!A \fCenter !A$
  \UnaryInf $!A \fCenter !A \lor (!A \lif \lfalse)$
  \Axiom $!B \fCenter !B$
  \BinaryInf$!A, !B\lif \fCenter \lfalse$
  \UnaryInf$(!B \fCenter !A \lif \lfalse$
  \UnaryInf$( !B \fCenter !A \lor (!A \lif \lfalse)$
  \Axiom$(!B \fCenter !B$
  \BinaryInf$(!B \fCenter \lfalse$
  \UnaryInf$\fCenter !B \lif \lfalse$
  \DisplayProof
\end{gather*}
where $!B$ is short for $(!A \lor (!A \lif \lfalse))\lif \lfalse$.

\end{document}
