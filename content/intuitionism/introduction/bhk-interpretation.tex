\documentclass[../../../include/open-logic-section]{subfiles}

\begin{document}

\olsection{BHK interpretation}

There is an informal constructive interpretation of the intuitionist
connectives, usually known as the Brouwer-Heyting-Kolmogorov
interpretation. Assuming our language ranges over natural numbers, it says:
\begin{enumerate}
\item We assume that we know what constitute a proof of an atomic statement.
\item A proof of $!A_1 \land !A_2$ is a pair $andi{M_1}{M_2}$ where $M_1$ is a
  proof of $!A_1$ and $M_2$ is a proof of $A_2$.
\item A proof of $!A_1 \lor !A_2$ is a pair $andi{s}{M}$ where $s$ is $1$
  and $M$ is a proof of $!A_1$, or $s$ is $2$ and $M$ is a proof of
  $!A_2$.
\item A proof of $!A \lif !B$ is a function that converts a proof
  of $!A$ into a proof of $!B$.
\item A proof of $(\lforall x ) !A(x)$ is a function that converts
  a natural number $n$ to a proof of $!A(n)$.
\item A proof of $(\lexists x) !A(x)$ is a pair $\andi{n}{M}$
  where $n$ is a natural number, and $M$ is a proof of $!A(M)$.
\item There is no proof for $\lfalse$ (absurdity).
\item $\lnot !A$ is defined as synonym for $!A \lif \lfalse$.
\end{enumerate}

Note how the negation of a proposition $!A$ is handled: it's defined to
be $!A \lif \lfalse$, which is a function converting a proof of $!A$
into $\lfalse$.

\begin{ex}
Take $\lnot 0=1$ for example. The reader may recall that $\lfalse$ has no
proof, which is not a problem as no one can provide a proof of $0=1$ in the
first place; thus we may use the identify function (or any other
function) as the proof, and our criteria for a function to be a proof
(returning $\lfalse$ given proof of $0=1$) vacuously holds.
\end{ex}

General speaking, $\lnot !A$ means ``A proof of $!A$ is impossible''.

\begin{ex}
  Let us prove $!A \lif \lnot \lnot !A$ for any proposition $!A$, which is  $!A \lif ((!A
\lif \lfalse) \lif \lfalse)$. The proof should be a function that,
given a proof of $!A$, returns a proof of $(!A \lif \lfalse) \lif
\lfalse$. Here is how we construct $(!A \lif \lfalse) \lif
\lfalse$: it's a function that accepts a proof of $!A \lif
\lfalse$, applies it to the proof of $!A$ (that we received earlier) to
get $\lfalse$, return the $\lfalse$ as required.
\end{ex}

Note in the above example we can't use the same trick when proving $\lnot (0=1)$,
since it's possible that the skeptic of the proposition indeed has
the proof of $!A$ or $\lnot !A$. Is it possible that $!A$ and $\lnot
!A$ is both provable? We prove the following formula:

\begin{ex}
Let us prove $\lnot(!A \land \lnot !A)$, which is $(!A \land (!A \lif \lfalse))
\lif \lfalse$. First we receive a proof of $!A \land (!A \lif \lfalse)$
which should be a pair whose first element is proof of $!A$ and second
proof is $!A \lif \lfalse$. Applying the second to the first, we get
$\lfalse$ as required.
\end{ex}

\begin{ex}
Let us prove $(!A \land !B \lif !C) \lif (!A \lif !B \lif !C)$: given $!A \land !B
\lif !C$ we give $!A \lif !B \lif !C$, which we construct as
follows: given $!A$ we give $!B \lif !C$, which we construct
as follows: given $!B$ we give $!C$, which we construct as follows:
construct $!A \land !B$ from the given $!A$ and $!B$, pass it to the
given $!A \land !B \lif !C$ and we get $!C$ as required.
\end{ex}

Note how this is related to currying we introduced in lambda
calculus: if we consider the actual proof terms, we will find our
proof a function currying functions; the converse of the formula is also
provable (the proof is left to the reader) and corresponds to uncurrying.

The statement $!A \lor \lnot !A$ is called LEM (Law of Excluded
Middle). We can prove it for some specific $!A$(say,
$0=1$ or $0=0$), but not in general because the
intuitionistic disjunction requires a proof of either side, but there
are statements neither proved or refuted(say, Goldbach's conjecture as
of 2016). However, you can't refute LEM either: that is, $\lnot
\lnot (!A \lor \lnot !A)$.

\begin{ex}
  Here is how we prove it:given $(!A \lor (!A \lif \lfalse))\lif \lfalse$, we are expected to give $\lfalse$. We
now try to construct $!A \lor (!A \lif \lfalse)$ so we can apply the given
function to get $\lfalse$.

But wait, haven't we just established that $!A \lor (!A \lif \lfalse)$
is not provable in general? But note that we are in a
position different from the general case: now we have $(!A \lor (!A \lif \lfalse))\lif \lfalse$
as premise, which is enough for us to derive LEM: we will construct the right side of the
disjunction, $!A \lif \lfalse$: we are given $!A$ and expected to give
$\lfalse$. Attention! From $!A$ we can derive $!A \lor (!A \lif
\lfalse)$; applying the mentioned premise to it we get 
$\lfalse$, as required.
\end{ex}

This kind of proving soon gets too twisted to construct and understand;
we thus need a more formal language to construct such proofs.
\end{document}