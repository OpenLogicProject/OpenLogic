% Part: computability
% Chapter: computability-theory
% Section: fixed-point-thm

\documentclass[../../../include/open-logic-section]{subfiles}

\begin{document}

\olfileid{cmp}{thy}{fix}
\olsection{The Fixed-Point Theorem}

Let's consider the halting problem again. As temporary
notation, let us write $\gn{\cfind{x}(y)}$ for $\tuple{x, y}$; think of
this as representing a ``name'' for the value $\cfind{x}(y)$. With this
notation, we can reword one of our proofs that the halting problem is
undecidable.

Question: is there a computable function $h$, with the
following property? For every $x$ and $y$,
\[
h(\gn{\cfind{x}(y)}) =
\begin{cases}
1 & \text{if $\cfind{x}(y) \fdefined$} \\
0 & \text{otherwise.}
\end{cases}
\]

Answer: No; otherwise, the partial function
\[
g(x) \simeq
\begin{cases}
0 & \text{if $h(\gn{\cfind{x}(x)}) = 0$} \\
\text{undefined} & \text{otherwise}
\end{cases}
\]
would be computable, and so have some index~$e$. But then we have
\[
\cfind{e}(e) \simeq
\begin{cases}
0 & \text{if $h(\gn{\cfind{e}(e)}) = 0$} \\
\text{undefined} & \text{otherwise,}
\end{cases}
\]
in which case $\cfind{e}(e)$ is defined if and only if it isn't, a
contradiction.

Now, take a look at the equation with $\cfind{e}$. There is an instance of
self-reference there, in a sense: we have arranged for the value of
$\cfind{e}(e)$ to depend on $\gn{\cfind{e}(e)}$, in a certain way. The
fixed-point theorem says that we {\em can} do this, in general---not
just for the sake of proving contradictions.

\olref{lem:fixed-equiv} gives two equivalent ways of stating the
fixed-point theorem. Logically speaking, the fact that the statements
are equivalent follows from the fact that they are both true; but what
we really mean is that each one follows straightforwardly from the
other, so that they can be taken as alternative statements of the same
theorem.

\begin{lem}
\ollabel{lem:fixed-equiv}
The following statements are equivalent:
\begin{enumerate}
\item For every partial computable function $g(x,y)$, there is an
  index~$e$ such that for every~$y$,
\[
\cfind{e}(y) \simeq g(e,y).
\]
\item For every computable function~$f(x)$, there is an index~$e$ such
  that for every~$y$,
\[
\cfind{e}(y) \simeq \cfind{f(e)}(y).
\]
\end{enumerate}
\end{lem}

\begin{proof}
$(1) \Rightarrow (2)$: Given $f$, define $g$ by $g(x,y) \simeq
\fn{Un}(f(x),y)$. Use (1) to get an index~$e$ such that for every~$y$,
\begin{align*}
\cfind{e}(y) & = \fn{Un}(f(e),y) \\
& = \cfind{f(e)}(y).
\end{align*}

$(2) \Rightarrow (1)$: Given $g$, use the $s$-$m$-$n$ theorem to get $f$ such
that for every $x$ and~$y$, $\cfind{f(x)}(y) \simeq g(x,y)$. Use (2) to
get an index~$e$ such that
\begin{align*}
\cfind{e}(y) & = \cfind{f(e)}(y) \\
& = g(e,y).
\end{align*}
This concludes the proof.
\end{proof}

\begin{explain}
Before showing that statement (1) is true (and hence (2) as well),
consider how bizarre it is. Think of $e$ as being a computer program;
statement (1) says that given any partial computable $g(x,y)$, you can
find a computer program $e$ that computes $g_e(y) \simeq g(e,y)$. In
other words, you can find a computer program that computes a function
that references the program itself.
\end{explain}

\begin{thm}
The two statements in \olref{lem:fixed-equiv} are
true. Specifically, for every partial computable function $g(x,y)$,
there is an index~$e$ such that for every~$y$,
\[
\cfind{e}(y) \simeq g(e,y).
\]
\end{thm}

\begin{proof}
The ingredients are already implicit in the discussion of the halting
problem above. Let $\fn{diag}(x)$ be a computable function which for each
$x$ returns an index for the function $f_x(y) \simeq \cfind{x}(x,y)$,
i.e.
\[
\cfind{\fn{diag}(x)}(y) \simeq \cfind{x}(x,y).
\]
Think of $\fn{diag}$ as a function that transforms a program for a 2-ary
function into a program for a 1-ary function, obtained by fixing the
original program as its first argument. The function $\fn{diag}$ can be
defined formally as follows: first define $s$ by
\[
s(x,y) \simeq \fn{Un}^2(x,x,y),
\]
where $\fn{Un}^2$ is a 3-ary function that is universal for partial computable
2-ary functions. Then, by the $s$-$m$-$n$ theorem, we can find a primitive
recursive function $\fn{diag}$ satisfying
\[
\cfind{\fn{diag}(x)}(y) \simeq s(x,y).
\]

Now, define the function $l$ by
\[
l(x,y) \simeq g(\fn{diag}(x),y).
\]
and let $\gn{l}$ be an index for $l$. Finally, let $e = \fn{diag}(\gn{l})$.
Then for every $y$, we have
\begin{align*}
\cfind{e}(y) & \simeq \cfind{\fn{diag}(\gn{l})}(y) \\
& \simeq \cfind{\gn{l}}(\gn{l}, y) \\
& \simeq l(\gn{l}, y) \\
& \simeq g(\fn{diag}(\gn{l}),y) \\
& \simeq g(e, y),
\end{align*}
as required.
\end{proof}

\begin{explain}
What's going on? Suppose you are given the task of writing a computer
program that prints itself out. Suppose further, however, that you are
working with a programming language with a rich and bizarre library of
string functions. In particular, suppose your programming language has
a function $\fn{diag}$ which works as follows: given an input
string~$s$, $\fn{diag}$ locates each instance of the symbol `x'
occurring in~$s$, and replaces it by a quoted version of the original
string. For example, given the string
\begin{quote}
\begin{verbatim}
hello x world
\end{verbatim}
\end{quote}
as input, the function returns
\begin{quote}
\begin{verbatim}
hello 'hello x world' world
\end{verbatim}
\end{quote}
as output. In that case, it is easy to write the desired program; you
can check that
\begin{quote}
\begin{verbatim}
print(diag('print(diag(x))'))
\end{verbatim}
\end{quote}
does the trick. For more common programming languages like C++ and
Java, the same idea (with a more involved implementation) still works.

We are only a couple of steps away from the proof of the fixed-point
theorem. Suppose a variant of the print function $\fn{print}(x,y)$
accepts a string $x$ and another numeric argument $y$, and prints the
string $x$ repeatedly, $y$ times. Then the ``program''
\begin{quote}
\begin{verbatim}
getinput(y);
print(diag('getinput(y); print(diag(x), y)'), y)
\end{verbatim}
\end{quote}
prints itself out $y$ times, on input $y$. Replacing the
$\fn{getinput}$---$\fn{print}$---$\fn{diag}$ skeleton by an
arbitrary function $g(x,y)$ yields
\begin{quote}
\begin{verbatim}
g(diag('g(diag(x), y)'), y)
\end{verbatim}
\end{quote}
which is a program that, on input $y$, runs $g$ on the program itself
and $y$. Thinking of ``quoting'' with ``using an index for,'' we have
the proof above.

For now, it is o.k.\ if you want to think of the proof as formal
trickery, or black magic. But you should be able to reconstruct the
details of the argument given above. When we prove the incompleteness
theorems (and the related ``fixed-point theorem'') we will discuss
other ways of understanding why it works.
\end{explain}

\begin{tagblock}{lambda}
\begin{digress}
The same idea can be used to get a ``fixed point'' combinator. Suppose
you have a lambda term $g$, and you want another term $k$ with the
property that $k$ is $\beta$-equivalent to $gk$. Define terms
\[
\fn{diag}(x) = xx
\]
and
\[
l(x) = g(\fn{diag}(x))
\]
using our notational conventions; in other words, $l$ is the term
$\lambd[x][g(xx)]$. Let $k$ be the term $ll$. Then we have
\begin{align*}
k & = (\lambd[x][g(xx)])(\lambd[x][g(xx)]) \\
& \red  g((\lambd[x][g(xx)])(\lambd[x][g(xx)])) \\
& = gk.
\end{align*}
If one takes
\[
Y = \lambd[g][((\lambd[x][g(xx)])(\lambd[x][g(xx)]))]
\]
then $Yg$ and $g(Yg)$ reduce to a common term; so $Yg \equiv_\beta
g(Yg)$. This is known as ``Curry's combinator.'' If instead one takes
\[
Y = (\lambd[xg][g(xxg)])(\lambd[xg][g(xxg)])
\]
then in fact $Yg$ reduces to $g(Yg)$, which is a stronger statement.
This latter version of $Y$ is known as ``Turing's combinator.''
\end{digress}
\end{tagblock}

\end{document}
