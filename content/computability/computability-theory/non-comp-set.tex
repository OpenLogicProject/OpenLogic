% Part: computability
% Chapter: computability-theory
% Section: non-comp-set

\documentclass[../../../include/open-logic-section]{subfiles}

\begin{document}

\olfileid{cmp}{thy}{ncp}
\olsection{There Are Non-Computable Sets}


We saw above that every computable set is computably enumerable. Is
the converse true? The following shows that, in general, it is not.

\begin{thm}\ollabel{thm:K-0}
Let $K_0$ be the set $\Setabs{\tuple{e, x}}{\cfind{e}(x) \fdefined}$.
Then $K_0$ is computably enumerable but not computable.
\end{thm}

\begin{proof}
To see that $K_0$ is computably enumerable, note that it is the
domain of the function~$f$ defined by
\[
f(z) = \umin{y}{(\len{z} = 2 \land T((z)_0, (z)_1, y))}.
\]
For, if $\cfind{e}(x)$ is defined, $f(\tuple{e, x})$ finds a halting
computation sequence; if $\cfind{e}(x)$ is undefined, so is
$f(\tuple{e, x})$; and if $z$ doesn't even code a pair, then $f(z)$ is
also undefined.

The fact that $K_0$ is not computable is just the undecidability of
the halting problem, \olref[hlt]{thm:halting-problem}.
\end{proof}

The set $K_0$ is the set of pairs $\tuple{e,x}$ such that
$\cfind{e}(x) \fdefined$, i.e., $\tuple{e,x} \in K_0$ iff $\cfind{e}$
is defined (halts) on input~$x$, so it is also called the ``halting
set.'' The set $K = \Setabs{e}{\cfind{e}(e) \fdefined}$ is the
``self-halting set.'' It is often used as a canonical undecidable set.

\begin{thm}\ollabel{thm:K}
The self-halting set $K = \Setabs{e}{\cfind{e}(e) \fdefined}$ is
!!{c.e.} but not decidable.
\end{thm}

\begin{proof}
  Suppose $K$ is decidable, i.e., its characteristic function
  $\Char{K}$ is computable. Let 
  \[d(e) = \begin{cases}
  1 & \text{if\/ $\Char{K}(e) = 0$}\\
  \fundefined & \text{otherwise.}
  \end{cases}
  \] 
  Let $k$ be the index of~$d$, i.e., $d \simeq \cfind{k}$. Then $d(k)
  \simeq \cfind{k}(k)$. This contradicts the fact that $d(k)
  \fdefined$ iff $\cfind{k}(k) \fundefined$, which follows from the
  definition of~$d$.

  $K$ is the domain of $f(x) = \umin{y}{T(x,x,y)}$ and so is !!{c.e.}
\end{proof}

\end{document}
