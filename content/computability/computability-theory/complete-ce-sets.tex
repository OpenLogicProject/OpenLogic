% Part: computability
% Chapter: computability-theory
% Section: complete-ce-sets

\documentclass[../../../include/open-logic-section]{subfiles}

\begin{document}

\olfileid{cmp}{thy}{cce}
\olsection{Complete Computably Enumerable Sets}

\begin{defn}
A set $A$ is a \emph{complete !!{computably enumerable} set}
(under many-one reducibility) if
\begin{enumerate}
\item $A$ is computably enumerable, and
\item for any other computably enumerable set $B$, $B \leq_m A$.
\end{enumerate}
\end{defn}

In other words, complete computably enumerable sets are the
``hardest'' computably enumerable sets possible. They allow one to
answer questions about \emph{any} computably enumerable set.

\begin{thm}
$K$, $K_0$, and $K_1$ are all complete computably enumerable sets.
\end{thm}

\begin{proof}
To see that $K_0$ is complete, let $B$ be any computably
enumerable set. Then for some index $e$,
\[
B = W_e = \Setabs{x}{\cfind{e}(x) \fdefined}.
\]
Let $f$ be the function $f(x) = \tuple{e, x}$. Then for every natural
number $x$, $x \in B$ if and only if $f(x) \in K_0$. In other words, $f$
reduces $B$ to~$K_0$.

To see that $K_1$ is complete, note that in the proof of
\olref[k1]{prop:k1} we reduced $K_0$ to it. So, by
\olref[ppr]{prop:trans-red}, any computably enumerable set can be
reduced to~$K_1$ as well.

$K$ can be reduced to $K_0$ in much the same way.
\end{proof}

\begin{prob}
Give a reduction of $K$ to $K_0$.
\end{prob}

\begin{digress}
So, it turns out that all the examples of computably enumerable sets
that we have considered so far are either computable, or complete.
This should seem strange!{} Are there any examples of computably
enumerable sets that are neither computable nor complete? The answer
is yes, but it wasn't until the middle of the 1950s that this was
established by Friedberg and Muchnik, independently.
\end{digress}

\end{document}

