% Part: computability
% Chapter: computability-theory
% Section: rice-theorem

\documentclass[../../../include/open-logic-section]{subfiles}

\begin{document}

\olfileid{cmp}{thy}{rce}
\olsection{Rice's Theorem}

If you think about it, you will see that the specifics of $\fn{Tot}$
do not play into the proof of \olref[tot]{prop:total}. We designed
$h(x,y)$ to act like the constant function $j(y) = 0$ exactly when $x$
is in $K$; but we could just as well have made it act like any other
partial computable function under those circumstances. This
observation lets us state a more general theorem, which says, roughly,
that no nontrivial property of computable functions is decidable.

Keep in mind that $\cfind{0}$, $\cfind{1}$, $\cfind{2}$,~\dots is our
standard enumeration of the partial computable functions.

\begin{thm}[Rice's Theorem]
  Let $C$ be any set of partial computable functions, and let $A =
  \Setabs{n}{\cfind{n} \in C}$. If $A$ is computable, then either $C$
  is $\emptyset$ or $C$ is the set of all the partial computable
  functions.
\end{thm}

An {\em index set} is a set $A$ with the property that if $n$ and $m$
are indices which ``compute'' the same function, then either both $n$
and $m$ are in $A$, or neither is. It is not hard to see that the
set~$A$ in the theorem has this property. Conversely, if $A$~is an
index set and $C$~is the set of functions computed by these indices,
then $A = \Setabs{n}{\cfind{n} \in C}$.

\begin{explain}
With this terminology, Rice's theorem is equivalent to saying that no
nontrivial index set is decidable. To understand what the theorem
says, it is helpful to emphasize the distinction between
\emph{programs} (say, in your favorite programming language) and the
functions they compute. There are certainly questions about programs
(indices), which are syntactic objects, that are computable: does this
program have more than 150 symbols? Does it have more than 22 lines?
Does it have a ``while'' statement? Does the string ``hello world''
every appear in the argument to a ``print'' statement? Rice's theorem
says that no nontrivial question about the program's \emph{behavior}
is computable. This includes questions like these: does the program
halt on input $0$? Does it ever halt? Does it ever output an even
number?
\end{explain}

\begin{proof}[Proof of Rice's theorem]
Suppose $C$ is neither $\emptyset$ nor the set of all the partial
computable functions, and let $A$ be the set of indices of functions
in~$C$. We will show that if $A$ were computable, we could solve the
halting problem; so $A$~is not computable.

Without loss of generality, we can assume that the function $f$ which
is nowhere defined is not in $C$ (otherwise, switch $C$ and its
complement in the argument below). Let $g$ be any function in~$C$. The
idea is that if we could decide~$A$, we could tell the difference
between indices computing~$f$, and indices computing~$g$; and then we
could use that capability to solve the halting problem.

Here's how. Using the universal computation predicate, we can define a
function
\[
h(x,y) \simeq
\begin{cases}
\text{undefined} & \text{if $\cfind{x}(x) \undefined$} \\
g(y) & \text{otherwise.}
\end{cases}
\]
To compute $h$, first we try to compute $\cfind{x}(x)$; if that
computation halts, we go on to compute $g(y)$; and if {\em that}
computation halts, we return the output. More formally, we can write
\[
h(x,y) \simeq \Proj{2}{0}(g(y),\fn{Un}(x,x)).
\]
where $\Proj{2}{0}(z_0, z_1) = z_0$ is the $2$-place projection
function returning the $0$-th argument, which is computable.

Then $h$ is a composition of partial computable functions, and the right
side is defined and equal to $g(y)$ just when $\fn{Un}(x,x)$ and
$g(y)$ are both defined.

Notice that for a fixed $x$, if $\cfind{x}(x)$ is undefined, then
$h(x,y)$ is undefined for every~$y$; and if $\cfind{x}(x)$ is defined,
then $h(x,y) \simeq g(y)$. So, for any fixed value of~$x$, either
$h(x,y)$ acts just like $f$ or it acts just like $g$, and deciding
whether or not $\cfind{x}(x)$ is defined amounts to deciding which of
these two cases holds. But this amounts to deciding whether or not
$h_x(y) \simeq h(x,y)$ is in $C$ or not, and if $A$ were computable,
we could do just that.

More formally, since $h$~is partial computable, it is equal to the
function $\cfind{k}$ for some index~$k$. By the $s$-$m$-$n$ theorem
there is a primitive recursive function $s$ such that for each $x$,
$\cfind{s(k,x)}(y) = h_x(y)$. Now we have that for each $x$, if
$\cfind{x}(x) \defined$, then $\cfind{s(k,x)}$ is the same function
as~$g$, and so $s(k,x)$ is in $A$. On the other hand, if $\cfind{x}(x)
\uparrow$, then $\cfind{s(k,x)}$ is the same function as $f$, and so
$s(k,x)$ is not in $A$. In other words we have that for every $x$, $x
\in K$ if and only if $s(k,x) \in A$. If $A$ were computable, $K$
would be also, which is a contradiction. So $A$ is not computable.
\end{proof}

Rice's theorem is very powerful. The following immediate corollary
shows some sample applications.
\begin{cor}
The following sets are undecidable.
\begin{enumerate}
\item $\Setabs{x}{\text{17 is in the range of $\cfind{x}$}}$
\item $\Setabs{x}{\text{$\cfind{x}$ is constant}}$
\item $\Setabs{x}{\text{$\cfind{x}$ is total}}$
\item $\Setabs{x}{\text{whenever $y < y'$, $\cfind{x}(y) \defined$, and
    if $\cfind{x}(y') \defined$, then $\cfind{x}(y) < \cfind{x}(y')$}}$
\end{enumerate}
\end{cor}

\begin{proof}These are all nontrivial index sets. \end{proof}

\end{document}

