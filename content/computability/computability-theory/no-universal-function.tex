% Part: computability
% Chapter: computability-theory
% Section: no-universal-function

\documentclass[../../../include/open-logic-section]{subfiles}

\begin{document}

\olfileid{cmp}{thy}{nou}
\olsection{No Universal Computable Function}

Although there is a partial computable function that is total for the
partial computable functions, there is no total computable
function that is universal for the total computable functions. 

\begin{thm}
\ollabel{thm:no-univ}
There is no universal computable function. In other words, any
function $\fn{Un}'(k, x)$ which is such that if $f(x)$ is a total
computable function, then there is a natural number~$k$ such that
$f(x) = \fn{Un}'(k,x)$ for every~$x$, is not computable.
\end{thm}

\begin{proof}
The proof is a simple diagonalization: if $\fn{Un}'(k,x)$ were total
and computable, then
\[
d(x) = \fn{Un}'(x, x) + 1
\]
would also be total and computable. However, by definition, $d(k)$ is
not equal to $\fn{Un}'(k,k)$. Hence, for every $k$, the values of
$d(x)$ and~$\fn{Un}'(k, x)$ differ for at least one~$x$, namely $x = k$.
\end{proof}

\begin{explain}
\olref[uni]{thm:univ-comp} above shows that we can get around this
diagonalization argument, but only at the expense of allowing the
universal function to be partial. That is, $\fn{Un}$ is universal for
the total computable functions, it just isn't total. The
diagonalization argument doesn't work in the partial case. 
\end{explain}

\begin{prob}
  To understand why the diagonalization argument in the proof of
  \olref{thm:no-univ} does not work in the partial
  case, consider the function $f(x) \simeq \fn{Un}(x,x)+1$. Is it
  partial computable? If so, it has an index~$e$, i.e., $f(x) \simeq
  \fn{Un}(e,x)$. What can you say about~$f(e)$?
\end{prob}


\end{document}
