% Part: computability
% Chapter: recursive-functions
% Section: sequences

\documentclass[../../../include/open-logic-section]{subfiles}

\begin{document}

\olfileid{cmp}{rec}{seq}
\olsection{Sequences}

The set of primitive recursive functions is remarkably robust. But we
will be able to do even more once we have developed a adequate means
of handling \emph{sequences}. We will identify finite sequences of
natural numbers with natural numbers in the following way: the
sequence $\langle a_0, a_1, a_2, \dots, a_k \rangle$ corresponds to
the number
\[
p_0^{a_0+1} \cdot p_1^{a_1+1} \cdot p_2^{a_2+1} \cdot \dots \cdot
p_k^{a_k+1}.
\]
We add one to the exponents to guarantee that, for example, the
sequences $\langle 2, 7, 3\rangle$ and $\langle 2, 7, 3, 0, 0 \rangle$
have distinct numeric codes. We can take both $0$ and~$1$ to code the
empty sequence; for concreteness, let $\emptyseq$ denote~$0$.

The reason that this coding of sequences works is the so-called
Fundamental Theorem of Arithmetic: every natural number $n \ge 2$ can
be written in one and only one way in the form
\[
n = p_0^{a_0} \cdot p_1^{a_1} \cdot \dots \cdot p_k^{a_k}
\]
with $a_k \ge 1$. This guarantees that the mapping $\tuple{}(a_0,
\dots, a_k) = \tuple{a_0, \dots, a_k}$ is injective: different
sequences are mapped to different numbers; to each number only at most
one sequence corresponds.

We'll now show that the operations of determining the length of a
sequence, determining its $i$th element, appending an element to a
sequence, and concatenating two sequences, are all primitive
recursive.

\begin{prop}
  The function $\len{s}$, which returns the length of the sequence
  $s$, is primitive recursive.
\end{prop}

\begin{proof}
  Let $R(i, s)$ be the relation defined by
  \[
  R(i, s) \text{ iff }
  p_i \mid s \land p_{i+1} \nmid s.
  \]
  $R$ is clearly primitive recursive. Whenever $s$ is the code of a
  non-empty sequence, i.e.,
  \[
  s = p_0^{a_0+1} \cdot \dots \cdot p_{k}^{a_{k}+1},
  \]
  $R(i,s)$ holds if $p_i$ is the largest prime such that
  $p_i \mid s$, i.e., $i = k$. The length of $s$ thus is $i+1$ iff
  $p_i$ is the largest prime that divides~$s$, so we can let
  \[
  \len{s} =
  \begin{cases}
    0 & \text{if $s = 0$ or $s = 1$} \\
    1 + \bmin{i < s}{R(i, s)} & \text{otherwise}
  \end{cases}
  \]
  We can use bounded minimization, since there is only one $i$ that
  satisfies $R(s, i)$ when $s$ is a code of a sequence, and if $i$
  exists it is less than~$s$ itself.
\end{proof}

\begin{prop}
  The function $\fn{append}(s,a)$, which returns the result of appending $a$ to
  the sequence $s$, is primitive recursive.
\end{prop}

\begin{proof}
  $\fn{append}$ can be defined by:
  \[
  \fn{append}(s,a) =
  \begin{cases}
    2^{a+1} & \text{if $s = 0$ or $s = 1$} \\
    s \cdot p_{\len{s}}^{a+1} & \text{otherwise.}
  \end{cases}
  \]
\end{proof}

\begin{prop}
  The function $\fn{element}(s,i)$, which returns the $i$th element of $s$
  (where the initial element is called the $0$th), or $0$ if $i$ is
  greater than or equal to the length of $s$, is primitive recursive.
\end{prop}

\begin{proof}
Note that $a$ is the $i$th element of~$s$ iff $p_i^{a+1}$ is
the largest power of~$p_i$ that divides~$s$, i.e., $p_i^{a+1} \mid s$
but $p_i^{a+2} \nmid s$. So:
  \[
  \fn{element}(s,i) =
  \begin{cases}
    0 & \mbox{if $i \geq \len{s}$} \\
    \bmin{a < s}{(p_i^{a+2} \nmid s)} & \text{otherwise.}
  \end{cases}
  \]
\end{proof}

Instead of using the official names for the functions defined above,
we introduce a more compact notation. We will use $(s)_i$ instead of
$\fn{element}(s,i)$, and $\tuple{s_0, \dots, s_k}$ to abbreviate
\[
\fn{append}(\fn{append}(\dots \fn{append}(\emptyseq,s_0)
\dots),s_k).
\]
Note that if $s$ has length~$k$, the elements of $s$ are
$(s)_0$, \dots,~$(s)_{k-1}$.

\begin{prop}
The function $\fn{concat}(s,t)$, which concatenates two
sequences, is primitive recursive.
\end{prop}

\begin{proof}
  We want a function $\fn{concat}$ with the property that
  \[
    \fn{concat}(\tuple{a_0, \dots, a_k}, \tuple{b_0, \dots, b_l}) =
    \tuple{a_0, \dots, a_k, b_0, \dots, b_l}.
  \]
  We'll use a ``helper'' function
  $\fn{hconcat}(s,t,n)$ which concatenates the first $n$ symbols of $t$
  to~$s$. This function can be defined by primitive recursion as
  follows:
  \begin{align*}
    \fn{hconcat}(s,t,0) & = s\\
    \fn{hconcat}(s,t,n+1) & = \fn{append}(\fn{hconcat}(s,t,n),(t)_n)
    \intertext{Then we can define $\fn{concat}$ by}
    \fn{concat}(s,t) & = \fn{hconcat}(s,t,\len{t}).
  \end{align*}
\end{proof}

We will write $s \concat t$ instead of $\fn{concat}(s,t)$.

It will be useful for us to be able to bound the numeric code of a
sequence in terms of its length and its largest element. Suppose $s$
is a sequence of length~$k$, each element of which is less than or equal
to some number~$x$. Then $s$ has at most $k$ prime factors, each at
most~$p_{k-1}$, and each raised to at most $x+1$ in the prime
factorization of~$s$. In other words, if we define
\[
\fn{sequenceBound}(x,k) = p_{k-1}^{k \cdot (x+1)},
\]
then the numeric code of the sequence~$s$ described above is at
most~$\fn{sequenceBound}(x,k)$.

Having such a bound on sequences gives us a way of defining new
functions using bounded search. For example, we can define
$\fn{concat}$ using bounded search. All we need to do is write down a
primitive recursive \emph{specification} of the object (number of the
concatenated sequence) we are looking for, and a bound on how far to
look. The following works:
\begin{align*}
  \fn{concat}(s,t) = {} & \bmin{v < \fn{sequenceBound}(s+t,\len{s} +
    \len{t})}{} \\
  & \quad(\len{v} = \len{s} + \len{t} \land {}\\
    & \qquad \bforall{i < \len{s}}{((v)_i = (s)_i) \land {} \\
      & \qquad \bforall{j < \len{t}}{((v)_{\len{s}+j} = (t)_j)})}
\end{align*}

\begin{prob}
Show that there is a primitive recursive function~$\fn{sconcat}(s)$
with the property that
\[
\fn{sconcat}(\tuple{s_0, \dots, s_k}) = s_0 \concat \dots \concat s_k.
\]
\end{prob}

\begin{prob}
Show that there is a primitive recursive function~$\fn{tail}(s)$
with the property that
\begin{align*}
  \fn{tail}(\emptyseq) & = 0 \text{ and}\\
  \fn{tail}(\tuple{s_0, \dots, s_{k}}) & = \tuple{s_1, \dots, s_{k}}.
\end{align*}
\end{prob}

\begin{prop}
  \ollabel{prop:subseq}
  The function $\fn{subseq}(s, i, n)$ which returns the subsequence
  of $s$ of length~$n$ beginning at the $i$th element, is primitive
  recursive.
\end{prop}

\begin{proof}
  Exercise.
\end{proof}

\begin{prob}
  Prove \olref[cmp][rec][seq]{prop:subseq}.
\end{prob}
  
\end{document}

