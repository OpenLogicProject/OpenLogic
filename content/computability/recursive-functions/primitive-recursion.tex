% Part: computability
% Chapter: recursive-functions
% Section: primitive-recursion

\documentclass[../../../include/open-logic-section]{subfiles}

\begin{document}

\olfileid{cmp}{rec}{pre}
\olsection{Primitive Recursion}

A characteristic of the natural numbers is that every natural number
can be reached from~$0$ by applying the successor operation~$+1$
finitely many times---any natural number is either~$0$ or the
successor of \dots{} the successor of~$0$. One way to specify a
function~$h\colon\Nat \to \Nat$ that makes use of this fact is this:
(a)~specify what the value of $h$ is for argument~$0$, and (b)~also
specify how to, given the value of $h(x)$, compute the value of
$h(x+1)$. For (a) tells us directly what $h(0)$ is, so $h$~is defined
for~$0$. Now, using the instruction given by (b) for $x=0$, we can
compute $h(1) = h(0+1)$ from~$h(0)$. Using the same instructions for
$x=1$, we compute $h(2) = h(1+1)$ from~$h(1)$, and so on. For every
natural number~$x$, we'll eventually reach the step where we define
$h(x)$ from $h(x+1)$, and so $h(x)$ is defined for all $x \in \Nat$.

For instance, suppose we specify $h\colon \Nat \to \Nat$ by the following
two equations:
\begin{align*}
h(0) & =  1\\
h(x+1) & =  2 \cdot h(x)
\end{align*}
If we already know how to multiply, then these equations give us the
information required for (a) and~(b) above. By successively applying
the second equation, we get that
\begin{align*}
  h(1) & = 2\cdot h(0) = 2,\\
  h(2) & = 2\cdot h(1) = 2\cdot 2,\\
  h(3) & = 2 \cdot h(2) = 2\cdot 2 \cdot 2,\\
  & \vdots
\end{align*}
We see that the function~$h$ we have specified is $h(x) = 2^x$.

The characteristic feature of the natural numbers guarantees that
there is only one function~$h$ that meets these two criteria.  A pair
of equations like these is called a \emph{definition by primitive
recursion} of the function~$h$. It is so-called because we define~$h$
``recursively,'' i.e., the definition, specifically the second
equation, involves $h$ itself on the right-hand-side. It is
``primitive'' because in defining $h(x+1)$ we only use the
value~$h(x)$, i.e., the immediately preceding value. This is the
simplest way of defining a function on~$\Nat$ recursively.

We can define even more fundamental functions like addition and
multiplication by primitive recursion. In these cases, however, the
functions in question are $2$-place. We fix one of the argument
places, and use the other for the recursion. E.g, to define
$\Add(x, y)$ we can fix~$x$ and define the value first for $y=0$
and then for $y+1$ in terms of~$y$. Since $x$ is fixed, it will appear
on the left and on the right side of the defining equations.
\begin{align*}
\Add(x,0) & =  x\\
\Add(x,y+1) & =  \Add(x,y)+1
\end{align*}
These equations specify the value of $\Add$ for all $x$
\emph{and}~$y$. To find $\Add(2,3)$, for instance, we apply the
defining equations for $x = 2$, using the first to find
$\Add(2,0) = 2$, then using the second to successively find
$\Add(2,1) = 2 + 1 = 3$, $\Add(2, 2) = 3 + 1 = 4$, $\Add(2,
3) = 4 + 1 = 5$.

In the definition of $\Add$ we used $+$ on the right-hand-side of
the second equation, but only to add~$1$. In other words, we used the
successor function $\Succ(z) = z+1$ and applied it to the previous value
$\Add(x,y)$ to define $\Add(x,y+1)$. So we can think of the
recursive definition as given in terms of a single function which we
apply to the previous value. However, it doesn't hurt---and sometimes
is necessary---to allow the function to depend not just on the previous
value but also on $x$ and~$y$. Consider:
\begin{align*}
  \Mult(x,0) & =  0 \\
  \Mult(x,y+1) & =  \Add(\Mult(x,y),x)
\end{align*}
This is a primitive recursive definition of a function $\Mult$ by
applying the function $\Add$ to both the preceding value
$\Mult(x,y)$ and the first argument~$x$. It also defines the
function~$\Mult(x,y)$ for all arguments $x$ and~$y$. For instance,
$\Mult(2,3)$ is determined by successively computing $\Mult(2,0)$,
$\Mult(2,1)$, $\Mult(2,2)$, and~$\Mult(2,3)$:
\begin{align*}
  \Mult(2,0) & = 0\\
  \Mult(2,1) & = \Mult(2,0+1) =
  \Add(\Mult(2,0), 2) = \Add(0, 2) = 2\\
  \Mult(2,2) & = \Mult(2,1+1) =
  \Add(\Mult(2,1), 2) = \Add(2, 2) = 4\\
  \Mult(2,3) & = \Mult(2,2+1) =
  \Add(\Mult(2,2), 2) = \Add(4, 2) = 6
\end{align*}

The general pattern then is this: to give a primitive recursive
definition of a function~$h(x_0, \dots, x_{k-1}, y)$, we provide two
equations. The first defines the value of $h(x_0, \dots, x_{k-1}, 0)$
without reference to~$h$. The second defines the value of $h(x_0,
\dots, x_{k-1}, y+1)$ in terms of $h(x_0, \dots, x_{k-1}, y)$, the
other arguments $x_0$, \dots,~$x_{k-1}$, and~$y$. Only the immediately
preceding value of~$h$ may be used in that second equation.  If we
think of the operations given by the right-hand-sides of these two
equations as themselves being functions $f$ and~$g$, then the general
pattern to define a new function~$h$ by primitive recursion is this:
\begin{align*}
  h(x_0, \dots, x_{k-1}, 0) & = f(x_0, \dots, x_{k-1})\\
  h(x_0, \dots, x_{k-1}, y+1) & =
  g(x_0, \dots, x_{k-1}, y, h(x_0, \dots, x_{k-1}, y))
\end{align*}  
In the case of $\Add$, we have $k=1$ and $f(x_0) = x_0$ (the
identity function), and $g(x_0, y, z) = z + 1$ (the $3$-place function
that returns the successor of its third argument):
\begin{align*}
  \Add(x_0, 0) & = f(x_0) = x_0\\
  \Add(x_0, y+1) & = g(x_0, y, \Add(x_0, y)) =
  \Succ(\Add(x_0, y))
\end{align*}
In the case of $\Mult$, we have $f(x_0) = 0$ (the constant
function always returning~$0$) and $g(x_0, y, z) = \Add(z,x_0)$
(the $3$-place function that returns the sum of its last and first
argument):
\begin{align*}
  \Mult(x_0,0) & =  f(x_0) = 0 \\
  \Mult(x_0,y+1) & =  g(x_0, y, \Mult(x_0,y)) =
  \Add(\Mult(x_0,y), x_0)
\end{align*}

\end{document}
