% Part: computability
% Chapter: recursive-functions
% Section: introduction

\documentclass[../../../include/open-logic-section]{subfiles}

\begin{document}

\olfileid{cmp}{rec}{int}
\olsection{Introduction}

In order to develop a mathematical theory of computability, one has to
first of all develop a \emph{model} of computability.  We now think of
computability as the kind of thing that computers do, and computers
work with symbols.  But at the beginning of the development of
theories of computability, the paradigmatic example of computation was
\emph{numerical} computation.  Mathematicians were always interested
in number-theoretic functions, i.e., functions $f\colon \Nat^n \to
\Nat$ that can be computed. So it is not surprising that at the
beginning of the theory of computability, it was such functions that
were studied.  The most familiar examples of computable numerical
functions, such as addition, multiplication, exponentiation (of
natural numbers) share an interesting feature: they can be defined
\emph{recursively}.  It is thus quite natural to attempt a general
definition of \emph{computable function} on the basis of recursive
definitions.  Among the many possible ways to define number-theoretic
functions recursively, one particulalry simple pattern of definition
here becomes central: so-called \emph{primitive recursion}.

In addition to computable functions, we might be interested in
computable sets and relations. A set is computable if we can compute
the answer to whether or not a given number is an !!{element} of the
set, and a relation is computable iff we can compute whether or not a
tuple $\tuple{n_1, \dots, n_k}$ is an !!{element} of the relation.  By
considering the \emph{characteristic function} of a set or relation,
discussion of computable sets and relations can be subsumed under that
of computable functions.  Thus we can define primitive recursive
relations as well, e.g., the relation ``$n$ evenly divides $m$'' is a
primitive recursive relation.

Primitive recursive functions---those that can be defined using just
primitive recursion---are not, however, the only computable
number-theoretic functions. Many generalizations of primitive
recursion have been considered, but the most powerful and
widely-accepted additional way of computing functions is by unbounded
search.  This leads to the definition of \emph{partial recursive
  functions}, and a related definition to \emph{general recursive
  functions}.  General recursive functions are computable and total,
and the definition characterizes exactly the partial recursive
functions that happen to be total.  Recursive functions can simulate
every other model of computation (Turing machines, lambda calculus,
etc.) and so represent one of the many accepted models of computation.


\end{document}
