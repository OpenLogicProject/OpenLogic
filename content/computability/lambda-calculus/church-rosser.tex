% Part: computability
% Chapter: lambda-calculus
% Section: church-rosser

\documentclass[../../../include/open-logic-section]{subfiles}

\begin{document}

\olfileid{cmp}{lam}{cr}
\olsection{The Church-Rosser Property}

\begin{thm}
\ollabel{thm:church-rosser}
Let $M$, $N_1$, and $N_2$ be terms, such that $M \red N_1$ and $M \red
N_2$. Then there is a term $P$ such that $N_1 \red P$ and $N_2 \red P$.
\end{thm}

\begin{cor}
Suppose $M$ can be reduced to normal form. Then this normal form is
unique.
\end{cor}

\begin{proof}
If $M \red N_1$ and $M \red N_2$, by the previous theorem there is a
term~$P$ such that $N_1$ and $N_2$ both reduce to~$P$. If $N_1$
and~$N_2$ are both in normal form, this can only happen if $N_1 = P =
N_2$.
\end{proof}

Finally, we will say that two terms $M$ and~$N$ are
\emph{$\beta$-equivalent}, or just \emph{equivalent}, if they reduce
to a common term; in other words, if there is some $P$ such that $M
\red P$ and $N \red P$. This is written $M \equiv N$. Using
\olref{thm:church-rosser}, you can check that $\equiv$ is an
equivalence relation, with the additional property that for every $M$
and~$N$, if $M \red N$ or $N \red M$, then $M \equiv N$. (In fact, one
can show that $\equiv$ is the \emph{smallest} equivalence relation
having this property.)

\end{document}

