% Part: model-theory
% Chapter: interpolation
% Section: introduction

\documentclass[../../../include/open-logic-section]{subfiles}

\begin{document}

\olfileid{mod}{int}{int}

\olsection{Introduction}

The interpolation theorem is the following result: Suppose
$\Entails !A \lif !B$. Then there is !!a{sentence} $!C$ such that
$\Entails !A \lif !C$ and $\Entails !C \lif !B$.  Moreover, every
!!{constant}, !!{function}, and !!{predicate} (other than $\eq$) in
$!C$ occurs both in $!A$ and~$!B$. The !!{sentence} $!C$ is called an
\emph{interpolant} of $!A$ and~$!B$.

The interpolation theorem is interesting in its own right, but its
main importance lies in the fact that it can be used to prove results
about definability in a theory, and the conditions under which
combining two consistent theories results in a consistent theory.  The
first result is known as the Beth definability theorem; the second,
Robinson's joint consistency theorem.

\end{document}
