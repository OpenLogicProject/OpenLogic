% Part: first-order-logic
% Chapter: lindstrom
% Section: ls-property

\documentclass[../../../include/open-logic-section]{subfiles}

\begin{document}

\olfileid{mod}{lin}{lsp}

\olsection{Compactness and L\"owenheim-Skolem Properties}

We now give the obvious extensions of compactness and
L\"owenheim-Skolem to the case of abstract logics. 

\begin{defn}
An abstract logic $\tuple{L, \models_L}$ has the \emph{Compactness
  Property} if each set $\Gamma$ of $L(\Lang{L})$-!!{sentence}s is
satisfiable whenever each finite $\Gamma_0 \subseteq \Gamma$ is
satisfiable.
\end{defn}

\begin{defn}
$\tuple{L, \models_L}$ has the \emph{Downward L\"owenheim-Skolem
  property} if any satisfiable $\Gamma$ has !!a{enumerable} model.
\end{defn}


The notion of partial isomorphism from
\olref[bas][pis]{defn:partialisom} is purely ``algebraic'' (i.e.,
given without reference to the !!{sentence}s of the language but only
to the constants provided by the !!{language}~$\Lang{L}$ of the
!!{structure}s), and hence it applies to the case of abstract
logics. In case of first-order logic, we know from
\olref[bas][pis]{thm:p-isom2} that if two !!{structure}s are partially
isomorphic then they are elementarily equivalent. That proof does not
carry over to abstract logics, for induction on !!{formula}s need not
be available for arbitrary $!E \in L(\Lang{L})$, but the theorem is
true nonetheless, provided the L\"owenheim-Skolem property holds.

\begin{thm}
\ollabel{thm:abstract-p-isom}
Suppose $\tuple{L, \models_L}$ is a normal logic with the
L\"owenheim-Skolem property. Then any two !!{structure}s that are
partially isomorphic are elementarily equivalent in $\tuple{L,
  \models_L}$.
\end{thm}

\begin{proof}
Suppose $\Struct{M} \simeq_p \Struct{N}$, but for some $!E$ also
$\Struct{M} \models_L !E$ while $\Struct{N} \not\models_L !E$. By the
Isomorphism Property we can assume that $\Domain{M}$ and $\Domain{N}$
are disjoint, and by the Expansion Property we can assume that $!E \in
L(\Lang{L})$ for a finite !!{language}~$\Lang{L}$. Let $\mathcal{I}$
be a set of partial isomorphisms between $\Struct{M}$ and
$\Struct{N}$, and with no loss of generality also assume that if $p
\in \mathcal{I}$ and $q \subseteq p$ then also $q \in \mathcal{I}$.

$\Domain{M}^{<\omega}$ is the set of finite sequences of !!{element}s
of~$\Domain{M}$. Let $S$ be the ternary relation over $\Domain{M}^{<\omega}$
representing concatenation, i.e., if $\mathbf{a}, \mathbf{b},
\mathbf{c} \in \Domain{M}^{<\omega}$ then $S(\mathbf{a}, \mathbf{b},
\mathbf{c})$ holds if and only if $\mathbf{c}$ is the concatenation of
$\mathbf{a}$ and $\mathbf{b}$; and let $T$ be the ternary relation
such that $T(\mathbf{a}, b, \mathbf{c})$ holds for $b \in M$ and
$\mathbf{a}, \mathbf{c} \in \Domain{M}^{<\omega}$ if and only if $\mathbf{a} =
a_1, \dots a_n$ and $\mathbf{c} = a_1, \dots a_n, b$.  Pick new
3-place !!{predicate}s $P$ and $Q$ and form the !!{structure}
$\Struct{M}^*$ having the universe $\Domain{M} \cup \Domain{M}^{<\omega}$, having
$\Struct{M}$ as a substructure, and interpreting $P$ and $Q$ by the
concatenation relations $S$ and $T$ (so $\Struct{M}^*$ is in the
!!{language} $\Lang{L} \cup \{ P, Q\}$).

Define $\Domain{N}^{<\omega}$, $S'$, $T'$, $P'$, $Q'$ and
$\Struct{N}^*$ analogously. Since by hypothesis $\Struct{M} \simeq_p
\Struct{N}$, there is a relation $I$ between $\Domain{M}^{<\omega}$
and $\Domain{N}^{<\omega}$ such that $I(\mathbf{a}, \mathbf{b})$
holds if and only if $\mathbf{a}$ and $\mathbf{b}$ are isomorphic and
satisfy the back-and-forth condition of
\olref[bas][pis]{defn:partialisom}.  Now, let $\Struct{M}$ be the
!!{structure} whose !!{domain} is the union of the !!{domain}s of
$\Struct{M}^*$ and $\Struct{N}^*$, having $\Struct{M}^*$ and
$\Struct{N}^*$ as sub!!{structure}s, in the !!{language} with one extra
binary !!{predicate}~$R$ interpreted by the relation~$I$ and
!!{predicate}s denoting the !!{domain}s $\Domain{M}^*$
and~$\Domain{N}*$.

\begin{figure}[h]
  \centering
  \begin{tikzpicture}[node distance=2cm, auto, thick, >=stealth']
    \draw [rounded corners] (0,0) -- (8,0) -- (8,4) -- (0,4) --  cycle;
    \draw (2,2) circle (0.5cm);
    \draw (2,2) circle (1.25cm);
    \draw (6,2) circle (0.5cm);
    \draw (6,2) circle (1.25cm);
    \path node at (0.75,3.5) {\large $\Struct{M}$};
    \path node at (2,2) {\large $\Struct{M}$};
    \path node at (6,2) {\large $\Struct{N}$};
    \path node at (3.5,1) {\large $\Struct{M}^*$};
    \path node at (7.5,1) {\large $\Struct{N}^*$};
    \node (Idom) at (2.8,2) {};
    \node (Irng) at (5.2,2) {};
    \draw[<->, bend left] (Idom) to node {\large $I$} (Irng) ;
  \end{tikzpicture} 
  \caption{The !!{structure}~$\Struct{M}$ with the internal
    partial isomorphism.}
\end{figure}

The crucial observation is that in the !!{language} of the
!!{structure}~$\Struct{M}$ there is a \emph{first-order} !!{sentence} $!D_1$
true in $\Struct{M}$ saying that $\Struct{M} \models_L !E$ and
$\Struct{N} \not\models_L !E$ (this requires the Relativization
Property), as well as a \emph{first-order} !!{sentence} $!D_2$ true in
$\Struct{M}$ saying that $\Struct{M} \simeq_p \Struct{N}$ via the
partial isomorphism~$I$. By the L\"owenheim-Skolem Property, $!D_1$
and $!D_2$ are jointly true in !!a{enumerable} model $\Struct{M}_0$
containing partially isomorphic substructures $\Struct{M}_0$ and
$\Struct{N}_0$ such that $\Struct{M}_0 \models_L !E$ and $\Struct{N}_0
\not\models_L !E$. But !!{enumerable} partially isomorphic !!{structure}s are
in fact isomorphic by \olref[bas][pis]{thm:p-isom1}, contradicting the
Isomorphism Property of normal abstract logics.
\end{proof}

\end{document}
