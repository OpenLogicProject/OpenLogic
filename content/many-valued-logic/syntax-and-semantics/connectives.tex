% Part: many-valued-logic
% Chapter: syntax-and-semantics
% Section: connectives

\documentclass[../../../include/open-logic-section]{subfiles}

\begin{document}

\olfileid{mvl}{syn}{con}

\olsection{Languages and Connectives}

Classical propositional logic, and many other logics, use a set supply
of \emph{propositional constants} and \emph{connectives}. For
instance, we use the following as primitives:
\begin{enumerate}
\tagitem{prvFalse}{The propositional constant for !!{falsity}~$\lfalse$.}{}
\tagitem{prvTrue}{The propositional constant for !!{truth}~$\ltrue$.}{}
\item The logical connectives:
  \startycommalist
  \iftag{prvNot}{\ycomma $\lnot$ (negation)}{}%
  \iftag{prvAnd}{\ycomma $\land$ (conjunction)}{}%
  \iftag{prvOr}{\ycomma $\lor$ (disjunction)}{}%
  \iftag{prvIf}{\ycomma $\lif$ (!!{conditional})}{}%
  \iftag{prvIff}{\ycomma $\liff$ (!!{biconditional})}{}%
\end{enumerate}
\iftag{defNot,defOr,defAnd,defIf,defIff,defTrue,defFalse,defEx,defAll}{%
In addition to the primitive connectives above, we also use symbols
defined as abbreviations, such as
\startycommalist
  \iftag{defNot}{\ycomma $\lnot$ (negation)}{}%
  \iftag{defAnd}{\ycomma $\land$ (conjunction)}{}%
  \iftag{defOr}{\ycomma $\lor$ (disjunction)}{}%
  \iftag{defIf}{\ycomma $\lif$ (!!{conditional})}{}%
  \iftag{defIff}{\ycomma $\liff$ (!!{biconditional})}{}%
  \iftag{defFalse}{\ycomma $\lfalse$ (!!{falsity})}{}%
  \iftag{defTrue}{\ycomma $\ltrue$ (!!{truth})}.}{}

The same connectives are used in many-valued logics as well. However,
it is often useful to include different versions of, say, conjunction,
in the same logic, and that would require different symbols to keep
the versions separate. Some many-valued logics also include
connectives that have no equivalent in classical logic. So, we'll be a
bit more general than usual.

\begin{defn}
  A \emph{propositional language} consists of a set $\Lang L$ of
  \emph{connectives}. Each connective $\star$ has an \emph{arity}; a
  connective of arity~$n$ is said to be \emph{$n$-place.}
  Connectives of arity~$0$ are also called \emph{constants};
  connectives of arity~$1$ are called \emph{unary}, and connectives of
  arity~$2$, \emph{binary}.
\end{defn}

\begin{ex}
  The standard language of propositional logic $\Lang L_0$ consists of
  the following connectives (with associated arities): 
  $\lfalse$~($0$)
  $\lnot$~($1$),
  $\land$~($2$),
  $\lor$~($2$),
  $\lif$~($2$). Most logics we consider will use this language. Some
  logics by tradition an convention use different symbols for some
  connectives. For instance, in product logic, the conjunction symbol
  is often $\odot$ instead of~$\land$. Sometimes it is convenient to
  add a new operator, e.g., the determinateness operator $\triangle$
  ($1$-place).
\end{ex}

\end{document}
