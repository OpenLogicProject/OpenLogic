% part: intuitionistic-logic
% chapter: propositions-as-types
% section: introduction

\documentclass[../../../include/open-logic-section]{subfiles}

\begin{document}

\olfileid{int}{pty}{int}

\olsection{Introduction}

Historically the lambda calculus and intuitionistic logic were
developed separately. Haskell Curry and William Howard independently
discovered a close similarity: types in a typed lambda calculus
correspond to formulas in intuitionistic logic in such a way that
!!a{derivation} of !!a{formula} corresponds directly to a typed lambda
term with that !!{formula} as its type. Moreover, beta reduction in
the typed lambda calculus corresponds to certain transformations of
!!{derivation}s.

For instance, !!a{derivation} of $!A \lif !B$ corresponds to a term
$\lambd[\typeof{x}{!A}][\typeof{N}{!B}]$, which has the function type
$!A \lif !B$. The inference rules of natural deduction correspond to
typing rules in the typed lambda calculus, e.g.,
\begin{prooftree}
  \AxiomC{$\Discharge{!A}{x}$}
  \DeduceC{$!B$}
  \DischargeRule{\Intro\lif}{x}
  \UnaryInfC{$!A \lif !B$}
  \DisplayProof\bottomAlignProof
  \quad\text{corresponds to}\quad
  \Axiom$x: !A \fCenter N: !B$
  \RightLabel{$\lambd$}
  \UnaryInf$ \fCenter \lambd[\typeof{x}{!A}][\typeof{N}{!B}] : !A \lif !B$
\end{prooftree}
where the rule on the right means that if $x$ is of type~$!A$ and $N$
is of type~$!B$, then $\lambd[\typeof{x}{!A}][N]$ is of type~$!A \lif
!B$.

The $\Elim\lif$ rule corresponds to the typing rule for composition
terms, i.e.,
\begin{prooftree}
  \AxiomC{$!A \lif !B$}
  \AxiomC{$!A$}
  \RightLabel{\Elim\lif}
  \BinaryInfC{$!B$}
  \DisplayProof
  \quad\text{corresponds to}\quad
  \Axiom$\fCenter P: !A \lif !B$
  \Axiom$\fCenter Q: !A$
  \RightLabel{app}
  \BinaryInf$ \fCenter \typeof{P}{!A \lif !B}\typeof{Q}{!A} : !B$
\end{prooftree}

If a \Intro\lif{} rule is followed immediately by a \Elim\lif{} rule,
the !!{derivation} can be simplified:
\begin{gather*}
  \AxiomC{$\Discharge{!A}{x}$}
  \DeduceC{$!B$}
  \DischargeRule{\Intro\lif}{x}
  \UnaryInfC{$!A \lif !B$}
  \AxiomC{}
  \DeduceC{$!A$}
  \RightLabel{\Elim\lif}
  \BinaryInfC{$!B$}
  \DisplayProof
  \quad\redone\quad
  \AxiomC{}
  \DeduceC{$!A$}
  \DeduceC{$!B$}
  \DisplayProof
\end{gather*}
which corresponds to the beta reduction of lambda terms
\[
(\lambd[\typeof{x}{!A}][\typeof{P}{!B}])Q \quad\redone\quad
\Subst{P}{Q}{x}.
\]

Similar correspondences hold between the rules for $\land$ and ``product''
types, and between the rules for $\lor$ and ``sum'' types.

This correspondence between terms in the simply typed lambda calculus
and natural deduction !!{derivation}s is called the ``Curry-Howard'',
or ``propositions as types'' correspondence.  In addition to
!!{formula}s (propositions) corresponding to types, and proofs to
terms, we can summarize the correspondences as follows:
\begin{center}
  \begin{tabular}{c c c}
    logic & program \\
    \hline
    proposition & type \\
    proof & term \\
    assumption & variable \\
    discharged assumption & bind variable \\
    not discharged assumption & free variable \\
    implication & function type \\
    conjunction & product type \\
    disjunction & sum type \\
    absurdity & bottom type \\
    \hline
  \end{tabular}
\end{center}

The Curry-Howard correspondence is one of the cornerstones of
automated proof assistants and type checkers for programs, since
checking a proof witnessing a proposition (as we did above) amounts to
checking if a program (term) has the declared type.
\end{document}
