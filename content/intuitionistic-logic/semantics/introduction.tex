% part: intuitionistic-logic
% chapter: semantics
% section: introduction

\documentclass[../../../include/open-logic-section]{subfiles}

\begin{document}

\olfileid{int}{sem}{int}

\olsection{Introduction}

No logic is satisfactorily described without a semantics, and
intuitionistic logic is no exception. Whereas for classical logic, the
semantics based on !!{valuation}s is canonical, there are several
competing semantics for intuitionistic logic. None of them are
completely satisfactory in the sense that they give an
intuitionistically acceptable account of the meanings of the
connectives.

The semantics based on !!{relational model}s, similar to the semantics
for modal logics, is perhaps the most popular one.  In this semantics,
!!{propositional variable}s are assigned to worlds, and these worlds
are related by an accessibility relation. That relation is always a
partial order, i.e., it is reflexive, antisymmetric, and
transitive.

Intuitively, you might think of these worlds as states of knowledge or
``evidentiary situations.''  A state~$w'$ is accessible from~$w$ iff,
for all we know, $w'$ is a possible (future) state of knowledge, i.e.,
one that is compatible with what's known at~$w$.  Once a proposition
is known, it can't become un-known, i.e., whenever $!A$ is known at~$w$
and~$Rww'$, $!A$ is known at~$w'$ as well. So ``knowledge'' is
monotonic with respect to the accessibility relation.

If we define ``$!A$ is known'' as in epistemic logic as ``true in all
epistemic alternatives,'' then $!A \land !B$ is known at~$w$ if in all
epistemic alternatives, both $!A$ and~$!B$ are known. But since
knowledge is monotonic and $R$ is reflexive, that means that $!A \land
!B$ is known at $w$ iff $!A$ and $!B$ are known at~$w$.  For the same
reason, $!A \lor !B$ is known at $w$ iff at least one of them is
known. So for $\land$ and $\lor$, the truth conditions of the
connectives coincide with those in classical logic.

The truth conditions for the conditional, however, differ from
classical logic. $!A \lif !B$ is known at~$w$ iff at no $w'$ with
$Rww'$, $!A$ is known without $!B$ also being known. This is not the
same as the condition that $!A$ is unknown or $!B$~is known
at~$w$. For if we know neither $!A$ nor $!B$ at~$w$, there might be a
future epistemic state~$w'$ with $Rww'$ such that at $w'$, $!A$ is
known without also coming to know~$!B$.

We know $\lnot !A$ only if there is no possible future epistemic state
in which we know~$!A$. Here the idea is that if $!A$ were knowable,
then in some possible future epistemic state~$!A$ becomes known. Since
we can't know $\lfalse$, in that future epistemic state, we would know
$!A$ but not know~$\lfalse$.

On this interpretation the principle of excluded middle fails. For
there are some $!A$ which we don't yet know, but which we might come
to know. For such !!a{formula}~$!A$, both $!A$ and~$\lnot !A$ are unknown, so
$!A \lor \lnot !A$ is not known. But we do know, e.g., that $\lnot(!A
\land \lnot !A)$. For no future state in which we know both $!A$ and
$\lnot !A$ is possible, and we know this independently of whether or
not we know~$!A$ or $\lnot !A$.

!!^{relational model}s are not the only available semantics for
intuitionistic logic. The topological semantics is another: here
propositions are interpreted as open sets in a topological space, and
the connectives are interpreted as operations on these sets (e.g.,
$\land$ corresponds to intersection).

\end{document}
