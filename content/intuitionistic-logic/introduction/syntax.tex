% part: intuitionistic-logic
% chapter: introduction
% section: syntax

\documentclass[../../../include/open-logic-section]{subfiles}

\begin{document}

\olfileid{int}{int}{syn}

\olsection{Syntax of Intuitionistic Logic}

The syntax of intuitionistic logic is the same as that for
propositional logic. In classical propositional logic it is possible
to define connectives by others, e.g., one can define $!A \lif !B$ by
$\lnot !A \lor !B$, or $!A \lor !B$ by $\lnot(\lnot !A \land \lnot
!B)$. Thus, presentations of classical logic often introduce some
connectives as abbreviations for these definitions. This is not so in
intuitionistic logic, with two exceptions: $\lnot !A$ can be---and
often is---defined as an abbreviation for $!A \lif \lfalse$. Then, of
course, $\lfalse$ must not itself be defined!{} Also, $!A \liff !B$
can be defined, as in classical logic, as $(!A \lif !B) \land (!B \lif
!A)$.

!!^{formula}s of propositional intuitionistic logic are built up from
\emph{!!{propositional variable}s} and the propositional
constant~$\lfalse$ using \emph{logical connectives}. We have:

\begin{enumerate}
\item !!^a{denumerable} set~$\PVar$ of !!{propositional variable}s $\Obj p_0$,
  $\Obj p_1$, \dots
  \item The propositional constant for !!{falsity}~$\lfalse$.
\item The logical connectives: $\land$
  (conjunction), $\lor$ (disjunction), $\lif$ (conditional)
\item Punctuation marks: (, ), and the comma.
\end{enumerate}

\begin{defn}[Formula]
\ollabel{defn:formulas} The set~$\Frm[L_0]$ of \emph{!!{formula}s} of
propositional intuitionistic logic is defined inductively as follows:
\begin{enumerate}
\item $\lfalse$ is an atomic !!{formula}.

\item Every !!{propositional variable}~$\Obj p_i$ is an atomic
  !!{formula}.

\item If $!A$ and $!B$ are !!{formula}s, then $(!A \land
  !B)$ is !!a{formula}.

\item If $!A$ and $!B$ are !!{formula}s, then $(!A \lor !B)$
  is !!a{formula}.

\item If $!A$ and $!B$ are !!{formula}s, then $(!A \lif !B)$
  is !!a{formula}.

\tagitem{limitClause}{Nothing else is !!a{formula}.}{}
\end{enumerate}
\end{defn}

In addition to the primitive connectives introduced above, we also use
the following \emph{defined} symbols: $\lnot$ (negation) and $\liff$
(!!{biconditional}).  Formulas constructed using the defined operators
are to be understood as follows:
\begin{enumerate}
\item $\lnot !A$ abbreviates $!A \lif \lfalse$.

\item $!A \liff !B$ abbreviates $(!A \lif !B) \land (!B
  \lif !A)$.
\end{enumerate}

Although $\lnot$ is officially treated as an abbreviation, we will
sometimes give explicit rules and clauses in definitions for~$\lnot$
as if it were primitive. This is mostly so we can state practice
problems.

\end{document}
